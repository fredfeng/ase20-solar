% This is samplepaper.tex, a sample chapter demonstrating the
% LLNCS macro package for Springer Computer Science proceedings;
% Version 2.20 of 2017/10/04
%
% \documentclass[sigconf,review,anonymous]{acmart}
\documentclass[sigconf]{acmart}
%
% \usepackage{subcaption} %% For complex figures with subfigures/subcaptions
                        %% http://ctan.org/pkg/subcaption

\copyrightyear{2020}
\acmYear{2020}
\setcopyright{rightsretained}
\acmConference[ASE '20]{35th IEEE/ACM International Conference on Automated Software Engineering}{September 21--25, 2020}{Virtual Event, Australia}
\acmBooktitle{35th IEEE/ACM International Conference on Automated Software Engineering (ASE '20), September 21--25, 2020, Virtual Event, Australia}\acmDOI{10.1145/3324884.3416646}
\acmISBN{978-1-4503-6768-4/20/09}

%% self-defined packages
\usepackage{xspace}
\usepackage{stmaryrd}
\usepackage{comment}

\usepackage{amsfonts}
\usepackage{amsmath}
\usepackage{amssymb}
\DeclareMathOperator*{\argmax}{arg\,max}
\DeclareMathOperator*{\argmin}{arg\,min}

\usepackage{xcolor}
\usepackage{listings}
\lstset{escapeinside={(*}{*)},mathescape}

\usepackage{multirow}
\usepackage{syntax} 

\usepackage{xparse}
\NewDocumentCommand{\codeword}{v}{%
% \texttt{\textcolor{blue}{#1}}%
  \texttt{#1}%
}
\lstset{language=Python,keywordstyle={\bfseries \color{blue}}}

\usepackage{algorithm}
\usepackage[noend]{algpseudocode}

\usepackage{threeparttable}
\usepackage{proof}
\usepackage{enumitem}
\usepackage{ulem}
\normalem

\usepackage{tikz}
\usetikzlibrary{positioning,shapes,arrows}
\usetikzlibrary{patterns}
\usepackage{pgfplots}
\newtheorem{vul}{Vulnerability}
% Used for displaying a sample figure. If possible, figure files should
% be included in EPS format.
%
% If you use the hyperref package, please uncomment the following line
% to display URLs in blue roman font according to Springer's eBook style:
% \renewcommand\UrlFont{\color{blue}\rmfamily}
\newcommand{\program}{P\xspace}
\newcommand{\abi}{I\xspace}
\newcommand{\query}{Q\xspace}
\newcommand{\lang}{\mathcal{LABIOMANCY}}
\newcommand{\toolname}{{\sc Solar}\xspace}
\newcommand{\oyente}{{\sc Oyente}\xspace}
\newcommand{\mythril}{{\sc Mythril}\xspace}
\newcommand{\teether}{{\sc teether}\xspace}
\newcommand{\rosette}{{\sc Rosette}\xspace}
\newcommand{\madmax}{{\sc Madmax}\xspace}
\newcommand{\contractfuzz}{{\sc ContractFuzzer}\xspace}
\newcommand{\pstate}{\Gamma}
\newcommand{\vulnerability}{\mathcal{V}}
\newcommand{\transition}{\mathcal{T}}
\newcommand{\contract}{\mathcal{C}}
\newcommand{\summary}{\mathcal{M}}
\newcommand{\pc}{\phi}
\newcommand{\etherscan}{{\sc Etherscan}\xspace}
\newcommand{\batchoverflow}{{\sc BatchOverflow}\xspace}
\newcommand{\reentrancy}{{\sc Reentrancy}\xspace}
\newcommand{\store}{\sigma}
\newcommand{\ppath}{\pi}
\newcommand{\assert}{\alpha}
\newcommand{\todo}[1]{{\color{red}{#1}}}
\newcommand{\eval}[1]{{[\![#1]\!]}}
\newcommand{\sumi}[1]{{\overline{#1}}}
\newcommand{\geval}[1]{{[\![#1]\!]}}
\newcommand{\peval}[1]{{[\![#1]\!]_\pstate}}
\newcommand{\reach}{\rightsquigarrow}

\newcommand{\hypo}{\mathcal{G}}
\newcommand{\trace}{\mathcal{R}}
\newcommand{\sketch}{\mathcal{S}}
\newcommand{\victim}{V}
\newcommand{\sample}{\Delta}
\newcommand{\comps}{\Upsilon}
\newcommand{\ex}{\mathcal{E}}
\newcommand{\worklist}{\mathcal{W}}
\newcommand{\prog}{\mathcal{P}}
\newcommand{\comp}{\mathcal{C}}
\newcommand{\tab}{\mathcal{T}}
\newcommand{\cmark}{\ding{51}}%
\newcommand{\xmark}{\ding{55}}%

\lstset{basicstyle=\footnotesize\ttfamily,breaklines=true,numbers=left,stepnumber=1}
\lstset{frame=bottomline}
\newcommand*\circled[1]{\tikz[baseline=(char.base)]{
            \node[shape=circle,draw,inner sep=0.3pt] (char) {#1};}}

\begin{document}
%
\title{Summary-Based Symbolic Evaluation for Smart Contracts}

\author{Yu Feng}
\email{yufeng@cs.ucsb.edu}
\affiliation{%
  \institution{University of California, Santa Barbara}
}
\author{Emina Torlak}
\email{emina@cs.washington.edu}
\affiliation{%
  \institution{University of Washington}
}
\author{Rastislav Bodik}
\email{bodik@cs.washington.edu}
\affiliation{%
  \institution{University of Washington}
}
%
%\titlerunning{Abbreviated paper title}
% If the paper title is too long for the running head, you can set
% an abbreviated paper title here
%
% \author{First Author\inst{1}\orcidID{0000-1111-2222-3333} \and
% Second Author\inst{2,3}\orcidID{1111-2222-3333-4444} \and
% Third Author\inst{3}\orcidID{2222--3333-4444-5555}}
%
% \authorrunning{F. Author et al.}
% First names are abbreviated in the running head.
% If there are more than two authors, 'et al.' is used.
%
% \institute{Princeton University, Princeton NJ 08544, USA \and
% Springer Heidelberg, Tiergartenstr. 17, 69121 Heidelberg, Germany
% \email{lncs@springer.com}\\
% \url{http://www.springer.com/gp/computer-science/lncs} \and
% ABC Institute, Rupert-Karls-University Heidelberg, Heidelberg, Germany\\
% \email{\{abc,lncs\}@uni-heidelberg.de}}
%
\begin{abstract}
% Smart contracts are programs running on top of blockchain platforms. They
% interact with each other through well-defined interfaces to perform financial
% transactions in a distributed system with no trusted third parties. But these
% interfaces also provide a favorable setting for attackers, who can exploit
% security vulnerabilities in smart contracts.
%to achieve financial gain. 

This paper presents \toolname, a system for automatic synthesis of adversarial
contracts that exploit vulnerabilities in a victim smart contract.
To make the synthesis tractable, we introduce a query language as well as \emph{summary-based
symbolic evaluation}, which significantly reduces the number of instructions
that our synthesizer needs to evaluate symbolically, without compromising the
precision of the vulnerability query. We encoded common
vulnerabilities of smart contracts and evaluated
\toolname on the entire data set from \etherscan. 
%with $>$25K smart contracts. 
Our experiments demonstrate the benefits of summary-based symbolic evaluation and
show that \toolname outperforms state-of-the-art smart contracts analyzers,
\teether, \mythril, and \contractfuzz, in terms of running time, precision, and soundness.\looseness=-1
\end{abstract}

\maketitle              % typeset the header of the contribution
%

%% Abstract
%% Note: \begin{abstract}...\end{abstract} environment must come
%% before \maketitle command

%
%
%

% !TEX root =  main.tex
\section{Introduction}\label{sec:intro}
Smart contracts are programs running on top of blockchain platforms such as
Bitcoin~\cite{bitcoin} and Ethereum~\cite{ethereum}.  They interact with each
other to perform effective financial transactions in a distributed system
without the intervention from trusted third parties (e.g., banks). A smart
contract is written in a high-level programming language (e.g.,
Solidity~\cite{solidity}), and it is typically comprised of a unique address,
persistent storage holding a certain amount of cryptocurrency (i.e., Ether in
Ethereum), and a set of functions that manipulate the persistent storage to
fulfill credible transactions without trusted parties. For contract-to-contract
interaction, some functions are public and callable by other contracts. Thanks
to the expressiveness afforded by the high-level programming languages and the
security guarantees from the underlying consensus protocol, smart contracts have
shown many attractive use cases, and their number has skyrocketed, with over 45
million~\cite{etherscan} instances covering financial products, online gaming,
real estate~\cite{case1}, shipping, and logistics~\cite{case2}.

Because all smart contracts deployed on a blockchain are freely accessible
through their public methods, any functional bugs or vulnerabilities inside the
contracts can lead to disastrous losses, as demonstrated by recent
attacks~\cite{attack1,attack2,attack3,attack4}. For instance, the code (simplified) in
Figure~\ref{fig:motivate} illustrates the notorious \reentrancy attack~\cite{attack1}. When the
victim program (3) issues a money transaction
to the attacker (2), it implicitly triggers the attacker's callback method, 
which invokes the victim's method (i.e., \texttt{withdraw}) again to make
another transaction without updating the victim's balance. The attack maliciously extracted tokens
from the victim and led to
a financial loss of \$150M in 2016. To make things worse, smart contracts are
immutable---once they are deployed, fixing their bugs is extremely difficult due
to the design of the consensus protocol.\looseness=-1

% Improving robustness of smart contracts is thus a pressing practical problem.
% It is also an active area of research, with several contract analysis 
% tools~\cite{oyente,securify,contractfuzzer,madmax,zeus,teether} developed in the
% past few years. However, these tools either soundly overapproximate the
% execution of smart contracts and report warnings~\cite{securify,madmax} that
% cannot be exploited in reality, or they precisely
% enumerate~\cite{teether,contractfuzzer,oyente} \emph{concrete traces} of smart
% contracts, so cannot scale their analyses to large programs.

% \todo{Improving robustness of smart contracts is thus a pressing practical problem. Unsurprisingly,
% a complex vulnerability like \reentrancy typically involves interactions between multiple contracts, 
% which requires an analyzer to \emph{precisely} model the inter-contracts communication 
% and reason about the execution in a \emph{precise} and \emph{scalable} way.
% But existing tools either soundly overapproximate the
% execution of smart contracts and report warnings~\cite{securify,madmax} that
% cannot be exploited in reality, or they precisely
% enumerate~\cite{teether,contractfuzzer,oyente} \emph{concrete traces} of smart
% contracts, so cannot scale their analyses to large programs.}

Improving robustness of smart contracts is thus a pressing practical problem. Unsurprisingly,
 a complex vulnerability like \reentrancy typically involves interactions between multiple contracts, 
 which requires an analyzer to model the inter-contracts communication 
 and reason about the execution in a \emph{precise} and \emph{scalable} way.
 But existing tools either aggressively \emph{overapproximate} the execution a smart contract and report warnings~\cite{securify,madmax} that do not correspond to 
 feasible paths and therefore cannot be exploited, or they precisely
 enumerate~\cite{teether,contractfuzzer,oyente} \emph{concrete traces} of a smart
 contract, so cannot scale to large programs with many paths.

% As a result, it is vital to develop precise and scalable analysis techniques to 
% ensure the safety and robustness of smart contracts, and identify potential 
% \emph{malicious attacks} before deployment. But precise analysis is especially 
% challenging in this setting because it depends on precisely modeling a 
% contract's interactions with other parties on the blockchain. 
% Without a precise  interaction model,  blindly applying traditional bug finding and 
% verification techniques to smart contracts leads to a high false positive rate
% and vulnerabilities that are infeasible to exploit. 

% \begin{figure}
%     % \centering
%     \begin{subfigure}[b]{0.4\textwidth}
%     \begin{lstlisting}
% contract Victim {
%   mapping (address => uint) public balances;

%   function withdraw() public {
%     uint amount = balances[msg.sender];
%     //call withdraw again
%     msg.sender.call.value(amount)(); 
%     balances[msg.sender] = 0;
%   }
% }
%     \end{lstlisting}
%     \caption{The Vulnerable Program}
%       \label{fig:dao-vic}
%     \end{subfigure} \\
%     \begin{subfigure}[b]{0.4\textwidth}
%     \begin{lstlisting}      
%   contract Attacker {
%     ...
%     function () payable {
%       Victim v;
%       v.withdraw();
%     }
%   }
% \end{lstlisting}
%     \caption{The Attack Program}
%       \label{fig:dao-attack}
%     \end{subfigure}
% \caption{Reentrancy Attack \label{fig:intro-example}}
% \end{figure}
% \begin{figure}
%   \centering
%   \includegraphics[scale=0.26]{reentracy.pdf}
% \caption{Reentrancy Attack \label{fig:intro-example}}
% \label{fig:batchcode}
% \end{figure}



% This paper presents \toolname, a tool that uses \emph{program synthesis} to
% automatically generate adversarial smart contracts (i.e., attack programs),
% which exploit common vulnerabilities in victim contracts. \todo{Our design choice is based on two key insights. First, an effective way to test the security properties of a smart contract is to construct \emph{real} exploits triggered by attack contracts, which faithfully simulates the untrusted execution environment on blockchains. Second, a security analyst can easily initialize boilerplate code (e.g., constructor of the victim with its address) for the basic communication channel between the attack and victim contracts, and then leverage constraint solvers to \emph{synthesize} an attack contract by efficiently exploring the search space defined by the victim's public interface.}
This paper presents \toolname, a new point in the design space of smart contract analysis tools that achieves an effective trade-off among expressiveness, precision, and scalability. \toolname provides the security analyst with a query language for expressing \emph{vulnerability patterns} 
that can be exploited in an attack, as well as an automatic engine for \emph{synthesizing} an attack program (if one exists) that exploits the given vulnerability. Our key insight is based on the observation that an attacker typically exploits the vulnerability by making a sequence of transitions (calls over public methods of the victim), in which storage states are preserved across different transitions. As shown in Section~\ref{sec:vul}, because most types of vulnerabilities can be overapproximated through assertions over storage variables, this insight motivates an effective summary-based symbolic evaluation technique where the summary of a method soundly models its side-effect over storage variables, which dramatically reduces the number of instructions that \toolname has to re-evaluate symbolically. As a result, \toolname is able to scale reasoning with better precision to large contracts that are out of reach of existing symbolic execution~\cite{teether,oyente} and fuzzing~\cite{contractfuzzer} tools. 
% Our engine employs a novel summary-based symbolic evaluation technique to scale precise reasoning to large contracts 
% that are out of reach of existing symbolic execution~\cite{teether,oyente} and fuzzing~\cite{contractfuzzer} tools. 
Furthermore, previous summarization techniques~\cite{AnandGT08,Godefroid07} rely on symbolic execution and 
can therefore lead to summaries that are exponential in program size. 
Our technique relies on Rosette~\cite{rosette}, a hybrid symbolic evaluator that combines symbolic execution and bounded model checking, 
to compute compact (i.e., polynomially-sized) and precise (i.e., encoding all feasible bounded paths) summaries at the procedure level. 
%In particular, the size of the summary is polynomial in the size of the procedure, and the summary precisely encodes 
%the meaning of {all} (bounded) paths through that procedure. 
Using these summaries, \toolname can perform precise all-paths analysis of 
 a given contract
while symbolically executing significantly fewer paths than Rosette alone.\looseness=-1

To use our tool, a
security analyst expresses a target vulnerability query (e.g., the reentrancy
vulnerability) as a declarative specification.  
\toolname then \emph{synthesizes} an attack program that exploits the victim's public
interface to satisfy the vulnerability query. Given this problem, a naive
approach is to enumerate all possible candidate programs and then symbolically
evaluate each of them to check if it satisfies the query. While precise, the
naive approach fails to scale to realistic contracts. 
%To tackle this challenge,
%we employ a novel \emph{summary-based symbolic evaluation}, which enables
%\toolname to both find real attacks and scale to large programs. 
\begin{figure}
  \centering
  \includegraphics[scale=0.40]{motivate.pdf}
\caption{Sample contracts to show the Reentrancy attack.}
\label{fig:motivate}
\end{figure}
% Fig~\ref{fig:overview} shows an overview of our approach. Given the public
% methods provided by the Application Binary Interface (ABI) of a smart contract,
% our system first \emph{symbolically} evaluates each method and generates a
% summary that soundly records the method's side-effects on the storage as well as
% other global state of the Blockchain. 
Even with summarization, the search space
is still too large for brute-force enumeration. To address this issue, we
partition the search space by case splitting on the range of symbolic variables,
which allows us to simultaneously explore multiple attack programs using Rosette's 
SMT-based symbolic evaluation engine~\cite{rosette}. 
% \toolname further reduces
% the search space by pruning infeasible candidates early, using their symbolic encoding to 
% quickly check for the absence of potentially exploitable paths. After that, our tool symbolically evaluates each
% remaining candidate to check if any of them satisfies the vulnerability query.
% If so, the candidate is returned as a potential exploit.\looseness=-1

%  we present an efficient and precise approach for 
% reasoning about security properties of smart contracts. Given a contract under 
% test, our tool will synthesize an \emph{attack program} (i.e., adversarial smart contract) that 
% exploits a vulnerability expressed in our query language. 
% The attack program models a feasible set of interactions between 
% the contract under testing and the outside world, making our technique precise. 
% The technique is also complete within a user-provided bound on the size of the attack program.

% To use our technique for finding vulnerabilities in a contract, 
% a security analyst provides its implementation (binary or source), 
% the Contract Application Binary Interface (or just ``ABI" for short) 
% listing the contract's public methods that can be accessed by a third-party contract, 
% and a vulnerability query $\query$ in the form of first-order formula over 
% \emph{program states}. For instance, the Reentracy vulnerability in Figure~\ref{fig:intro-example} can be expressed as the
% following query: synthesize an attack program such that: 1) the program can invoke a sequence of public APIs provided by 
% the victim contract. 2) On executing the attack program, there exists a trace such that it contains 
% \emph{at least two consecutive} \texttt{calls} before a store operation. For instance, the attack program in 
% Figure~\ref{fig:dao-attack} can generate a trace like the following: \texttt{call call call store}
% which eventually leads to a DAO attack. 
%  \todo{Can we add a sentence here about the expressive power of the query language? What kinds of attacks can it capture?}

% Given these inputs, our technique iteratively enumerates \emph{sketches}~\cite{sketch-paper} of possible attack programs. 
% Each sketch $\sketch$ is composed of method calls to the public methods from the ABI, where   
% the arguments of all method calls are \emph{symbolic}.
% The technique then evaluates the sketch $\sketch$ using an off-the-shelf symbolic virtual machine~\cite{rosette}, 
% obtaining a symbolic program state $\pstate$ that encodes all reachable states 
% of $\sketch$. Since the original vulnerability query $\query$ is written in a formula over program 
% state $\pstate$, we leverage an SMT solver to check if there exists a concrete 
% interpretation $I_b$ for $\pstate$ where $\query$ holds. If the answer is yes,  
% we obtain a concrete attack program from $I_b$. Otherwise, we try other 
% candidate sketches (up to a user-specified bound on attack program size). 
% \todo{Can we add a sentence here describing the key technical novelty: is it the sketch enumeration algorithm or some other way to carefully control the search space to ensure scalability?}

We have evaluated \toolname on the entire data set ($>$25K) from
\etherscan~\cite{etherscan}, showing that our tool is expressive, efficient, and
effective. \toolname's query specification language is expressive in that it is
rich enough to encode common vulnerabilities found in the literature (such as the
Reentrancy attack~\cite{attack1}, Time manipulation~\cite{attack-time}, and
malicious access control~\cite{teether}), Security Best
Practices~\cite{best-practice}, as well as the recent \batchoverflow
Bug~\cite{attack-int} (CVE-2018–10299), which allows the attacker to create an
arbitrary amount of cryptocurrency. \toolname is efficient: on average it takes
only 8 seconds to analyze a smart contract from \etherscan, which is four times faster than \teether~\cite{teether} and two orders of magnitude
faster than \contractfuzz~\cite{contractfuzzer}. \toolname is also effective in
that it significantly outperforms state-of-the-art smart contracts
analyzers, namely, \teether, \mythril, and \contractfuzz, in terms of false positive and false negative rates. The approximate queries also enable \toolname to generate compact summaries and explore deeper vulnerabilities in exchange of a minor loss in precision.
% We have implemented our proposed technique as a tool called \toolname, a synthesizer 
% for automatically generating attack programs that exploit vulnerable smart contracts. 
% To demonstrate its effectiveness, we encode common vulnerabilities using our query 
% language and evaluate \toolname on a large data set from the Ethereum blockchain. 
% The results show that \toolname\ significantly outperforms two state-of-the-art
% smart contract analyzers, namely, \oyente and \contractfuzz, in terms of both false positives and false negatives. Furthermore, \toolname is an order of magnitude faster
% than \oyente and two orders of magnitude faster than \contractfuzz.

%\paragraph{Contributions.} % <--- This looks odd ...
% \begin{figure*}
%   \centering
%   \includegraphics[width=\textwidth]{overview-ex.pdf}
% \caption{Overview of \toolname}
% \label{fig:overview}
% \end{figure*}
In summary, this paper makes the following contributions:\looseness=-1
\begin{itemize}
% \item We formalize the problem of finding smart contracts vulnerabilities as the problem 
% of synthesizing an attack program, which allows us to precisely model all possible interactions provided by the public ABI
% of smart contracts.
% \item  We present a simple but general query language for describing common vulnerabilities over program states in smart contracts.
% \item We develop an efficient synthesis algorithm and symbolic reasoning system for precisely 
% synthesizing adversarial contracts that exploits vulnerabilities specifying by the queries.
% \item  We implement the proposed ideas in a  synthesis tool called \toolname and demonstrate its effectiveness on a large data set from the Ethereum blockchain.
\item We formalize the problem of exploit generation as a program synthesis problem 
and provide a query language for expressing common vulnerabilities in smart contracts as declarative
specifications (Section~\ref{sec:vul}).
\item We propose a new summary-based symbolic evaluation technique for smart contracts that significantly reduces the number of paths that \toolname has to execute symbolically (Section~\ref{sec:sum}).
\item We develop an efficient attack synthesizer based on the summary-based symbolic 
evaluation, which incorporates a novel combination of search space partitioning, parallel
symbolic execution, and early pruning based on the semantics of
candidate programs (Section~\ref{sec:parallel}).
\item We perform a systematic evaluation of \toolname on the entire data set
from \etherscan. Our experiments demonstrate the substantial benefits of our
technique and show that \toolname outperforms three state-of-the-art smart
contracts analyzers in terms of running time, precision, and soundness
(Section~\ref{sec:eval}).
\end{itemize}

% !TEX root =  main.tex
\section{Background}\label{sec:background}
% This section briefly reviews the background on blockchains and smart contracts.
We first review necessary background on smart contracts.

% \subsection{Blockchain and Ethereum}
% Blockchain, invented by Satoshi Nakamoto in 2008, is a distributed public ledger that stores transactions between different parties. A blockchain is comprised of a growing list of blocks, each of which contains the hash of the previous block, a timestamp when the block is appended, and transaction value. Due to the decentralized consensus protocol, each block is inherently resistant to modification once it is created.\looseness=-1 

% While Satoshi's original blockchain proposes a peer to peer e-cash system that offers secure transactions, the Ethereum~\cite{yellowpaper} blockchain provides a more powerful distributed computing platform that can execute custom code in the form of \emph{smart contracts}. In addition to the crypto tokens (i.e., Ether) that are transferred among parties during a transaction, Ethereum also implements a \emph{gas} scheme (explained in Section~\ref{sec:abi}) to incentivize \emph{miners} who perform the computationally expensive creation of new blocks. 
% Gas is simply a small amount of cryptocurrency paid to the miners to perform the computation. A user specifies the gas along with the executable code (i.e., smart contracts) during a transaction.

% \subsection{Smart Contract}
\paragraph{Smart Contract.}Smart contracts are programs that are stored and
executed on the blockchain. They are created through the transaction system on
the blockchain and are immutable once deployed. Each smart contract is
associated with a unique 160-bit address; a private persistent storage; a
certain amount of cryptocurrency, expressed as a balance (i.e., Ether in
Ethereum) held by the contract; and a piece of executable code that fulfills
complex computations to manipulate the storage and balance. The code is
typically written in a high-level Turing-complete programming language such as
Serpent~\cite{serpent}, Vyper~\cite{vyper}, and Solidity~\cite{solidity}, and
then compiled to the Ethereum Virtual Machine (EVM) bytecode~\cite{yellowpaper},
a low-level stack-based language. For instance, Figure~\ref{fig:motivate} shows
two smart contracts written in the Solidity programming language~\cite{solidity}.

% \subsection{ABI and Transactions}\label{sec:abi}
\paragraph{Application Binary Interface.}
In the Ethereum ecosystem, smart contracts communicate with each other using the
Contract Application Binary Interface (ABI), which defines the signatures of
public functions provided by the hosted contract. While ABI offers a flexible
mechanism for communication, it also creates an attack surface for exploits that
use  the ABI of a given smart contract. 
%We will elaborate on this in the following section. For instance, Figure~\ref{fig:attack-abi} shows the ABI for the smart contract in Figure~\ref{fig:attack-vic}.
% \begin{table}[]
% \centering
% \begin{tabular}{|l|l|}
% \hline
% ...         & ...                                                                \\ \hline
% From:       & 0x7d5c8c59837357e541bc7d87dee53fcbba55ba65                         \\ \hline
% To:         & 0x8811fffcfc266844e8c36418389f7cda76c77ab7 \\ \hline
% Value:      & 0.05 Ether                                                         \\ \hline
% Gas Limit:  & 31602                                                              \\ \hline
% Input Data: & 0x687474703a2f2f6c6f63616c686f73743a38353435                       \\ \hline
% \end{tabular}
% \caption{A sample transaction~\cite{sample-trans} obtained from \etherscan}
% \label{fig:trans-sample}
% \end{table}

% All interactions between smart contracts are fulfilled 
% by transactions. 
% Table~\ref{fig:trans-sample} shows a sample transaction obtained
% from \etherscan. Here, the important fields are \texttt{From},
% \texttt{To}, \texttt{Value}, \texttt{Gas Limit}, and \texttt{Input Data}. In 
% particular, \texttt{From} and \texttt{To} represent the sender and recipient, respectively.
% \texttt{Value} denotes the amount transferred from one smart contract to another. 
% \texttt{Input Data} contains the function's signature (obtained from the ABI) and its arguments. 
% Finally, the \texttt{Gas Limit} field specifies the amount of cryptocurrency which a miner 
% gets for conveying the transaction. The Ethereum protocol~\cite{yellowpaper} defines
% the gas cost for each bytecode instruction. For instance, an integer division operation 
% costs 5 units of gas while a store operation on the storage can cost up to 20000.
% As we will see in Section~\ref{sec:eval}, 
% the gas mechanism plays a key role in several different types of vulnerabilities.


% \subsection{Threat Model}
\paragraph{Threat Model.} To synthesize an adversarial contract,  we assume that
we can obtain the victim contract's bytecode and the ABI specifying its public
methods. To confirm an adversarial contract is indeed an exploit, we must also
be able to invoke public methods by submitting transactions over the Ethereum
Blockchain. These requirements are easy to satisfy in practice.

\section{Overview}\label{sec:overview}
In this section, we give an overview of our approach with the aid of a motivating example.

\subsection{Smart Contract Vulnerabilities}\label{sec:overview-vul}

A security analyst, Alice, can specify various types of vulnerabilities that may appear
in a smart contract. 
% For instance, a \emph{Reentrancy vulnerability}~\cite{attack1} 
% occurs when an attacker's previous invocation is allowed to make new calls to the 
% victim contract before the previous execution is complete. This means that if the 
% call involves money transactions, the attacker can repeatedly 
% trigger many transactions until the current procedure runs out of gas. 
% A \emph{Timestamp dependence vulnerability}~\cite{attack-time}, on the other hand, 
% happens when a transaction relies on a certain timestamp, which allows malicious miners to gain advantage by choosing a suitable timestamp. 
For instance, Figure~\ref{fig:motivate} shows a simplified example of a
\reentrancy attack. The \texttt{withdraw} function does two
steps: \circled{1} send a given amount of Ether to the caller, and \circled{2}
update the storage state to reflect the new balance. At any point, the total
amount of balances of the victim and attacker should remain the same (i.e.,
$B_v+B_a=C$). However, since \circled{1} happens before updating the state in
\circled{2}, an attacker can re-enter the \texttt{withdraw} function again
through the anonymous callback function triggered by \circled{1}. As a result,
the execution of the attack program can lead to an inconsistent state (i.e.,
$B_v' + B_a' > C$), which enables the attacker to extract a large amount of
Ether from the victim.\footnote{Ethereum's gas mechanism ensures that this
callback loop terminates.}\looseness=-1 

To automatically generate exploits for the \reentrancy vulnerability, Alice
first specifies a \emph{query} that \emph{characterizes} the semantics of
\reentrancy. As shown in the lower part of Figure~\ref{fig:motivate}, the attack
can be summarized using a sequence of key statements between the victim and the
attacker, i.e., two or more \texttt{call} instructions followed by a
\texttt{store} operation, which can be expressed using the first-order
formula in Figure~\ref{fig:motivate}.

% This section uses the most recent \batchoverflow 
% vulnerability (CVE-2018–10299)~\cite{attack-int} as a motivating example.
% Exploits due to this vulnerability have resulted in the creation of 
% trillions of \emph{invalid} Ethereum Tokens in 2018~\cite{batch-news}, causing major exchanges to temporary halt until all tokens could be reassessed.
% As shown in Figure~\ref{fig:batchcode}, the \texttt{batchTransfer} function 
% performs a multiplication that can overflow 256 bits, 
% which results in a small value that passes the check at line 12 and 
% further transfers a large amount of tokens to the attacker (line 20).\looseness=-1

% \begin{lstlisting}[xleftmargin=.2\textwidth]      
% $\exists\ arg_0, arg_1, r_1, r_2, r_3, call$ 
% (&& (= $r_3$ ($\otimes$ $r_1$ $r_2$)) 
%     (> $\geval{r_2}$ $\geval{r_3}$)
%     (interfere? $r_2$ call.value) 
%     (interfere? $arg_0$ call.addr) 
%     (interfere? $arg_1$ call.value)) 
% where $\otimes \in${+,$\times$}
% \end{lstlisting}


% \begin{figure}
%     \begin{lstlisting}[xleftmargin=0em,linewidth=6cm]  
% $\exists\ arg_0, arg_1, r_1, r_2, r_3, call$ 
% (&& (= $r_3$ ($\otimes$ $r_1$ $r_2$)) 
%     (> $\geval{r_2}$ $\geval{r_3}$)
%     (interfere? $r_2$ call.value) 
%     (interfere? $arg_0$ call.addr) 
%     (interfere? $arg_1$ call.value)) 
% where $\otimes \in${+,$\times$}
%       \end{lstlisting}
%   \caption{Query for the BatchOverFlow}
%     \label{fig:sum-gen}
%   \end{figure} 
%   \begin{figure}
%     \begin{lstlisting}[xleftmargin=1em,linewidth=5.5cm]     
% batchTransfer($arg_0$, $arg_1$) {
%   ... 
%   $r_3$ = $r_1$ $\otimes$ $r_2$;
%   ... 
%   call(gas, addr, value, ...);
%   ...
% }
%       \end{lstlisting}
%     \caption{The key pattern of the BatchOverflow Vulnerability}
%     \label{fig:batchcode}
% \end{figure}
\begin{figure}[!t]
  \centering
  \includegraphics[scale=0.23]{batchoverflow.png}
\caption{An example to show the \batchoverflow attack.}
\label{fig:batchcode}
% \vspace{-0.2in}
\end{figure}
% \toolname's specifications are assertions expressed in the Racket 
% language~\cite{racket}. In particular, the \batchoverflow vulnerability
% can be expressed as the query in Figure~\ref{fig:batchcode}. To highlight 
% the key concepts in our query language,
% we also visualize this vulnerability pattern in Figure~\ref{fig:batchcode}. 
% Here, $arg_i$, $r_j$, and \texttt{call} represent function arguments, 
% registers, and the \texttt{CALL} instruction (to perform a transaction in Solidity), respectively. 
% We use $\geval{r_i}$ to denote the value 
% (either concrete or symbolic) in the register $r_i$.
% The \texttt{interfere?} function, which is defined
% in section~\ref{sec:problem}, checks the interference between two 
% expressions. The interference~\cite{non-interfere} 
% (denoted by an arrow in Figure~\ref{fig:batchcode}) in our system precisely
% captures the data- and control-dependency. For instance,   
% the vulnerability states that, there exists a \texttt{CALL} instruction
% for which the beneficiary (i.e., recipient's address) and value are controlled by the attacker (line 5, 6).
% Furthermore, the transaction's value is influenced by a register (line 4) used in an arithmetic operation that overflows (line 2,3).

% We leverage the constraint solver Z3. Yet we cannot
% simply pass our set of collected constraints as is, as the
% constraint solver is unaware of the special semantics of
% Keccak-256 results and symbolic-read objects.
Once Alice expresses the \reentrancy vulnerability, the next step is to
construct an attack to confirm that the vulnerability indeed exists in the
victim contract. Alice can leverage existing symbolic execution
tools~\cite{mythril,oyente,teether} to generate exploits for simple properties such as 
attack-control~\cite{teether}) in a \emph{single contract}. But for
complex vulnerabilities that require reasoning about interactions among multiple
contracts (e.g., attacker versus victim in \reentrancy or caller versus callee in
Parity Multisig~\cite{multisig}), existing tools provide either no 
support~\cite{teether} or very limited support that 
leads to high rates~\cite{oyente} of false positives and negatives (as shown in
Section~\ref{sec:oyente}). % due to the increasing search space imposed by multiple contracts. 
Yet Alice can easily initialize the boilerplate code
for basic interactions, like the ``attack template" on the left hand side of
Figure~\ref{fig:motivate}. What she needs is an efficient way to  
fill in the details of the attack program, which involves exploring the space 
of all programs that can be obtained by completing the template with the methods 
from the victim's interface.
% Doing so manually is challenging, however, because the analyst has to understand the semantics of the smart contract and simulate all possible interactions that an attacker may perform.  As a result, the analysis process is both tedious and error-pone.
% has to figure out the following details:
% \begin{itemize}
% \item Figure out a list of public methods from the Application Binary Interface (ABI)
% of the victim contract~\ref{fig:attack-abi}. 
% \item Construct attack candidates by trying all possible combinations of function
% calls up to size $K$. Note that even for this simple benchmark with only 16 public
% methods, there are already 65536 candidates of size 4!
% \item For each candidate from the previous step, she has to execute it and manually
% check if there is any trace satisfying the constraint of the vulnerability. Note that
% the number of traces will be exponential to the number of branches of each candidate.
% \end{itemize}
% Since the entire process is tedious and error-pone, as the vulnerabilities get more 
% complex and the size of the benchmarks become larger,
% it becomes more difficult to construct the attack programs manually.
% \begin{figure}
%     % \centering
%     \begin{subfigure}[b]{0.4\textwidth}
%     \begin{lstlisting}[escapechar=@]  
% contract PausableToken {
% bool flag = false; 

% function makeFlag(bool fg) { 
%  flag = fg; 
% } 

% function batchTransfer(address[] _receivers, uint256 _value) {
%     uint cnt = _receivers.length;
%     uint256 amount = @\textbf{uint256(cnt) * _value}@;
%     require(@\textbf{flag}@);
%     require(balances[msg.sender] @\textbf{>= amount)}@;

%     balances[msg.sender] = 
%       balances[msg.sender].sub(amount);
%     for (uint i = 0; i < cnt; i++) {
%       address recv = _receivers[i];
%       balances[recv] = 
%         balances[recv].add(_value);
%       @\textbf{Transfer(msg.sender, recv, _value)}@;
%     }
%     return true;
%   }
% }
% \end{lstlisting}
%     \caption{The Vulnerable Program}
%       \label{fig:attack-vic}
%     \end{subfigure} \\
%     \begin{subfigure}[b]{0.4\textwidth}
%     \begin{lstlisting}      
%   contract Attacker {
%     ...
%     function exploit() {
%       VulContract v;
%       v.makeFlag(true);
%       v.batchTransfer([0x123, 0x456], $2^{256} - 1$);
%     }
%   }
% \end{lstlisting}
%     \caption{An Attack Program}
%       \label{fig:attack-code}
%     \end{subfigure} \\
%     \begin{subfigure}[b]{0.4\textwidth}
%     \begin{lstlisting} 
%  {
%   "inputs": [
%   {"name": "_receivers", "type": "address[]"},
%   {"name": "_value", "type": "uint256"}
%   ],
%   "name": "batchTransfer", "type": "function"
%   ...
% },{ 
%  "inputs": [{"name": "fg", "type": "bool"}],
%  "name": "makeFlag", "type": "function"}
%   \end{lstlisting}
%       \caption{Contract Application Binary Interface (ABI) for the vulnerable contract in Fig~\ref{fig:attack-vic}}
%       \label{fig:attack-abi}
%     \end{subfigure}
% \caption{A running example to show the \batchoverflow Vulnerability\label{fig:overview-example}}
% \end{figure}

\subsection{\toolname}
\vspace{-0.04in}
\toolname helps automate this process by searching for attacks that exploit a
given vulnerability in a victim contract. The tool takes as input a potential
vulnerability $\vulnerability$ expressed as a declarative specification. If
$\vulnerability$ exists in the victim contract, \toolname\ automatically
synthesizes an \emph{attack program} that exploits $\vulnerability$. An attacker
interacts with a vulnerable contract through its public methods defined in the
ABI. Therefore, our goal is to construct an attack program that exploits the
victim's ABI and that contains at least one concrete trace where
$\vulnerability$ holds. 

To achieve this goal, \toolname  models 
the executions of a smart contract as \emph{state transitions} over registers, memory, and storage. 
The vulnerability $\vulnerability$ 
is expressed in Racket~\cite{racket} as a boolean predicate over these state transitions. 
The technical challenge addressed by \toolname is to efficiently search for an attack program where $\vulnerability$ 
holds. 

To illustrate the difficulty of this task, consider the problem of synthesizing an attack 
program that exploits 
the \batchoverflow vulnerability (CVE-2018–10299)~\cite{attack-int} in Figure~\ref{fig:batchcode}. 
The attack program performs a complex three-step 
interaction with the victim contract.  First, the attacker must set the storage variable 
\texttt{flag} to \texttt{true} to pass the check at line 12. Next, it needs to 
assign a large number to \texttt{v} that leads to an overflow at line 11. 
Finally, it specifies the attacker's address as the beneficiary of the transaction (line 17). 
Synthesizing this attack program involves discovering which methods to call, in what order, and with what arguments. 

%Given those requirements,
%one straightforward way is to use an off-the-shelf constraint solver to 
%generate a counter-example as the attack~\cite{oyente}. 
%However, since a constraint solver does not understand the special semantics 
%in smart contracts, such as the
%Keccak-256 hash functions and symbolic-read over storage, the naive approach will lead to poor coverage and fail to find
%the desired attack. To bypass this limitation, one way is to simulate an real attacker's behavior which constructs the 
%attack program through a sequence of calls to the public methods.

The naive approach to solving this problem is to generate 
all possible \emph{concrete programs} and explore the space of 
their \emph{concrete traces}. This approach suffers from two sources of exponential explosion. 
First, there are $O(n^k)$ concrete programs of length $k$ for a victim contract with $n$ public methods.  
Second, the number of concrete traces in each of these programs is exponential in the size of the 
program's global control-flow graph obtained by inlining all method calls. 

To address the trace explosion challenge, \toolname\ employs a novel summary-based 
symbolic evaluation technique presented in Section~\ref{sec:sum}. Intuitively,
this technique enables \toolname to preserve only those state transitions that 
are persistent across different transactions and are \emph{sufficient} to answer
the vulnerability query.\looseness=-1

To address the program explosion challenge, Section~\ref{sec:impl} introduces three additional 
optimizations. First, instead of exploring the space of concrete programs, we
leverage \rosette~\cite{rosette} to partition this space into a small set of 
\emph{symbolic programs} 
(Section~\ref{sec:rosette}). 
% Second, instead of eagerly exploring the space of
% symbolic programs,  we design a simple but effective \emph{early pruning}
% strategy that allows \toolname to prune \emph{infeasible} symbolic candidates
% before executing them (Section~\ref{sec:prune}). 
Second, instead of executing
each symbolic program \emph{sequentially}, we partition the search space by case
splitting on the range of symbolic variables, which enables \toolname to
simultaneously explore multiple symbolic candidates
(Section~\ref{sec:parallel}).

\section{Problem Formulation}\label{sec:problem}

% This section formalizes the semantics of smart contracts, shows how to express Smart Contract vulnerabilities, and defines what it means for a vulnerability to appear in a smart contract.\looseness=-1
This section formalizes the semantics of smart contracts, shows how to express smart contract vulnerabilities in \toolname, and 
defines the problem of synthesizing an attack contract that exploits a given vulnerability.%, and defines what they mean in a smart contract.\looseness=-1

\subsection{Smart Contract Language}\label{sec:lang}

Figure~\ref{fig:grammar} shows the core features of our intermediate language
for smart contracts. This language is a superset of the EVM language. It includes
standard EVM bytecode instructions such as assignment (\texttt{x := e}), memory operations
(\texttt{mstore,mload}), storage operations (\texttt{sstore,sload}), hash
operation (\texttt{sha3}), sequential composition (\texttt{$s_1;s_2$}),
conditional (\texttt{jumpi}) and unconditional jump (\texttt{jump}). It also
includes the EVM instructions specific to smart contracts: \texttt{transfer}
denotes all functions that send tokens between different addresses, \texttt{balance} accesses the current account
balance, and \texttt{selfdestruct} terminates a contract and transfers its
balance to a given address. Finally, our language extends EVM with features that
facilitate symbolic evaluation, including \emph{symbolic variables} (introduced
by \texttt{def-sym}) and \emph{symbolic expressions} (obtained by operating on
symbolic variables) whose concrete values will be determined by an off-the-shelf
SMT solver~\cite{NiemetzPreinerBiere-JSAT15}.

\begin{figure}[!t]
%   \begin{minipage}{1.0\linewidth}
    \setlength{\grammarparsep}{0em}
    \begin{grammar}
    \small
<var>  ::= \texttt{def-sym} id $\tau$  \ \ where 
$\tau\in\{\textbf{boolean}, \textbf{number}\}\quad$

<pc>   ::= <const> | <var>

<expr> ::= <const> | <var> | <expr> $\oplus$ <expr> \\
($\oplus\in\{+, -, \times, /, \vee, \wedge, ...\}$)

<stmt> ::=  <var> := <expr> 
  \alt <var> := \textbf{mload} <var> | \textbf{mstore} <var> <var>
  \alt <var> := \textbf{sload} <var> | \textbf{sstore} <var> <var>
  \alt <var> := \{\textbf{balance}, \textbf{gas}, \textbf{address} \}

<stmts> ::= <stmt> | <stmt>; <stmts> | \textbf{sha3} <var> <var>
\alt \textbf{jumpI} <pc> <expr> | \textbf{jump} <pc> | \textbf{no-op}
\alt \textbf{transfer} <var> <var> <...> | \textbf{selfdestruct} <var>

<param> ::= <var> 

<params> ::= <param> | <param>, <params> 

<prog> ::= $\lambda$<params>. <stmts> 
\end{grammar}
% \end{minipage}
% \vspace{-0.2in} 
\caption{Intermediate language for smart contract}
\vspace{-0.1in}
\label{fig:grammar}
\end{figure}


% \begin{figure}

  
% \begin{prooftree}
% \AxiomC{$s = \textbf{no-op}$}
% \LeftLabel{\texttt{no-op}}
% \UnaryInfC{$\pstate \vdash s: \pstate'[\text{pc++}]$}
% \end{prooftree}
% \vspace{0.05in}

% \begin{prooftree}
% \AxiomC{$s=(\textbf{jumpi } d \ e)$}
% \noLine
% \UnaryInfC{$\pstate \vdash e: v$}
% \AxiomC{$i=(v=0)?(pc+1):d$}
% \noLine
% \UnaryInfC{$\pstate' = \pstate[x\gets v,pc\gets i]$}
% \LeftLabel{\texttt{jmp}}
% \BinaryInfC{$\pstate \vdash s: \pstate'$}
% \end{prooftree}
% \vspace{0.05in}

% \begin{prooftree}
% \AxiomC{$s = (param := \textbf{def-sym}(e,\tau))$}
% \AxiomC{$v = |(e,\tau)|$}
% \noLine
% \UnaryInfC{$\pstate' = \pstate[param\gets v]$}
% \LeftLabel{\texttt{sym}}
% \BinaryInfC{$\pstate \vdash s: \pstate', v$}
% \end{prooftree}
% \vspace{0.05in}

% \begin{prooftree}
% \AxiomC{$s = (x := e)$}
% \AxiomC{$\pstate \vdash e: v$}
% \AxiomC{$\pstate' = \pstate[x\gets v]$}
% \LeftLabel{\texttt{assign}}
% \TrinaryInfC{$\pstate \vdash s: \pstate'[\text{pc++}], v$}
% \end{prooftree}
% \vspace{0.05in}

% \begin{prooftree}
% \AxiomC{$s = (x:=e_1 \oplus e_2)(\oplus \in \{+,-,/,\times\})$ }
% \noLine
% \UnaryInfC{$\pstate \vdash e_1: v_1$}
% \noLine
% \UnaryInfC{$\pstate \vdash e_2: v_2$}
% \AxiomC{$v=v_1 \oplus v_2$}
% \noLine
% \UnaryInfC{$\pstate'[x\gets v]$}
% \LeftLabel{\texttt{biop}}
% \BinaryInfC{$\pstate \vdash s: \pstate'[\text{pc++}], v$}
% \end{prooftree}
% \vspace{0.05in}

% \begin{prooftree}
% \AxiomC{$s = (s_1;s_2)$ }
% \AxiomC{$\pstate \vdash s_1: \pstate_1, v_1$}
% \noLine
% \UnaryInfC{$\pstate_1\vdash s_2: \pstate_2, v_2$}
% \LeftLabel{\texttt{seq}}
% \BinaryInfC{$\pstate \vdash s: \pstate_2, v_2$}
% \end{prooftree}
% \vspace{0.05in}

% \begin{prooftree}
% \AxiomC{$s = (x:=\textbf{sload } e)$ }
% \noLine
% \UnaryInfC{$\pstate \vdash e: \mu$}
% \AxiomC{$\pstate \vdash \mu: v$}
% \noLine
% \UnaryInfC{$\pstate'[x\gets v]$}
% \LeftLabel{\texttt{sload}}
% \BinaryInfC{$\pstate \vdash s: \pstate'[\text{pc++}],v$}
% \end{prooftree}
% \vspace{0.05in}

% \begin{prooftree}
% \AxiomC{$s = (\textbf{sstore } \mu \ e)$ }
% \noLine
% \UnaryInfC{$\pstate \vdash e: v$}
% \AxiomC{$\pstate'[\mu\gets v]$}
% \LeftLabel{\texttt{sstore}}
% \BinaryInfC{$\pstate \vdash s: \pstate'[\text{pc++}]$}
% \end{prooftree}
% \vspace{0.05in}

% \begin{prooftree}
% \AxiomC{$s = (r_l:=\textbf{call }(e_1 \ e_2 \ e_3))$ }
% \noLine
% \UnaryInfC{$\pstate \vdash e_1: v_1$}
% \noLine
% \UnaryInfC{$\pstate \vdash e_2: v_2$}
% \AxiomC{$\pstate \vdash e_3: v_3$}
% \noLine
% \UnaryInfC{$v=\textbf{call}_i(v_1,v_2,v_3)$}
% \noLine
% \UnaryInfC{$\pstate'[r_l\gets v]$}
% \LeftLabel{\texttt{call}}
% \BinaryInfC{$\pstate \vdash s: \pstate'[\text{pc++}], r_l$}
% \end{prooftree}
% \vspace{0.05in}

% \begin{prooftree}
% \AxiomC{$s = (x:= \textbf{sha3 }m \ e)$ }
% \noLine
% \UnaryInfC{$\pstate \vdash m: v_1$}
% \noLine
% \AxiomC{$\pstate \vdash e: v_2$}	
% \noLine
% \UnaryInfC{$v=\textbf{sha3}_i(v_1,v_2)$}
% \noLine
% \UnaryInfC{$\pstate'[x\gets v]$}
% \LeftLabel{\texttt{sha}}
% \BinaryInfC{$\pstate \vdash s: \pstate'[\text{pc++}],v$}
% \end{prooftree}
% \caption{Operational semantics for our language 
% in Figure~\ref{fig:grammar}}
% \label{fig:sem}
% \end{figure}

% We define the semantics of the language operationally, as 
% shown in Figure~\ref{fig:sem}. 
We define the operational semantics of each statement in Figure~\ref{fig:grammar} 
based on the standard defined by the EVM yellow paper~\cite{evm-yellow}. 
The semantics is lifted to work on symbolic values in the standard way~\cite{rosette}.
The meaning of a statement 
is given by a \emph{state transition} rule that specifies the statement's 
effect on the \emph{program state}. We define states and transitions as follows. 
%
\begin{definition}{{(\bf Program State)}}
The \emph{Program State} $\pstate$ consists of a stack $E$, memory $M$,
 persistent storage $S$, global properties (e.g., balance, address, timestamp)
 of a smart contract, and the program counter \texttt{pc}. We use $e_i$, $m_i$,
 and $\mu_i$ to denote variables from the stack, memory, and storage,
 respectively. 
\end{definition}
%
\noindent A program state also includes a model of the gas system in EVM, but we
omit this part of the semantics to simplify the presentation. If a state maps a
variable to a symbolic expression, we call it a \emph{symbolic state}.  
%
\begin{definition}{{(\bf State transition over statement $s$)}}
  A \emph{State Transition} $\transition$ over a statement $s$ is
  denoted by a judgment of the form $\pstate \vdash s: \pstate', v$. 
  The meaning of this judgment is the following: assuming we successfully execute $s$ under program 
  state $\pstate$, it will result in value $v$ and the new state is $\pstate'$. 
  %We use $\pstate \vdash s: \bot$ to indicate failure.
\end{definition}

% Most of the rules in Figure~\ref{fig:sem} specify the standard semantics of EVM 
% instructions. For example, the \texttt{biop} rule describes the meaning of binary operations: 
% it first looks up the values (concrete or symbolic) of the 
% operands in the current program state $\pstate$, applies the binary operator to those
% values (i.e., $v_1, v_2$), and then binds the result to the target variable, 
% increases the program counter, and produces a new program state $\pstate'$. 
% The  \texttt{sstore}, \texttt{sload}, \texttt{jmp}, and \texttt{seq} rules are also standard.


% The  \texttt{sym},  \texttt{sha3}, and  \texttt{call} rules, on the other hand, are 
% tailored for (efficient) symbolic evaluation. 
% The \texttt{sym} rule introduces symbolic values into the program state. 
% The construct $|(e,\tau)|$ denotes a fresh symbolic variable $e$ of type $\tau$,
% which is bound to the \texttt{def-sym} parameter in the new program state
% $\pstate'$. Here, we do not increase the program counter as the symbolic binding
% is not an EVM instruction.
% The \texttt{sha3} and \texttt{call} instructions are part of EVM, but we overapproximate their 
% semantics with  \emph{uninterpreted functions} to produce more tractable vulnerability queries. 

% The standard semantics of the \texttt{sha3} instruction is to obtain a memory 
% location by computing a Keccak-256 hash over a variable length data and its offset.

% However, applying hashing functions to symbolic arguments results in hard-to-solve queries. 
% The \texttt{sha3} rule therefore uses an uninterpreted function, denoted by $\texttt{sha3}_i$,  
% to model the original hash function. 
%In particular, we 
%first look up the values of the arguments, apply $\texttt{sha3}_i$ to them, 
%and bind the result to the variable on the left hand side.

% As mentioned earlier, the \texttt{call} instruction is used to initiate 
% a transaction with another contract, whose address is specified as an argument.
% The \texttt{call} rule uses an 
% uninterpreted function, denoted by $\texttt{call}_i$, to model the effect of the \texttt{call} instruction. 
% Note that the rule also records the return value of each \texttt{call} using a special variable 
% $r_l$ in $\pstate$, where $l$ is the location of the \texttt{call} command. 
% This handling of \texttt{call} instructions is key to our summary-based symbolic evaluation, as explained 
% in Section~\ref{sec:sum}.\looseness=-1

%Finally, the \texttt{seq} rule is used to compose the program state for multiple
%statements, and we recursively apply the rule of each individual statement 
%and return the program state of the last statement.

    % \begin{figure}
    %   \lstinputlisting[language=Java,linewidth=7cm]{EubChainIco.sol}
    %   \caption{A Smart Contract written in Solidity.}
    %   \label{fig:code-sum}
    % \end{figure}
\begin{example}
Figure~\ref{fig:sum-interp}a shows a smart contract written in Solidity. To analyze this 
contract, \toolname first translates it to the program in Figure~\ref{fig:sum-interp}b,
using the intermediate language in Figure~\ref{fig:grammar}. The resulting program is then 
evaluated symbolically in an environment $\pstate$ that binds \texttt{_amount} to a 
fresh symbolic number. % using \todo{the operational semantics of EVM bytecode defined by the yellow paper~\cite{evm-yellow}.}
% in Figure~\ref{fig:sem}. 
For instance, after executing line 2 in 
Figure~\ref{fig:sum-interp}b, register \texttt{r1} holds a symbolic value represented 
by $\pstate[\mathtt{\_amount}] - 1$. Since \toolname does not model the event 
system in Solidity, we turn the corresponding instructions (e.g., line 7 in Figure~\ref{fig:sum-interp}b) 
into \texttt{no-op}s.\looseness=-1
\end{example}

\begin{definition}{{(\bf Abstract execution trace)}}
An abstract execution trace $\trace$ contains a list of events (i.e., statements) that are of interest. Each event has an event type representing the type of statement, and a list
of attributes.
\end{definition}
    \begin{figure}[t!]
\begin{lstlisting}[title={(a) Solidity program}]
  require(_amount > 0);
  vesting.amount = _amount.sub(1);
  transfer(msg.sender,_to,vesting.amount);
  uint256 v1 = _amount - 15;
  uint256 wei = v1;
  uint t1 = vesting.startTime;
  emit VestTransfer(msg.sender, _to, wei, t1, _);
\end{lstlisting}
\begin{lstlisting}[title={(b) Symbolic evaluation}]
  assert(_amount > 0);
  r1 (*\textbf{:=}*) _amount - 1;
  (*\textbf{sstore}*)(vesting.amount, _amount - 1);
  (*\textbf{transfer}*)(msg.sender, _to, _amount - 1);
  r2 (*\textbf{:=}*) amount - 15;
  r3 (*\textbf{:=}*) amount - 15;
  r4 (*\textbf{:=}*) (*\textbf{sload}*)(vesting.startTime);
  (*\textbf{no-op}*);
\end{lstlisting}
\begin{lstlisting}[title={(c) Summary extraction}]
  $\sumi{sstore}(\mathtt{vesting.amount}, \pstate_S[\mathtt{\_amount}] - 1)@(\pstate_S[\mathtt{\_amount}]>0)$;
  $\sumi{transfer}(\pstate_S[\mathtt{msg.sender}], \pstate_S[\mathtt{\_to}], \pstate_S[\mathtt{\_amount}] - 1)@(\pstate_S[\mathtt{\_amount}]>0)$;
\end{lstlisting}
\begin{lstlisting}[title={(d) Summary interpretation}]
  if ($\pstate[\mathtt{\_amount}] > 0$) (*\textbf{sstore}*)$(\mathtt{vesting.amount}, \pstate[\mathtt{\_amount}] - 1)$;
  if ($\pstate[\mathtt{\_amount}] > 0$) (*\textbf{transfer}*)$(\pstate[\mathtt{msg.sender}], \pstate[\mathtt{\_to}], \pstate[\mathtt{\_amount}] - 1)$;
\end{lstlisting}
    \caption{\small From Standard to Summary-Based Symbolic Evaluation\looseness=-1}
    \label{fig:sum-interp}
    \end{figure}

% With slight abuse of notation, we write $\overset{\#}{\pstate}$ 
% to denote the statement that leads to program state $\pstate$.
% That is, $\overset{\#}{\pstate} = s$ if 
% \[
% \exists \pstate_1. \pstate_1 \vdash s:\pstate \]

\subsection{Smart Contract Vulnerabilities}\label{sec:vul}
We now describe how to express smart contract vulnerabilities in \toolname and what it means for a vulnerability to appear in a program.\looseness=-1


Figure~\ref{fig:query-lang} shows our  query language over program traces. In particular, a query consists of three parts: The \textbf{uses} block declares typed variables, which will be matches against variables or statements appearing in the program. The \textbf{matchses} block declares a subsequence of trace that satisfy the pattern. The \textbf{where} clause further refines the search criteria by imposing constraints over program trace. 
\paragraph{Query variables} Query variables in the \textbf{uses} block correspond to variables or statements in the program trace. Common variables include statements, storage variables, arguments, etc.

\paragraph{Statements} Statements in the query language correspond to events in the execution trace discussed in Section~\ref{sec:problem}. In particular, an event is of type record whose fields are properties of that event. Table~\ref{tbl:stmt-fields} lists the fields of some representative statements appearing in the query. Furthermore, \texttt{seqStmt} (i.e., a; b) specifies that event a happens before b. Finally, An event may be forbidden using the exclusion operator ``!''.

\paragraph{Conditional clauses} The criteria of a query can be further refined using the \emph{conditional clauses} in the \textbf{where} block. In particular, a conditional clause is a boolean expression whose sub-expressions are constants, query variables, fields of query variables, or custom predicate like \texttt{interfere} which we introduce later.

% Here, \texttt{arg}, \texttt{reg}, \texttt{mem}, and
% \texttt{store} are variables from function arguments, registers, memory, and
% storage, respectively. Furthermore, variables can also refer to global
% properties (e.g., balance, address, timestamp) of a smart contract. We use
% \geval{var} to denote the concrete or symbolic value held by \texttt{var}. The predicate
% \texttt{(interfere? var e)} determines whether \texttt{var} can interfere with
% \texttt{e}, as specified in Definition~\ref{def:interfere}. 
% The expression \texttt{(inst opcode var var ...)} represents an instruction. 
% For instance, \texttt{(inst call $v_1$ $v_2$ $v_3$ $l$)} denotes a call
% instruction at location $l$ where variables $v_1$, $v_2$, and $v_3$ represent
% operands that hold the gas, address, and value of the call. 
% %Note that we use location $l$ to encode the order of each instruction during symbolic evaluation.
% With slight abuse of notation, we will  use
% \texttt{inst.operand} to refer to the operand of an instruction \texttt{inst}.
% For instance, in the previous instruction, the symbolic expression held by $v_1$
% can be referenced as $call$.gas. Finally, we can express more complex queries by
% composing simple expressions with logical operators ($\neq, \vee, \wedge,
% \exists$, etc.). For queries that contain quantifiers, we use skolemization to make them 
% quantifier-free  (or reject them if they cannot be skolemized).\looseness=-1

% \begin{figure}[hbt!]
%   \setlength{\grammarparsep}{0em}
%     \begin{grammar}
%     <var>  ::=  <arg> | <reg> | <mem> | <store> | timestamp | ...  

%     <opcode> ::= call | jmp | store | ...  
  
%     <E> ::= <const> | \geval{var} | <var> | <E> $\oplus$ <E> | $\neg$<E>
%     | $\forall$<var>. <E> \\
%     | $\exists$<var>. <E> | (interfere? <var> <E>) \\
%     | (inst <opcode> <var> <var> ...) \\
%     ($\oplus\in\{+, -, >, =, \neq, \vee, \wedge, ...\}$)
%         \end{grammar}
%   \caption{Query language for \toolname}
%   \label{fig:query-lang}
%   \end{figure}

\begin{figure}[hbt!]
  \setlength{\grammarparsep}{0em}
    \begin{grammar}
    <query> ::= <uses declList;>  \\
    | <matches {seqStmt}> \\
    | <where cond> \\
    
    <declList> ::= <typeName id (,id)*>
    
    <typeName> ::= <id> 
    
    <stmt> ::= <transfer> | <sstore> | <jump> | <binaryExp> | <!stmt> ...
    
    <seqStmt> ::= <stmt> | <stmt;stmt>  
    
    <cond> ::=  <E> $\oplus$ <E> 
 ($\oplus\in\{+, -, >, \neq, \vee, \wedge, ...\}$)
 
   <E> ::= <const> | \geval{var} | <var> \\
   | <fieldAccess> | (interfere? <E> <E>)
   
   <var> ::= <local> | <argument>
   
   <fieldAccess> ::= <id.id>
   
   <id> ::= <A-Za-z>*

    \end{grammar}
  \caption{Query language for \toolname}
  \label{fig:query-lang}
  \end{figure}

\paragraph{Compilation of query} 
Once a query is given, \toolname automatically converts it into its corresponding FOL formulas through a syntax-directed translation. For queries that contain quantifiers, we use skolemization to make them quantifier-free  (or reject them if they cannot be skolemized).\looseness=-1

% \begin{definition}{{(\bf Vulnerability)}}
% A \emph{Vulnerability} $\vulnerability$ is a predicate over a set of 
% variables $V$ in the program state. 
% %where $V$ represents values in persistent storage, 
% %function arguments, and other vulnerability-specific 
% %values (timestamp, blockNumber, etc). 
% A vulnerability $\vulnerability$ appears in the program $P$ 
% if the execution of $P$ can reach a program state $\pstate'$ that satisfies $\vulnerability$: $\pstate' \models \vulnerability$.
% % \[
% %     \pstate' \models \vulnerability
% % \]
% \end{definition}

\begin{table}[]
\begin{tabular}{|l|l|}
\hline
\multicolumn{2}{|l|}{\textbf{Fields of transfer statement}} \\ \hline \hline
\texttt{sender}        & sender's address                   \\ \hline
\texttt{recipient}     & target's address                   \\ \hline
\texttt{loc}     & program counter of the statement                   \\ \hline
\texttt{gas}           & gas budget for the transfer                          \\ \hline
\texttt{amount}        & amount of tokens                   \\ \hline
\texttt{ret}        & return value of the statement                   \\ \hline \hline
\multicolumn{2}{|l|}{\textbf{Fields of jump statement}}     \\ \hline \hline
\texttt{condVar}     & condition variable of jump statement                     \\ \hline
\texttt{target}        & target address                     \\ \hline \hline
\multicolumn{2}{|l|}{\textbf{Fields of sstore statement}}    \\ \hline \hline
\texttt{name}           & name of storage variable           \\ \hline
\texttt{value}         & new value that is used             \\ \hline \hline
\multicolumn{2}{|l|}{\textbf{Fields of binary statement}}   \\ \hline \hline
\texttt{lhs}           & variable that is assigned          \\ \hline
\texttt{opcode}        & opcode of the binary statement     \\ \hline
\texttt{oprand1}       & the first operand                  \\ \hline
\texttt{oprand2}       & the second operand                 \\ \hline
\end{tabular}
    \caption{Fields of core statements appearing in the query language}
    \label{tbl:stmt-fields}
\end{table}

The rest of this section introduces a few representative vulnerabilities,
and shows how they are encoded as formulas in \toolname.
But first, we introduce an auxiliary function \texttt{interfere?} 
which will be used by several vulnerabilities.

\begin{definition}{{(\bf Interference)}}\label{def:interfere}
% Following the negation of the standard non-interference~\cite{non-interfere}, <--- This is confusing ...
A symbolic variable $v$ interferes with a symbolic expression $e$ if they satisfy
the following constraint: $\exists v_0,v_1. \ e[v_0/v] \neq e[v_1/v] \land (v_0 \neq v_1)$
% \[
%     \exists v_0,v_1. \ e[v_0/v] \neq e[v_1/v] \land (v_0 \neq v_1)
% \]
\end{definition}
\noindent Intuitively, changing $v$'s value will also affect $e$'s output,
which is denoted as ``(interfere? $v$ $e$)". Interference precisely captures 
the data- and control-dependencies between two expressions and turns out to be 
the \emph{necessary condition} of many exploits.

\medskip
Section~\ref{sec:overview} describes the \batchoverflow vulnerability, 
which enables an attacker to perform a multiplication that 
overflows and transfers a large amount of tokens on the attacker's behalf. 
This vulnerability can be formalized as follows:
\begin{vul}{{\bf \batchoverflow}}  
% \begin{equation*}
% \small
% \begin{split}
% \exists arg_0, arg_1, r_1, r_2, r_3, call \ \
%           r_3 = (r_1 \otimes r_2) \ \land \ \geval{r_2} > \geval{r_3} \\
%           \land \ (\text{interfere?} \ r_2 \ call.value) \ \land \ (\text{interfere?} \ arg_0 \ call.addr) \\
%           \land \ (\text{interfere?} \ arg_1 \ call.value)
% \end{split}
% \end{equation*}
\begin{lstlisting}[numbers=none,morekeywords={uses,matches,where,interfere}]
uses Transfer $t_1$; BinaryExp e; Argument $a_1,a_2$;
matches {e; $t_1$;} where 
($e.opcode == "\times" \land \geval{e.oprand1} > \geval{e.lhs}$ 
$\land$ (interfere? $e.oprand1$ $t_1.amount$)
$\land$ (interfere? $a_1$ $t_1.recipient$) $\land$ (interfere? $a_2$ $t_1.amount$))
\end{lstlisting}
\end{vul}
\noindent In other words, the victim program contains a \texttt{transfer} instruction whose beneficiary and value 
can be  controlled by the attacker. Furthermore, the transaction value is also influenced by a variable 
from an arithmetic operation that overflows.

% A Timestamp Dependency vulnerability occurs if a transaction depends on a timestamp: 
% \begin{vul}{{\bf Timestamp Dependency}} 
% \begin{equation*}
% \small
% \begin{split}
% \exists \ \text{timestamp}, call. \ call.value > 0 \ \land 
%           (\text{interfere?} \ \text{timestamp} \ call.value)
% \end{split}
% \end{equation*}
% \end{vul}
% \noindent This vulnerability enables a malicious miner to gain an advantage by choosing a suitable timestamp for a block.

An \emph{Unchecked-send Vulnerability} occurs when the programmer fails to check 
the return values of critical instructions such as \texttt{delegatecall} and \texttt{call}. If these 
instructions result in runtime errors, the programmer is responsible for manually 
checking their return values and restoring the program state. Failing to do so can lead to unexpected
behavior~\cite{gasless}. We formalize the absence of this check as follows: 
\begin{vul}{{\bf Unchecked-send (Gasless-send)}}
%An Unchecked-send vulnerability occurs if a return value of a \texttt{call} 
%instruction is not checked by the programmer.
%We represent the vulnerability using the following formula:
% \begin{equation*}
% \small
% \begin{split}
% \neg \forall \ call, \exists \ jmp \ \
%   (\text{interfere?} \ call.ret \ jmp.var)
% \end{split}
% \end{equation*}
\begin{lstlisting}[numbers=none,morekeywords={uses,matches,where,interfere}]
uses Transfer $t$; Jump j;
matches { t; ~j;} where ((interfere? $t.ret$ $j.condVar$))
\end{lstlisting}

\end{vul}
\noindent Here, the return value of a \texttt{transfer} instruction does not 
\emph{interfere with} the conditional variables of any \emph{conditional jump} statements. 
In other words, this return value is not checked.

The \reentrancy vulnerability (introduced in Section~\ref{sec:intro}) 
 occurs when an attacker's call is allowed to 
repeatedly make new calls to the same victim contract without updating the victim's balance. 
It can be overapproximated as follows:
\begin{vul}{{\bf Reentrancy}} 
% \begin{equation*}
% \small
% \begin{split}
% \exists \ \text{arg}, i, j, k, call, call', store. \ \ i < j < k \ \land
%           call_{loc} = i \land call'_{loc} = j \\
%           \land \ store_{loc} = k \ \land \
%           call.\text{gas} > 2300 \land (\text{interfere? arg} \ call.\text{addr})
% \end{split}
% \end{equation*}
% \begin{equation*}
% \small
% \begin{split}
% \textbf{Approximate query:} \\
% \neg \forall \ i, k, \exists j, \ \trace.  \ \land
%           \trace[i] = \trace[k] = \mathsf{call} \land \mathsf{call.gas} > 2300 \implies \\
%           i < j < k \ \land \ \trace[j] = \mathsf{store} 
%         %   \land (\mathsf{interfere? \ arg} \ call.\mathsf{addr})
% \end{split}
% \end{equation*}
\begin{lstlisting}[numbers=none,morekeywords={uses,matches,where,interfere,within}]
uses Transfer $t_1$, $t_2$; Store s; Argument a;
matches {$t_1$; ~s; $t_2$;} where ($t_1.loc == t_2.loc \land t_2.gas>2300$
  $\land$ (interfere? a $t_2.recipient$))
\end{lstlisting}

% \begin{equation*}
% \small
% \begin{split}
% \textbf{Precise query:} \\
% \neg \forall \ i, k, \exists j, \ \trace.  \ \land
%           \trace[i] = \trace[k] = \mathsf{call} \land \mathsf{call.gas} > 2300 \implies \\
%           i < j < k \ \land \ \trace[j] = \mathsf{store} \ \land \ \mathsf{store.token \ge call.token} \\
%           \land \ \mathsf{(interfere? \ store.token \ call.token)}
%         %   \land (\mathsf{interfere? \ arg} \ call.\mathsf{addr})
% \end{split}
% \end{equation*}

\end{vul}
\noindent In other words, let trace $\trace$ contains a sequence instructions that include multiple \texttt{transfer} statements that share the same program counter, if there is no \texttt{store} statement between the two \texttt{transfer} functions that has the minimum gas (i.e., 2300), then there may exist a Reentrancy vulnerability.
\subsection{Attack Synthesis}\label{sec:attack}

Given a vulnerability query, we are interested in synthesizing an attack 
program that can exploit this vulnerability in a victim contract. 
The basic building blocks of an attack program are called 
\emph{components}, and each component $\comp$ corresponds to a public method provided by the  
victim contract. We use $\comps$ to denote the union of all publicly available 
methods.

\begin{definition}{{(\bf Component)}}
A \emph{Component} $\comp$ from an ABI configuration is a pair $(f, \tau)$ where: 1) $f$ is $\comp$'s name, and 2) $\tau$ is the type signature of $\comp$.
% \begin{itemize}
% \item $f$ is $\comp$'s name. 
% \item $\tau$ is the type signature of $\comp$.
% \end{itemize}
\end{definition}

\begin{example}
Consider the ABI configuration in Figure~\ref{fig:batchcode}. Its first element 
declares a component for the problematic \texttt{batchTransfer} method. 
This component takes inputs as an 
array of \texttt{address} and a 256-bit integer (\texttt{uint256}). 
\end{example}


%Given a component $\comp(t_1,...,t_n)$ where $t_i$ are symbolic values, we write
%$\peval{\comp(t_1,...,t_n)}$ to represent the result of executing $\comp(t_1,...,t_n)$ on
%program state $\pstate$.

We represent a set of candidate attack programs as a \emph{symbolic program}, 
which is a sequence of \emph{holes} to be filled with components from $\comps$. 
The synthesizer fills these holes to obtain 
a \emph{concrete program} that exploits a given vulnerability.
\begin{definition}{{(\bf Symbolic Attack Program)}}\label{def:symbolic-attack}
Given a set of components $\comps=\{(f_1,\tau_1),\ldots,(f_N,\tau_N)\}$, a \emph{symbolic attack program} $\sketch$ 
for $\comps$ is a sequence of \emph{statement holes} of the form\looseness=-1
$$
\mathtt{choose}(f_1({\vec{v}_{\tau_1}}), \ldots, f_N({\vec{v}_{\tau_N}}));
$$
where $f_i({\vec{v}_{\tau_i}})$ stands for the application of the
$i$-th component to fresh symbolic values of types specified by $\tau_i$. 
\end{definition}
% Our synthesis procedure starts with the most general sketch $\sketch$ and 
% iteratively refines it to obtain concrete programs. Specifically, we express sketch using 
% a \emph{refinement list} whose elements and size can change dynamically. 
% Adequately, the list represents a partial program in our sketch
% language of which context-free grammar is defined in Figure~\ref{fig:sketch}.
% According to the grammar, we can apply two different rules on a sketch $\sketch$,
% namely, \emph{instantiation}, whose goal is to pick a component $\comp$ from $\comps$ 
% and set all the arguments to be symbolic, and \emph{expansion}, whose goal is to 
% expand current sketch. 
\begin{definition}{{(\bf Concrete Attack Program)}}
A \emph{concrete attack program} for a symbolic program $\sketch$ 
replaces each hole in $\sketch$ with one of the specified function calls, 
and each symbolic argument to a function call is replaced with a concrete value.
\end{definition}
\begin{example}
Here is a symbolic program that captures the attack candidate in
Fig~\ref{fig:batchcode}:
\begin{lstlisting}[numbers=none,frame=none,basicstyle=\footnotesize\ttfamily]
choose(makeFlag($x_1$), batchTransfer($y_1$,$z_1$));  
choose(makeFlag($x_2$), batchTransfer($y_2$,$z_2$)); 
\end{lstlisting}
And here is a concrete attack program for this symbolic attack:  
\begin{lstlisting}[numbers=none,frame=none,basicstyle=\footnotesize\ttfamily]
makeFlag(true); 
batchTransfer([0x123,0x345], $2^{256}-1$);  
\end{lstlisting}

\end{example}

The $\mathtt{choose}$ construct is a notational shorthand for a conditional statement that guards the specified choices with fresh
symbolic booleans. For example, $\mathtt{choose}(e_1, e_2)$ stands for the statement $\mathtt{if}\ b_1\ \mathtt{then}\ e_1\
\mathtt{else}\ e_2$, where $b_1$ is a fresh symbolic boolean value. A concrete
attack program therefore substitutes concrete values for the implicit $\mathtt{choose}$ guards and the
explicit function arguments of a symbolic attack program. 


% \begin{figure}[t]
% \[
% \begin{array}{lll}
% {\rm Term} \ t:\tau & := & (\texttt{define-symbolic freshId }  \tau?)  \\
% {\rm Sketch} \ \sketch & := &   \sketch \ | \ \sketch; \sketch \ \ {\rm -expansion}\\
% & & | \ \comp(t_1:\tau_1,...,t_n:\tau_n) \ \ (\comp \in \comps) \ \ {\rm -instantiation}
% \end{array}
% \]
% \caption{Context-free grammar for sketch}\label{fig:sketch}
% \end{figure}

% \begin{figure}
% \lstinputlisting[language=Java]{code.sol}
% \caption{A Smart Contract written in Solidity.}
% \label{fig:code}
% \end{figure}

% \begin{figure}
% \lstinputlisting[language=Java]{abi.json}
% \caption{Contract Application Binary Interface (ABI) for the example in Figure~\ref{fig:code}.}
% \label{fig:abi}
% \end{figure}



%Since attack programs are valid programs in our language, 
%their semantics is given by the rules in Figure~\ref{fig:sem}. 
%\begin{definition}{{(\bf State Transition over symbolic program $\sketch$)}}
%  Since a symbolic program $\sketch$ is a sequence of statements, a
%  \emph{State Transition} $\transition^{*}$ over $\sketch$ is obtained 
%  by sequentially invoking the rules in Figure~\ref{fig:sem}.
%  We denote the transition using $\pstate \reach_{\sketch} \pstate'$. 
%\end{definition} 

%We write $\peval{\sketch}$ to denote the result of executing 
%a symbolic attack program $\sketch$ from the program state $\pstate$.
%The result $\peval{\sketch}$
%represents the set of states reachable by all concrete programs for $\sketch$ starting from the state $\pstate$. 
The goal of attack synthesis is to  
find a concrete program $P$ for a given symbolic program $\sketch$ 
such that $P$ reaches a state satisfying a desired vulnerability query.


\begin{definition}{{(\bf Problem Specification)}}
The specification for our \emph{attack synthesis} problem is a tuple ($\pstate_0$, $\vulnerability$, $\sketch$) where:
\begin{itemize}
\item $\sketch$ is a symbolic attack program for the set of components $\comps$
of a victim contract $\victim$.
\item $\pstate_0$ is the initial state of the symbolic attack program, obtained by executing the victim's initialization code.
\item $\vulnerability$ is a first-order formula over the (symbolic) program state
$\peval{\sketch}$ reachable from $\pstate_0$ by the attack program $\sketch$.
\end{itemize}
\end{definition}
\begin{definition}{{(\bf Attack Synthesis)}}
Given a specification ($\pstate_0$, $\vulnerability$, $\sketch$), the 
\emph{Attack Synthesis problem} is to find a \emph{concrete attack program} $P$ for
$\sketch$ such that: 1) $\geval{ P }_{\pstate_0} = \pstate$, and 2) $\pstate \models \vulnerability$.
In other words, executing 
% the concrete attack 
$P$ from the initial state $\pstate_0$
results in a program state $\pstate$ that satisfies $\vulnerability$.
\end{definition}


% !TEX root =  main.tex
\section{Summary-based Symbolic Evaluation}\label{sec:sum}

Solving the attack synthesis problem involves searching for a concrete program
$P$ in the space of candidate attacks defined by a symbolic program $\sketch$.
\toolname delegates this search to an off-the-shelf SMT solver, by using
symbolic evaluation to reduce the attack synthesis problem to a satisfiability
query. Given a specification  $(\pstate_0, \vulnerability, \sketch)$,  \toolname
evaluates $\sketch$ on the state $\pstate_0$ to obtain the state
$\geval{\sketch}_{\pstate_0}$, and then uses the solver to check the
satisfiability of the  formula $\exists \vec{v} .
\vulnerability(\geval{\sketch}_{\pstate_0})$, where $\vec{v}$ denotes the
symbolic variables in $\sketch$. A model of this formula, if it exists, binds
every variable in $\vec{v}$ to a concrete value, and so represents a concrete
attack program $P$ for $\sketch$ that triggers the vulnerability
$\vulnerability$.\looseness=-1
\begin{figure}[!t]
    \centering
  \begin{minipage}{0.5\linewidth}
\begin{lstlisting}[linewidth=8cm] 
(define (get-summary s $\pc$)
  (match s
    [transfer(x, y, z) $\sumi{transfer}$($\pstate_S(x)$, $\pstate_S[y]$, $\pstate_S[z]$)@$\pc$]
    [sstore(x, y)  $\sumi{sstore}$(x, $\pstate_S[y]$)@$\pc$]
    [_             #f]))
		\end{lstlisting}
		\end{minipage}
		\vspace{-0.1in}
    \caption{Procedure for summary generation. }
    \vspace{-0.1in}
      \label{fig:sum-gen}
\end{figure}
But computing $\geval{\sketch}_{\pstate_0}$ is expensive as it relies on
symbolic evaluation~\cite{rosette}. In particular, evaluating a
$\mathtt{choose}$ statement in $\sketch$ involves symbolically evaluating each
function call in that statement. So, for a symbolic program of length $K$, every
public function in the victim contract must be symbolically executed $K$ times
on different symbolic arguments. As we will see in section~\ref{sec:eval}, this
direct approach to evaluating $\sketch$ does not scale to real contracts that
contain a large number of complex public functions. To mitigate this issue, we
use a summary-based symbolic evaluation that performs symbolic execution of each
public method only once. 

Our approach is based on the following insight. An attack program performs a
sequence of transactions---i.e., method invocations---that manipulate the
victim's persistent storage and global properties. The transactions that
comprise an attack exchange data and influence each other's control flow
exclusively through these two parts of the program state. So, if we can
faithfully summarize the effects of a public method on the persistent storage
and global properties, evaluating this summary on the symbolic arguments passed
to the method is equivalent to symbolically executing the method
itself.\looseness=-1

\begin{definition}
A summary $\summary$ in our system is a pair $s@\pc$ where $s$ represents a
statement that has a side effect on the persistent state (i.e., storage and
global properties) of a smart contract, and $\pc$ denotes the path condition of
executing $s$.
\end{definition}

We generate such faithful method summaries in two steps. First, we evaluate the
method on a program state $\pstate_S$ that maps every state variable (i.e.,
persistent storage location, global property, etc.) to a fresh symbolic variable
of the right type. This step produces a path condition and symbolic inputs for
each instruction that capture every possible way to reach and execute the
instruction within the given method. Next, we use the procedure in
Figure~\ref{fig:sum-gen} to generate the method summary.\footnote{We omit 
the details of other side-effecting instructions for
simplicity.\looseness=-1} Given a storage-store instruction \texttt{sstore(x,y)}
and its path condition, we generate a ``summary sstore" statement (i.e.,
$\sumi{sstore}$) that takes as input the name of the storage variable (i.e.,
$x$) and the symbolic expression $\pstate_S[y]$ held in the register $y$. Similarly,
given a \texttt{call(gas,addr,value)} instruction and path condition, we emit
its ``summary call" statement (i.e., $\sumi{call}$) that takes as input the
symbolic expressions of the instruction's gas consumption, recipient address,
and amount of cryptocurrency, respectively. All other instructions are omitted
from the summary since they have no effect on the persistent state. By
construction, our summary therefore precisely  captures all of the method's
effects on the persistent state, and the summaries are polynomially-sized as guaranteed by Rosette's symbolic evaluator~\cite{rosette}.\looseness=-1

\begin{example}
Recall that we introduce the following code snippet in Figure~\ref{fig:sum-interp}b:
\begin{lstlisting}[linewidth=9cm,frame=none, numbers=none]
  assert(_amount > 0);
  r1 (*\textbf{:=}*) _amount - 1;
  (*\textbf{sstore}*)(vesting.amount, _amount - 1);
  (*\textbf{transfer}*)(msg.sender, _to, _amount - 1);
  r2 (*\textbf{:=}*) amount - 15;
  r3 (*\textbf{:=}*) amount - 15;
  r4 (*\textbf{:=}*) (*\textbf{sload}*)(vesting.startTime);
  (*\textbf{no-op}*);
\end{lstlisting}
Then using the rule in Figure~\ref{fig:sum-gen}, \toolname generates the following summary: 
\begin{lstlisting}[linewidth=9cm,frame=none, numbers=none]
  $\sumi{sstore}(\mathtt{vesting.amount}, \pstate_S[\mathtt{\_amount}] - 1)@(\pstate_S[\mathtt{\_amount}]>0)$;
  $\sumi{transfer}(\pstate_S[\mathtt{msg.sender}], \pstate_S[\mathtt{\_to}], \pstate_S[\mathtt{\_amount}] - 1)@(\pstate_S[\mathtt{\_amount}]>0)$;
\end{lstlisting}
In particular, our
tool summarizes the side effects of the \texttt{call} and \texttt{sstore}
instructions at lines 2 and 3 in Figure~\ref{fig:sum-interp}b, respectively.  The remaining instructions (E.g., statements from line 5 to 8) are
omitted from the summary because they have no persistent side effects.\looseness=-1
\end{example}

Once \toolname generates the summary for each procedure, we still need to adjust the symbolic evaluation engine to cope with the summaries. In particular, given a method summary and a program state $\pstate$, we use the procedure in
Figure~\ref{fig:sum-inter} to reproduce the effects of executing the method
symbolically on $\pstate$ as follows. Recall that we generate the summary by
executing the method on a fully symbolic state $\pstate_S=\{x_1\mapsto
v_1,\ldots, x_n\mapsto v_n\}$, so every path
condition and symbolic expression in the summary is given in terms of the
symbolic variables $v_1,\ldots,v_n$. Our summary interpretation procedure works by
substituting each $v_i$ in an instruction's path condition and inputs with its
corresponding value in $\pstate$, i.e., $\pstate[x_i]$. The resulting instruction summary
$s_{\pstate}@\pc_{\pstate}$ is therefore expressed in terms of $\pstate$, so applying its
side effects $s_{\pstate}$ under the path condition $\pc_{\pstate}$ is equivalent to
executing the instruction $s$ in the original method on the state $\pstate$. 
Since we interpret every instruction in the summary in this way, the
combined effect on the persistent state is equivalent to executing the original
method symbolically on $\pstate$.\looseness=-1

% Because our rules
% for summary generation preserve each instruction's data- and
% control-dependencies (in the symbolic inputs and path condition, respectively),
% executing the summary in this way has the same effect on the persistent state as
% executing the original method symbolically on $E$.\looseness=-1

%Once our system obtains the summary of the original program, it will further apply the 
%abstract interpretation procedure in Figure~\ref{fig:sum-inter} to evaluate its side effects
%over the storage states, using the same operational semantics in Figure~\ref{fig:sem}. The only difference is,
%we also have to stitch the original path condition $\pc$ to its corresponding statement to 
%soundly model the control dependency for each instruction.
\begin{example}
Figure~\ref{fig:sum-interp}d shows an example for interpreting the summary in
Figure~\ref{fig:sum-interp}c by applying the procedure in
Figure~\ref{fig:sum-inter}. Specifically, given an environment $\pstate$ and the
\texttt{transfer} summary at line 2 in Figure~\ref{fig:sum-interp}c, we first
generate an \texttt{if} statement guarded by the path condition $\pc$ in $\pstate$, 
then in the body of the \texttt{if} statement we symbolically
evaluate the \texttt{call} statement in the environment $\pstate$.
\end{example}


%Thanks to the soundness of our summaries, we can state the following theorem:
%\begin{theorem}
% Suppose we have a program $\sketch$, its corresponding summary $\sumi{\sketch}$,
% initial state $\pstate_0$, states $\pstate_1$ and $\pstate_2$ by executing $\sketch$ and 
% $\sumi{\sketch}$, respectively, then if a query $\query$ holds on $\pstate_1$, it will
% also hold on $\pstate_2$.
%\end{theorem}

\begin{figure}[!t]
  \centering
  \begin{minipage}{0.5\linewidth}
\begin{lstlisting}[linewidth=8cm] 
(define (interpret-summary $s$@$\pc$ ${\pstate}$) 
  (define $s_{\pstate}$@$\pc_{\pstate}$ (substitute $s$@$\pc$ ${\pstate}$))
  (match $s_{\pstate}$
    [$\sumi{transfer}(x_{\pstate}, y_{\pstate}, z_{\pstate})$  (when $\pc_{\pstate}$ transfer($x_{\pstate}$, $y_{\pstate}$, $z_{\pstate}$))]
    [$\sumi{sstore}(x, y_{\pstate})$     (when $\pc_{\pstate}$ sstore(x, $y_{\pstate}$))]
    [_  no-op]))
\end{lstlisting}
\end{minipage}
\vspace{-0.1in}
		\caption{Procedure for summary interpretation}
		\vspace{-0.1in}
      \label{fig:sum-inter}
\end{figure}
% !TEX root =  main.tex
\section{Implementation}\label{sec:impl}

This section discusses the design and implementation of 
\toolname, as well as two key optimizations that enable our tool 
to efficiently solve the synthesis attack problem.\looseness=-1


\subsection{Symbolic Computation Using \rosette}\label{sec:rosette}

\toolname leverages \rosette~\cite{rosette} to symbolically search for attack
programs. \rosette is a programming language that provides facilities for
symbolic evaluation.  \rosette programs use
assertions and symbolic values to formulate queries about program behavior,
which are then solved with off-the-shelf SMT solvers. For example, the
\texttt{(solve expr)} query searches for a binding of symbolic variables to
concrete values that satisfies the assertions encountered during the symbolic
evaluation of the program expression \texttt{expr}. \toolname uses the \texttt{solve}
query to search for a concrete attack program.


\begin{figure}
\centering
  \begin{minipage}{0.5\linewidth}
\begin{lstlisting}[linewidth=8cm] 
(define (solar $\vulnerability$  $\comps$ $K$)
 (define program (for/list ([i K]) (apply choose* $\comps$)))
 (define i-pstate (get-initial-state $\comps$))
 (define o-pstate (interpret program i-state))
 (define binding (solve (assert ($\vulnerability$ o-pstate))))
 (evaluate program binding))
\end{lstlisting}
\end{minipage}
\vspace{-0.1in}
\caption{\toolname implementation in \rosette. 
 }
\label{fig:sketch-overview}
\vspace{-0.1in}
\end{figure}

Figure~\ref{fig:sketch-overview} shows the implementation of \toolname in Rosette. 
The tool takes as input a vulnerability specification
$\vulnerability$, the components $\comps$ of a victim program, and a bound $K$ 
on the length of the attack program. Given these inputs, line 2 uses $\comps$ to 
construct a symbolic attack \texttt{program} of length $K$. 
Next, lines 3 runs the victim's initialization code to obtain the 
initial program state, \texttt{i-pstate}, for the attack. 
Then, line 4 evaluates the symbolic attack \texttt{program} on the initial state to 
obtain a symbolic output state, \texttt{o-pstate}. 
Finally, lines 5-6 use the \texttt{solve} query to search for a concrete
attack program that satisfies the vulnerability assertion.

The core of our tool is the \emph{interpreter} for our smart contract language
(Figure~\ref{fig:grammar}), which implements the semantics from the EVM
yellow paper~\cite{evm-yellow}. We use this interpreter to compute the symbolic
summaries of the victim's public methods (Section~\ref{sec:sum}) and to evaluate
symbolic attack programs. The interpreter itself does not implement symbolic
execution; instead, it uses \rosette's symbolic evaluation engine to execute
programs in our language on symbolic values. 

Another key component of \toolname is the \emph{translator} that converts EVM
bytecode into our language (Figure~\ref{fig:grammar}). The translator leverages the Vandal
Decompiler~\cite{madmax} to soundly convert the stack-based EVM bytecode into
its corresponding three-address format in our language. The jump targets are resolved through abstract
interpretation~\cite{CousotC77}.  We use the translator to convert victim
contracts to the \toolname language for attack synthesis. Both the translator and the
interpreter support all the instructions defined in the Ethereum
specification~\cite{yellowpaper}.\looseness=-1



% To use \toolname, a security analyst expresses specifications in Racket as 
% boolean predicates over storage variables. For instance, the time dependency vulnerability
% from section~\ref{sec:vul} can be expressed as follows:
% \begin{lstlisting}
% (define (time-dependent $V_{time}$ $V_{ret}^{call}$)
% (apply ||
%   (for*/list ([x $V_{time}$][y $V_{ret}^{call}$])
%     (interferes? x y))))
% \end{lstlisting}

\subsection{Parallel Synthesis using Hoisting}\label{sec:parallel}

\toolname uses summary-based symbolic evaluation to efficiently reduce attack
synthesis problems to satisfiability queries. But the resulting queries can
still be too difficult %(for both \rosette and the underlying solver) 
to solve
in practice, especially when the victim contract has many public methods. To
further improve performance, \toolname exploits the structure of symbolic attack
programs (Definition~\ref{def:symbolic-attack}) to decompose the single
\texttt{solve} query in Figure~\ref{fig:sketch-overview} into multiple smaller
queries that can be solved quickly and in parallel, without missing any concrete
attacks.\looseness=-1

The basic idea is as follows. Given a set of $N$ components and a bound $K$ on 
the length of the attack, line 2 creates a symbolic attack program of the
following form:
\begin{lstlisting}[numbers=none,frame=none,basicstyle=\footnotesize\ttfamily]
  choose$_1$($f_1({\vec{v_1}_{\tau_1}}), \ldots, f_N({\vec{v_1}_{\tau_N}})$);  
  $\vdots$
  choose$_K$($f_1({\vec{v_K}_{\tau_1}}), \ldots, f_N({\vec{v_K}_{\tau_N}})$);   
\end{lstlisting}
This symbolic attack encodes a set of concrete attacks that can also be
expressed using $N^K$ symbolic programs that fix the choice of the method to
call at each line, but leave the arguments symbolic. So, we can enumerate these
$N^K$ programs and solve the vulnerability query for each of them, instead of
solving the single query at line 5. This approach essentially \emph{hoists} the
symbolic boolean guards out of the \texttt{choose} statements in the original query, and
\toolname explores all possible values for these guards explicitly, rather than
via SMT solving.\footnote{For practical efficiency, our implementation hoists 
the guards to generate $N^K/c$ symbolic programs, where $c$ is  
the number of available cores.} 
As we show in Section~\ref{sec:eval},
hoisting the guards leads to significantly faster synthesis, 
both because it enables parallel solving of the smaller queries, and 
because the smaller queries can be solved quickly.\looseness=-1



% \subsection{SMT-based Early Pruning}\label{sec:prune}

% In addition to hoisting, we also design a simple but effective \emph{early pruning}
% strategy that allows \toolname to prune \emph{infeasible} symbolic programs 
% before executing them. The intuition behind our strategy is that
% all attacks expressible in \toolname (e.g.,~\cite{attack1,attack2,attack3}) invoke at least one public method that 
% manipulates persistent storage and at least one public method that transfers 
% cryptocurrency using the \texttt{call} instruction. In other words,  
% a successful attack executes at least one store instruction followed 
% by at least one \texttt{call} instruction. We express our early pruning strategy using the following 
% \rosette program:\looseness=-1
% % \begin{lstlisting}[xleftmargin=.2\textwidth,linewidth=9.6cm]
% \begin{lstlisting}
% (define (may-store-and-call? p)
%  (solve (exists (list i j)
%   (and (< i j) (= (type p[i]) 'store)
% 	       (= (type p[j]) 'call)))))
% \end{lstlisting}
% This procedure queries the solver to find out if the given symbolic program $p$ contains 
% any concrete attack program that executes a \texttt{call} after a store. 
% This query is much faster to solve than a vulnerability query, so if $p$ contains no 
% feasible candidate, \toolname does not run the vulnerability query for it.\looseness=-1

\subsection{Practical EVM fragment}\label{sec:fragment}
In this section, we briefly illustrate how \toolname handles other challenging EVM fragments.

\paragraph{Loops} Similar to other analyzers that based on symbolic execution, for loops whose iterations are determined by potentially unbounded collections, \toolname will unroll the loops for $K$ time and $K$ is set to 2 by default.

\paragraph{SHA and Storage access}
\toolname directly analyzes the EVM bytecode. For complex data structures such as arrays and maps that are ubiquitous, the address of an element is determined by the follow function:
\[
a[i] := \textit{SHA-256(id(a))} + n \times i
\]
In other words, to access element \texttt{a[i]} whose size is $n$, the address is determined by the
SHA-256 hash of its identifier and size. \toolname leverages uninterpreted functions to model both the SHA-256 and address computation function. I.e., two addresses are the same as long as they share the same array identifier, index, and element size.

\paragraph{Gas consumption} \toolname's program state keeps tracking of the gas usage by accumulating the cost of instructions during symbolic evaluation. If a transaction runs out of gas in the middle of the evaluation,  \toolname terminates the current exploration with an “out of gas” assertion failure.
% !TEX root =  main.tex
\section{Evaluation}\label{sec:eval}
We evaluated \toolname by conducting a set of experiments that are designed to answer the following questions: 
\begin{itemize}
% \item Q1: \emph{Expressiveness}: Can \toolname express the specifications of real world vulnerabilities? 
\item RQ1: \emph{Effectiveness}: How does \toolname compare against state-of-the-art analyzers for smart contracts?
\item RQ2: \emph{Efficiency}: How much does summary-based symbolic 
evaluation improve the performance of \toolname?
% \item Q3: \emph{Expressiveness}: Can \toolname express the specifications of recent vulnerabilities? 
\end{itemize}

To answer these questions, we perform a systematic evaluation by running
\toolname on the entire set of smart contracts from \etherscan~\cite{etherscan}.
Using a snapshot from Feb 13 2019, we obtained a total number of 25,983 smart
contracts (duplicate contracts were removed) whose source code are publicly available. \toolname starts from attack programs of size one and gradually increases
the size until finding the exploit or running out of time. All experiments in this section are
conducted on a \texttt{t3.2xlarge} machine on Amazon EC2 with an Intel Xeon
Platinum 8000 CPU and 32G of memory, running the Ubuntu 18.04 operating system
and using a timeout of 10 minutes for each smart contract.\looseness=-1

\subsection{Comparison with Existing Tools}\label{sec:comp}
To show the advantages of our proposed approach, we compare \toolname against three
state-of-the-art analyzers for exploits generation: \mythril and \teether, based on 
symbolic execution, and \contractfuzz, based on dynamic random testing.
% ~\footnote{ 
% We also did a comparison with the \oyente tool, and the results are included in the supplemental 
% material.}

% !TEX root =  main.tex
% \section{Comparison with \oyente}\label{sec:oyente}
\paragraph{Comparison with \mythril}\label{sec:oyente}
% \begin{figure}
%   \centering
%   \begin{subfigure}[b]{0.22\textwidth}
%     \includegraphics[width=\textwidth]{time.pdf}
%     \caption{Timestamp Dependency}
%     \label{fig:eval-oyente-time}
%   \end{subfigure}
%   %
%   \begin{subfigure}[b]{0.22\textwidth}
%     \includegraphics[width=\textwidth]{dao.pdf}
%     \caption{Reentrancy}
%     \label{fig:eval-oyente-dao}
%   \end{subfigure}
% \caption{Comparing \toolname against \oyente}% on obfuscated apps}
% \label{fig:eval-oyente}
% \end{figure}

% \begin{table}[]
% \begin{tabular}{|l|l|l|l|l|l|l|l|l|}
% \hline
% \multirow{2}{*}{Vulnerability} & \multicolumn{4}{l|}{\toolname} & \multicolumn{4}{l|}{\oyente} \\ \cline{2-9} 
%                                & No.     & Chk    & FP   & FN   & No.    & Chk   & FP   & FN   \\ \hline
% Timestamp                      & 1245    & 20     & 0    & 1    & 660    & 20    & 18   & 20   \\ \hline
% Reentracy                      & 248     & 20     & 0    & 0    & 90     & 20    & 5    & 3    \\ \hline
% \end{tabular}
% \caption{Comparing \toolname against \oyente}% on obfuscated apps}
% \label{fig:eval-oyente}
% \end{table}

% We first compare with \oyente~\cite{oyente}, which takes as input 
% a smart contract and checks whether there are concrete traces that match
% the tool's predefined security properties. If so, the tool returns a counterexample
% as the exploit. We evaluate \oyente and \toolname on the \etherscan data set, and 
% both systems use a timeout of ten minutes.

We first compare with \mythril~\cite{mythril}~\footnote{Since both \toolname and \mythril are general-purpose analyzers for common vulnerabilities in smart contracts, for fair comparison, we only enable the relevant queries in the evaluation.ol} by generating exploits for the reentrancy vulnerability. \mythril takes as input 
a smart contract and checks whether there are concrete traces that match
the tool's predefined security properties. If so, the tool returns a counterexample
as the exploit. We evaluate \mythril and \toolname on the \etherscan data set, and 
both systems use a timeout of 10 minutes.\looseness=-1

% The \oyente tool supports four different types of vulnerabilities, namely, 
% call-stack-limit, Timestamp dependency~\cite{attack4}, Reentrancy~\cite{attack1}, 
% and Transaction-Ordering dependency (TOD)~\cite{attack4}. Since the call-stack-limit
% vulnerability had already been fixed by the Solidity team and the TOD vulnerability
% requires synthesizing multiple programs, we will cover the 
% remaining two vulnerabilities.

\paragraph{Summary of results}
% The results of our evaluation are summarized in Table~\ref{fig:eval-oyente}. In particular, for the Timestamp dependency vulnerability, there
% are 485 benchmarks where both tools report a vulnerability and find the exploits.
% 39 benchmarks are flagged as vulnerable by \oyente but \toolname can not find 
% the exploits. We manually inspected the source code of those benchmarks and 
% confirm that 30 of them are false positives. On the other hand, 842
% benchmarks are flagged as safe by \oyente while \toolname manages to find
% their exploits. To verify the reports of our tool, we randomly select 
% 20 benmarks and confirm 18 of them are actually vulnerable. In the meantime,
% we also contacted the author of \oyente and confirmed our report.

% For the Reentrancy vulnerability, 49 benchmarks are flagged by both tools.
% 41 benchmarks are flagged as vulnerable by \oyente while \toolname cannot
% find the exploits. After manual inspection, we confirm all of them are false 
% positives. In contrast, 128 benchmarks are marked as safe but \toolname 
% successfully finds their exploits, and we manage to reproduce 102 of the attacks 
% in our testbed.

For 156 contracts flagged as \reentrancy vulnerablity
by at least one tool, we manually determine the ground truth and summarize the results in Figure~\ref{fig:eval-oyente}. 
% As shown in Table~\ref{fig:eval-oyente-fp-fn}, for 
% the Timestamp vulnerability, the FN and FP rates of \toolname are 7\% and 10\%,
% while the FN and FP rates of \oyente on our selected data set are 36\% and 35\%.
% The result on the Reentrancy vulnerability is similar: 
The false negative (FN) and false positive (FP) rates of \toolname are 7\% and 3\%,
while the FN and FP rates of \mythril  are 26\% and 12\%.

\paragraph{Performance}
\mythril takes an average of 23 seconds to analyze a contract, 
while \toolname takes an average of 8 seconds for this data set.
\paragraph{Discussion} 
% To understand why \oyente has higher false positive and negative rates than
% \toolname, we manually inspected 20 randomly chosen samples from each category. 
% The results of this analysis are as follows.

The high false negative rate in \mythril is caused by low coverage on the
corresponding benchmarks. In the presence of large and complex
methods, \mythril fails to generate traces that trigger the vulnerability.
Moreover, \mythril does not support cross-function re-entrancy, i.e., re-entrancy attacks span over multiple functions of the
victim contract.\looseness=-1 

% The false positives in \mythril can be attributed to two root causes. The first
% is that the tool does not model the semantics of the gas system, and its query
% language cannot reason about gas consumption in a smart contract. For instance,
% \mythril will report spurious Reentrancy vulnerabilities even though the gas
% specified by the victim is insufficient for an attacker to generate the exploit.
% On the other hand, since \toolname precisely models the semantics of the gas
% system, we are able to achieve a low false positive rate. The second cause of
% false positives is due to the exploration of paths that an attacker cannot
% trigger. 
%The vulnerable functions of the second type of smart contracts
%strictly check whether the caller of them is the owner of the
%smart contract specified during contract creation. Since there is
%no way for an external account to invoke the function containing
%ether transfer, reentrant attack is also not possible. 
% For instance, \oyente marks the following code as Reentrancy vulnerability 
% even though an attacker has no permission to trigger it.
% \begin{lstlisting}[escapechar=@]
% public function mintETHRewards(
%   address _contract, uint256 _amount) 
%   @\textbf{onlyManager}@() {
%   require(_contract.call.value(_amount)());}
% \end{lstlisting}

We also investigated the cause of false positives 
reported by \toolname. It turns out that the false positives are 
caused by the imprecision of our queries. 
% Recall from Section~\ref{sec:vul}
% that we use a specific pattern of traces to \emph{overapproximate} the
In particular, we use a specific pattern of traces to \emph{overapproximate} the 
behavior of the Reentrancy attack. While effective and 
efficient in practice, our query may generate spurious 
exploits that are infeasible. To mitigate this limitation, one 
compelling approach for developing secure smart contracts is to 
ask the developers to provide invariants that the tool can use to 
rule out infeasible attacks. 
%for preventing the vulnerabilities, 
%and then use \toolname to search for exploits that violate the invariants. 
\definecolor{bblue}{HTML}{0064FF}
\definecolor{rred}{HTML}{C0504D}
\definecolor{ggreen}{HTML}{9BBB59}
\definecolor{ppurple}{HTML}{9F4C7C}

\definecolor{b1}{HTML}{BEE9E8}
\definecolor{b2}{HTML}{188FA7}

\begin{figure}[!t]
\centering
\scalebox{0.85}{
\begin{tikzpicture}
    \begin{axis}[
        width  = 8cm,
        height = 8cm,
        y=0.11cm,
        x=2.8cm,
        major x tick style = transparent,
        ybar=2*\pgflinewidth,
        bar width=24pt,
        ymajorgrids = true,
        ylabel = {Percentage \%},
        ylabel style={yshift=-4mm},
        symbolic x coords={FN,FP},
        ytick={0,10,20,30,40},
        xtick = data,
        scaled y ticks = false,
        enlarge x limits=0.6,
        ymin=0,
        ymax=40,
        legend cell align=left,
        legend entries={\toolname, \mythril},
        legend style={
                at={(0.5,1.14)},
                legend columns=-1,
                anchor=north,
        %        column sep=1ex
        }
    ]
    %\hspace*{-2mm}
        \addplot[style={ggreen,fill=ggreen,mark=none}]
        coordinates{(FN,3)(FP,7)};

        \addplot[style={ppurple,fill=ppurple,mark=none}]
        coordinates{(FN,26)(FP,12)};
    \end{axis}
\end{tikzpicture}
}
% \vspace{-0.2in}
\caption{Comparing \toolname against \mythril}
\vspace{-0.2in}
\label{fig:eval-oyente}
\end{figure}

% \begin{table}
% \centering
% \begin{tabular}{|c|c|c|c|c|}
% \hline
% \multirow{2}{*}{Vulnerability} & \multicolumn{2}{c|}{\toolname} & \multicolumn{2}{c|}{\oyente} \\ \cline{2-5} 
%                               & FP             & FN            & FP            & FN           \\ \hline
% Timestamp                      & 7\%            & 10\%            & 36\%             & 35\%            \\ \hline
% Reentrancy                     & 14\%             & 5\%           & 43\%             & 37\%            \\ \hline
% \end{tabular}
% \caption{Analysis of the results based on full inspection on 20 random samples 
% from $S \cup O$}
% \label{fig:eval-oyente-fp-fn}
% \end{table}

% \begin{table}
% \centering
% \begin{tabular}{|l|l|l|l|}
% \hline
% \multicolumn{1}{|c|}{\multirow{2}{*}{Vulnerability}} & \multicolumn{3}{l|}{Number of vulnerable contracts} \\ \cline{2-4} 
% \multicolumn{1}{|c|}{}                               & \multicolumn{1}{c|}{$S \land O$}            & \multicolumn{1}{c|}{$S - O$}           & \multicolumn{1}{c|}{$O - S$}           \\ \hline
% Timestamp   &\multicolumn{1}{c|}{485}  &\multicolumn{1}{c|}{842}   &\multicolumn{1}{c|}{39}                \\ \hline
% Reentracy   &\multicolumn{1}{c|}{49}  &\multicolumn{1}{c|}{128}   &\multicolumn{1}{c|}{41}                \\ \hline
% \end{tabular}
% \caption{Comparing \toolname ($S$) against \oyente ($O$).
% $S \land O$, $S - O$, and $O - S$ represent \# of benchmarks 
% reported by both tools, $S$ only, and $O$ only, respectively.}% on obfuscated apps}
% \label{fig:eval-oyente}
% % \vspace{-0.1in}
% \end{table}

% !TEX root =  main.tex
\paragraph{\bf{Comparison with \teether}}\label{sec:teether}
We next compare \toolname against \teether~\cite{teether}, the most recent
tool using dynamic symbolic execution for generating exploits that would enable
the attacker to control the money transactions of a victim contract. In
particular, \teether looks for so-called \emph{critical instructions}
(i.e., \texttt{call}, \texttt{selfdestruct}, etc.) that include recipients' addresses, 
which can be manipulated by the attacker to withdraw
tokens from a vulnerable contract. 

\paragraph{Summary of results}
% The results of our evaluation are summarized in Figure~\ref{fig:plot-attack} and 
% Table~\ref{fig:eval-teether-fp-fn}. 
% In particular, as shown in Figure~\ref{fig:plot-attack}, \toolname and \teether flag
% 198 and 179 benchmarks as vulnerable, respectively. Here, the vulnerable benchmarks
% reported by \teether are a subset of the ones flagged by \toolname.
% To obtain the ground truth for those 198 contracts, we set up a private test 
% blockchain where we can deploy the contracts and validate their corresponding exploits.
% In the end, it turns out that 181 benchmarks are indeed vulnerable (i.e., true positives 
% in the second column of Table~\ref{fig:eval-teether-fp-fn}). Specifically,
% both tools maintain a low false positives with 17 in \toolname and 
% 19 in \teether. That is not very surprising because both tools are based on symbolic 
% execution and use the same query for driving the evaluation. On the other hand, 
% \toolname manages to find 21 exploits that cannot be generated by the \teether tool. 

In total, there are 198 contracts that are marked as vulnerable by at
least one tool. While \toolname covers all exploits generated by
\teether, \toolname also finds 21 \emph{extra} exploits that cannot be generated
by \teether. 

% \begin{table}[]
% \centering
% \begin{tabular}{|c|c|c|c|c|c|}
% \hline
% \multirow{2}{*}{Vulnerability}  &\multirow{2}{*}{\#TP}     & \multicolumn{2}{c|}{\toolname} & \multicolumn{2}{c|}{\teether} \\ \cline{3-6} 
%                                  &    & \#FP             & \#FN            & \#FP            & \#FN           \\ \hline
% Attack Control                 & 181  & 17            & 0            & 19             & 21            \\ \hline
% \end{tabular}
% \caption{Analysis of the results based on full inspection on 198 suspicious contracts from \etherscan}
% \label{fig:eval-teether-fp-fn}
% \end{table}

\paragraph{Performance}
\teether takes an average of 31 seconds to analyze the \etherscan data set, 
while \toolname takes an average of 8 seconds per contract.

\paragraph{Discussion} 
% To understand the cause of false positives and false negatives in \teether and
% \toolname, we manually inspected all the problematic benchmarks that lead
% to either false positives or false negatives.

The missing exploits in \teether are caused by low coverage on the
corresponding benchmarks. For the 21 benchmarks with exploits that cannot be generated by \teether, 
14 involve attack programs with four method calls, and each of the remaining 7 
benchmarks contains over 3000 lines of source code with complex control flow.
As a result,  \teether  fails to explore sufficiently many \emph{concrete traces} to 
find the exploits, even if we increase the timeout from 10 minutes to 1 hour.\looseness=-1  
% Moreover, since the Keccak-256 hash function is ubiquitous in smart contracts,
% and hard for the solver to reason about, \teether may fail to cover the code regions
% that have dependencies on the hash function. 

% The false positives in \teether can be attributed to several root causes. The major
% reason, which is also discussed in the \teether's paper, is due to the
% inconsistency of the persistent states between the initial exploit generation 
% and exploit validation. This issue can be further exacerbated if the exploits 
% depend on parts of the global state that are subject to implicit invariants 
% unknown to the exploit generator. 
% We note that \toolname also shares this limitation. To mitigate it, one 
% compelling approach for developing secure smart contracts is to 
% ask the developers to provide these invariants. % for preventing the vulnerabilities, 
% %and then use \toolname to search for exploits that violate the invariants. 
% Finally, the two extra false positives from \teether are caused by its imprecise 
% modeling of the gas system. In particular, \teether sets the gas price of all
% operations to 0, which leads to some infeasible exploits.


\paragraph{\bf{Comparison with \contractfuzz}}\label{sec:fuzz}
We further compared \toolname against \contractfuzz~\cite{contractfuzzer}, a recent 
smart contract analyzer based on dynamic fuzzing. 
\contractfuzz takes as input the ABI interfaces of smart
contracts and \emph{randomly} generates inputs invoking the public methods 
provided by the ABI. To verify the correctness of the exploits, \contractfuzz
implements oracles for different vulnerabilities by instrumenting
the Ethereum Virtual Machine (EVM) with extra assertions.

% \begin{figure}
%   \centering
%   \begin{subfigure}[b]{0.22\textwidth}
%     \includegraphics[width=\textwidth]{time-fuzz.pdf}
%     \caption{Timestamp Dependency}
%     \label{fig:eval-fuzz-time}
%   \end{subfigure}
%   %
%   \begin{subfigure}[b]{0.22\textwidth}
%     \includegraphics[width=\textwidth]{gasless.pdf}
%     \caption{Gasless Vulnerability}
%     \label{fig:eval-fuzz-gas}
%   \end{subfigure}
% \caption{Comparing \toolname against \contractfuzz}% on obfuscated apps}
% \label{fig:eval-fuzz}
% \end{figure}

\begin{table}[]
\centering
\begin{tabular}{|l|l|l|l|l|l|l|}
\hline
\multirow{2}{*}{Vulnerability} & \multicolumn{3}{l|}{\toolname} & \multicolumn{3}{l|}{\contractfuzz} \\ \cline{2-7} 
                               & No.       & FP       & FN      & No.        & FP        & FN        \\ \hline
Timestamp                      & 16        & 0        & 1       & 13         & 4         &7         \\ \hline
Gasless Send                   & 17        & 0        & 0       & 14         & 3         & 6         \\ \hline
Bad Random                     & 9        & 0        & 0       & 5          & 1         & 5         \\ \hline
\end{tabular}
\caption{Comparing \toolname against \contractfuzz}% on obfuscated apps}
\vspace{-0.2in}
\label{fig:eval-fuzz}
\end{table}

We use the docker image~\cite{fuzz-docker} provided by the author of \contractfuzz.
The original paper does not discuss the performance of the tool, 
but from our experience, \contractfuzz is slow, taking more than 
10 mins to fuzz a smart contract. Since it would be time-consuming to 
run \contractfuzz on the \etherscan data set, we evaluate both tools 
on the 33 benchmarks from the \contractfuzz artifact~\cite{fuzz-data} plus another
67 random samples from \etherscan for which we know the ground truth.

\paragraph{Summary of results}
The results of our evaluation are summarized in Table~\ref{fig:eval-fuzz}.
For the timestamp dependency, \contractfuzz flags 13 benchmarks 
as vulnerable. However, 4 of them are false alarms, and \contractfuzz fails to detect 7 
vulnerable benchmarks. On the other hand, \toolname detects most of 
the benchmarks with only one false negative, which is caused by a timeout of the Vandal
decompiler~\cite{madmax}.\looseness=-1

Similarly, for the Gasless-send vulnerability, 14 benchmarks are flagged by \contractfuzz.
However, 3 of them are false positives, and 6 vulnerable benchmarks can not be detected 
within 10 minutes. In contrast, \toolname successfully generates exploits for 
all the vulnerable benchmarks.

\paragraph{Performance}
On average, \contractfuzz takes 10 mins to analyze a smart contract. 
\toolname takes an average of 11 seconds on this data set.

\paragraph{Discussion}
The cause of false negatives in \contractfuzz is easy to understand as it is
based on random, rather than exhaustive, exploration of an extremely large
search space. So if there are relatively few inputs in this space that lead to
an attack, \contractfuzz is unlikely to find one in reasonable time. The false
positives in \contractfuzz are caused by the limited expressiveness of its
assertion language. For instance, the Time Dependency is defined as the
following assertion in \contractfuzz: 
\[
\textbf{TimestampOp} \wedge (\textbf{SendCall} \vee \textbf{EtherTransfer})
\]
The assertion raises a Time Dependency vulnerability if the smart contract
contains the \texttt{timestamp} and \texttt{call} instructions. It is
easy to raise false alarms with this assertion if the \texttt{call} instruction 
does not depend on \texttt{timestamp}. 
%On the other hand, the \texttt{interfere?} function enables \toolname to reason about this dependency precisely.

\fbox{
\begin{minipage}{0.9\linewidth}
{\bf Result for RQ1:}
\toolname outperforms three state-of-the-art analyzers in terms of running time, false positives, and false negatives.
\end{minipage}
} \\
\subsection{Impact of Summary-based Symbolic Evaluation}\label{sec:expr}
% \begin{table}[]
% % \vspace{-0.3in}
% \centering
% \begin{tabular}{|l|l|l|l|l|}
% \hline
% \multirow{2}{*}{$S^{\dagger}$-mean} & \multirow{2}{*}{$S^{\diamond}$-mean} & \multicolumn{3}{c|}{\# of Benchmarks Timeout} \\ \cline{3-5} 
%                     &                      & $S^{\dagger} \land S^{\diamond}$   & $S^{\dagger} - S^{\diamond}$  & $S^{\diamond} - S^{\dagger}$ \\ \hline
% 8s                   & 35s                    & 1846         & 548        & 17454     \\ \hline
% \end{tabular}
%   \caption{Comparison between summary-based ($S^{\dagger}$) and non-summary ($S^{\diamond}$). $S^{\dagger} \land S^{\diamond}$, $S^{\dagger} - S^{\diamond},$ and $S^{\diamond} - S^{\dagger}$ represent number of benchmarks timeout on both, $S^{\dagger}$ only, and $S^{\diamond}$ only, respectively.}
% %   \vspace{-0.3in}
%   \label{fig:summary}
% \end{table}

To understand the impact of our summary-based symbolic evaluation described 
in Section~\ref{sec:sum} and how are queries of different granularity related to the performance, we run \toolname on the \etherscan data set with different queries. In particular, $\vulnerability$ denotes the approximate query discussed in Section~\ref{sec:vul}, and $\vulnerability^{\diamond}$ corresponds to more precise queries adopted from the previous work~\cite{Hirai17}. Specifically, the more precise queries in $\vulnerability^{\diamond}$ include not only constraints in $\vulnerability$ but also extra constraints that encode the invariant of storage balance. In other words, $\vulnerability^{\diamond}$ implies $\vulnerability$ (I.e., $\vulnerability^{\diamond} \implies \vulnerability$).
To speed up the evaluation, for both settings, we enable the
parallel synthesis optimization discussed in Section~\ref{sec:impl}. 
%
\begin{figure}
    \centering
    \begin{tikzpicture}[scale=0.93]
    \begin{axis}[%
    xmode=log,
    ymode=log,
    xmin=0.1,
    xmax=300,
    ymin=0.1,
    ymax=300,
    scatter/classes={%
        a={mark=o,draw=black}}]
    \draw (1, 5) node {Non-summary is better};
    \draw (3, -2) node {Summary-based is better};
    \addplot[mark=none, line width=1.5pt,blue] coordinates {(0.01,0.01) (300,300)};    
    \addplot[mark size=1.5pt,scatter,only marks,%
        scatter src=explicit symbolic]%
    table {
    x y 
    1546 12917 690 1846 Intersect:  142 s1 - s2: 548 s2 - s1: 1704
    20078.244707347447
    6018.453788592101
    35683.66937570942 335584.2864018161
    22025 22025
    0.953 0.269
    0.889 0.315
    2.252 0.434
    14.194 1.034
    360.0 1.765
    0.196 0.154
    2.508 0.518
    3.445 0.707
    1.375 0.407
    7.914 1.405
    2.359 0.527
    360.0 5.685
    360.0 2.442
    1.574 0.406
    4.756 0.712
    360.0 360.0
    360.0 2.581
    8.345 0.571
    1.899 0.383
    360.0 4.538
    1.388 0.341
    15.153 2.353
    360.0 360.0
    360.0 1.995
    1.579 0.487
    1.327 0.378
    360.0 5.523
    0.666 0.266
    12.695 0.818
    360.0 360.0
    10.996 1.337
    2.871 0.758
    13.725 1.284
    0.373 0.21
    55.654 1.787
    0.279 0.132
    0.367 0.164
    360.0 7.182
    0.344 0.198
    1.161 0.421
    1.653 0.516
    360.0 7.75
    11.035 2.817
    116.57 1.67
    2.109 0.607
    0.363 0.135
    360.0 1.733
    1.812 0.476
    1.14 0.315
    4.017 3.132
    1.257 3.31
    0.416 0.153
    1.422 0.395
    58.04 2.363
    360.0 360.0
    360.0 1.435
    360.0 9.468
    3.828 1.018
    360.0 1.489
    1.756 0.937
    360.0 360.0
    1.161 0.567
    1.153 0.653
    1.505 0.869
    360.0 6.346
    2.001 0.656
    360.0 4.322
    360.0 360.0
    360.0 2.71
    5.123 3.312
    39.151 360.0
    360.0 360.0
    360.0 7.118
    258.129 31.65
    77.53 2.724
    2.16 1.007
    1.302 0.423
    360.0 10.361
    0.303 0.38
    1.33 0.917
    7.008 1.212
    360.0 360.0
    1.036 0.699
    0.384 0.241
    2.549 1.13
    1.705 0.84
    1.355 0.864
    3.455 0.982
    360.0 10.454
    1.393 0.966
    41.784 3.993
    1.122 0.927
    1.041 0.864
    1.432 0.867
    360.0 3.782
    360.0 3.539
    1.273 0.705
    90.611 3.333
    360.0 4.503
    8.787 2.474
    56.444 360.0
    360.0 5.055
    360.0 360.0
    0.367 0.277
    360.0 4.268
    96.329 360.0
    360.0 5.113
    8.484 2.229
    360.0 4.997
    360.0 3.817
    8.968 2.925
    0.865 0.651
    0.273 0.331
    4.577 1.114
    113.311 2.684
    360.0 360.0
    360.0 360.0
    360.0 360.0
    360.0 4.437
    360.0 360.0
    4.826 3.018
    360.0 3.374
    2.253 0.64
    4.373 0.803
    1.19 0.563
    360.0 11.284
    0.344 0.338
    1.325 0.647
    1.207 0.779
    360.0 360.0
    164.686 4.312
    360.0 360.0
    78.161 4.142
    360.0 4.815
    360.0 6.051
    360.0 0.867
    1.068 0.616
    360.0 6.798
    1.027 0.873
    8.854 1.781
    93.723 4.169
    4.445 1.787
    55.48 2.71
    360.0 7.833
    0.816 0.552
    360.0 5.819
    1.053 0.516
    17.625 3.054
    360.0 6.24
    1.456 0.826
    1.604 0.717
    1.423 0.69
    1.49 0.989
    0.887 0.601
    3.026 1.142
    1.169 0.504
    360.0 18.403
    1.296 0.605
    1.904 2.372
    360.0 5.035
    11.428 8.085
    1.343 0.48
    10.107 5.253
    3.505 0.993
    12.955 2.325
    360.0 3.786
    1.796 0.77
    80.95 4.39
    1.581 0.583
    360.0 4.378
    360.0 7.022
    0.96 0.657
    52.402 12.362
    1.216 0.515
    360.0 7.834
    1.267 0.643
    0.375 0.34
    3.013 1.333
    360.0 26.286
    60.744 2.914
    360.0 360.0
    360.0 20.56
    1.624 0.535
    360.0 5.371
    360.0 5.144
    1.255 0.699
    360.0 4.072
    360.0 3.485
    4.261 1.209
    360.0 285.136
    0.902 0.565
    38.4 360.0
    1.528 0.757
    360.0 5.719
    360.0 5.214
    4.238 1.574
    1.341 0.942
    360.0 5.237
    1.283 0.853
    1.384 0.823
    360.0 9.529
    360.0 26.372
    0.396 0.256
    1.72 0.85
    360.0 2.845
    360.0 2.897
    0.314 0.371
    360.0 360.0
    56.067 2.593
    0.667 0.383
    1.086 0.701
    360.0 216.736
    360.0 5.0
    1.325 0.731
    360.0 7.757
    1.742 0.999
    22.397 2.024
    4.416 1.119
    1.254 0.836
    360.0 3.453
    360.0 4.698
    0.505 0.44
    360.0 7.596
    2.25 0.951
    6.496 3.043
    1.504 0.789
    12.945 2.73
    1.953 1.102
    1.377 0.595
    360.0 360.0
    42.74 3.487
    0.909 0.514
    360.0 360.0
    337.575 5.153
    1.594 0.819
    360.0 2.0
    13.261 4.068
    1.435 0.974
    1.952 0.926
    1.026 0.607
    360.0 4.322
    360.0 2.475
    6.162 1.765
    360.0 4.021
    0.971 0.628
    360.0 3.665
    9.708 2.635
    5.755 1.159
    3.961 0.891
    360.0 15.285
    8.91 2.245
    0.997 0.652
    360.0 33.038
    360.0 3.882
    62.019 2.646
    9.247 7.912
    1.818 0.875
    0.989 0.671
    1.387 0.513
    1.516 0.749
    190.256 2.255
    7.174 1.997
    360.0 26.986
    360.0 5.374
    360.0 48.606
    0.317 0.357
    1.363 0.882
    360.0 4.268
    360.0 39.609
    360.0 4.641
    0.717 0.403
    360.0 3.661
    0.38 0.311
    360.0 4.664
    360.0 4.696
    360.0 1.043
    360.0 5.005
    360.0 3.579
    360.0 360.0
    1.954 0.642
    360.0 3.631
    9.669 2.067
    58.123 32.114
    84.074 2.261
    360.0 5.827
    360.0 2.107
    360.0 3.776
    360.0 0.644
    0.885 0.47
    360.0 2.812
    0.363 0.37
    360.0 5.751
    1.749 0.919
    360.0 4.528
    1.599 0.812
    360.0 8.085
    360.0 4.032
    360.0 4.982
    4.866 1.679
    1.505 0.535
    0.816 0.559
    360.0 5.303
    360.0 4.676
    360.0 3.431
    360.0 360.0
    360.0 3.304
    55.447 2.564
    50.332 3.021
    360.0 360.0
    60.777 2.93
    360.0 360.0
    0.884 0.664
    0.408 0.484
    1.212 0.667
    360.0 360.0
    360.0 3.96
    360.0 0.385
    2.243 0.709
    360.0 4.973
    360.0 4.681
    1.962 0.814
    360.0 15.63
    3.911 0.972
    1.189 0.565
    3.679 1.279
    0.87 0.519
    4.093 3.696
    360.0 9.406
    0.637 0.489
    1.251 5.902
    360.0 10.424
    0.323 0.362
    2.025 0.823
    0.366 0.311
    360.0 3.829
    0.607 0.485
    360.0 6.862
    3.656 1.322
    1.017 0.289
    1.176 0.634
    360.0 4.474
    360.0 4.31
    4.472 1.346
    360.0 5.084
    4.414 0.942
    0.911 0.622
    360.0 4.268
    13.712 1.608
    1.431 0.898
    360.0 360.0
    360.0 3.366
    2.327 1.141
    360.0 5.607
    360.0 3.99
    2.17 1.312
    1.297 0.886
    360.0 360.0
    0.843 0.594
    360.0 16.416
    1.015 0.727
    360.0 3.806
    7.421 1.103
    57.568 2.375
    360.0 6.395
    16.564 2.431
    360.0 360.0
    360.0 5.528
    360.0 5.877
    1.35 0.622
    105.356 360.0
    1.766 0.899
    360.0 360.0
    5.501 0.81
    360.0 14.836
    0.831 0.596
    113.289 4.081
    360.0 4.404
    360.0 360.0
    5.412 2.238
    1.262 0.491
    2.443 0.853
    20.913 3.51
    1.601 0.853
    5.997 3.173
    360.0 5.127
    3.984 2.126
    360.0 2.687
    360.0 360.0
    360.0 18.001
    16.304 1.943
    2.209 0.984
    360.0 4.159
    360.0 21.356
    360.0 8.021
    1.546 0.808
    360.0 3.964
    4.628 0.986
    13.304 2.205
    0.67 0.582
    360.0 15.743
    14.094 1.63
    181.004 2.708
    1.216 0.596
    360.0 4.374
    0.377 0.328
    360.0 4.327
    1.164 0.628
    2.03 1.172
    1.272 0.743
    2.138 1.017
    360.0 4.365
    1.424 0.725
    360.0 282.959
    2.447 1.045
    360.0 3.741
    14.062 1.769
    360.0 8.506
    1.467 0.808
    360.0 360.0
    8.036 2.263
    360.0 360.0
    3.537 0.985
    360.0 360.0
    1.977 0.734
    0.909 0.61
    41.783 2.17
    360.0 29.116
    82.032 4.283
    116.284 2.983
    55.948 2.391
    4.194 1.143
    360.0 22.811
    12.973 1.553
    360.0 4.316
    360.0 3.139
    360.0 360.0
    360.0 4.457
    13.546 1.807
    360.0 360.0
    1.413 0.931
    360.0 5.547
    59.624 2.618
    54.437 2.757
    1.298 0.628
    56.348 360.0
    360.0 2.454
    1.758 1.045
    1.792 0.826
    360.0 3.29
    1.671 0.826
    0.372 0.357
    15.741 2.02
    2.485 0.999
    0.644 0.5
    0.987 0.987
    1.009 360.0
    360.0 360.0
    0.707 0.659
    4.773 1.177
    360.0 9.882
    0.386 0.416
    4.087 1.056
    18.77 2.749
    360.0 3.822
    360.0 4.028
    360.0 4.557
    360.0 3.116
    3.037 1.51
    360.0 4.91
    1.608 0.966
    13.734 1.71
    360.0 0.687
    2.223 1.003
    58.364 2.775
    360.0 10.421
    2.186 1.045
    5.237 1.25
    360.0 3.627
    360.0 4.8
    4.043 1.31
    360.0 5.278
    5.101 1.072
    1.792 0.565
    12.664 1.853
    1.841 0.596
    1.663 0.673
    360.0 13.325
    51.34 2.226
    2.14 0.864
    2.152 1.118
    1.263 0.748
    51.027 19.01
    24.624 2.722
    1.45 0.898
    4.203 1.436
    4.973 1.543
    1.872 0.983
    360.0 3.815
    360.0 360.0
    1.642 0.752
    1.36 0.85
    360.0 4.034
    360.0 4.729
    0.558 0.487
    21.757 360.0
    55.76 2.528
    1.857 0.739
    4.173 1.294
    360.0 3.776
    360.0 5.229
    56.946 2.424
    1.535 0.797
    58.548 9.614
    0.948 0.603
    12.812 1.851
    55.89 3.294
    14.859 4.072
    15.658 1.626
    57.218 3.221
    252.473 360.0
    360.0 2.993
    0.781 0.641
    360.0 0.796
    2.828 1.334
    360.0 2.597
    12.496 360.0
    0.657 0.491
    360.0 4.356
    5.516 1.018
    15.17 1.719
    360.0 0.585
    360.0 5.114
    360.0 360.0
    55.996 2.587
    2.013 0.887
    360.0 2.522
    360.0 4.719
    4.182 0.955
    360.0 3.554
    360.0 3.38
    115.698 2.979
    360.0 360.0
    0.369 0.273
    5.953 1.367
    1.651 0.821
    58.827 4.846
    1.539 1.08
    360.0 3.595
    28.016 4.098
    3.613 1.096
    19.464 360.0
    360.0 317.244
    360.0 11.402
    57.678 2.558
    0.896 0.592
    1.24 0.71
    0.357 0.348
    7.39 1.841
    56.153 2.9
    360.0 4.042
    360.0 3.69
    360.0 360.0
    1.643 0.823
    3.635 1.238
    1.677 0.946
    17.024 2.387
    360.0 360.0
    1.801 0.938
    360.0 3.54
    9.121 1.794
    117.756 2.685
    3.888 1.31
    1.446 0.802
    360.0 4.517
    1.519 0.772
    58.561 2.804
    360.0 8.219
    1.368 0.824
    360.0 5.434
    1.039 4.746
    1.334 0.786
    4.248 0.832
    360.0 2.373
    3.809 1.268
    360.0 5.359
    1.084 0.701
    360.0 360.0
    0.904 0.578
    104.63 4.248
    360.0 0.836
    84.62 2.832
    360.0 6.847
    360.0 344.458
    1.842 0.876
    112.471 2.771
    360.0 7.611
    360.0 4.324
    360.0 2.739
    360.0 4.594
    360.0 4.307
    360.0 7.596
    360.0 4.373
    57.49 2.69
    66.526 3.006
    2.486 360.0
    360.0 12.166
    27.855 2.741
    29.963 3.038
    1.225 0.847
    0.72 0.289
    1.213 0.615
    57.043 2.044
    7.986 17.722
    1.184 0.681
    16.127 5.616
    1.267 0.994
    360.0 5.105
    116.598 3.244
    1.479 0.701
    1.51 0.588
    0.985 0.666
    56.677 360.0
    360.0 7.662
    54.438 11.25
    99.542 3.056
    0.939 0.646
    360.0 3.919
    45.026 2.56
    1.12 0.832
    360.0 0.498
    1.469 0.756
    1.132 0.681
    63.93 35.807
    1.379 360.0
    360.0 5.394
    80.094 2.661
    360.0 360.0
    360.0 3.69
    2.256 0.932
    360.0 360.0
    312.267 1.699
    2.126 0.989
    0.363 0.35
    360.0 3.035
    3.661 1.26
    360.0 3.435
    360.0 44.532
    1.602 0.615
    0.793 0.658
    360.0 9.334
    1.679 4.887
    360.0 4.837
    360.0 29.095
    360.0 3.214
    14.238 4.235
    360.0 2.514
    1.916 0.833
    58.285 360.0
    109.741 2.869
    0.374 0.372
    1.233 0.782
    2.533 0.794
    2.185 0.954
    7.627 1.878
    19.458 3.562
    1.287 0.839
    6.591 3.621
    1.759 0.976
    4.124 1.262
    229.053 2.87
    360.0 4.534
    0.347 0.27
    1.423 0.608
    360.0 360.0
    1.795 0.782
    360.0 15.059
    360.0 360.0
    360.0 230.606
    0.66 0.65
    2.09 0.911
    360.0 4.408
    360.0 12.505
    360.0 4.178
    1.069 0.666
    360.0 3.633
    0.643 0.601
    0.577 0.462
    1.806 0.959
    0.916 0.477
    2.301 0.579
    54.388 4.343
    360.0 5.429
    345.885 35.349
    53.889 3.161
    13.242 1.425
    360.0 4.011
    87.181 2.601
    2.248 0.714
    360.0 6.015
    1.216 0.631
    360.0 4.473
    360.0 3.211
    360.0 4.251
    1.628 0.686
    360.0 4.513
    360.0 5.446
    360.0 5.142
    360.0 4.052
    6.15 1.943
    360.0 4.473
    18.197 1.882
    57.764 2.856
    360.0 45.581
    2.415 1.039
    1.341 0.885
    1.879 0.542
    1.385 0.849
    360.0 7.019
    360.0 4.087
    60.713 3.628
    172.906 4.2
    1.114 0.676
    85.802 2.986
    1.185 0.508
    8.461 1.835
    4.489 1.059
    12.222 1.613
    4.137 1.047
    360.0 17.869
    1.185 0.815
    360.0 12.682
    360.0 6.79
    1.098 0.585
    13.767 1.745
    360.0 1.824
    1.035 0.723
    85.873 2.151
    1.433 0.567
    2.918 0.809
    1.004 0.478
    360.0 360.0
    0.334 0.386
    58.972 19.937
    360.0 2.573
    0.916 0.645
    0.977 0.607
    4.048 1.403
    0.367 0.374
    1.142 0.725
    1.421 0.774
    57.119 2.765
    11.93 3.213
    360.0 13.513
    360.0 4.29
    360.0 13.329
    0.202 0.407
    61.327 2.797
    360.0 5.084
    240.43 4.563
    1.081 0.549
    110.941 2.814
    0.925 0.57
    1.472 0.726
    0.613 0.507
    64.882 2.713
    360.0 93.163
    14.709 2.734
    60.406 2.886
    1.227 0.596
    2.271 0.967
    360.0 360.0
    360.0 5.66
    1.522 0.856
    2.729 1.082
    360.0 360.0
    1.237 0.727
    360.0 3.651
    1.207 0.847
    360.0 4.398
    360.0 5.768
    25.122 3.958
    360.0 3.083
    360.0 12.042
    12.571 2.501
    58.549 2.627
    360.0 3.664
    360.0 360.0
    1.079 0.686
    360.0 54.901
    15.256 1.587
    1.413 0.751
    360.0 4.802
    360.0 360.0
    360.0 4.637
    360.0 2.611
    1.539 0.733
    2.041 0.996
    1.407 0.492
    13.014 2.805
    360.0 4.061
    360.0 5.312
    193.856 17.012
    0.29 0.381
    0.351 0.345
    360.0 4.323
    1.533 0.926
    360.0 3.502
    0.183 0.378
    0.275 0.397
    1.213 0.679
    5.493 1.34
    2.674 0.841
    1.49 0.792
    360.0 2.625
    59.254 2.925
    3.641 1.254
    360.0 14.272
    360.0 4.341
    1.679 0.992
    4.31 1.445
    2.5 0.66
    1.357 0.788
    23.787 3.002
    360.0 360.0
    360.0 360.0
    2.354 0.676
    1.059 0.682
    2.107 0.875
    58.92 2.652
    360.0 23.427
    1.4 0.837
    3.503 1.683
    360.0 4.444
    2.224 0.988
    360.0 3.827
    0.997 0.673
    360.0 13.549
    360.0 4.105
    360.0 43.814
    360.0 3.973
    360.0 2.464
    0.374 0.32
    7.165 360.0
    360.0 360.0
    360.0 11.636
    360.0 4.03
    13.616 1.492
    360.0 360.0
    360.0 360.0
    2.068 1.051
    58.444 2.842
    0.949 0.592
    360.0 4.791
    360.0 360.0
    0.831 0.53
    2.712 0.957
    360.0 3.474
    1.989 0.793
    1.952 0.934
    1.452 0.985
    360.0 360.0
    1.196 0.474
    360.0 360.0
    54.616 2.701
    1.991 1.109
    360.0 360.0
    1.03 0.378
    1.155 0.762
    4.41 1.225
    86.896 2.463
    10.127 2.924
    0.831 0.731
    5.982 1.049
    14.963 360.0
    360.0 4.122
    0.298 0.357
    124.756 4.204
    13.487 1.584
    56.943 5.138
    2.359 0.808
    3.778 1.163
    16.784 2.045
    0.428 0.367
    0.984 0.441
    360.0 5.181
    1.007 0.684
    1.232 0.665
    360.0 360.0
    360.0 3.191
    1.716 0.806
    360.0 360.0
    3.847 1.12
    360.0 6.795
    360.0 5.728
    360.0 360.0
    360.0 5.678
    360.0 360.0
    360.0 5.778
    0.972 0.596
    360.0 13.001
    360.0 4.521
    0.383 0.372
    1.81 0.794
    360.0 5.197
    360.0 66.977
    54.136 2.912
    360.0 360.0
    0.919 0.745
    360.0 360.0
    1.386 0.807
    1.57 0.828
    58.067 3.22
    360.0 360.0
    0.403 0.321
    360.0 7.568
    29.927 4.308
    360.0 3.75
    1.354 0.773
    0.893 0.271
    1.482 0.701
    360.0 13.219
    360.0 360.0
    58.508 3.225
    360.0 3.578
    78.519 6.678
    0.386 0.362
    360.0 4.624
    2.047 0.971
    360.0 2.634
    360.0 360.0
    1.764 0.904
    141.935 4.208
    360.0 3.959
    360.0 3.827
    1.455 0.65
    1.584 0.728
    360.0 2.917
    5.882 0.937
    9.064 3.644
    1.526 0.654
    3.39 1.052
    15.363 1.731
    1.618 0.798
    360.0 3.196
    0.942 0.73
    0.232 0.333
    1.057 0.726
    360.0 3.557
    360.0 4.789
    360.0 6.74
    360.0 6.343
    9.926 1.535
    53.926 3.265
    360.0 360.0
    360.0 4.651
    10.53 4.091
    1.423 0.903
    360.0 2.942
    97.244 27.991
    11.795 2.243
    360.0 4.071
    3.754 1.0
    4.274 1.206
    0.362 0.241
    1.703 1.011
    1.513 0.785
    1.224 0.688
    360.0 4.06
    3.582 0.923
    360.0 5.858
    360.0 6.65
    1.617 0.704
    360.0 360.0
    14.993 2.708
    360.0 4.87
    1.798 0.734
    10.25 2.529
    0.98 0.653
    1.812 0.867
    360.0 4.939
    56.116 2.411
    5.151 1.004
    1.349 0.65
    360.0 47.85
    56.894 2.021
    360.0 6.133
    360.0 2.113
    1.21 0.702
    1.209 0.56
    15.588 3.566
    360.0 17.005
    2.487 0.691
    360.0 18.366
    360.0 4.787
    1.204 0.628
    360.0 360.0
    1.271 0.852
    1.723 1.001
    0.7 0.584
    360.0 4.311
    360.0 4.116
    1.679 0.816
    0.545 0.469
    3.783 81.256
    360.0 360.0
    360.0 4.042
    34.671 8.789
    0.939 0.573
    360.0 360.0
    360.0 4.177
    2.086 0.82
    1.521 0.734
    360.0 3.644
    154.748 360.0
    5.825 1.783
    0.832 0.503
    0.85 360.0
    55.363 2.597
    360.0 0.641
    1.245 0.506
    42.14 3.586
    1.801 0.686
    360.0 217.574
    4.021 1.254
    1.658 0.701
    57.68 2.825
    5.014 1.123
    360.0 15.713
    360.0 360.0
    54.702 2.859
    360.0 5.253
    1.098 0.544
    0.354 0.342
    3.719 1.279
    360.0 360.0
    1.102 0.648
    1.53 0.781
    360.0 7.563
    9.654 4.217
    1.027 0.729
    360.0 5.426
    2.501 1.045
    3.172 360.0
    1.269 12.904
    1.12 0.527
    1.794 0.667
    1.796 0.803
    60.602 2.521
    40.557 4.373
    1.333 0.508
    3.6 0.927
    0.336 0.333
    360.0 39.551
    360.0 360.0
    0.375 0.406
    360.0 3.214
    360.0 13.207
    360.0 3.695
    80.091 2.765
    360.0 5.374
    360.0 3.74
    360.0 5.188
    360.0 8.664
    3.695 1.168
    1.185 0.488
    360.0 360.0
    360.0 5.22
    2.339 1.01
    360.0 360.0
    15.61 3.299
    1.197 0.681
    360.0 50.698
    5.812 1.553
    360.0 12.558
    360.0 2.881
    360.0 4.234
    360.0 3.826
    0.876 0.49
    360.0 3.946
    360.0 2.834
    360.0 21.744
    58.992 2.75
    1.03 0.573
    1.398 0.76
    0.433 0.398
    360.0 37.672
    4.693 1.735
    4.276 1.002
    0.804 0.653
    360.0 360.0
    360.0 24.039
    13.109 2.084
    360.0 5.917
    2.126 0.919
    360.0 3.374
    0.408 0.469
    17.865 2.005
    57.296 2.86
    1.245 0.655
    19.48 8.627
    360.0 5.409
    360.0 7.113
    0.927 0.541
    0.387 0.35
    1.697 0.493
    360.0 4.98
    1.901 0.684
    360.0 24.222
    4.009 1.012
    58.623 3.574
    54.063 2.413
    2.022 0.888
    360.0 102.988
    360.0 3.905
    1.646 0.672
    4.243 1.302
    4.104 1.076
    360.0 360.0
    1.894 1.074
    12.41 6.548
    10.816 2.806
    360.0 1.031
    1.266 0.593
    19.409 5.057
    360.0 360.0
    12.611 2.202
    10.271 2.035
    1.582 0.911
    12.46 1.557
    1.316 0.76
    3.949 1.558
    1.567 0.878
    360.0 19.271
    0.872 0.663
    360.0 4.933
    41.906 11.289
    360.0 4.245
    360.0 360.0
    1.654 0.903
    360.0 4.268
    1.721 0.657
    1.588 0.728
    360.0 360.0
    360.0 360.0
    1.595 1.013
    360.0 28.912
    360.0 4.585
    0.343 0.342
    1.998 0.973
    0.906 0.496
    0.984 0.624
    1.775 0.774
    360.0 6.333
    360.0 3.299
    360.0 5.063
    5.369 2.409
    5.321 1.797
    360.0 6.729
    1.181 0.572
    4.721 0.91
    0.554 0.223
    360.0 3.391
    152.334 12.875
    360.0 360.0
    360.0 2.853
    1.718 0.975
    360.0 6.31
    1.938 0.942
    360.0 2.192
    1.577 0.877
    360.0 3.642
    360.0 3.279
    360.0 360.0
    0.928 0.619
    0.408 0.258
    55.456 2.474
    2.712 1.275
    360.0 2.044
    360.0 3.54
    360.0 3.707
    0.39 0.341
    360.0 3.569
    0.812 0.378
    360.0 15.95
    3.547 1.019
    1.989 0.733
    1.807 0.96
    58.357 3.216
    1.222 0.815
    360.0 6.672
    0.307 0.428
    0.972 0.619
    0.301 0.374
    22.282 2.319
    1.865 0.979
    360.0 1.356
    360.0 4.131
    3.397 1.359
    360.0 11.636
    1.61 0.647
    4.654 1.296
    1.469 0.565
    360.0 4.543
    3.052 1.073
    112.789 3.459
    1.157 0.686
    15.277 1.694
    0.371 0.352
    1.108 0.874
    3.303 1.25
    0.316 0.409
    1.719 0.704
    360.0 2.609
    0.784 0.486
    15.524 1.291
    114.025 3.301
    1.357 0.621
    4.19 1.073
    360.0 3.181
    116.639 3.097
    1.777 0.797
    360.0 360.0
    360.0 2.92
    1.381 0.667
    0.387 0.365
    1.971 0.587
    117.974 7.723
    330.388 360.0
    1.475 0.904
    360.0 360.0
    3.773 1.212
    2.906 41.276
    1.168 0.852
    1.006 0.796
    1.76 0.647
    3.773 1.117
    1.516 0.614
    2.66 1.155
    0.781 0.41
    3.932 1.134
    0.874 0.616
    31.98 3.85
    360.0 31.193
    1.009 0.559
    10.181 5.325
    360.0 360.0
    1.294 0.43
    360.0 7.546
    117.74 3.147
    0.975 0.563
    9.36 2.232
    1.912 1.13
    3.905 1.351
    1.742 0.94
    3.907 0.857
    1.236 0.533
    360.0 360.0
    360.0 4.035
    1.752 1.543
    1.824 0.575
    360.0 4.798
    1.817 0.811
    2.348 0.771
    1.023 0.67
    10.878 2.082
    1.787 1.082
    4.691 1.063
    360.0 5.306
    12.499 2.237
    360.0 247.137
    5.232 360.0
    1.45 0.705
    2.383 0.76
    338.716 6.194
    0.38 0.38
    178.78 2.866
    1.818 0.802
    1.592 0.489
    1.535 0.677
    182.4 9.042
    2.353 0.865
    2.057 0.797
    360.0 5.393
    57.601 2.657
    1.083 0.617
    0.738 0.476
    7.872 2.062
    3.725 0.984
    3.272 1.601
    360.0 5.677
    1.005 0.652
    360.0 7.136
    130.889 2.257
    360.0 3.283
    360.0 2.35
    1.493 0.796
    2.487 0.814
    360.0 360.0
    5.878 1.521
    13.73 9.994
    1.418 0.777
    10.373 5.422
    360.0 4.079
    360.0 6.169
    2.209 0.957
    360.0 12.095
    1.844 0.957
    2.229 1.388
    360.0 7.811
    1.743 0.557
    360.0 4.592
    1.15 0.446
    1.661 0.911
    360.0 360.0
    1.265 0.587
    360.0 7.996
    360.0 1.042
    0.827 0.421
    21.605 360.0
    360.0 360.0
    4.099 1.565
    360.0 4.134
    360.0 14.636
    54.311 4.275
    85.277 360.0
    3.379 1.443
    4.611 0.884
    1.243 0.531
    1.051 0.603
    360.0 360.0
    4.95 1.193
    0.346 0.377
    0.9 0.603
    72.077 31.44
    360.0 4.647
    1.218 0.62
    0.393 0.357
    1.322 0.848
    360.0 6.972
    1.767 0.89
    1.245 0.555
    360.0 2.627
    360.0 2.883
    360.0 271.167
    0.743 0.654
    360.0 6.969
    360.0 360.0
    4.68 0.929
    1.359 0.539
    13.255 2.162
    360.0 4.005
    360.0 4.839
    0.783 0.545
    360.0 3.819
    0.322 0.394
    77.673 4.563
    6.899 1.799
    13.84 1.411
    360.0 8.233
    48.401 2.455
    10.715 1.939
    1.071 0.675
    360.0 2.595
    56.895 16.006
    1.289 0.808
    89.177 3.541
    6.4 3.324
    0.807 0.551
    360.0 8.103
    1.967 0.8
    59.228 2.763
    360.0 5.948
    0.38 0.395
    4.504 1.318
    0.499 0.501
    1.875 1.04
    1.737 0.843
    360.0 255.844
    0.338 0.406
    360.0 4.183
    1.129 0.6
    1.951 1.039
    360.0 4.518
    0.798 0.673
    0.402 0.365
    360.0 360.0
    360.0 14.555
    1.384 0.759
    0.769 0.573
    360.0 9.13
    360.0 4.912
    11.265 1.744
    350.53 6.532
    0.362 0.384
    0.905 0.461
    360.0 3.222
    0.413 0.407
    360.0 4.381
    0.707 0.44
    360.0 6.974
    360.0 4.08
    1.283 0.605
    56.872 2.621
    0.352 0.372
    1.218 0.743
    1.664 0.691
    360.0 360.0
    0.267 0.405
    360.0 2.412
    0.884 0.582
    1.958 0.625
    1.195 0.731
    8.693 1.985
    0.828 0.482
    360.0 360.0
    360.0 1.878
    12.518 1.511
    1.401 0.678
    360.0 0.686
    114.71 2.847
    1.377 0.708
    360.0 9.891
    360.0 174.097
    2.389 0.488
    360.0 21.218
    82.244 2.498
    360.0 3.751
    360.0 360.0
    10.684 2.247
    360.0 1.224
    1.777 2.838
    360.0 2.754
    0.578 0.348
    1.275 0.678
    1.186 0.299
    2.253 0.742
    0.701 0.486
    360.0 4.557
    360.0 4.209
    1.544 360.0
    1.849 0.711
    1.54 0.947
    360.0 3.894
    0.377 0.269
    360.0 10.877
    360.0 7.468
    360.0 6.957
    360.0 71.783
    2.141 1.049
    360.0 4.73
    360.0 33.351
    360.0 4.943
    360.0 2.247
    1.438 0.631
    360.0 8.46
    0.369 0.411
    360.0 3.585
    2.01 0.797
    1.303 0.726
    360.0 4.307
    360.0 3.947
    1.346 0.887
    149.354 10.588
    0.363 0.36
    32.007 3.025
    0.272 0.37
    360.0 4.008
    1.723 1.038
    1.059 0.386
    360.0 360.0
    1.724 1.162
    360.0 4.126
    360.0 360.0
    33.467 3.387
    360.0 6.504
    74.713 2.86
    360.0 3.878
    1.117 0.581
    1.535 0.949
    360.0 360.0
    57.298 3.117
    360.0 77.015
    14.598 1.677
    0.596 0.485
    360.0 2.215
    102.005 360.0
    2.173 0.963
    1.656 0.682
    12.988 9.175
    4.159 1.175
    116.712 360.0
    60.436 2.76
    0.596 0.525
    1.145 0.955
    169.218 2.957
    360.0 9.426
    360.0 0.559
    360.0 360.0
    3.73 1.282
    360.0 4.727
    360.0 360.0
    1.509 1.019
    2.082 1.097
    360.0 360.0
    360.0 360.0
    360.0 3.762
    360.0 85.744
    360.0 3.123
    360.0 2.54
    360.0 5.863
    360.0 5.543
    1.52 0.698
    79.762 10.926
    0.383 0.304
    5.961 2.034
    0.805 0.484
    2.19 0.929
    1.587 0.781
    6.051 1.686
    0.168 0.381
    1.286 0.657
    360.0 2.816
    1.089 0.733
    4.135 1.518
    110.254 3.955
    360.0 2.882
    360.0 3.964
    1.225 0.815
    1.381 0.748
    360.0 28.526
    7.359 2.485
    0.365 0.451
    1.369 0.904
    1.183 0.531
    1.207 0.604
    55.96 2.743
    360.0 4.932
    360.0 4.083
    1.369 0.637
    3.25 0.811
    360.0 360.0
    0.378 0.33
    360.0 14.162
    0.964 0.612
    360.0 3.279
    8.317 2.447
    360.0 6.694
    68.724 4.347
    56.8 360.0
    0.987 0.509
    360.0 29.83
    360.0 3.633
    360.0 3.138
    1.71 1.022
    99.927 14.42
    1.161 0.88
    360.0 360.0
    56.63 4.276
    74.512 2.378
    1.844 0.822
    360.0 360.0
    0.337 0.357
    1.968 0.886
    54.706 2.72
    1.863 0.611
    8.1 4.473
    14.566 2.947
    360.0 5.105
    360.0 3.389
    360.0 63.455
    360.0 4.772
    1.287 0.506
    1.681 0.799
    360.0 2.982
    360.0 7.189
    1.33 0.954
    360.0 4.398
    54.47 2.178
    0.763 0.462
    360.0 22.776
    0.391 0.363
    1.622 0.607
    6.264 0.726
    360.0 6.633
    360.0 4.749
    0.805 0.591
    0.254 0.364
    360.0 4.711
    61.426 2.987
    28.329 22.39
    311.163 360.0
    29.657 2.092
    360.0 3.47
    360.0 5.066
    4.21 1.065
    360.0 4.603
    360.0 2.767
    0.933 0.579
    1.576 0.699
    56.111 2.571
    58.781 2.526
    12.882 5.859
    1.457 0.685
    41.926 2.882
    114.106 2.117
    35.155 6.616
    50.472 4.463
    85.412 3.387
    1.238 0.634
    360.0 14.32
    360.0 360.0
    1.955 0.741
    3.127 1.099
    1.104 0.561
    0.967 0.638
    0.949 0.491
    360.0 360.0
    1.222 0.532
    2.552 0.798
    360.0 110.748
    13.203 2.562
    5.694 1.998
    110.635 1.958
    3.799 21.257
    360.0 4.77
    55.836 2.743
    11.166 1.515
    0.394 0.381
    360.0 3.562
    360.0 13.784
    4.149 1.288
    360.0 1.278
    360.0 8.368
    37.02 17.411
    0.962 0.73
    1.168 0.645
    8.845 3.781
    360.0 360.0
    0.639 0.509
    360.0 4.631
    360.0 360.0
    8.55 2.384
    360.0 3.842
    0.649 0.608
    360.0 17.421
    13.34 1.987
    14.43 3.969
    55.24 2.683
    1.496 0.692
    9.048 4.469
    4.215 1.005
    1.952 0.997
    360.0 10.481
    360.0 3.558
    1.917 0.758
    1.144 0.634
    360.0 43.905
    360.0 2.34
    360.0 2.182
    59.001 2.86
    117.944 2.829
    1.454 0.733
    200.624 3.29
    4.3 1.438
    360.0 12.576
    123.271 2.409
    91.499 4.53
    19.364 9.028
    360.0 3.822
    56.996 2.586
    360.0 3.211
    360.0 12.737
    360.0 10.891
    360.0 3.827
    0.447 0.539
    1.836 0.671
    12.419 1.055
    3.019 1.796
    360.0 360.0
    17.38 4.793
    360.0 360.0
    1.446 0.976
    360.0 3.31
    1.759 1.067
    360.0 360.0
    25.538 3.355
    360.0 30.959
    360.0 0.635
    360.0 360.0
    0.372 0.272
    1.103 0.771
    9.472 2.493
    174.622 4.988
    2.208 1.036
    56.819 2.969
    176.382 22.17
    1.503 0.558
    0.958 0.644
    57.043 2.77
    360.0 360.0
    360.0 2.287
    263.634 360.0
    3.765 1.036
    360.0 3.751
    360.0 4.981
    1.214 0.827
    0.523 360.0
    1.664 0.679
    13.243 2.201
    360.0 2.336
    360.0 3.205
    1.256 0.914
    1.832 0.898
    1.336 360.0
    1.557 0.682
    360.0 11.57
    360.0 2.193
    0.773 0.58
    0.803 0.653
    360.0 4.943
    360.0 360.0
    2.066 0.922
    360.0 4.382
    360.0 3.185
    5.005 1.054
    28.661 360.0
    10.831 1.701
    1.586 3.313
    360.0 6.533
    1.23 3.74
    15.155 1.664
    112.145 6.346
    0.339 0.401
    1.833 0.973
    0.629 0.523
    360.0 4.754
    360.0 360.0
    115.404 2.902
    0.375 0.36
    55.635 2.778
    4.671 1.757
    0.393 0.422
    2.02 0.783
    230.287 20.352
    360.0 1.222
    360.0 360.0
    360.0 7.652
    40.021 3.678
    360.0 2.723
    1.06 0.696
    360.0 7.05
    0.702 0.522
    1.369 0.856
    1.678 0.639
    360.0 360.0
    360.0 5.031
    1.34 0.716
    360.0 5.277
    360.0 4.324
    1.934 1.076
    360.0 32.133
    0.932 0.458
    360.0 5.333
    3.266 1.053
    360.0 13.951
    57.787 2.747
    360.0 3.709
    1.798 1.054
    0.395 0.183
    1.174 0.908
    360.0 4.483
    6.881 2.515
    0.621 0.678
    60.75 3.259
    360.0 3.201
    360.0 4.663
    360.0 360.0
    1.066 0.834
    1.585 0.966
    3.729 1.292
    1.581 0.725
    4.028 1.388
    360.0 4.402
    112.495 4.198
    12.007 1.623
    0.342 0.451
    3.774 2.222
    360.0 2.748
    360.0 4.993
    0.179 0.363
    14.919 1.643
    0.97 0.52
    3.674 0.94
    59.391 2.448
    1.048 0.464
    0.789 0.582
    13.351 2.068
    2.139 1.022
    1.601 0.93
    0.801 0.605
    360.0 4.11
    1.935 0.979
    360.0 4.469
    360.0 2.507
    1.639 0.994
    0.37 0.363
    360.0 11.499
    3.202 1.603
    3.865 0.877
    360.0 3.703
    360.0 45.106
    0.665 0.487
    5.254 4.718
    2.307 0.776
    360.0 4.216
    54.869 2.568
    1.336 0.724
    360.0 0.848
    7.981 2.462
    1.91 0.956
    0.959 0.695
    4.306 1.28
    0.731 0.496
    4.57 2.584
    59.102 2.701
    360.0 6.292
    351.388 85.278
    360.0 6.438
    360.0 17.465
    58.519 3.218
    360.0 360.0
    360.0 3.27
    360.0 2.16
    360.0 21.18
    1.311 360.0
    360.0 9.6
    57.457 3.36
    101.315 4.615
    1.8 0.895
    0.38 0.292
    1.207 0.692
    3.595 0.879
    360.0 4.387
    360.0 5.598
    0.388 0.418
    9.094 2.987
    1.619 0.978
    360.0 2.08
    360.0 4.116
    349.262 4.348
    360.0 9.4
    360.0 4.779
    360.0 8.645
    1.799 0.892
    360.0 6.025
    1.3 0.379
    1.974 0.873
    5.566 1.125
    360.0 35.715
    2.283 0.901
    1.856 0.664
    360.0 3.86
    360.0 4.152
    1.813 0.675
    360.0 5.273
    360.0 22.929
    1.968 0.674
    360.0 0.695
    360.0 3.796
    1.577 0.811
    360.0 9.248
    360.0 2.255
    1.839 0.874
    360.0 4.378
    360.0 3.598
    1.669 0.695
    1.784 0.984
    1.055 0.498
    1.131 0.721
    360.0 3.149
    0.734 0.638
    4.237 1.553
    1.103 0.44
    1.236 0.552
    59.043 2.987
    55.024 2.716
    1.173 0.706
    25.234 2.369
    360.0 18.323
    360.0 360.0
    1.405 0.829
    1.189 0.715
    1.657 0.504
    0.82 0.448
    0.759 0.616
    360.0 7.613
    1.464 0.642
    178.145 2.544
    3.452 4.076
    360.0 77.588
    360.0 360.0
    360.0 360.0
    360.0 49.203
    360.0 360.0
    56.072 3.204
    360.0 360.0
    0.972 0.67
    1.837 0.691
    360.0 2.354
    3.976 0.947
    22.835 3.794
    360.0 5.378
    112.118 2.759
    360.0 4.748
    14.124 2.372
    0.849 0.632
    0.395 0.281
    18.798 5.212
    360.0 15.309
    1.656 0.811
    1.055 0.629
    360.0 12.002
    360.0 5.592
    18.775 2.715
    8.823 1.598
    1.093 0.634
    1.087 0.538
    90.322 2.952
    112.769 2.516
    360.0 8.941
    1.808 0.935
    38.023 4.714
    110.62 2.611
    360.0 3.994
    360.0 4.854
    0.25 0.386
    360.0 24.533
    360.0 3.816
    17.482 1.88
    360.0 4.713
    360.0 360.0
    360.0 360.0
    0.891 0.643
    360.0 360.0
    17.191 2.0
    360.0 10.083
    360.0 6.42
    360.0 2.474
    360.0 5.809
    360.0 318.395
    0.264 0.374
    360.0 3.688
    0.922 0.652
    360.0 4.264
    117.151 10.431
    1.311 0.677
    360.0 33.234
    360.0 4.489
    0.547 0.377
    9.905 3.002
    360.0 0.591
    4.951 2.378
    36.176 2.846
    2.1 0.773
    105.504 6.326
    2.18 0.815
    360.0 13.463
    0.563 0.467
    1.72 0.771
    360.0 23.248
    55.386 2.955
    0.379 0.379
    3.839 1.17
    360.0 2.847
    360.0 3.968
    56.27 2.845
    360.0 2.424
    1.12 0.663
    360.0 26.988
    360.0 3.793
    34.483 4.377
    3.179 1.087
    360.0 0.522
    360.0 4.415
    360.0 4.63
    0.388 0.361
    360.0 3.638
    360.0 8.696
    360.0 10.039
    1.063 0.683
    360.0 54.474
    360.0 4.697
    360.0 5.743
    360.0 360.0
    360.0 21.799
    360.0 0.707
    117.999 2.8
    360.0 360.0
    360.0 1.137
    0.705 0.425
    57.859 1.58
    360.0 0.402
    1.549 0.597
    2.048 0.658
    360.0 7.43
    68.058 6.435
    1.218 0.658
    360.0 296.536
    116.266 2.812
    360.0 3.262
    360.0 4.505
    360.0 3.643
    107.198 2.749
    360.0 4.312
    360.0 10.992
    360.0 0.789
    3.932 1.27
    5.047 1.109
    360.0 2.718
    1.826 0.695
    360.0 5.235
    19.482 2.765
    1.328 0.634
    12.329 1.96
    0.285 0.309
    0.342 0.374
    40.263 360.0
    1.19 0.416
    23.075 3.877
    8.474 2.368
    360.0 3.075
    1.6 0.73
    360.0 360.0
    360.0 7.367
    38.883 360.0
    360.0 0.681
    4.731 1.335
    360.0 0.602
    360.0 0.589
    2.356 1.482
    23.305 4.045
    360.0 5.886
    360.0 1.931
    0.972 0.6
    360.0 40.241
    156.631 360.0
    360.0 3.114
    360.0 3.416
    360.0 37.869
    360.0 26.754
    1.406 33.643
    1.345 0.718
    360.0 0.843
    0.851 0.633
    0.373 0.249
    360.0 3.61
    360.0 2.913
    55.471 3.79
    360.0 3.911
    1.352 0.87
    1.703 0.981
    360.0 360.0
    2.159 0.994
    116.38 360.0
    360.0 12.243
    360.0 4.913
    329.878 4.25
    360.0 0.981
    1.107 0.562
    1.161 0.669
    1.309 0.683
    360.0 0.836
    58.386 3.143
    115.204 2.978
    360.0 4.049
    5.256 1.274
    1.249 0.513
    360.0 2.982
    360.0 5.939
    1.597 0.823
    4.194 1.031
    1.801 0.902
    360.0 360.0
    1.219 0.853
    1.223 0.491
    360.0 4.348
    0.814 0.431
    360.0 0.868
    2.077 0.89
    360.0 0.524
    360.0 0.705
    1.599 0.852
    360.0 0.398
    360.0 360.0
    360.0 0.609
    360.0 5.378
    360.0 0.987
    360.0 2.735
    360.0 174.376
    360.0 5.273
    360.0 5.93
    19.278 2.87
    360.0 0.864
    360.0 360.0
    0.677 0.663
    360.0 2.888
    3.193 0.825
    360.0 360.0
    360.0 4.455
    360.0 2.602
    360.0 5.072
    0.975 0.654
    7.931 4.466
    1.081 0.798
    360.0 16.598
    0.887 0.621
    360.0 2.692
    61.525 2.612
    10.028 2.655
    139.515 3.752
    3.209 1.369
    1.338 0.772
    112.273 5.177
    1.408 360.0
    0.652 0.47
    13.173 1.555
    1.722 0.744
    1.789 0.869
    1.033 0.287
    2.368 1.053
    0.573 0.479
    360.0 360.0
    1.331 0.884
    360.0 15.103
    360.0 5.618
    360.0 0.764
    57.914 2.101
    71.661 6.569
    11.336 1.788
    11.203 360.0
    4.778 2.063
    1.413 0.753
    3.622 1.103
    7.941 4.359
    360.0 12.627
    1.301 0.658
    360.0 360.0
    0.979 0.687
    360.0 360.0
    0.386 0.378
    0.35 0.296
    360.0 5.096
    360.0 2.709
    0.637 0.586
    13.211 1.614
    360.0 0.467
    1.445 0.546
    47.129 38.014
    360.0 1.901
    6.78 2.911
    3.518 1.333
    1.584 2.945
    1.554 0.45
    360.0 1.038
    360.0 360.0
    1.69 0.968
    1.346 0.869
    360.0 1.144
    11.34 1.325
    27.236 4.723
    360.0 2.826
    360.0 5.825
    126.618 3.72
    22.052 4.853
    360.0 3.805
    8.955 4.024
    360.0 4.016
    360.0 0.73
    1.698 0.581
    142.63 6.453
    4.015 1.293
    360.0 0.367
    84.884 8.038
    2.007 0.879
    3.621 0.639
    360.0 9.25
    1.976 0.969
    360.0 6.166
    1.914 0.801
    360.0 3.595
    1.603 0.803
    360.0 3.277
    55.216 2.195
    360.0 360.0
    55.557 1.679
    21.894 2.559
    360.0 0.499
    0.92 0.543
    360.0 0.715
    1.21 0.634
    1.384 0.857
    360.0 3.0
    360.0 360.0
    360.0 5.271
    360.0 1.298
    12.151 1.758
    1.761 0.658
    360.0 360.0
    1.573 0.795
    360.0 5.726
    3.73 1.003
    360.0 2.693
    1.327 6.018
    360.0 5.366
    0.759 0.542
    3.069 1.273
    10.048 1.639
    5.329 1.043
    1.734 0.646
    360.0 35.787
    4.008 1.024
    360.0 0.608
    20.421 3.893
    0.631 0.52
    1.547 0.736
    1.752 0.841
    3.766 0.969
    19.578 6.809
    5.136 2.502
    0.367 0.355
    58.897 2.292
    0.415 0.251
    54.125 360.0
    1.086 0.627
    360.0 18.37
    360.0 360.0
    360.0 92.253
    3.234 1.864
    1.883 0.883
    360.0 0.649
    120.343 2.932
    360.0 0.781
    360.0 11.034
    196.813 4.526
    1.229 0.58
    360.0 8.382
    2.225 0.746
    360.0 2.382
    360.0 3.595
    18.138 3.063
    360.0 2.779
    1.03 0.583
    57.383 2.801
    360.0 5.955
    1.275 0.649
    360.0 69.363
    26.172 2.816
    360.0 5.356
    360.0 1.082
    360.0 4.938
    360.0 0.62
    360.0 0.99
    15.906 12.826
    360.0 8.695
    360.0 6.073
    12.421 2.839
    172.579 5.734
    360.0 360.0
    55.196 2.786
    1.445 0.784
    360.0 19.642
    34.811 360.0
    53.13 2.248
    360.0 0.826
    360.0 51.112
    1.191 0.635
    360.0 6.661
    360.0 3.818
    181.241 4.496
    360.0 0.337
    1.678 7.124
    360.0 4.037
    1.47 0.659
    360.0 2.402
    3.904 1.336
    0.401 0.516
    3.665 0.992
    1.216 0.616
    360.0 2.579
    9.748 360.0
    2.004 0.97
    360.0 0.339
    360.0 4.525
    360.0 0.563
    1.482 0.908
    360.0 1.665
    4.419 0.913
    4.414 1.299
    360.0 1.537
    1.212 0.288
    360.0 2.076
    4.346 1.152
    360.0 0.905
    14.234 1.524
    360.0 360.0
    360.0 0.441
    2.196 0.734
    360.0 4.531
    360.0 0.515
    56.045 5.018
    360.0 0.763
    1.991 1.018
    360.0 1.139
    15.393 1.705
    360.0 4.664
    4.943 1.4
    360.0 1.82
    2.452 0.713
    19.662 2.554
    360.0 2.041
    2.144 0.928
    360.0 0.643
    55.859 2.378
    360.0 14.951
    0.744 0.37
    360.0 0.648
    360.0 1.927
    289.087 2.535
    360.0 1.173
    360.0 2.657
    360.0 0.341
    360.0 2.284
    360.0 2.248
    360.0 10.288
    360.0 360.0
    360.0 0.591
    360.0 360.0
    360.0 14.485
    360.0 0.538
    360.0 3.175
    360.0 0.614
    360.0 1.96
    360.0 4.756
    360.0 0.506
    360.0 4.937
    360.0 0.817
    360.0 4.234
    360.0 4.474
    360.0 4.26
    360.0 1.08
    360.0 19.922
    360.0 3.34
    14.269 3.677
    360.0 3.68
    1.528 0.818
    360.0 0.361
    360.0 1.041
    0.372 0.365
    360.0 3.161
    360.0 3.369
    1.174 0.618
    360.0 4.615
    22.771 2.927
    360.0 1.615
    360.0 0.655
    360.0 4.81
    360.0 360.0
    360.0 7.175
    360.0 0.444
    2.642 0.932
    4.86 0.95
    360.0 3.604
    360.0 3.484
    3.954 1.087
    360.0 52.247
    360.0 2.272
    360.0 3.935
    13.687 360.0
    360.0 0.635
    360.0 0.405
    2.742 1.488
    15.211 3.155
    360.0 5.719
    57.074 2.66
    360.0 1.074
    360.0 38.115
    360.0 7.407
    360.0 3.209
    360.0 1.03
    360.0 1.103
    360.0 11.341
    360.0 3.967
    360.0 1.288
    360.0 15.765
    360.0 0.339
    360.0 0.603
    360.0 6.399
    360.0 0.514
    360.0 0.785
    360.0 0.903
    360.0 0.688
    360.0 4.261
    360.0 360.0
    360.0 0.352
    360.0 43.239
    360.0 3.989
    360.0 1.005
    360.0 71.567
    360.0 0.485
    360.0 0.532
    69.245 2.191
    3.962 1.023
    360.0 7.039
    360.0 2.11
    360.0 4.477
    360.0 6.736
    0.205 0.389
    360.0 1.752
    360.0 360.0
    113.013 1.641
    360.0 10.861
    360.0 0.823
    360.0 12.829
    360.0 2.727
    360.0 0.838
    360.0 3.04
    360.0 4.049
    360.0 6.344
    76.172 24.068
    145.569 4.165
    0.387 0.428
    360.0 4.366
    0.802 0.703
    45.892 1.841
    360.0 2.785
    360.0 4.567
    360.0 3.926
    9.366 2.321
    1.272 0.623
    2.471 0.851
    5.532 1.433
    55.052 2.974
    360.0 5.657
    4.329 1.172
    0.335 0.286
    360.0 5.046
    1.842 0.516
    360.0 360.0
    1.073 0.616
    1.343 1.111
    360.0 1.21
    360.0 4.155
    6.398 2.579
    1.246 0.847
    4.383 0.825
    14.196 1.763
    1.265 0.587
    1.737 1.148
    360.0 12.908
    9.471 2.182
    360.0 0.573
    360.0 1.408
    2.681 0.846
    360.0 360.0
    360.0 0.91
    3.589 1.054
    16.618 3.958
    1.407 0.642
    0.277 0.337
    0.403 0.351
    0.284 0.382
    360.0 1.59
    0.365 0.315
    360.0 0.662
    360.0 4.393
    360.0 2.823
    8.899 2.256
    13.856 1.73
    360.0 7.291
    360.0 360.0
    0.377 0.371
    360.0 6.392
    360.0 5.494
    360.0 4.978
    1.426 0.69
    360.0 0.751
    0.385 0.365
    360.0 0.58
    360.0 6.596
    0.358 0.364
    360.0 4.799
    360.0 0.91
    360.0 0.619
    2.962 4.153
    0.483 0.321
    360.0 2.106
    360.0 7.368
    0.675 0.452
    360.0 8.582
    360.0 2.555
    6.182 2.451
    7.845 2.014
    0.795 0.588
    360.0 3.64
    360.0 3.584
    360.0 0.75
    25.529 3.178
    360.0 5.24
    56.627 2.477
    360.0 0.883
    3.143 1.059
    360.0 3.355
    360.0 5.144
    360.0 0.396
    360.0 1.387
    360.0 1.49
    360.0 0.793
    360.0 0.766
    360.0 0.471
    360.0 94.405
    360.0 6.182
    4.008 1.296
    360.0 46.864
    360.0 3.994
    0.381 0.385
    2.703 0.915
    1.74 0.883
    20.373 1.582
    0.986 0.587
    0.642 0.735
    13.126 2.592
    4.034 1.432
    1.059 6.408
    23.495 14.471
    360.0 4.56
    2.941 1.041
    0.499 0.475
    0.266 0.391
    5.82 1.467
    360.0 5.689
    360.0 360.0
    4.18 1.04
    360.0 2.26
    360.0 0.372
    0.464 0.228
    360.0 0.478
    109.031 2.534
    360.0 360.0
    360.0 0.452
    1.262 0.622
    1.408 0.568
    1.494 0.709
    2.442 0.756
    360.0 3.753
    0.903 0.682
    360.0 0.382
    1.091 0.742
    360.0 4.172
    0.644 0.552
    360.0 9.753
    360.0 5.59
    0.348 0.274
    360.0 16.066
    1.837 0.904
    0.76 0.513
    360.0 360.0
    1.369 0.637
    360.0 0.35
    360.0 0.802
    360.0 1.253
    360.0 360.0
    360.0 5.604
    0.359 0.355
    360.0 0.972
    7.789 2.004
    360.0 1.182
    360.0 4.078
    64.909 3.147
    360.0 0.954
    360.0 3.703
    360.0 3.208
    360.0 6.088
    76.353 7.701
    360.0 2.636
    360.0 1.261
    1.531 0.86
    360.0 39.105
    360.0 2.124
    18.183 3.071
    360.0 0.187
    0.815 0.522
    360.0 0.84
    360.0 0.718
    360.0 0.891
    360.0 13.924
    360.0 0.968
    360.0 17.147
    360.0 2.742
    360.0 1.018
    360.0 1.875
    360.0 3.251
    360.0 2.095
    360.0 1.107
    360.0 0.959
    360.0 4.358
    360.0 4.568
    360.0 0.79
    360.0 11.576
    360.0 0.828
    360.0 360.0
    360.0 1.352
    1.806 0.799
    360.0 0.608
    360.0 13.041
    360.0 0.664
    360.0 2.645
    360.0 5.595
    360.0 3.175
    360.0 1.091
    17.382 2.21
    360.0 3.533
    3.585 1.186
    59.379 2.677
    360.0 3.766
    360.0 360.0
    12.912 2.567
    360.0 7.479
    360.0 1.196
    360.0 360.0
    360.0 1.0
    360.0 17.925
    360.0 2.191
    106.306 3.439
    360.0 0.84
    0.948 0.658
    1.44 0.403
    360.0 1.084
    360.0 98.405
    1.069 0.561
    360.0 3.582
    360.0 360.0
    0.903 0.605
    0.352 0.375
    1.341 0.711
    360.0 360.0
    360.0 221.57
    360.0 360.0
    360.0 130.571
    360.0 4.666
    1.42 0.661
    1.357 0.762
    13.679 2.202
    84.341 5.66
    9.78 3.028
    1.238 0.759
    360.0 360.0
    54.479 2.732
    360.0 4.409
    0.384 0.25
    360.0 4.825
    1.025 0.63
    360.0 3.21
    4.0 1.037
    14.269 1.451
    360.0 360.0
    360.0 6.846
    360.0 1.961
    114.175 2.908
    360.0 0.389
    1.236 0.749
    360.0 3.913
    360.0 360.0
    360.0 0.568
    15.341 11.597
    360.0 360.0
    360.0 195.222
    1.372 0.823
    360.0 1.865
    360.0 0.537
    360.0 4.953
    360.0 45.855
    360.0 0.74
    360.0 0.815
    360.0 0.636
    360.0 1.602
    360.0 360.0
    360.0 0.37
    360.0 5.714
    360.0 0.547
    360.0 5.994
    360.0 2.234
    360.0 0.678
    360.0 1.874
    360.0 0.527
    360.0 48.234
    360.0 11.817
    360.0 5.646
    360.0 15.269
    360.0 2.479
    19.892 5.24
    360.0 4.052
    360.0 4.016
    1.215 0.61
    360.0 0.636
    360.0 1.76
    117.814 3.451
    360.0 1.712
    360.0 4.269
    360.0 1.08
    360.0 2.536
    360.0 0.373
    360.0 3.784
    360.0 0.794
    360.0 360.0
    360.0 7.676
    360.0 0.81
    360.0 360.0
    360.0 0.666
    0.274 0.329
    360.0 5.648
    360.0 1.081
    360.0 2.791
    360.0 3.113
    1.098 0.513
    360.0 360.0
    0.358 0.39
    360.0 2.516
    25.209 1.871
    0.848 0.697
    360.0 0.879
    0.315 0.179
    360.0 0.78
    360.0 5.072
    32.166 4.7
    360.0 3.966
    360.0 16.325
    360.0 1.511
    7.002 0.907
    82.24 3.48
    360.0 5.552
    360.0 183.502
    360.0 4.656
    360.0 3.081
    360.0 1.049
    360.0 360.0
    360.0 0.843
    14.118 1.891
    360.0 3.045
    0.589 0.643
    360.0 26.33
    360.0 7.495
    360.0 6.283
    360.0 338.298
    360.0 2.939
    1.274 0.881
    360.0 4.221
    360.0 3.538
    0.301 0.364
    360.0 4.942
    360.0 5.121
    0.842 0.457
    1.537 0.838
    360.0 3.403
    360.0 10.429
    360.0 1.289
    360.0 4.484
    360.0 0.625
    11.17 11.575
    360.0 4.643
    1.383 0.921
    360.0 0.495
    1.167 0.642
    1.058 0.68
    3.865 1.064
    360.0 6.37
    360.0 4.49
    83.899 3.36
    360.0 2.925
    360.0 4.214
    360.0 0.319
    360.0 24.635
    360.0 360.0
    360.0 3.96
    360.0 2.753
    360.0 2.882
    360.0 0.496
    360.0 27.521
    360.0 0.693
    360.0 0.502
    360.0 0.305
    360.0 1.967
    360.0 89.69
    360.0 0.92
    360.0 360.0
    360.0 0.747
    360.0 2.758
    360.0 1.546
    360.0 0.753
    360.0 0.673
    360.0 0.655
    360.0 360.0
    360.0 1.035
    360.0 0.662
    360.0 5.994
    360.0 0.736
    360.0 0.378
    360.0 3.0
    360.0 0.41
    360.0 0.629
    360.0 0.664
    360.0 0.605
    360.0 2.614
    360.0 0.786
    360.0 5.44
    360.0 0.663
    360.0 0.816
    360.0 121.455
    360.0 3.218
    360.0 0.743
    360.0 3.028
    360.0 0.362
    360.0 360.0
    360.0 0.686
    360.0 3.716
    360.0 4.41
    360.0 0.837
    360.0 1.22
    360.0 11.671
    360.0 360.0
    360.0 4.679
    360.0 0.872
    360.0 2.65
    360.0 0.729
    360.0 1.075
    360.0 0.916
    360.0 6.597
    360.0 8.449
    360.0 0.66
    360.0 2.521
    360.0 1.319
    360.0 4.48
    360.0 0.68
    360.0 0.712
    360.0 2.151
    360.0 0.582
    360.0 0.191
    360.0 1.0
    360.0 0.318
    360.0 0.781
    360.0 0.996
    360.0 3.538
    360.0 4.9
    360.0 0.854
    360.0 2.411
    360.0 29.524
    360.0 0.635
    360.0 0.676
    360.0 0.489
    360.0 3.95
    360.0 0.549
    360.0 0.776
    360.0 3.205
    360.0 9.523
    360.0 5.168
    360.0 3.865
    360.0 3.363
    360.0 3.557
    360.0 0.874
    360.0 36.923
    360.0 1.368
    360.0 360.0
    360.0 1.249
    360.0 6.372
    360.0 0.293
    360.0 0.798
    360.0 0.694
    360.0 0.979
    360.0 1.769
    360.0 1.326
    360.0 0.605
    360.0 0.977
    360.0 86.391
    360.0 2.551
    360.0 1.266
    360.0 0.901
    360.0 0.653
    360.0 0.664
    360.0 20.89
    360.0 1.659
    360.0 2.71
    360.0 28.469
    360.0 5.181
    360.0 0.554
    360.0 360.0
    360.0 7.141
    360.0 360.0
    360.0 91.863
    360.0 5.318
    360.0 2.503
    360.0 0.399
    360.0 0.537
    360.0 360.0
    360.0 2.868
    360.0 0.794
    360.0 4.158
    360.0 3.584
    360.0 2.143
    360.0 5.553
    360.0 2.229
    360.0 3.817
    360.0 3.501
    360.0 6.481
    360.0 6.612
    360.0 2.324
    360.0 4.038
    360.0 3.05
    360.0 0.356
    360.0 0.726
    360.0 0.585
    360.0 1.314
    360.0 0.858
    360.0 360.0
    360.0 0.933
    360.0 0.44
    360.0 0.905
    360.0 2.781
    360.0 0.906
    360.0 2.402
    360.0 100.519
    360.0 22.264
    360.0 0.779
    360.0 0.908
    360.0 21.772
    360.0 3.117
    360.0 45.867
    360.0 1.64
    360.0 0.648
    360.0 28.619
    360.0 0.787
    360.0 6.515
    360.0 4.129
    360.0 40.172
    360.0 1.541
    360.0 11.299
    360.0 253.038
    360.0 3.817
    360.0 4.483
    360.0 0.937
    360.0 9.385
    360.0 0.762
    360.0 1.976
    360.0 1.32
    360.0 0.783
    360.0 360.0
    360.0 3.227
    360.0 40.83
    360.0 360.0
    360.0 5.062
    360.0 2.609
    360.0 1.049
    360.0 360.0
    360.0 35.073
    360.0 360.0
    360.0 21.714
    360.0 0.993
    360.0 2.492
    360.0 360.0
    360.0 4.078
    360.0 360.0
    360.0 4.7
    360.0 3.758
    360.0 3.55
    360.0 3.771
    360.0 1.0
    360.0 4.008
    360.0 1.639
    360.0 11.52
    360.0 3.115
    360.0 0.946
    360.0 2.54
    360.0 0.664
    360.0 0.596
    360.0 3.488
    360.0 6.548
    360.0 1.632
    360.0 1.758
    360.0 2.731
    360.0 2.595
    360.0 5.06
    360.0 1.859
    360.0 0.6
    360.0 1.065
    360.0 0.842
    360.0 0.803
    360.0 2.041
    360.0 360.0
    360.0 1.31
    360.0 2.691
    360.0 2.026
    360.0 4.926
    360.0 0.862
    360.0 0.875
    360.0 2.655
    360.0 1.666
    360.0 3.657
    360.0 0.655
    360.0 0.752
    360.0 0.669
    360.0 360.0
    360.0 3.319
    360.0 2.544
    360.0 16.649
    360.0 0.63
    360.0 0.746
    360.0 27.48
    360.0 0.714
    360.0 360.0
    360.0 3.718
    360.0 5.956
    360.0 2.762
    360.0 0.354
    360.0 5.133
    360.0 5.163
    360.0 1.601
    360.0 0.368
    360.0 360.0
    360.0 2.997
    360.0 2.632
    360.0 8.737
    360.0 0.73
    360.0 0.516
    360.0 0.49
    360.0 0.774
    360.0 4.873
    360.0 0.644
    360.0 0.74
    360.0 0.954
    360.0 3.274
    360.0 360.0
    360.0 0.669
    360.0 2.334
    360.0 1.043
    360.0 0.683
    360.0 2.801
    360.0 0.488
    360.0 4.576
    360.0 5.424
    360.0 5.544
    360.0 4.047
    360.0 0.631
    360.0 0.742
    360.0 0.706
    360.0 2.355
    360.0 1.829
    360.0 2.832
    360.0 2.985
    360.0 100.588
    360.0 6.885
    360.0 2.078
    360.0 2.941
    360.0 0.829
    360.0 0.644
    360.0 0.376
    360.0 2.341
    360.0 0.646
    360.0 4.543
    360.0 4.209
    360.0 2.561
    360.0 1.07
    360.0 360.0
    360.0 0.825
    360.0 5.796
    360.0 360.0
    360.0 2.47
    360.0 17.565
    360.0 0.411
    360.0 0.93
    360.0 4.199
    360.0 4.912
    360.0 73.724
    360.0 4.551
    360.0 1.154
    360.0 1.93
    360.0 0.961
    360.0 0.516
    360.0 3.187
    360.0 3.118
    360.0 5.276
    360.0 27.574
    360.0 3.344
    360.0 0.516
    360.0 0.695
    360.0 1.637
    360.0 360.0
    360.0 0.496
    360.0 3.466
    360.0 4.622
    360.0 0.785
    360.0 4.678
    360.0 6.803
    360.0 2.537
    360.0 360.0
    360.0 3.118
    360.0 4.093
    360.0 0.371
    360.0 2.854
    360.0 5.356
    360.0 53.943
    360.0 1.11
    360.0 20.176
    360.0 3.538
    360.0 5.165
    360.0 360.0
    360.0 0.522
    360.0 40.288
    360.0 0.654
    360.0 2.694
    360.0 0.977
    360.0 360.0
    360.0 1.29
    360.0 0.768
    360.0 0.607
    360.0 0.792
    360.0 2.937
    360.0 0.407
    360.0 1.565
    360.0 16.017
    360.0 4.143
    360.0 4.48
    360.0 0.468
    360.0 4.227
    360.0 3.189
    360.0 0.698
    360.0 0.521
    360.0 0.606
    360.0 0.695
    360.0 5.253
    360.0 12.755
    360.0 0.745
    360.0 19.278
    360.0 0.92
    360.0 6.808
    360.0 1.135
    360.0 4.262
    360.0 1.569
    360.0 1.335
    360.0 2.027
    360.0 1.022
    360.0 2.99
    360.0 0.998
    360.0 360.0
    360.0 1.295
    360.0 2.877
    360.0 1.089
    360.0 0.896
    360.0 26.052
    360.0 2.415
    360.0 0.641
    360.0 0.711
    360.0 3.652
    360.0 1.457
    360.0 0.592
    360.0 0.714
    360.0 1.306
    360.0 3.593
    360.0 2.738
    360.0 0.513
    360.0 1.061
    360.0 0.852
    360.0 3.219
    360.0 1.671
    360.0 0.293
    360.0 4.048
    360.0 51.711
    360.0 5.347
    360.0 3.954
    360.0 0.6
    360.0 0.884
    360.0 0.737
    360.0 0.665
    360.0 0.685
    360.0 5.806
    360.0 0.435
    360.0 1.008
    360.0 1.186
    360.0 360.0
    360.0 360.0
    360.0 5.992
    360.0 0.587
    360.0 3.587
    360.0 1.534
    360.0 360.0
    360.0 1.541
    360.0 1.785
    360.0 0.492
    360.0 79.281
    360.0 2.075
    360.0 0.736
    360.0 70.401
    360.0 3.125
    360.0 4.554
    360.0 2.482
    360.0 0.723
    360.0 2.755
    360.0 2.426
    360.0 0.452
    360.0 0.869
    360.0 151.476
    360.0 1.663
    360.0 3.493
    360.0 0.616
    360.0 56.935
    360.0 0.935
    360.0 16.103
    360.0 0.578
    360.0 2.955
    360.0 29.157
    360.0 0.51
    360.0 0.749
    360.0 2.398
    360.0 3.747
    360.0 3.651
    360.0 3.738
    360.0 360.0
    360.0 3.737
    360.0 360.0
    360.0 360.0
    360.0 360.0
    360.0 360.0
    360.0 360.0
    360.0 0.688
    360.0 0.499
    360.0 0.717
    360.0 360.0
    360.0 3.53
    360.0 0.898
    360.0 0.392
    360.0 3.694
    360.0 0.648
    360.0 5.35
    360.0 9.195
    360.0 6.101
    360.0 0.724
    360.0 0.764
    360.0 0.348
    360.0 2.635
    360.0 1.186
    360.0 2.553
    360.0 3.858
    360.0 0.163
    360.0 360.0
    360.0 2.286
    360.0 0.647
    360.0 0.602
    360.0 19.095
    360.0 0.98
    360.0 360.0
    360.0 44.812
    360.0 0.48
    360.0 2.492
    360.0 360.0
    360.0 5.181
    360.0 3.997
    360.0 52.15
    360.0 4.039
    360.0 0.463
    360.0 116.2
    360.0 2.425
    360.0 36.377
    360.0 6.096
    360.0 4.726
    360.0 0.848
    360.0 1.845
    360.0 4.606
    360.0 1.198
    360.0 41.515
    360.0 2.011
    360.0 0.872
    360.0 1.317
    360.0 3.479
    360.0 0.77
    360.0 0.927
    360.0 0.357
    360.0 0.578
    360.0 0.508
    360.0 0.607
    360.0 3.695
    360.0 0.698
    360.0 5.208
    360.0 360.0
    360.0 2.51
    360.0 3.719
    360.0 3.678
    360.0 4.651
    360.0 0.536
    360.0 1.105
    360.0 2.379
    360.0 0.971
    360.0 55.525
    360.0 157.606
    360.0 0.972
    360.0 0.359
    360.0 1.472
    360.0 3.274
    360.0 360.0
    360.0 1.951
    360.0 0.298
    360.0 18.266
    360.0 0.411
    360.0 0.835
    360.0 5.125
    360.0 3.794
    360.0 1.935
    360.0 6.396
    360.0 360.0
    360.0 5.155
    360.0 6.944
    360.0 0.612
    360.0 0.83
    360.0 0.932
    360.0 7.622
    360.0 0.795
    360.0 1.915
    360.0 0.293
    360.0 0.525
    360.0 3.382
    360.0 2.01
    360.0 360.0
    360.0 2.843
    360.0 0.749
    360.0 1.345
    360.0 0.657
    360.0 3.26
    360.0 6.271
    360.0 3.372
    360.0 1.213
    360.0 1.157
    360.0 3.825
    360.0 0.659
    360.0 0.361
    360.0 2.909
    360.0 2.905
    360.0 0.292
    360.0 3.185
    360.0 14.035
    360.0 3.997
    360.0 5.367
    360.0 3.823
    360.0 3.164
    360.0 360.0
    360.0 3.179
    360.0 360.0
    360.0 0.476
    360.0 0.394
    360.0 0.83
    360.0 6.034
    360.0 360.0
    360.0 4.48
    360.0 2.685
    360.0 1.005
    360.0 14.661
    360.0 0.336
    360.0 3.879
    360.0 1.308
    360.0 73.583
    360.0 1.205
    360.0 1.058
    360.0 0.518
    360.0 3.289
    360.0 2.499
    360.0 0.755
    360.0 360.0
    360.0 8.991
    360.0 1.62
    360.0 1.566
    360.0 1.51
    360.0 0.681
    360.0 0.897
    360.0 2.861
    360.0 3.743
    360.0 28.021
    360.0 0.771
    360.0 3.659
    360.0 0.866
    360.0 0.797
    360.0 0.844
    360.0 1.192
    360.0 1.16
    360.0 0.599
    360.0 65.429
    360.0 5.176
    360.0 0.374
    360.0 3.819
    360.0 13.18
    360.0 5.32
    360.0 0.497
    360.0 360.0
    360.0 360.0
    360.0 0.92
    360.0 0.71
    360.0 0.357
    360.0 360.0
    360.0 360.0
    360.0 5.647
    360.0 1.056
    360.0 1.623
    360.0 8.148
    360.0 4.686
    360.0 4.656
    360.0 0.91
    360.0 13.829
    360.0 1.05
    360.0 8.945
    360.0 1.957
    360.0 1.008
    360.0 0.615
    360.0 0.722
    360.0 2.786
    360.0 0.387
    360.0 0.604
    360.0 1.169
    360.0 2.546
    360.0 1.816
    360.0 4.632
    360.0 1.026
    360.0 0.942
    360.0 0.68
    360.0 2.088
    360.0 0.49
    360.0 0.689
    360.0 7.382
    360.0 179.516
    360.0 0.851
    360.0 0.73
    360.0 0.588
    360.0 0.386
    360.0 5.904
    360.0 0.856
    360.0 1.546
    360.0 4.542
    360.0 0.607
    360.0 0.78
    360.0 360.0
    360.0 360.0
    360.0 1.969
    360.0 8.739
    360.0 0.653
    360.0 0.605
    360.0 3.144
    360.0 0.432
    360.0 76.371
    360.0 47.28
    360.0 0.833
    360.0 0.822
    360.0 0.902
    360.0 0.493
    360.0 5.211
    360.0 0.651
    360.0 3.311
    360.0 0.543
    360.0 0.824
    360.0 0.386
    360.0 4.206
    360.0 8.331
    360.0 0.67
    360.0 1.204
    360.0 0.455
    360.0 3.723
    360.0 0.673
    360.0 360.0
    360.0 169.502
    360.0 42.358
    360.0 1.558
    360.0 360.0
    360.0 0.545
    360.0 360.0
    360.0 40.571
    360.0 0.239
    360.0 1.084
    360.0 3.064
    360.0 5.889
    360.0 43.533
    360.0 0.689
    360.0 3.885
    360.0 0.728
    360.0 3.017
    360.0 5.814
    360.0 4.466
    360.0 360.0
    360.0 5.325
    360.0 0.646
    360.0 3.089
    360.0 1.19
    360.0 0.775
    360.0 10.311
    360.0 3.124
    360.0 0.358
    360.0 0.891
    360.0 41.029
    360.0 360.0
    360.0 0.416
    360.0 12.923
    360.0 2.7
    360.0 1.398
    360.0 0.411
    360.0 1.867
    360.0 0.798
    360.0 10.7
    360.0 0.988
    360.0 0.729
    360.0 1.933
    360.0 0.531
    360.0 360.0
    360.0 4.648
    360.0 0.808
    360.0 3.408
    360.0 7.13
    360.0 0.638
    360.0 0.64
    360.0 360.0
    360.0 1.264
    360.0 8.184
    360.0 2.189
    360.0 4.458
    360.0 0.651
    360.0 2.939
    360.0 3.364
    360.0 1.485
    360.0 0.684
    360.0 4.082
    360.0 0.796
    360.0 360.0
    360.0 360.0
    360.0 1.895
    360.0 0.829
    360.0 0.982
    360.0 0.814
    360.0 4.441
    360.0 5.59
    360.0 0.662
    360.0 0.782
    360.0 6.163
    360.0 6.146
    360.0 0.395
    360.0 6.531
    360.0 360.0
    360.0 3.76
    360.0 0.845
    360.0 0.637
    360.0 3.606
    360.0 2.825
    360.0 1.041
    360.0 0.643
    360.0 0.69
    360.0 360.0
    360.0 0.645
    360.0 0.499
    360.0 0.284
    360.0 3.124
    360.0 1.0
    360.0 7.751
    360.0 1.668
    360.0 0.882
    360.0 2.658
    360.0 4.421
    360.0 1.039
    360.0 0.705
    360.0 360.0
    360.0 4.164
    360.0 3.257
    360.0 0.602
    360.0 3.572
    360.0 0.382
    360.0 4.968
    360.0 360.0
    360.0 0.978
    360.0 0.781
    360.0 2.462
    360.0 0.802
    360.0 0.678
    360.0 0.616
    360.0 0.494
    360.0 2.469
    360.0 0.857
    360.0 0.648
    360.0 0.918
    360.0 5.87
    360.0 5.286
    360.0 0.554
    360.0 0.69
    360.0 3.425
    360.0 0.785
    360.0 360.0
    360.0 0.904
    360.0 2.532
    360.0 4.52
    360.0 3.723
    360.0 2.334
    360.0 3.249
    360.0 0.873
    360.0 0.721
    360.0 0.732
    360.0 360.0
    360.0 360.0
    360.0 1.681
    360.0 0.488
    360.0 7.2
    360.0 5.433
    360.0 0.743
    360.0 5.055
    360.0 67.107
    360.0 0.705
    360.0 128.413
    360.0 0.439
    360.0 0.734
    360.0 27.115
    360.0 8.468
    360.0 2.005
    360.0 3.382
    360.0 0.671
    360.0 2.624
    360.0 3.772
    360.0 7.916
    360.0 1.262
    360.0 4.291
    360.0 0.811
    360.0 5.179
    360.0 0.681
    360.0 4.948
    360.0 0.89
    360.0 0.378
    360.0 12.022
    360.0 0.409
    360.0 3.293
    360.0 4.414
    360.0 2.86
    360.0 8.088
    360.0 4.002
    360.0 0.771
    360.0 0.887
    360.0 6.941
    360.0 3.539
    360.0 3.566
    360.0 0.665
    360.0 1.4
    360.0 4.002
    360.0 3.985
    360.0 2.81
    360.0 0.385
    360.0 3.928
    360.0 0.852
    360.0 2.917
    360.0 0.748
    360.0 1.056
    360.0 22.96
    360.0 3.009
    360.0 2.903
    360.0 0.913
    360.0 0.646
    360.0 0.885
    360.0 4.686
    360.0 1.805
    360.0 3.771
    360.0 1.461
    360.0 4.266
    360.0 9.152
    360.0 12.533
    360.0 2.778
    360.0 1.76
    360.0 0.504
    360.0 0.598
    360.0 2.838
    360.0 0.859
    360.0 1.456
    360.0 0.364
    360.0 0.552
    360.0 0.749
    360.0 4.875
    360.0 0.656
    360.0 0.697
    360.0 8.019
    360.0 0.824
    360.0 2.775
    360.0 0.873
    360.0 0.336
    360.0 2.685
    360.0 0.9
    360.0 0.753
    360.0 360.0
    360.0 3.292
    360.0 2.107
    360.0 1.445
    360.0 360.0
    360.0 3.703
    360.0 0.682
    360.0 0.857
    360.0 1.708
    360.0 0.327
    360.0 0.362
    360.0 3.963
    360.0 2.388
    360.0 0.873
    360.0 2.678
    360.0 0.408
    360.0 4.172
    360.0 3.099
    360.0 4.13
    360.0 26.911
    360.0 360.0
    360.0 4.689
    360.0 0.668
    360.0 0.766
    360.0 0.765
    360.0 3.463
    360.0 1.01
    360.0 360.0
    360.0 0.646
    360.0 0.37
    360.0 0.386
    360.0 6.912
    360.0 19.055
    360.0 7.06
    360.0 3.599
    360.0 18.811
    360.0 38.559
    360.0 1.705
    360.0 10.197
    360.0 0.944
    360.0 4.724
    360.0 0.897
    360.0 1.728
    360.0 3.167
    360.0 0.378
    360.0 11.824
    360.0 7.264
    360.0 2.88
    360.0 256.16
    360.0 1.814
    360.0 0.758
    360.0 11.189
    360.0 2.719
    360.0 0.685
    360.0 7.673
    360.0 360.0
    360.0 3.136
    360.0 6.561
    360.0 360.0
    360.0 0.918
    360.0 7.195
    360.0 3.138
    360.0 58.0
    360.0 0.501
    360.0 0.287
    360.0 4.465
    360.0 3.643
    360.0 0.805
    360.0 4.587
    360.0 3.783
    360.0 360.0
    360.0 1.46
    360.0 0.564
    360.0 0.486
    360.0 0.633
    360.0 1.379
    360.0 1.043
    360.0 0.738
    360.0 0.864
    360.0 1.054
    360.0 360.0
    360.0 0.397
    360.0 0.877
    360.0 1.208
    360.0 0.789
    360.0 0.665
    360.0 1.182
    360.0 0.391
    360.0 0.679
    360.0 360.0
    360.0 360.0
    360.0 5.32
    360.0 1.767
    360.0 2.334
    360.0 2.214
    360.0 116.452
    360.0 2.885
    360.0 0.705
    360.0 23.522
    360.0 0.956
    360.0 4.008
    360.0 1.743
    360.0 4.486
    360.0 0.721
    360.0 0.269
    360.0 9.975
    360.0 360.0
    360.0 0.661
    360.0 2.569
    360.0 1.861
    360.0 360.0
    360.0 2.426
    360.0 4.68
    360.0 1.685
    360.0 360.0
    360.0 0.84
    360.0 1.394
    360.0 4.749
    360.0 0.717
    360.0 360.0
    360.0 5.413
    360.0 37.35
    360.0 1.356
    360.0 1.071
    360.0 0.217
    360.0 1.207
    360.0 0.461
    360.0 15.873
    360.0 360.0
    360.0 3.017
    360.0 0.525
    360.0 360.0
    360.0 0.757
    360.0 360.0
    360.0 7.765
    360.0 3.204
    360.0 0.895
    360.0 1.891
    360.0 360.0
    360.0 0.491
    360.0 6.334
    360.0 1.191
    360.0 1.041
    360.0 0.815
    360.0 360.0
    360.0 0.417
    360.0 1.298
    360.0 1.596
    360.0 2.489
    360.0 3.052
    360.0 0.457
    360.0 0.519
    360.0 0.693
    360.0 14.331
    360.0 85.826
    360.0 1.492
    360.0 4.03
    360.0 31.872
    360.0 4.91
    360.0 2.604
    360.0 2.829
    360.0 360.0
    360.0 4.875
    360.0 1.053
    360.0 1.788
    360.0 6.491
    360.0 0.768
    360.0 2.089
    360.0 0.257
    360.0 11.815
    360.0 0.38
    360.0 4.061
    360.0 10.599
    360.0 0.67
    360.0 360.0
    360.0 2.105
    360.0 0.269
    360.0 291.968
    360.0 10.755
    360.0 1.269
    360.0 1.048
    360.0 1.094
    360.0 360.0
    360.0 0.778
    360.0 360.0
    360.0 0.272
    360.0 360.0
    360.0 3.035
    360.0 2.507
    360.0 0.945
    360.0 1.056
    360.0 0.801
    360.0 0.742
    360.0 2.496
    360.0 3.976
    360.0 0.64
    360.0 3.921
    360.0 4.307
    360.0 1.803
    360.0 2.213
    360.0 360.0
    360.0 3.646
    360.0 1.075
    360.0 2.066
    360.0 2.873
    360.0 0.833
    360.0 1.649
    360.0 1.804
    360.0 1.248
    360.0 3.417
    360.0 0.929
    360.0 2.768
    360.0 0.642
    360.0 1.943
    360.0 4.125
    360.0 2.022
    360.0 5.741
    360.0 0.798
    360.0 3.16
    360.0 1.334
    360.0 48.401
    360.0 4.368
    360.0 2.13
    360.0 0.585
    360.0 0.599
    360.0 360.0
    360.0 0.892
    360.0 4.666
    360.0 0.249
    360.0 0.873
    360.0 0.58
    360.0 3.487
    360.0 28.158
    360.0 0.783
    360.0 15.43
    360.0 6.155
    360.0 360.0
    360.0 0.974
    360.0 1.24
    360.0 0.666
    360.0 0.29
    360.0 360.0
    360.0 4.884
    360.0 1.694
    360.0 0.492
    360.0 51.603
    360.0 0.713
    360.0 6.361
    360.0 360.0
    360.0 360.0
    360.0 0.376
    360.0 0.824
    360.0 37.686
    360.0 0.391
    360.0 10.596
    360.0 0.537
    360.0 5.832
    360.0 1.042
    360.0 2.94
    360.0 0.599
    360.0 0.711
    360.0 7.288
    360.0 0.759
    360.0 0.555
    360.0 0.732
    360.0 0.508
    360.0 0.815
    360.0 1.092
    360.0 1.364
    360.0 0.703
    360.0 5.833
    360.0 360.0
    360.0 50.603
    360.0 0.835
    360.0 2.766
    360.0 1.273
    360.0 360.0
    360.0 0.946
    360.0 2.762
    360.0 0.775
    360.0 2.163
    360.0 0.542
    360.0 360.0
    360.0 2.663
    360.0 2.471
    360.0 8.885
    360.0 0.451
    360.0 360.0
    360.0 0.65
    360.0 0.718
    360.0 4.822
    360.0 0.278
    360.0 2.737
    360.0 3.899
    360.0 10.501
    360.0 0.904
    360.0 2.251
    360.0 360.0
    360.0 3.114
    360.0 1.986
    360.0 0.372
    360.0 6.278
    360.0 48.175
    360.0 0.507
    360.0 0.768
    360.0 1.556
    360.0 0.503
    360.0 2.933
    360.0 4.808
    360.0 360.0
    360.0 2.227
    360.0 3.934
    360.0 0.903
    360.0 0.543
    360.0 1.789
    360.0 5.038
    360.0 182.725
    360.0 3.846
    360.0 2.165
    360.0 0.309
    360.0 1.11
    360.0 1.538
    360.0 1.657
    360.0 4.664
    360.0 0.887
    360.0 2.301
    360.0 0.693
    360.0 17.479
    360.0 1.18
    360.0 0.641
    360.0 3.982
    360.0 0.859
    360.0 0.52
    360.0 3.174
    360.0 360.0
    360.0 1.543
    360.0 0.556
    360.0 2.694
    360.0 0.971
    360.0 4.97
    360.0 0.799
    360.0 5.962
    360.0 360.0
    360.0 0.654
    360.0 0.763
    360.0 1.516
    360.0 3.824
    360.0 360.0
    360.0 1.53
    360.0 0.846
    360.0 2.419
    360.0 0.384
    360.0 360.0
    360.0 3.193
    360.0 0.695
    360.0 0.848
    360.0 7.051
    360.0 2.775
    360.0 0.692
    360.0 4.374
    360.0 0.535
    360.0 4.544
    360.0 1.291
    360.0 1.55
    360.0 0.468
    360.0 0.955
    360.0 0.41
    360.0 8.398
    360.0 0.549
    360.0 1.396
    360.0 0.743
    360.0 3.843
    360.0 2.841
    360.0 1.233
    360.0 0.65
    360.0 360.0
    360.0 0.689
    360.0 0.38
    360.0 0.876
    360.0 0.52
    360.0 0.878
    360.0 36.123
    360.0 4.312
    360.0 0.675
    360.0 3.888
    360.0 0.481
    360.0 0.967
    360.0 0.273
    360.0 11.154
    360.0 0.911
    360.0 360.0
    360.0 0.79
    360.0 4.412
    360.0 360.0
    360.0 2.844
    360.0 0.687
    360.0 0.362
    360.0 0.587
    360.0 360.0
    360.0 3.422
    360.0 0.886
    360.0 0.639
    360.0 5.17
    360.0 0.879
    360.0 7.09
    360.0 1.779
    360.0 0.762
    360.0 2.039
    360.0 0.408
    360.0 2.784
    360.0 0.806
    360.0 1.034
    360.0 4.505
    360.0 2.683
    360.0 25.998
    360.0 0.764
    360.0 0.854
    360.0 0.756
    360.0 5.227
    360.0 5.891
    360.0 6.752
    360.0 360.0
    360.0 13.232
    360.0 0.814
    360.0 0.404
    360.0 1.786
    360.0 2.217
    360.0 3.445
    360.0 0.244
    360.0 0.567
    360.0 0.53
    360.0 1.486
    360.0 0.537
    360.0 1.316
    360.0 0.378
    360.0 9.066
    360.0 0.522
    360.0 3.863
    360.0 1.143
    360.0 9.286
    360.0 33.836
    360.0 3.541
    360.0 3.329
    360.0 1.033
    360.0 1.735
    360.0 0.826
    360.0 360.0
    360.0 0.26
    360.0 26.97
    360.0 5.704
    360.0 0.403
    360.0 0.686
    360.0 58.83
    360.0 0.445
    360.0 0.715
    360.0 0.38
    360.0 0.412
    360.0 1.812
    360.0 3.112
    360.0 0.471
    360.0 3.237
    360.0 0.74
    360.0 0.845
    360.0 0.706
    360.0 2.687
    360.0 2.17
    360.0 2.174
    360.0 4.993
    360.0 32.161
    360.0 0.295
    360.0 0.716
    360.0 0.468
    360.0 360.0
    360.0 2.746
    360.0 0.368
    360.0 3.291
    360.0 4.1
    360.0 0.665
    360.0 4.234
    360.0 3.442
    360.0 0.938
    360.0 0.285
    360.0 0.875
    360.0 0.913
    360.0 360.0
    360.0 0.656
    360.0 0.592
    360.0 3.727
    360.0 0.439
    360.0 0.856
    360.0 5.265
    360.0 3.006
    360.0 126.002
    360.0 1.092
    360.0 3.053
    360.0 3.005
    360.0 2.336
    360.0 360.0
    360.0 4.121
    360.0 9.536
    360.0 0.545
    360.0 3.093
    360.0 0.346
    360.0 7.388
    360.0 27.923
    360.0 0.326
    360.0 3.193
    360.0 2.349
    360.0 6.241
    360.0 0.493
    360.0 0.679
    360.0 0.671
    360.0 0.81
    360.0 0.598
    360.0 4.155
    360.0 1.673
    360.0 5.031
    360.0 0.978
    360.0 360.0
    360.0 172.374
    360.0 5.096
    360.0 1.134
    360.0 360.0
    360.0 2.277
    360.0 30.757
    360.0 3.773
    360.0 1.321
    360.0 360.0
    360.0 2.158
    360.0 3.298
    360.0 2.485
    360.0 0.516
    360.0 360.0
    360.0 4.668
    360.0 0.486
    360.0 0.961
    360.0 4.827
    360.0 360.0
    360.0 1.372
    360.0 0.848
    360.0 1.288
    360.0 3.198
    360.0 4.434
    360.0 0.53
    360.0 0.481
    360.0 1.243
    360.0 0.78
    360.0 0.416
    360.0 6.438
    360.0 360.0
    360.0 4.05
    360.0 1.067
    360.0 360.0
    360.0 13.662
    360.0 0.916
    360.0 0.736
    360.0 3.746
    360.0 5.76
    360.0 6.909
    360.0 0.548
    360.0 0.731
    360.0 0.883
    360.0 0.512
    360.0 24.062
    360.0 360.0
    360.0 310.176
    360.0 0.625
    360.0 0.489
    360.0 5.145
    360.0 4.45
    360.0 360.0
    360.0 2.782
    360.0 3.903
    360.0 0.643
    360.0 0.462
    360.0 1.957
    360.0 0.269
    360.0 0.953
    360.0 0.521
    360.0 1.46
    360.0 3.702
    360.0 2.715
    360.0 360.0
    360.0 0.613
    360.0 0.384
    360.0 2.351
    360.0 4.046
    360.0 0.713
    360.0 0.438
    360.0 0.983
    360.0 44.887
    360.0 0.382
    360.0 2.591
    360.0 1.106
    360.0 0.671
    360.0 0.629
    360.0 0.841
    360.0 0.638
    360.0 0.83
    360.0 2.659
    360.0 2.92
    360.0 3.171
    360.0 0.333
    360.0 1.25
    360.0 1.12
    360.0 1.282
    360.0 3.839
    360.0 2.28
    360.0 0.394
    360.0 6.105
    360.0 130.176
    360.0 8.743
    360.0 0.905
    360.0 3.913
    360.0 0.915
    360.0 7.549
    360.0 2.992
    360.0 1.001
    360.0 1.207
    360.0 0.769
    360.0 0.34
    360.0 2.544
    360.0 0.368
    360.0 0.826
    360.0 0.6
    360.0 3.707
    360.0 0.459
    360.0 2.775
    360.0 5.022
    360.0 3.764
    360.0 0.761
    360.0 360.0
    360.0 0.734
    360.0 0.535
    360.0 5.126
    360.0 0.936
    360.0 4.892
    360.0 2.628
    360.0 0.693
    360.0 0.615
    360.0 0.783
    360.0 0.976
    360.0 360.0
    360.0 360.0
    360.0 0.903
    360.0 2.511
    360.0 0.879
    360.0 9.697
    360.0 1.28
    360.0 0.522
    360.0 4.196
    360.0 6.493
    360.0 0.681
    360.0 0.674
    360.0 4.601
    360.0 360.0
    360.0 5.634
    360.0 360.0
    360.0 0.824
    360.0 3.965
    360.0 1.034
    360.0 4.04
    360.0 2.212
    360.0 360.0
    360.0 2.006
    360.0 1.007
    360.0 4.848
    360.0 0.713
    360.0 2.826
    360.0 10.501
    360.0 3.954
    360.0 0.975
    360.0 46.888
    360.0 0.468
    360.0 8.986
    360.0 0.854
    360.0 0.553
    360.0 2.02
    360.0 360.0
    360.0 0.889
    360.0 2.424
    360.0 0.374
    360.0 5.341
    360.0 0.989
    360.0 0.696
    360.0 3.582
    360.0 0.958
    360.0 3.625
    360.0 0.688
    360.0 2.579
    360.0 0.64
    360.0 0.765
    360.0 3.685
    360.0 2.874
    360.0 0.469
    360.0 2.912
    360.0 2.759
    360.0 2.163
    360.0 1.491
    360.0 4.273
    360.0 1.327
    360.0 4.381
    360.0 5.878
    360.0 0.59
    360.0 5.841
    360.0 6.745
    360.0 0.713
    360.0 0.643
    360.0 0.756
    360.0 2.892
    360.0 1.792
    360.0 2.695
    360.0 2.827
    360.0 3.79
    360.0 0.36
    360.0 360.0
    360.0 1.011
    360.0 0.328
    360.0 0.595
    360.0 0.636
    360.0 0.494
    360.0 3.639
    360.0 0.474
    360.0 0.486
    360.0 4.203
    360.0 6.535
    360.0 4.126
    360.0 0.782
    360.0 3.415
    360.0 3.309
    360.0 5.019
    360.0 7.034
    360.0 5.654
    360.0 4.174
    360.0 6.269
    360.0 0.325
    360.0 0.697
    360.0 0.631
    360.0 2.909
    360.0 360.0
    360.0 0.722
    360.0 1.995
    360.0 0.873
    360.0 4.802
    360.0 360.0
    360.0 0.699
    360.0 360.0
    360.0 5.435
    360.0 360.0
    360.0 0.477
    360.0 154.553
    360.0 5.329
    360.0 12.318
    360.0 0.675
    360.0 2.01
    360.0 360.0
    360.0 2.87
    360.0 0.579
    360.0 0.295
    360.0 74.077
    360.0 0.713
    360.0 4.517
    360.0 1.038
    360.0 1.605
    360.0 0.544
    360.0 0.642
    360.0 6.079
    360.0 1.618
    360.0 0.538
    360.0 8.886
    360.0 360.0
    360.0 0.676
    360.0 10.553
    360.0 4.119
    360.0 0.738
    360.0 153.645
    360.0 2.446
    360.0 0.617
    360.0 360.0
    360.0 0.914
    360.0 2.598
    360.0 360.0
    360.0 7.552
    360.0 6.245
    360.0 0.884
    360.0 0.931
    360.0 360.0
    360.0 0.566
    360.0 0.502
    360.0 0.954
    360.0 0.575
    360.0 0.848
    360.0 0.524
    360.0 0.555
    360.0 7.246
    360.0 0.651
    360.0 360.0
    360.0 0.876
    360.0 36.45
    360.0 2.793
    360.0 1.545
    360.0 0.729
    360.0 360.0
    360.0 6.816
    360.0 41.686
    360.0 4.116
    360.0 0.588
    360.0 9.585
    360.0 3.215
    360.0 3.023
    360.0 0.74
    360.0 0.684
    360.0 0.845
    360.0 0.877
    360.0 0.37
    360.0 1.267
    360.0 2.738
    360.0 27.982
    360.0 0.875
    360.0 0.92
    360.0 0.661
    360.0 28.481
    360.0 360.0
    360.0 0.77
    360.0 2.958
    360.0 2.558
    360.0 0.326
    360.0 0.524
    360.0 360.0
    360.0 47.29
    360.0 2.876
    360.0 5.582
    360.0 2.669
    360.0 3.658
    360.0 203.766
    360.0 0.556
    360.0 7.372
    360.0 0.73
    360.0 22.603
    360.0 360.0
    360.0 3.008
    360.0 0.61
    360.0 0.35
    360.0 360.0
    360.0 360.0
    360.0 0.835
    360.0 0.604
    360.0 42.39
    360.0 360.0
    360.0 360.0
    360.0 2.456
    360.0 360.0
    360.0 360.0
    360.0 0.315
    360.0 3.082
    360.0 4.018
    360.0 0.52
    360.0 0.782
    360.0 3.831
    360.0 5.507
    360.0 0.946
    360.0 0.984
    360.0 4.705
    360.0 1.067
    360.0 0.81
    360.0 8.986
    360.0 8.095
    360.0 2.911
    360.0 0.649
    360.0 4.88
    360.0 0.845
    360.0 6.056
    360.0 0.889
    360.0 0.47
    360.0 19.732
    360.0 0.546
    360.0 360.0
    360.0 1.282
    360.0 3.007
    360.0 360.0
    360.0 1.054
    360.0 0.898
    360.0 3.896
    360.0 0.784
    360.0 2.641
    360.0 0.528
    360.0 4.407
    360.0 0.761
    360.0 7.7
    360.0 3.955
    360.0 1.294
    360.0 0.733
    360.0 41.812
    360.0 360.0
    360.0 4.194
    360.0 360.0
    360.0 5.215
    360.0 3.07
    360.0 0.445
    360.0 360.0
    360.0 6.191
    360.0 4.279
    360.0 0.69
    360.0 3.366
    360.0 0.391
    360.0 0.878
    360.0 0.467
    360.0 0.401
    360.0 0.843
    360.0 1.316
    360.0 0.369
    360.0 2.883
    360.0 3.634
    360.0 0.459
    360.0 10.996
    360.0 0.902
    360.0 4.135
    360.0 4.733
    360.0 360.0
    360.0 360.0
    360.0 3.451
    360.0 0.848
    360.0 0.805
    360.0 0.444
    360.0 3.236
    360.0 0.489
    360.0 19.415
    360.0 0.488
    360.0 360.0
    360.0 0.411
    360.0 1.454
    360.0 4.238
    360.0 0.762
    360.0 0.812
    360.0 4.884
    360.0 4.717
    360.0 1.262
    360.0 0.643
    360.0 0.854
    360.0 0.934
    360.0 1.539
    360.0 3.074
    360.0 1.133
    360.0 1.868
    360.0 0.85
    360.0 0.891
    360.0 360.0
    360.0 3.536
    360.0 3.159
    360.0 3.096
    360.0 4.258
    360.0 0.354
    360.0 0.547
    360.0 2.431
    360.0 0.704
    360.0 0.395
    360.0 2.938
    360.0 360.0
    360.0 2.571
    360.0 1.747
    360.0 3.692
    360.0 1.149
    360.0 12.617
    360.0 1.108
    360.0 0.799
    360.0 0.867
    360.0 0.967
    360.0 4.203
    360.0 0.383
    360.0 1.595
    360.0 0.974
    360.0 1.11
    360.0 360.0
    360.0 0.913
    360.0 4.637
    360.0 1.091
    360.0 2.099
    360.0 1.032
    360.0 0.955
    360.0 2.794
    360.0 2.792
    360.0 2.645
    360.0 1.307
    360.0 23.538
    360.0 0.873
    360.0 0.843
    360.0 0.887
    360.0 6.701
    360.0 8.312
    360.0 0.938
    360.0 2.29
    360.0 3.024
    360.0 5.356
    360.0 0.623
    360.0 360.0
    360.0 0.868
    360.0 0.868
    360.0 0.896
    360.0 1.017
    360.0 3.085
    360.0 0.677
    360.0 3.04
    360.0 3.835
    360.0 1.014
    360.0 360.0
    360.0 4.134
    360.0 3.686
    360.0 0.929
    360.0 0.585
    360.0 11.147
    360.0 6.094
    360.0 1.017
    360.0 3.738
    360.0 0.822
    360.0 1.312
    360.0 24.298
    360.0 1.873
    360.0 4.164
    360.0 14.071
    360.0 25.034
    360.0 360.0
    360.0 2.659
    360.0 3.156
    360.0 0.984
    360.0 2.064
    360.0 1.119
    360.0 360.0
    360.0 2.683
    360.0 6.68
    360.0 11.837
    360.0 6.797
    360.0 1.096
    360.0 1.646
    360.0 0.732
    360.0 0.795
    360.0 3.941
    360.0 1.004
    360.0 0.889
    360.0 0.564
    360.0 0.809
    360.0 2.328
    360.0 2.482
    360.0 43.522
    360.0 1.21
    360.0 0.795
    360.0 0.395
    360.0 1.423
    360.0 10.452
    360.0 3.612
    360.0 2.782
    360.0 360.0
    360.0 1.131
    360.0 0.945
    360.0 1.073
    360.0 3.341
    360.0 2.945
    360.0 360.0
    360.0 0.537
    360.0 0.636
    360.0 2.579
    360.0 2.063
    360.0 3.808
    360.0 3.545
    360.0 360.0
    360.0 360.0
    360.0 360.0
    360.0 4.353
    360.0 2.095
    360.0 3.214
    360.0 1.079
    360.0 0.631
    360.0 360.0
    360.0 0.669
    360.0 1.035
    360.0 0.371
    360.0 0.797
    360.0 0.972
    360.0 3.151
    360.0 1.553
    360.0 1.311
    360.0 1.758
    360.0 0.347
    360.0 3.165
    360.0 1.025
    360.0 0.549
    360.0 5.099
    360.0 1.156
    360.0 0.727
    360.0 0.813
    360.0 0.344
    360.0 360.0
    360.0 1.299
    360.0 8.147
    360.0 0.798
    360.0 0.663
    360.0 0.678
    360.0 3.909
    360.0 3.214
    360.0 360.0
    360.0 2.345
    360.0 0.831
    360.0 1.036
    360.0 1.029
    360.0 0.944
    360.0 3.529
    360.0 0.636
    360.0 2.439
    360.0 4.975
    360.0 1.996
    360.0 0.409
    360.0 0.778
    360.0 1.584
    360.0 2.818
    360.0 0.36
    360.0 360.0
    360.0 10.435
    360.0 0.575
    360.0 2.759
    360.0 4.007
    360.0 360.0
    360.0 0.359
    360.0 3.624
    360.0 3.648
    360.0 1.265
    360.0 62.231
    360.0 5.564
    360.0 2.306
    360.0 360.0
    360.0 15.477
    360.0 0.474
    360.0 3.501
    360.0 360.0
    360.0 29.897
    360.0 1.129
    360.0 4.723
    360.0 0.794
    360.0 0.841
    360.0 42.167
    360.0 2.402
    360.0 0.573
    360.0 0.487
    360.0 1.543
    360.0 0.846
    360.0 0.644
    360.0 1.641
    360.0 0.738
    360.0 1.771
    360.0 0.353
    360.0 3.147
    360.0 2.908
    360.0 2.431
    360.0 6.32
    360.0 0.356
    360.0 1.201
    360.0 0.743
    360.0 12.905
    360.0 4.093
    360.0 0.637
    360.0 0.354
    360.0 2.938
    360.0 0.516
    360.0 0.362
    360.0 11.656
    360.0 1.254
    360.0 0.737
    360.0 3.031
    360.0 4.799
    360.0 0.65
    360.0 1.277
    360.0 6.316
    360.0 2.566
    360.0 197.823
    360.0 0.851
    360.0 0.641
    360.0 1.994
    360.0 0.535
    360.0 1.332
    360.0 0.899
    360.0 2.004
    360.0 2.989
    360.0 4.028
    360.0 3.753
    360.0 1.152
    360.0 1.38
    360.0 3.788
    360.0 360.0
    360.0 0.561
    360.0 0.264
    360.0 0.742
    360.0 0.886
    360.0 0.832
    360.0 1.692
    360.0 0.459
    360.0 2.209
    360.0 3.781
    360.0 360.0
    360.0 2.543
    360.0 0.664
    360.0 0.847
    360.0 3.042
    360.0 2.543
    360.0 14.375
    360.0 0.487
    360.0 360.0
    360.0 0.669
    360.0 2.058
    360.0 16.114
    360.0 4.461
    360.0 2.629
    360.0 0.431
    360.0 4.146
    360.0 2.089
    360.0 360.0
    360.0 1.967
    360.0 3.464
    360.0 0.956
    360.0 2.945
    360.0 0.611
    360.0 0.487
    360.0 15.299
    360.0 0.906
    360.0 3.582
    360.0 6.604
    360.0 0.662
    360.0 0.964
    360.0 0.258
    360.0 61.981
    360.0 24.473
    360.0 4.088
    360.0 3.154
    360.0 0.366
    360.0 5.636
    360.0 0.797
    360.0 1.208
    360.0 1.406
    360.0 0.855
    360.0 360.0
    360.0 3.593
    360.0 4.591
    360.0 1.793
    360.0 0.83
    360.0 360.0
    360.0 4.119
    360.0 1.183
    360.0 7.833
    360.0 3.021
    360.0 1.187
    360.0 17.129
    360.0 4.754
    360.0 360.0
    360.0 0.836
    360.0 0.912
    360.0 0.76
    360.0 5.555
    360.0 360.0
    360.0 1.881
    360.0 0.622
    360.0 0.257
    360.0 1.0
    360.0 0.375
    360.0 3.522
    360.0 360.0
    360.0 0.711
    360.0 13.568
    360.0 0.523
    360.0 3.089
    360.0 1.291
    360.0 360.0
    360.0 2.208
    360.0 0.368
    360.0 3.387
    360.0 11.732
    360.0 360.0
    360.0 0.446
    360.0 4.065
    360.0 0.689
    360.0 5.459
    360.0 4.225
    360.0 3.732
    360.0 38.219
    360.0 2.768
    360.0 3.893
    360.0 6.895
    360.0 1.705
    360.0 0.829
    360.0 2.464
    360.0 3.776
    360.0 0.761
    360.0 0.795
    360.0 3.308
    360.0 0.447
    360.0 1.798
    360.0 0.394
    360.0 0.967
    360.0 3.097
    360.0 0.534
    360.0 360.0
    360.0 3.94
    360.0 2.388
    360.0 2.239
    360.0 1.999
    360.0 0.879
    360.0 1.504
    360.0 2.02
    360.0 2.978
    360.0 0.35
    360.0 0.393
    360.0 9.768
    360.0 5.744
    360.0 3.273
    360.0 25.042
    360.0 1.956
    360.0 5.614
    360.0 0.539
    360.0 17.007
    360.0 0.573
    360.0 7.141
    360.0 0.722
    360.0 0.987
    360.0 161.064
    360.0 7.384
    360.0 5.26
    360.0 1.598
    360.0 6.141
    360.0 0.82
    360.0 9.514
    360.0 2.956
    360.0 360.0
    360.0 72.609
    360.0 0.724
    360.0 12.774
    360.0 121.501
    360.0 3.317
    360.0 0.466
    360.0 0.951
    360.0 360.0
    360.0 0.322
    360.0 1.221
    360.0 0.677
    360.0 0.718
    360.0 3.242
    360.0 0.763
    360.0 13.238
    360.0 11.091
    360.0 0.387
    360.0 34.122
    360.0 1.816
    360.0 0.493
    360.0 5.336
    360.0 0.687
    360.0 360.0
    360.0 3.165
    360.0 360.0
    360.0 2.975
    360.0 5.93
    360.0 3.031
    360.0 360.0
    360.0 0.236
    360.0 360.0
    360.0 2.506
    360.0 0.782
    360.0 2.79
    360.0 1.942
    360.0 0.995
    360.0 0.53
    360.0 0.631
    360.0 0.898
    360.0 2.678
    360.0 2.674
    360.0 1.476
    360.0 0.948
    360.0 0.95
    360.0 2.342
    360.0 1.207
    360.0 8.569
    360.0 5.711
    360.0 0.804
    360.0 0.374
    360.0 0.972
    360.0 0.427
    360.0 0.448
    360.0 10.854
    360.0 2.526
    360.0 2.476
    360.0 0.436
    360.0 360.0
    360.0 2.89
    360.0 360.0
    360.0 0.395
    360.0 3.625
    360.0 0.494
    360.0 2.589
    360.0 360.0
    360.0 0.777
    360.0 360.0
    360.0 205.022
    360.0 2.308
    360.0 360.0
    360.0 0.849
    360.0 0.41
    360.0 0.959
    360.0 360.0
    360.0 0.832
    360.0 32.923
    360.0 3.302
    360.0 1.594
    360.0 0.728
    360.0 360.0
    360.0 0.901
    360.0 4.659
    360.0 0.756
    360.0 5.373
    360.0 1.147
    360.0 0.953
    360.0 360.0
    360.0 2.05
    360.0 0.258
    360.0 2.794
    360.0 20.049
    360.0 3.01
    360.0 7.343
    360.0 0.479
    360.0 0.813
    360.0 2.028
    360.0 0.794
    360.0 0.914
    360.0 4.271
    360.0 3.824
    360.0 4.082
    360.0 6.601
    360.0 4.206
    360.0 39.429
    360.0 4.711
    360.0 4.504
    360.0 6.115
    360.0 5.11
    360.0 8.888
    360.0 0.278
    360.0 360.0
    360.0 2.548
    360.0 0.368
    360.0 2.2
    360.0 4.58
    360.0 1.028
    360.0 1.947
    360.0 360.0
    360.0 0.807
    360.0 0.549
    360.0 3.667
    360.0 360.0
    360.0 2.516
    360.0 0.757
    360.0 0.662
    360.0 1.221
    360.0 2.093
    360.0 2.44
    360.0 0.839
    360.0 0.603
    360.0 4.875
    360.0 2.365
    360.0 1.855
    360.0 13.447
    360.0 0.35
    360.0 2.489
    360.0 0.173
    360.0 0.941
    360.0 3.01
    360.0 2.962
    360.0 1.315
    360.0 0.634
    360.0 0.263
    360.0 2.855
    360.0 5.364
    360.0 4.282
    360.0 1.029
    360.0 1.666
    360.0 1.208
    360.0 2.897
    360.0 4.028
    360.0 1.634
    360.0 0.489
    360.0 3.31
    360.0 2.92
    360.0 0.587
    360.0 0.656
    360.0 360.0
    360.0 6.89
    360.0 0.585
    360.0 0.772
    360.0 0.767
    360.0 0.873
    360.0 1.122
    360.0 1.1
    360.0 32.246
    360.0 4.779
    360.0 360.0
    360.0 2.586
    360.0 0.292
    360.0 9.113
    360.0 1.399
    360.0 4.495
    360.0 3.062
    360.0 8.13
    360.0 0.871
    360.0 0.528
    360.0 1.161
    360.0 0.903
    360.0 360.0
    360.0 0.692
    360.0 0.79
    360.0 0.92
    360.0 2.647
    360.0 2.416
    360.0 1.756
    360.0 0.592
    360.0 0.384
    360.0 1.073
    360.0 5.438
    360.0 1.123
    360.0 0.418
    360.0 31.25
    360.0 0.688
    360.0 1.909
    360.0 0.754
    360.0 0.818
    360.0 4.042
    360.0 0.355
    360.0 0.42
    360.0 0.607
    360.0 0.414
    360.0 7.98
    360.0 3.19
    360.0 7.377
    360.0 3.109
    360.0 1.262
    360.0 360.0
    360.0 2.259
    360.0 2.506
    360.0 12.282
    360.0 133.746
    360.0 0.748
    360.0 3.143
    360.0 0.615
    360.0 3.133
    360.0 10.21
    360.0 0.831
    360.0 4.764
    360.0 0.864
    360.0 249.567
    360.0 360.0
    360.0 0.342
    360.0 0.581
    360.0 4.927
    360.0 0.899
    360.0 10.946
    360.0 0.665
    360.0 108.591
    360.0 0.388
    360.0 0.665
    360.0 6.839
    360.0 0.559
    360.0 0.672
    360.0 2.591
    360.0 8.329
    360.0 5.17
    360.0 1.486
    360.0 0.28
    360.0 3.01
    360.0 360.0
    360.0 0.694
    360.0 5.212
    360.0 0.56
    360.0 1.082
    360.0 1.032
    360.0 1.039
    360.0 2.04
    360.0 1.929
    360.0 4.518
    360.0 0.714
    360.0 4.668
    360.0 360.0
    360.0 6.061
    360.0 3.365
    360.0 360.0
    360.0 0.999
    360.0 360.0
    360.0 2.718
    360.0 1.35
    360.0 0.653
    360.0 11.242
    360.0 360.0
    360.0 0.641
    360.0 0.689
    360.0 1.339
    360.0 2.654
    360.0 0.454
    360.0 24.784
    360.0 1.043
    360.0 5.26
    360.0 9.149
    360.0 360.0
    360.0 360.0
    360.0 3.036
    360.0 360.0
    360.0 2.71
    360.0 3.049
    360.0 1.182
    360.0 1.675
    360.0 360.0
    360.0 0.551
    360.0 0.431
    360.0 1.274
    360.0 0.369
    360.0 6.028
    360.0 1.162
    360.0 360.0
    360.0 0.784
    360.0 0.629
    360.0 1.091
    360.0 0.458
    360.0 19.749
    360.0 0.516
    360.0 2.749
    360.0 0.724
    360.0 0.752
    360.0 5.281
    360.0 0.799
    360.0 1.282
    360.0 3.848
    360.0 360.0
    360.0 0.393
    360.0 4.327
    360.0 4.397
    360.0 10.537
    360.0 0.631
    360.0 1.266
    360.0 0.833
    360.0 1.04
    360.0 0.648
    360.0 1.537
    360.0 5.12
    360.0 63.247
    360.0 3.343
    360.0 25.961
    360.0 1.367
    360.0 6.442
    360.0 0.849
    360.0 0.567
    360.0 2.153
    360.0 360.0
    360.0 3.821
    360.0 15.149
    360.0 3.284
    360.0 1.898
    360.0 5.741
    360.0 0.369
    360.0 2.755
    360.0 360.0
    360.0 1.474
    360.0 6.664
    360.0 3.329
    360.0 360.0
    360.0 0.883
    360.0 360.0
    360.0 0.815
    360.0 0.377
    360.0 0.573
    360.0 0.727
    360.0 2.601
    360.0 2.427
    360.0 3.621
    360.0 2.964
    360.0 360.0
    360.0 0.93
    360.0 1.112
    360.0 3.238
    360.0 4.15
    360.0 1.098
    360.0 3.018
    360.0 3.443
    360.0 1.313
    360.0 360.0
    360.0 0.792
    360.0 4.084
    360.0 2.926
    360.0 360.0
    360.0 1.16
    360.0 0.708
    360.0 1.209
    360.0 8.495
    360.0 360.0
    360.0 0.646
    360.0 3.203
    360.0 7.042
    360.0 0.544
    360.0 360.0
    360.0 11.914
    360.0 4.743
    360.0 3.583
    360.0 4.752
    360.0 3.756
    360.0 1.781
    360.0 0.972
    360.0 360.0
    360.0 3.173
    360.0 0.92
    360.0 0.481
    360.0 0.362
    360.0 1.446
    360.0 0.776
    360.0 2.516
    360.0 3.586
    360.0 3.688
    360.0 360.0
    360.0 0.676
    360.0 0.676
    360.0 0.695
    360.0 3.663
    360.0 5.103
    360.0 4.328
    360.0 1.552
    360.0 5.456
    360.0 0.33
    360.0 11.676
    360.0 0.321
    360.0 0.89
    360.0 1.965
    360.0 0.721
    360.0 1.133
    360.0 2.519
    360.0 0.648
    360.0 0.99
    360.0 1.301
    360.0 15.353
    360.0 2.567
    360.0 3.0
    360.0 3.126
    360.0 2.12
    360.0 5.679
    360.0 3.12
    360.0 1.602
    360.0 3.764
    360.0 2.964
    360.0 2.219
    360.0 360.0
    360.0 0.999
    360.0 2.903
    360.0 2.881
    360.0 0.865
    360.0 3.008
    360.0 0.749
    360.0 3.043
    360.0 0.381
    360.0 0.793
    360.0 0.674
    360.0 0.634
    360.0 360.0
    360.0 6.397
    360.0 4.12
    360.0 0.612
    360.0 3.77
    360.0 360.0
    360.0 1.181
    360.0 0.779
    360.0 0.349
    360.0 0.865
    360.0 1.897
    360.0 9.183
    360.0 0.358
    360.0 5.919
    360.0 3.78
    360.0 1.055
    360.0 1.111
    360.0 0.894
    360.0 0.83
    360.0 1.038
    360.0 0.698
    360.0 2.395
    360.0 360.0
    360.0 360.0
    360.0 2.579
    360.0 40.432
    360.0 0.871
    360.0 0.64
    360.0 360.0
    360.0 0.385
    360.0 3.135
    360.0 0.651
    360.0 1.855
    360.0 360.0
    360.0 0.806
    360.0 2.063
    360.0 0.309
    360.0 4.499
    360.0 0.734
    360.0 9.03
    360.0 1.441
    360.0 0.58
    360.0 1.003
    360.0 0.498
    360.0 6.082
    360.0 3.446
    360.0 0.976
    360.0 0.766
    360.0 1.795
    360.0 360.0
    360.0 0.363
    360.0 3.183
    360.0 4.338
    360.0 2.31
    360.0 3.064
    360.0 0.475
    360.0 0.809
    360.0 56.407
    360.0 3.513
    360.0 4.579
    360.0 3.131
    360.0 0.532
    360.0 0.811
    360.0 1.047
    360.0 0.55
    360.0 2.604
    360.0 360.0
    360.0 0.5
    360.0 0.828
    360.0 5.594
    360.0 2.504
    360.0 0.696
    360.0 1.404
    360.0 0.495
    360.0 360.0
    360.0 1.025
    360.0 1.066
    360.0 4.039
    360.0 360.0
    360.0 0.709
    360.0 0.893
    360.0 4.294
    360.0 0.533
    360.0 5.94
    360.0 3.997
    360.0 0.393
    360.0 0.989
    360.0 3.207
    360.0 4.214
    360.0 0.577
    360.0 4.561
    360.0 0.364
    360.0 0.362
    360.0 360.0
    360.0 3.42
    360.0 0.474
    360.0 0.654
    360.0 0.948
    360.0 1.47
    360.0 0.403
    360.0 360.0
    360.0 2.658
    360.0 0.619
    360.0 26.153
    360.0 4.083
    360.0 6.739
    360.0 6.012
    360.0 0.324
    360.0 2.93
    360.0 0.781
    360.0 0.491
    360.0 3.608
    360.0 0.747
    360.0 0.786
    360.0 0.769
    360.0 42.27
    360.0 2.025
    360.0 360.0
    360.0 2.984
    360.0 1.421
    360.0 0.326
    360.0 2.958
    360.0 0.55
    360.0 99.991
    360.0 0.691
    360.0 7.585
    360.0 5.193
    360.0 0.692
    360.0 5.099
    360.0 0.992
    360.0 6.788
    360.0 0.563
    360.0 1.305
    360.0 2.741
    360.0 6.406
    360.0 4.659
    360.0 4.762
    360.0 5.537
    360.0 0.289
    360.0 5.047
    360.0 208.923
    360.0 0.386
    360.0 0.783
    360.0 2.647
    360.0 1.181
    360.0 3.203
    360.0 0.558
    360.0 8.655
    360.0 0.36
    360.0 0.34
    360.0 2.161
    360.0 360.0
    360.0 360.0
    360.0 360.0
    360.0 8.274
    360.0 13.205
    360.0 0.626
    360.0 4.862
    360.0 0.557
    360.0 1.293
    360.0 6.555
    360.0 3.006
    360.0 8.46
    360.0 0.322
    360.0 1.328
    360.0 34.153
    360.0 5.975
    360.0 2.626
    360.0 2.377
    360.0 6.428
    360.0 0.629
    360.0 360.0
    360.0 3.331
    360.0 3.26
    360.0 3.406
    360.0 360.0
    360.0 2.825
    360.0 0.412
    360.0 2.356
    360.0 14.038
    360.0 1.06
    360.0 0.767
    360.0 360.0
    360.0 2.87
    360.0 18.842
    360.0 3.412
    360.0 2.572
    360.0 11.748
    360.0 2.99
    360.0 0.791
    360.0 1.605
    360.0 0.407
    360.0 0.579
    360.0 2.597
    360.0 2.961
    360.0 0.367
    360.0 1.118
    360.0 0.784
    360.0 1.165
    360.0 2.885
    360.0 0.88
    360.0 1.101
    360.0 2.254
    360.0 4.13
    360.0 3.093
    360.0 2.603
    360.0 5.412
    360.0 0.92
    360.0 1.36
    360.0 3.302
    360.0 24.404
    360.0 2.656
    360.0 3.686
    360.0 16.051
    360.0 0.409
    360.0 0.738
    360.0 3.614
    360.0 5.558
    360.0 360.0
    360.0 0.722
    360.0 2.338
    360.0 5.624
    360.0 5.816
    360.0 1.463
    360.0 0.668
    360.0 0.479
    360.0 0.375
    360.0 0.629
    360.0 0.764
    360.0 4.665
    360.0 1.215
    360.0 20.8
    360.0 4.593
    360.0 0.891
    360.0 360.0
    360.0 6.434
    360.0 3.446
    360.0 6.094
    360.0 8.974
    360.0 4.92
    360.0 0.959
    360.0 360.0
    360.0 1.644
    360.0 11.205
    360.0 0.529
    360.0 2.379
    360.0 3.963
    360.0 0.74
    360.0 8.202
    360.0 4.087
    360.0 12.223
    360.0 1.141
    360.0 2.599
    360.0 0.518
    360.0 2.929
    360.0 0.366
    360.0 1.346
    360.0 0.347
    360.0 0.761
    360.0 0.752
    360.0 0.467
    360.0 0.617
    360.0 5.041
    360.0 0.352
    360.0 360.0
    360.0 81.352
    360.0 5.515
    360.0 3.079
    360.0 0.631
    360.0 2.509
    360.0 360.0
    360.0 5.207
    360.0 4.057
    360.0 3.575
    360.0 0.813
    360.0 4.669
    360.0 1.619
    360.0 18.285
    360.0 3.334
    360.0 0.523
    360.0 0.717
    360.0 11.584
    360.0 3.728
    360.0 14.995
    360.0 0.585
    360.0 5.009
    360.0 360.0
    360.0 3.882
    360.0 360.0
    360.0 0.889
    360.0 0.621
    360.0 0.92
    360.0 2.602
    360.0 0.685
    360.0 0.518
    360.0 1.112
    360.0 0.688
    360.0 1.037
    360.0 0.963
    360.0 3.586
    360.0 2.312
    360.0 32.607
    360.0 0.661
    360.0 28.651
    360.0 318.729
    360.0 0.821
    360.0 0.418
    360.0 3.612
    360.0 54.008
    360.0 190.324
    360.0 0.676
    360.0 0.461
    360.0 0.37
    360.0 2.982
    360.0 8.765
    360.0 3.44
    360.0 9.828
    360.0 0.319
    360.0 1.3
    360.0 0.658
    360.0 33.043
    360.0 1.217
    360.0 4.686
    360.0 360.0
    360.0 9.537
    360.0 4.554
    360.0 360.0
    360.0 5.631
    360.0 0.764
    360.0 0.829
    360.0 0.748
    360.0 0.824
    360.0 360.0
    360.0 0.872
    360.0 4.49
    360.0 360.0
    360.0 1.212
    360.0 3.157
    360.0 1.887
    360.0 3.953
    360.0 0.514
    360.0 4.368
    360.0 3.696
    360.0 47.808
    360.0 92.513
    360.0 1.722
    360.0 360.0
    360.0 7.369
    360.0 0.771
    360.0 3.669
    360.0 42.403
    360.0 0.538
    360.0 5.246
    360.0 2.679
    360.0 6.422
    360.0 360.0
    360.0 4.921
    360.0 1.523
    360.0 7.924
    360.0 4.926
    360.0 4.294
    360.0 0.389
    360.0 0.969
    360.0 2.285
    360.0 360.0
    360.0 10.76
    360.0 3.325
    360.0 0.675
    360.0 6.224
    360.0 5.395
    360.0 1.883
    360.0 0.846
    360.0 0.904
    360.0 0.395
    360.0 14.625
    360.0 0.808
    360.0 1.219
    360.0 2.478
    360.0 1.491
    360.0 0.581
    360.0 0.888
    360.0 3.254
    360.0 2.666
    360.0 1.516
    360.0 0.896
    360.0 0.636
    360.0 0.725
    360.0 0.538
    360.0 14.34
    360.0 1.096
    360.0 15.366
    360.0 360.0
    360.0 37.952
    360.0 360.0
    360.0 0.872
    360.0 11.269
    360.0 0.644
    360.0 360.0
    360.0 0.713
    360.0 1.124
    360.0 0.841
    360.0 12.014
    360.0 8.956
    360.0 0.656
    360.0 4.702
    360.0 5.269
    360.0 0.827
    360.0 0.377
    360.0 2.239
    360.0 0.601
    360.0 360.0
    360.0 1.651
    360.0 0.832
    360.0 2.103
    360.0 7.564
    360.0 1.572
    360.0 0.887
    360.0 6.426
    360.0 3.56
    360.0 3.33
    360.0 22.489
    360.0 1.506
    360.0 0.726
    360.0 3.48
    360.0 3.697
    360.0 6.236
    360.0 2.336
    360.0 0.683
    360.0 1.489
    360.0 1.959
    360.0 0.702
    360.0 0.217
    360.0 1.041
    360.0 3.272
    360.0 0.84
    360.0 2.853
    360.0 4.61
    360.0 360.0
    360.0 0.978
    360.0 0.731
    360.0 0.85
    360.0 1.007
    360.0 4.911
    360.0 1.812
    360.0 0.38
    360.0 0.858
    360.0 2.519
    360.0 360.0
    360.0 0.789
    360.0 8.293
    360.0 4.312
    360.0 0.733
    360.0 1.303
    360.0 360.0
    360.0 0.852
    360.0 3.002
    360.0 5.133
    360.0 1.863
    360.0 215.761
    360.0 17.383
    360.0 1.272
    360.0 0.478
    360.0 6.581
    360.0 4.441
    360.0 0.647
    360.0 3.938
    360.0 360.0
    360.0 0.364
    360.0 5.993
    360.0 0.499
    360.0 2.772
    360.0 3.752
    360.0 0.245
    360.0 2.623
    360.0 3.543
    360.0 0.717
    360.0 3.725
    360.0 0.71
    360.0 3.274
    360.0 0.615
    360.0 4.643
    360.0 0.619
    360.0 2.192
    360.0 0.982
    360.0 1.837
    360.0 9.319
    360.0 5.977
    360.0 4.666
    360.0 3.386
    360.0 0.302
    360.0 0.987
    360.0 0.748
    360.0 5.558
    360.0 4.431
    360.0 59.245
    360.0 360.0
    360.0 4.168
    360.0 3.545
    360.0 7.908
    360.0 1.085
    360.0 1.469
    360.0 0.755
    360.0 360.0
    360.0 0.705
    360.0 36.028
    360.0 5.207
    360.0 0.398
    360.0 1.388
    360.0 0.947
    360.0 2.679
    360.0 1.741
    360.0 33.866
    360.0 0.501
    360.0 0.713
    360.0 3.997
    360.0 3.593
    360.0 10.038
    360.0 360.0
    360.0 4.669
    360.0 360.0
    360.0 4.472
    360.0 0.406
    360.0 14.433
    360.0 1.02
    360.0 11.456
    360.0 1.017
    360.0 0.314
    360.0 5.912
    360.0 17.017
    360.0 4.684
    360.0 5.248
    360.0 0.801
    360.0 1.449
    360.0 2.422
    360.0 2.889
    360.0 2.478
    360.0 0.743
    360.0 360.0
    360.0 0.961
    360.0 0.556
    360.0 360.0
    360.0 4.531
    360.0 360.0
    360.0 360.0
    360.0 4.642
    360.0 4.335
    360.0 1.013
    360.0 13.399
    360.0 0.76
    360.0 0.997
    360.0 0.512
    360.0 6.798
    360.0 4.165
    360.0 0.839
    360.0 2.289
    360.0 2.576
    360.0 5.281
    360.0 4.755
    360.0 2.37
    360.0 2.458
    360.0 0.878
    360.0 0.385
    360.0 3.031
    360.0 0.465
    360.0 360.0
    360.0 2.809
    360.0 0.366
    360.0 0.395
    360.0 7.186
    360.0 6.041
    360.0 11.466
    360.0 0.854
    360.0 7.778
    360.0 8.292
    360.0 1.057
    360.0 4.893
    360.0 0.476
    360.0 6.941
    360.0 0.931
    360.0 0.632
    360.0 4.568
    360.0 1.04
    360.0 0.548
    360.0 14.778
    360.0 0.689
    360.0 0.615
    360.0 0.576
    360.0 0.608
    360.0 4.312
    360.0 360.0
    360.0 1.416
    360.0 6.586
    360.0 4.108
    360.0 4.495
    360.0 14.077
    360.0 28.803
    360.0 7.838
    360.0 360.0
    360.0 0.636
    360.0 34.489
    360.0 0.835
    360.0 0.445
    360.0 2.641
    360.0 0.706
    360.0 0.738
    360.0 5.982
    360.0 3.603
    360.0 2.066
    360.0 0.934
    360.0 0.674
    360.0 0.641
    360.0 1.764
    360.0 3.107
    360.0 360.0
    360.0 360.0
    360.0 4.454
    360.0 0.377
    360.0 5.381
    360.0 1.721
    360.0 4.969
    360.0 1.559
    360.0 4.787
    360.0 2.964
    360.0 360.0
    360.0 0.935
    360.0 0.863
    360.0 0.486
    360.0 0.415
    360.0 0.855
    360.0 7.079
    360.0 360.0
    360.0 0.861
    360.0 5.842
    360.0 360.0
    360.0 5.805
    360.0 0.732
    360.0 0.278
    360.0 0.407
    360.0 0.787
    360.0 1.275
    360.0 2.52
    360.0 6.064
    360.0 2.898
    360.0 1.901
    360.0 0.638
    360.0 4.218
    360.0 0.435
    360.0 3.145
    360.0 1.099
    360.0 3.038
    360.0 0.803
    360.0 0.717
    360.0 0.295
    360.0 1.746
    360.0 0.308
    360.0 360.0
    360.0 360.0
    360.0 2.936
    360.0 2.915
    360.0 0.762
    360.0 0.612
    360.0 0.841
    360.0 1.145
    360.0 0.906
    360.0 0.411
    360.0 0.549
    360.0 2.969
    360.0 0.804
    360.0 3.032
    360.0 5.762
    360.0 2.587
    360.0 3.206
    360.0 0.643
    360.0 3.243
    360.0 0.661
    360.0 0.361
    360.0 5.972
    360.0 1.559
    360.0 11.395
    360.0 0.743
    360.0 360.0
    360.0 360.0
    360.0 360.0
    360.0 27.235
    360.0 4.554
    360.0 360.0
    360.0 360.0
    360.0 4.173
    360.0 0.853
    360.0 4.757
    360.0 360.0
    360.0 0.644
    360.0 2.36
    360.0 0.688
    360.0 0.367
    360.0 4.064
    360.0 0.667
    360.0 0.663
    360.0 1.18
    360.0 360.0
    360.0 0.381
    360.0 0.361
    360.0 6.531
    360.0 1.227
    360.0 360.0
    360.0 0.51
    360.0 6.717
    360.0 0.604
    360.0 0.476
    360.0 0.521
    360.0 0.173
    360.0 0.474
    360.0 1.262
    360.0 1.987
    360.0 0.691
    360.0 0.836
    360.0 2.279
    360.0 8.452
    360.0 0.471
    360.0 0.487
    360.0 4.353
    360.0 4.767
    360.0 0.679
    360.0 360.0
    360.0 2.732
    360.0 20.58
    360.0 16.742
    360.0 0.306
    360.0 3.068
    360.0 1.281
    360.0 1.678
    360.0 1.536
    360.0 9.333
    360.0 2.764
    360.0 0.427
    360.0 0.724
    360.0 0.51
    360.0 0.6
    360.0 0.676
    360.0 7.822
    360.0 360.0
    360.0 24.231
    360.0 360.0
    360.0 0.476
    360.0 50.63
    360.0 1.866
    360.0 0.656
    360.0 360.0
    360.0 3.54
    360.0 3.561
    360.0 2.779
    360.0 2.832
    360.0 0.627
    360.0 0.708
    360.0 4.901
    360.0 0.825
    360.0 0.406
    360.0 2.536
    360.0 3.624
    360.0 0.763
    360.0 0.292
    360.0 2.554
    360.0 0.747
    360.0 2.838
    360.0 3.053
    360.0 297.694
    360.0 15.135
    360.0 2.619
    360.0 0.617
    360.0 22.807
    360.0 0.353
    360.0 0.687
    360.0 8.853
    360.0 1.444
    360.0 360.0
    360.0 0.971
    360.0 0.311
    360.0 8.532
    360.0 1.209
    360.0 1.564
    360.0 2.599
    360.0 10.425
    360.0 0.391
    360.0 30.303
    360.0 0.432
    360.0 0.771
    360.0 0.839
    360.0 0.677
    360.0 1.014
    360.0 0.784
    360.0 1.027
    360.0 4.09
    360.0 3.159
    360.0 9.648
    360.0 5.22
    360.0 360.0
    360.0 360.0
    360.0 0.836
    360.0 0.613
    360.0 4.063
    360.0 0.364
    360.0 37.194
    360.0 12.489
    360.0 14.205
    360.0 1.709
    360.0 0.504
    360.0 56.561
    360.0 0.594
    360.0 360.0
    360.0 0.407
    360.0 6.642
    360.0 2.222
    360.0 1.908
    360.0 0.712
    360.0 0.74
    360.0 0.948
    360.0 360.0
    360.0 4.355
    360.0 2.936
    360.0 3.73
    360.0 3.711
    360.0 0.622
    360.0 0.583
    360.0 0.917
    360.0 2.789
    360.0 0.246
    360.0 8.541
    360.0 0.419
    360.0 0.34
    360.0 3.176
    360.0 360.0
    360.0 0.638
    360.0 23.607
    360.0 0.519
    360.0 3.372
    360.0 360.0
    360.0 0.661
    360.0 0.375
    360.0 360.0
    360.0 3.057
    360.0 2.731
    360.0 360.0
    360.0 2.898
    360.0 43.453
    360.0 4.477
    360.0 0.426
    360.0 0.492
    360.0 0.513
    360.0 31.458
    360.0 3.969
    360.0 0.528
    360.0 0.657
    360.0 0.662
    360.0 0.523
    360.0 7.309
    360.0 2.504
    360.0 54.577
    360.0 2.464
    360.0 5.085
    360.0 4.897
    360.0 6.174
    360.0 9.699
    360.0 2.079
    360.0 3.93
    360.0 0.477
    360.0 0.593
    360.0 0.61
    360.0 3.768
    360.0 1.274
    360.0 4.57
    360.0 3.015
    360.0 1.255
    360.0 2.99
    360.0 25.091
    360.0 0.412
    360.0 2.251
    360.0 1.258
    360.0 0.839
    360.0 0.382
    360.0 0.88
    360.0 0.396
    360.0 4.601
    360.0 0.841
    360.0 1.046
    360.0 0.828
    360.0 0.779
    360.0 2.858
    360.0 2.381
    360.0 360.0
    360.0 5.096
    360.0 1.028
    360.0 3.076
    360.0 4.321
    360.0 0.811
    360.0 360.0
    360.0 1.078
    360.0 0.781
    360.0 2.903
    360.0 1.29
    360.0 2.509
    360.0 2.763
    360.0 0.615
    360.0 1.268
    360.0 0.823
    360.0 240.71
    360.0 0.754
    360.0 1.116
    360.0 53.379
    360.0 0.926
    360.0 6.427
    360.0 1.094
    360.0 0.514
    360.0 1.003
    360.0 4.496
    360.0 5.192
    360.0 2.781
    360.0 2.434
    360.0 1.993
    360.0 0.719
    360.0 0.516
    360.0 3.129
    360.0 7.57
    360.0 0.61
    360.0 0.462
    360.0 0.69
    360.0 3.662
    360.0 0.988
    360.0 1.289
    360.0 0.838
    360.0 6.983
    360.0 360.0
    360.0 0.651
    360.0 11.449
    360.0 3.055
    360.0 0.718
    360.0 0.65
    360.0 1.309
    360.0 1.808
    360.0 1.3
    360.0 0.52
    360.0 3.879
    360.0 3.316
    360.0 3.498
    360.0 0.371
    360.0 6.064
    360.0 4.711
    360.0 5.345
    360.0 2.76
    360.0 2.692
    360.0 0.721
    360.0 2.376
    360.0 0.581
    360.0 11.053
    360.0 0.848
    360.0 0.484
    360.0 3.189
    360.0 360.0
    360.0 360.0
    360.0 7.661
    360.0 3.377
    360.0 9.456
    360.0 0.189
    360.0 0.892
    360.0 5.566
    360.0 360.0
    360.0 0.854
    360.0 0.606
    360.0 2.477
    360.0 0.379
    360.0 0.821
    360.0 0.481
    360.0 0.813
    360.0 1.086
    360.0 2.212
    360.0 0.732
    360.0 6.133
    360.0 4.643
    360.0 0.889
    360.0 0.56
    360.0 0.564
    360.0 1.173
    360.0 3.918
    360.0 0.698
    360.0 4.155
    360.0 4.297
    360.0 4.66
    360.0 0.364
    360.0 360.0
    360.0 0.666
    360.0 17.999
    360.0 9.415
    360.0 1.024
    360.0 6.197
    360.0 1.989
    360.0 360.0
    360.0 360.0
    360.0 0.392
    360.0 4.254
    360.0 0.632
    360.0 1.017
    360.0 2.397
    360.0 15.05
    360.0 0.574
    360.0 4.047
    360.0 360.0
    360.0 6.813
    360.0 1.18
    360.0 4.204
    360.0 4.553
    360.0 6.823
    360.0 0.607
    360.0 0.526
    360.0 5.811
    360.0 4.721
    360.0 2.491
    360.0 8.088
    360.0 0.627
    360.0 1.849
    360.0 1.401
    360.0 0.645
    360.0 0.397
    360.0 2.353
    360.0 1.685
    360.0 0.857
    360.0 102.789
    360.0 0.324
    360.0 4.345
    360.0 3.536
    360.0 1.135
    360.0 1.331
    360.0 0.408
    360.0 360.0
    360.0 0.469
    360.0 360.0
    360.0 0.899
    360.0 0.449
    360.0 1.356
    360.0 0.657
    360.0 360.0
    360.0 9.431
    360.0 1.839
    360.0 1.161
    360.0 0.277
    360.0 7.481
    360.0 2.997
    360.0 4.907
    360.0 16.033
    360.0 3.368
    360.0 3.399
    360.0 119.663
    360.0 0.548
    360.0 3.756
    360.0 2.169
    360.0 2.147
    360.0 0.599
    360.0 0.878
    360.0 4.441
    360.0 360.0
    360.0 0.671
    360.0 0.752
    360.0 360.0
    360.0 360.0
    360.0 0.91
    360.0 1.052
    360.0 0.976
    360.0 1.583
    360.0 2.081
    360.0 0.57
    360.0 360.0
    360.0 2.803
    360.0 16.074
    360.0 3.228
    360.0 3.097
    360.0 0.3
    360.0 0.63
    360.0 0.637
    360.0 0.658
    360.0 0.359
    360.0 2.604
    360.0 1.383
    360.0 7.202
    360.0 3.778
    360.0 4.078
    360.0 4.421
    360.0 0.384
    360.0 0.795
    360.0 3.338
    360.0 360.0
    360.0 2.376
    360.0 0.359
    360.0 0.452
    360.0 3.026
    360.0 0.386
    360.0 0.695
    360.0 58.672
    360.0 7.624
    360.0 28.333
    360.0 0.826
    360.0 0.519
    360.0 4.81
    360.0 0.452
    360.0 1.523
    360.0 0.299
    360.0 0.349
    360.0 0.639
    360.0 0.573
    360.0 59.87
    360.0 4.568
    360.0 21.305
    360.0 360.0
    360.0 0.578
    360.0 0.958
    360.0 3.569
    360.0 1.732
    360.0 360.0
    360.0 4.574
    360.0 2.827
    360.0 0.893
    360.0 1.108
    360.0 3.374
    360.0 0.473
    360.0 4.616
    360.0 0.369
    360.0 0.76
    360.0 0.772
    360.0 5.981
    360.0 3.628
    360.0 10.064
    360.0 18.031
    360.0 1.888
    360.0 3.047
    360.0 0.625
    360.0 360.0
    360.0 1.177
    360.0 3.468
    360.0 4.513
    360.0 0.542
    360.0 0.676
    360.0 3.361
    360.0 2.775
    360.0 45.366
    360.0 17.769
    360.0 360.0
    360.0 1.739
    360.0 0.694
    360.0 360.0
    360.0 4.917
    360.0 360.0
    360.0 2.871
    360.0 3.481
    360.0 0.876
    360.0 0.864
    360.0 0.888
    360.0 6.007
    360.0 0.68
    360.0 0.59
    360.0 122.096
    360.0 2.499
    360.0 11.923
    360.0 0.789
    360.0 3.496
    360.0 360.0
    360.0 4.095
    360.0 0.908
    360.0 7.676
    360.0 2.512
    360.0 24.948
    360.0 3.561
    360.0 4.539
    360.0 0.888
    360.0 2.54
    360.0 360.0
    360.0 0.845
    360.0 22.518
    360.0 360.0
    360.0 1.438
    360.0 4.86
    360.0 4.643
    360.0 2.864
    360.0 3.277
    360.0 2.326
    360.0 1.164
    360.0 0.666
    360.0 1.543
    360.0 6.02
    360.0 1.917
    360.0 3.153
    360.0 3.076
    360.0 0.722
    360.0 5.799
    360.0 0.494
    360.0 1.06
    360.0 0.725
    360.0 0.536
    360.0 11.247
    360.0 360.0
    360.0 0.919
    360.0 0.88
    360.0 0.645
    360.0 12.551
    360.0 6.101
    360.0 0.686
    360.0 2.308
    360.0 2.3
    360.0 2.047
    360.0 19.454
    360.0 0.384
    360.0 2.19
    360.0 360.0
    360.0 0.754
    360.0 5.556
    360.0 7.257
    360.0 3.774
    360.0 1.011
    360.0 4.089
    360.0 1.825
    360.0 3.055
    360.0 1.501
    360.0 0.451
    360.0 0.698
    360.0 1.94
    360.0 33.386
    360.0 0.496
    360.0 0.706
    360.0 1.813
    360.0 2.684
    360.0 5.351
    360.0 1.2
    360.0 0.937
    360.0 0.63
    360.0 3.792
    360.0 0.808
    360.0 0.403
    360.0 5.304
    360.0 0.87
    360.0 3.36
    360.0 6.152
    360.0 55.486
    360.0 0.375
    360.0 3.34
    360.0 360.0
    360.0 2.233
    360.0 1.177
    360.0 2.355
    360.0 2.712
    360.0 0.581
    360.0 0.983
    360.0 0.806
    360.0 360.0
    360.0 2.093
    360.0 360.0
    360.0 0.659
    360.0 6.451
    360.0 360.0
    360.0 1.286
    360.0 2.523
    360.0 1.08
    360.0 0.938
    360.0 360.0
    360.0 1.165
    360.0 0.452
    360.0 360.0
    360.0 0.68
    360.0 0.907
    360.0 0.392
    360.0 4.381
    360.0 3.994
    360.0 0.694
    360.0 0.673
    360.0 4.339
    360.0 1.053
    360.0 2.81
    360.0 0.712
    360.0 0.938
    360.0 19.004
    360.0 0.324
    360.0 0.881
    360.0 1.595
    360.0 1.25
    360.0 0.787
    360.0 0.721
    360.0 60.979
    360.0 24.316
    360.0 3.218
    360.0 3.7
    360.0 3.84
    360.0 1.445
    360.0 2.849
    360.0 4.427
    360.0 2.578
    360.0 6.93
    360.0 3.897
    360.0 5.421
    360.0 2.766
    360.0 1.024
    360.0 2.398
    360.0 0.691
    360.0 2.215
    360.0 12.661
    360.0 7.818
    360.0 29.298
    360.0 0.516
    360.0 1.072
    360.0 2.562
    360.0 0.995
    360.0 4.322
    360.0 4.033
    360.0 3.392
    360.0 2.145
    360.0 360.0
    360.0 0.955
    360.0 6.518
    360.0 4.569
    360.0 1.803
    360.0 360.0
    360.0 1.087
    360.0 4.492
    360.0 1.514
    360.0 3.82
    360.0 0.636
    360.0 0.71
    360.0 4.913
    360.0 0.857
    360.0 3.075
    360.0 0.557
    360.0 360.0
    360.0 3.133
    360.0 16.768
    360.0 6.482
    360.0 0.932
    360.0 4.853
    360.0 6.002
    360.0 1.038
    360.0 0.544
    360.0 0.986
    360.0 0.54
    360.0 0.546
    360.0 0.638
    360.0 1.284
    360.0 21.141
    360.0 0.697
    360.0 1.324
    360.0 3.282
    360.0 4.188
    360.0 1.891
    360.0 17.44
    360.0 1.044
    360.0 1.256
    360.0 0.599
    360.0 0.583
    360.0 0.92
    360.0 0.305
    360.0 0.429
    360.0 4.898
    360.0 0.674
    360.0 3.224
    360.0 4.31
    360.0 1.5
    360.0 0.818
    360.0 0.869
    360.0 0.706
    360.0 15.903
    360.0 3.117
    360.0 10.194
    360.0 2.171
    360.0 2.703
    360.0 0.985
    360.0 2.322
    360.0 5.74
    360.0 360.0
    360.0 4.774
    360.0 15.044
    360.0 2.113
    360.0 0.586
    360.0 0.709
    360.0 4.01
    360.0 0.446
    360.0 0.504
    360.0 3.04
    360.0 3.184
    360.0 4.042
    360.0 4.757
    360.0 6.014
    360.0 0.523
    360.0 2.003
    360.0 360.0
    360.0 4.043
    360.0 360.0
    360.0 0.393
    360.0 0.98
    360.0 2.581
    360.0 1.151
    360.0 3.804
    360.0 3.47
    360.0 10.257
    360.0 1.559
    360.0 0.545
    360.0 4.781
    360.0 3.428
    360.0 360.0
    360.0 13.392
    360.0 0.442
    360.0 0.472
    360.0 3.327
    360.0 9.316
    360.0 0.6
    360.0 18.446
    360.0 2.96
    360.0 4.322
    360.0 3.122
    360.0 0.877
    360.0 18.987
    360.0 0.675
    360.0 3.448
    360.0 6.409
    360.0 7.794
    360.0 3.204
    360.0 2.942
    360.0 2.331
    360.0 18.301
    360.0 0.71
    360.0 1.192
    360.0 0.788
    360.0 1.663
    360.0 1.279
    360.0 0.857
    360.0 15.186
    360.0 0.514
    360.0 1.725
    360.0 36.911
    360.0 1.496
    360.0 0.565
    360.0 4.644
    360.0 360.0
    360.0 0.958
    360.0 360.0
    360.0 0.286
    360.0 2.573
    360.0 1.913
    360.0 3.032
    360.0 0.855
    360.0 1.006
    360.0 0.755
    360.0 25.648
    360.0 3.144
    360.0 5.681
    360.0 2.884
    360.0 0.783
    360.0 0.591
    360.0 0.633
    360.0 4.498
    360.0 13.606
    360.0 3.66
    360.0 0.879
    360.0 0.9
    360.0 0.261
    360.0 0.744
    360.0 3.155
    360.0 28.265
    360.0 109.365
    360.0 11.791
    360.0 3.76
    360.0 0.937
    360.0 0.62
    360.0 0.585
    360.0 360.0
    360.0 2.203
    360.0 2.367
    360.0 0.986
    360.0 3.233
    360.0 0.985
    360.0 6.34
    360.0 3.4
    360.0 360.0
    360.0 3.35
    360.0 1.313
    360.0 0.914
    360.0 0.385
    360.0 0.871
    360.0 0.368
    360.0 4.891
    360.0 2.701
    360.0 1.244
    360.0 0.397
    360.0 360.0
    360.0 360.0
    360.0 165.297
    360.0 13.804
    360.0 6.179
    360.0 3.305
    360.0 1.349
    360.0 360.0
    360.0 2.501
    360.0 3.017
    360.0 3.866
    360.0 6.134
    360.0 0.853
    360.0 0.79
    360.0 1.722
    360.0 4.21
    360.0 3.942
    360.0 10.203
    360.0 0.745
    360.0 0.995
    360.0 4.865
    360.0 5.506
    360.0 360.0
    360.0 360.0
    360.0 0.876
    360.0 2.366
    360.0 1.126
    360.0 2.69
    360.0 0.969
    360.0 4.232
    360.0 360.0
    360.0 5.68
    360.0 0.866
    360.0 1.167
    360.0 0.359
    360.0 2.771
    360.0 32.437
    360.0 0.676
    360.0 2.059
    360.0 0.516
    360.0 2.541
    360.0 204.516
    360.0 4.084
    360.0 0.679
    360.0 2.189
    360.0 1.124
    360.0 2.897
    360.0 1.152
    360.0 2.245
    360.0 2.764
    360.0 0.254
    360.0 360.0
    360.0 3.777
    360.0 2.931
    360.0 1.288
    360.0 0.718
    360.0 360.0
    360.0 0.306
    360.0 5.089
    360.0 360.0
    360.0 360.0
    360.0 0.983
    360.0 0.355
    360.0 3.883
    360.0 0.785
    360.0 0.912
    360.0 1.096
    360.0 8.432
    360.0 360.0
    360.0 9.224
    360.0 360.0
    360.0 0.895
    360.0 4.499
    360.0 10.27
    360.0 0.767
    360.0 1.519
    360.0 0.346
    360.0 0.674
    360.0 0.519
    360.0 4.74
    360.0 360.0
    360.0 0.544
    360.0 2.079
    360.0 0.353
    360.0 11.633
    360.0 4.289
    360.0 1.254
    360.0 16.7
    360.0 3.127
    360.0 6.261
    360.0 5.615
    360.0 0.702
    360.0 0.41
    360.0 5.723
    360.0 3.604
    360.0 1.623
    360.0 0.36
    360.0 3.949
    360.0 1.903
    360.0 1.94
    360.0 0.981
    360.0 4.634
    360.0 1.065
    360.0 3.921
    360.0 360.0
    360.0 360.0
    360.0 0.444
    360.0 0.348
    360.0 0.701
    360.0 0.863
    360.0 1.018
    360.0 21.774
    360.0 14.814
    360.0 3.356
    360.0 3.454
    360.0 121.382
    360.0 1.173
    360.0 1.436
    360.0 3.599
    360.0 20.2
    360.0 2.985
    360.0 1.965
    360.0 0.918
    360.0 360.0
    360.0 0.764
    360.0 35.963
    360.0 6.318
    360.0 0.807
    360.0 1.351
    360.0 0.417
    360.0 0.358
    360.0 3.562
    360.0 6.009
    360.0 0.996
    360.0 4.129
    360.0 0.532
    360.0 360.0
    360.0 0.503
    360.0 0.337
    360.0 1.705
    360.0 0.961
    360.0 1.475
    360.0 0.888
    360.0 0.846
    360.0 0.367
    360.0 1.227
    360.0 1.286
    360.0 360.0
    360.0 3.107
    360.0 0.776
    360.0 2.626
    360.0 0.604
    360.0 0.7
    360.0 4.462
    360.0 360.0
    360.0 0.559
    360.0 0.829
    360.0 0.728
    360.0 0.664
    360.0 5.513
    360.0 0.726
    360.0 0.989
    360.0 1.918
    360.0 0.481
    360.0 44.93
    360.0 360.0
    360.0 1.126
    360.0 0.506
    360.0 2.671
    360.0 0.391
    360.0 1.713
    360.0 16.106
    360.0 0.876
    360.0 0.847
    360.0 1.183
    360.0 1.777
    360.0 0.728
    360.0 0.807
    360.0 3.468
    360.0 27.437
    360.0 4.776
    360.0 3.235
    360.0 360.0
    360.0 0.814
    360.0 360.0
    360.0 5.083
    360.0 3.438
    360.0 5.604
    360.0 0.814
    360.0 0.566
    360.0 360.0
    360.0 2.975
    360.0 3.392
    360.0 0.662
    360.0 0.858
    360.0 1.698
    360.0 57.229
    360.0 3.651
    360.0 0.568
    360.0 0.933
    360.0 0.54
    360.0 360.0
    360.0 1.003
    360.0 1.048
    360.0 4.185
    360.0 3.409
    360.0 1.051
    360.0 360.0
    360.0 1.438
    360.0 8.219
    360.0 4.01
    360.0 2.782
    360.0 3.97
    360.0 3.047
    360.0 0.612
    360.0 0.422
    360.0 0.745
    360.0 360.0
    360.0 8.384
    360.0 360.0
    360.0 3.463
    360.0 1.64
    360.0 360.0
    360.0 0.825
    360.0 0.645
    360.0 0.362
    360.0 360.0
    360.0 1.039
    360.0 0.55
    360.0 360.0
    360.0 360.0
    360.0 2.872
    360.0 0.766
    360.0 12.111
    360.0 0.318
    360.0 0.71
    360.0 3.074
    360.0 7.096
    360.0 4.051
    360.0 1.105
    360.0 0.955
    360.0 3.079
    360.0 4.0
    360.0 8.676
    360.0 1.826
    360.0 0.798
    360.0 360.0
    360.0 0.258
    360.0 1.0
    360.0 2.593
    360.0 1.286
    360.0 6.131
    360.0 0.677
    360.0 0.635
    360.0 360.0
    360.0 360.0
    360.0 0.464
    360.0 7.833
    360.0 360.0
    360.0 0.799
    360.0 0.578
    360.0 3.086
    360.0 0.993
    360.0 2.624
    360.0 1.196
    360.0 1.938
    360.0 1.046
    360.0 360.0
    360.0 360.0
    360.0 0.614
    360.0 0.658
    360.0 1.702
    360.0 2.466
    360.0 0.75
    360.0 1.018
    360.0 0.999
    360.0 0.667
    360.0 0.38
    360.0 2.911
    360.0 0.405
    360.0 1.096
    360.0 1.22
    360.0 4.826
    360.0 6.627
    360.0 0.99
    360.0 4.835
    360.0 3.24
    360.0 8.455
    360.0 0.694
    360.0 5.825
    360.0 8.386
    360.0 5.107
    360.0 5.715
    360.0 1.821
    360.0 4.158
    360.0 0.709
    360.0 0.527
    360.0 21.155
    360.0 5.166
    360.0 5.44
    360.0 0.651
    360.0 1.639
    360.0 0.664
    360.0 3.046
    360.0 13.41
    360.0 0.873
    360.0 0.547
    360.0 360.0
    360.0 0.854
    360.0 4.621
    360.0 3.868
    360.0 0.814
    360.0 0.645
    360.0 4.261
    360.0 1.187
    360.0 1.026
    360.0 1.398
    360.0 5.929
    360.0 243.03
    360.0 0.898
    360.0 29.46
    360.0 0.777
    360.0 0.741
    360.0 3.641
    360.0 3.671
    360.0 0.732
    360.0 0.368
    360.0 3.899
    360.0 8.693
    360.0 183.873
    360.0 0.525
    360.0 4.831
    360.0 0.98
    360.0 3.08
    360.0 360.0
    360.0 0.669
    360.0 4.451
    360.0 0.683
    360.0 0.859
    360.0 15.148
    360.0 0.405
    360.0 23.659
    360.0 12.459
    360.0 0.579
    360.0 3.126
    360.0 0.255
    360.0 7.084
    360.0 10.967
    360.0 2.059
    360.0 66.121
    360.0 0.362
    360.0 1.007
    360.0 0.663
    360.0 0.805
    360.0 2.347
    360.0 0.354
    360.0 17.771
    360.0 4.526
    360.0 30.096
    360.0 3.272
    360.0 0.606
    360.0 2.792
    360.0 4.38
    360.0 360.0
    360.0 5.71
    360.0 4.557
    360.0 0.633
    360.0 3.702
    360.0 0.82
    360.0 0.981
    360.0 5.203
    360.0 5.774
    360.0 0.843
    360.0 2.96
    360.0 0.507
    360.0 4.712
    360.0 4.622
    360.0 0.762
    360.0 225.423
    360.0 3.734
    360.0 3.519
    360.0 1.033
    360.0 8.469
    360.0 3.907
    360.0 3.461
    360.0 6.266
    360.0 3.597
    360.0 0.837
    360.0 0.901
    360.0 0.671
    360.0 360.0
    360.0 3.754
    360.0 4.469
    360.0 360.0
    360.0 0.976
    360.0 5.156
    360.0 2.653
    360.0 0.988
    360.0 1.207
    360.0 1.049
    360.0 4.027
    360.0 4.293
    360.0 4.613
    360.0 3.663
    360.0 3.966
    360.0 0.989
    360.0 2.953
    360.0 0.552
    360.0 3.045
    360.0 0.363
    360.0 1.739
    360.0 2.174
    360.0 6.106
    360.0 4.795
    360.0 2.835
    360.0 1.045
    360.0 1.048
    360.0 0.837
    360.0 1.717
    360.0 0.364
    360.0 1.032
    360.0 1.07
    360.0 0.614
    360.0 0.505
    360.0 3.582
    360.0 1.352
    360.0 2.515
    360.0 1.961
    360.0 0.417
    360.0 360.0
    360.0 0.661
    360.0 15.104
    360.0 2.425
    360.0 0.347
    360.0 6.284
    360.0 17.871
    360.0 6.293
    360.0 1.745
    360.0 3.194
    360.0 2.433
    360.0 0.81
    360.0 2.501
    360.0 0.804
    360.0 3.83
    360.0 0.941
    360.0 360.0
    360.0 0.382
    360.0 5.856
    360.0 0.803
    360.0 2.397
    360.0 0.831
    360.0 12.771
    360.0 2.769
    360.0 360.0
    360.0 3.352
    360.0 0.958
    360.0 2.594
    360.0 1.038
    360.0 0.864
    360.0 0.763
    360.0 2.004
    360.0 1.021
    360.0 3.219
    360.0 0.41
    360.0 0.915
    360.0 4.965
    360.0 85.005
    360.0 1.934
    360.0 0.384
    360.0 0.376
    360.0 0.619
    360.0 0.158
    360.0 13.665
    360.0 1.326
    360.0 2.544
    360.0 2.476
    360.0 4.939
    360.0 2.479
    360.0 5.176
    360.0 1.982
    360.0 3.912
    360.0 0.411
    360.0 0.484
    360.0 0.626
    360.0 1.737
    360.0 0.585
    360.0 0.311
    360.0 1.792
    360.0 360.0
    360.0 6.55
    360.0 8.221
    360.0 2.489
    360.0 5.094
    360.0 1.013
    360.0 3.213
    360.0 0.943
    360.0 0.672
    360.0 8.649
    360.0 360.0
    360.0 0.623
    360.0 0.806
    360.0 0.728
    360.0 4.345
    360.0 0.756
    360.0 0.548
    360.0 0.852
    360.0 4.239
    360.0 2.99
    360.0 3.727
    360.0 0.942
    360.0 7.873
    360.0 4.579
    360.0 0.929
    360.0 10.015
    360.0 360.0
    360.0 5.948
    360.0 0.797
    360.0 2.946
    360.0 2.887
    360.0 6.586
    360.0 0.286
    360.0 2.948
    360.0 0.959
    360.0 11.983
    360.0 26.642
    360.0 360.0
    360.0 3.1
    360.0 0.596
    360.0 0.315
    360.0 360.0
    360.0 0.864
    360.0 9.413
    360.0 11.995
    360.0 2.809
    360.0 2.932
    360.0 0.468
    360.0 4.066
    360.0 5.053
    360.0 6.014
    360.0 3.087
    360.0 3.102
    360.0 2.959
    360.0 0.591
    360.0 0.542
    360.0 0.873
    360.0 6.535
    360.0 0.697
    360.0 0.96
    360.0 0.453
    360.0 3.3
    360.0 0.734
    360.0 1.053
    360.0 4.236
    360.0 0.705
    360.0 360.0
    360.0 2.796
    360.0 2.542
    360.0 0.596
    360.0 360.0
    360.0 5.6
    360.0 3.823
    360.0 0.779
    360.0 1.681
    360.0 0.823
    360.0 2.33
    360.0 0.298
    360.0 2.48
    360.0 107.295
    360.0 0.61
    360.0 360.0
    360.0 25.952
    360.0 5.283
    360.0 0.335
    360.0 0.885
    360.0 0.772
    360.0 88.789
    360.0 1.239
    360.0 1.485
    360.0 360.0
    360.0 0.458
    360.0 3.269
    360.0 12.18
    360.0 4.003
    360.0 0.634
    360.0 2.883
    360.0 3.531
    360.0 1.45
    360.0 0.776
    360.0 0.753
    360.0 1.357
    360.0 10.732
    360.0 41.262
    360.0 0.898
    360.0 0.427
    360.0 0.753
    360.0 2.619
    360.0 0.865
    360.0 5.955
    360.0 4.172
    360.0 0.578
    360.0 2.04
    360.0 3.694
    360.0 0.966
    360.0 2.878
    360.0 2.397
    360.0 11.411
    360.0 2.578
    360.0 0.637
    360.0 5.415
    360.0 5.569
    360.0 41.316
    360.0 12.844
    360.0 1.946
    360.0 5.405
    360.0 2.957
    360.0 4.068
    360.0 3.892
    360.0 2.843
    360.0 360.0
    360.0 0.39
    360.0 360.0
    360.0 316.219
    360.0 0.697
    360.0 0.41
    360.0 5.383
    360.0 0.735
    360.0 19.327
    360.0 1.907
    360.0 2.392
    360.0 0.997
    360.0 22.458
    360.0 2.735
    360.0 0.627
    360.0 1.788
    360.0 4.98
    360.0 360.0
    360.0 3.649
    360.0 0.647
    360.0 3.323
    360.0 360.0
    360.0 15.742
    360.0 1.009
    360.0 4.513
    360.0 360.0
    360.0 3.749
    360.0 0.913
    360.0 3.732
    360.0 1.277
    360.0 2.733
    360.0 0.231
    360.0 6.067
    360.0 1.62
    360.0 3.21
    360.0 0.634
    360.0 3.186
    360.0 3.523
    360.0 3.188
    360.0 3.249
    360.0 2.51
    360.0 0.948
    360.0 2.861
    360.0 0.779
    360.0 8.012
    360.0 1.379
    360.0 13.893
    360.0 360.0
    360.0 1.049
    360.0 360.0
    360.0 31.046
    360.0 50.038
    360.0 11.105
    360.0 9.42
    360.0 6.494
    360.0 3.176
    360.0 4.716
    360.0 1.702
    360.0 0.784
    360.0 0.57
    360.0 13.814
    360.0 1.396
    360.0 360.0
    360.0 0.36
    360.0 1.091
    360.0 4.317
    360.0 0.311
    360.0 4.01
    360.0 4.276
    360.0 3.451
    360.0 10.202
    360.0 82.71
    360.0 4.523
    360.0 3.967
    360.0 7.322
    360.0 360.0
    360.0 5.011
    360.0 3.89
    360.0 6.697
    360.0 1.247
    360.0 4.94
    360.0 4.183
    360.0 2.498
    360.0 3.701
    360.0 2.316
    360.0 2.227
    360.0 0.364
    360.0 360.0
    360.0 1.126
    360.0 3.399
    360.0 4.8
    360.0 0.784
    360.0 2.665
    360.0 0.991
    360.0 0.885
    360.0 3.745
    360.0 1.206
    360.0 0.359
    360.0 0.735
    360.0 0.666
    360.0 0.38
    360.0 0.707
    360.0 0.368
    360.0 1.094
    360.0 1.889
    360.0 0.78
    360.0 1.06
    360.0 0.99
    360.0 34.845
    360.0 0.882
    360.0 35.417
    360.0 0.831
    360.0 0.846
    360.0 3.869
    360.0 1.246
    360.0 0.945
    360.0 3.175
    360.0 360.0
    360.0 0.817
    360.0 360.0
    360.0 5.645
    360.0 14.427
    360.0 0.935
    360.0 0.664
    360.0 3.33
    360.0 360.0
    360.0 0.369
    360.0 0.516
    360.0 2.494
    360.0 3.394
    360.0 3.484
    360.0 40.768
    360.0 360.0
    360.0 2.755
    360.0 0.395
    360.0 4.112
    360.0 2.446
    360.0 1.868
    360.0 360.0
    360.0 4.373
    360.0 21.001
    360.0 1.168
    360.0 279.696
    360.0 0.712
    360.0 0.575
    360.0 2.837
    360.0 5.177
    360.0 0.637
    360.0 2.227
    360.0 0.579
    360.0 8.877
    360.0 1.717
    360.0 0.969
    360.0 0.947
    360.0 3.259
    360.0 0.578
    360.0 0.896
    360.0 0.606
    360.0 0.552
    360.0 4.132
    360.0 9.805
    360.0 0.755
    360.0 2.695
    360.0 3.966
    360.0 0.687
    360.0 0.933
    360.0 0.672
    360.0 0.84
    360.0 1.338
    360.0 0.89
    360.0 2.337
    360.0 1.06
    360.0 2.545
    360.0 1.264
    360.0 0.951
    360.0 0.695
    360.0 4.049
    360.0 360.0
    360.0 360.0
    360.0 3.69
    360.0 0.665
    360.0 360.0
    360.0 0.319
    360.0 5.094
    360.0 0.973
    360.0 5.562
    360.0 2.438
    360.0 0.83
    360.0 2.512
    360.0 11.277
    360.0 360.0
    360.0 0.664
    360.0 3.039
    360.0 360.0
    360.0 0.708
    360.0 0.996
    360.0 3.983
    360.0 1.122
    360.0 0.951
    360.0 2.661
    360.0 2.027
    360.0 0.384
    360.0 360.0
    360.0 10.396
    360.0 0.697
    360.0 3.102
    360.0 4.132
    360.0 360.0
    360.0 1.707
    360.0 4.662
    360.0 4.491
    360.0 4.724
    360.0 0.698
    360.0 360.0
    360.0 360.0
    360.0 1.066
    360.0 0.913
    360.0 61.582
    360.0 0.729
    360.0 0.659
    360.0 4.852
    360.0 4.173
    360.0 4.515
    360.0 3.154
    360.0 0.655
    360.0 0.804
    360.0 1.604
    360.0 0.873
    360.0 0.306
    360.0 26.371
    360.0 0.848
    360.0 2.501
    360.0 0.344
    360.0 1.271
    360.0 360.0
    360.0 7.313
    360.0 0.781
    360.0 20.36
    360.0 0.748
    360.0 1.3
    360.0 360.0
    360.0 6.765
    360.0 5.12
    360.0 11.069
    360.0 0.36
    360.0 12.643
    360.0 1.508
    360.0 11.701
    360.0 2.843
    360.0 360.0
    360.0 6.532
    360.0 4.026
    360.0 1.751
    360.0 5.513
    360.0 2.753
    360.0 0.969
    360.0 13.453
    360.0 2.625
    360.0 3.359
    360.0 0.841
    360.0 3.21
    360.0 3.546
    360.0 4.115
    360.0 0.601
    360.0 3.314
    360.0 360.0
    360.0 2.619
    360.0 3.036
    360.0 1.955
    360.0 1.427
    360.0 29.536
    360.0 1.32
    360.0 0.676
    360.0 8.153
    360.0 3.542
    360.0 0.5
    360.0 12.419
    360.0 360.0
    360.0 0.796
    360.0 360.0
    360.0 0.767
    360.0 4.609
    360.0 0.575
    360.0 6.796
    360.0 7.907
    360.0 0.944
    360.0 0.37
    360.0 3.178
    360.0 1.193
    360.0 0.597
    360.0 6.002
    360.0 0.734
    360.0 25.191
    360.0 1.543
    360.0 0.422
    360.0 3.397
    360.0 3.734
    360.0 0.644
    360.0 3.793
    360.0 95.267
    360.0 360.0
    360.0 5.463
    360.0 0.544
    360.0 1.039
    360.0 1.059
    360.0 0.864
    360.0 61.671
    360.0 0.313
    360.0 1.952
    360.0 1.23
    360.0 6.028
    360.0 360.0
    360.0 13.077
    360.0 1.067
    360.0 2.798
    360.0 0.694
    360.0 5.84
    360.0 360.0
    360.0 3.643
    360.0 1.237
    360.0 10.138
    360.0 1.002
    360.0 0.641
    360.0 360.0
    360.0 360.0
    360.0 2.645
    360.0 2.801
    360.0 0.707
    360.0 4.733
    360.0 3.681
    360.0 2.738
    360.0 3.948
    360.0 0.849
    360.0 0.653
    360.0 9.414
    360.0 0.966
    360.0 2.938
    360.0 0.891
    360.0 1.076
    360.0 6.41
    360.0 5.475
    360.0 0.781
    360.0 0.645
    360.0 2.901
    360.0 2.543
    360.0 0.623
    360.0 0.713
    360.0 360.0
    360.0 0.875
    360.0 0.791
    360.0 360.0
    360.0 360.0
    360.0 2.951
    360.0 360.0
    360.0 0.939
    360.0 0.542
    360.0 2.738
    360.0 360.0
    360.0 1.988
    360.0 5.071
    360.0 2.741
    360.0 19.096
    360.0 3.485
    360.0 9.882
    360.0 7.224
    360.0 0.843
    360.0 0.423
    360.0 1.183
    360.0 19.4
    360.0 360.0
    360.0 3.363
    360.0 0.578
    360.0 51.557
    360.0 6.544
    360.0 360.0
    360.0 0.374
    360.0 0.366
    360.0 1.798
    360.0 56.533
    360.0 2.382
    360.0 6.324
    360.0 0.689
    360.0 0.876
    360.0 0.716
    360.0 2.519
    360.0 9.055
    360.0 2.813
    360.0 0.658
    360.0 5.56
    360.0 0.956
    360.0 4.099
    360.0 0.702
    360.0 0.56
    360.0 1.33
    360.0 2.675
    360.0 2.063
    360.0 3.654
    360.0 0.771
    360.0 0.36
    360.0 360.0
    360.0 0.784
    360.0 1.02
    360.0 1.596
    360.0 0.706
    360.0 0.413
    360.0 5.463
    360.0 3.069
    360.0 4.316
    360.0 5.06
    360.0 0.655
    360.0 0.796
    360.0 1.24
    360.0 0.643
    360.0 0.54
    360.0 1.02
    360.0 0.679
    360.0 0.651
    360.0 0.88
    360.0 4.154
    360.0 1.561
    360.0 2.945
    360.0 0.556
    360.0 27.463
    360.0 0.816
    360.0 3.122
    360.0 0.336
    360.0 0.891
    360.0 4.314
    360.0 7.307
    360.0 0.426
    360.0 11.743
    360.0 0.846
    360.0 2.182
    360.0 4.136
    360.0 0.595
    360.0 5.239
    360.0 0.493
    360.0 1.57
    360.0 2.463
    360.0 13.109
    360.0 0.386
    360.0 0.349
    360.0 0.851
    360.0 4.263
    360.0 0.373
    360.0 13.896
    360.0 3.111
    360.0 19.229
    360.0 11.647
    360.0 0.651
    360.0 5.024
    360.0 0.361
    360.0 360.0
    360.0 1.421
    360.0 5.087
    360.0 201.703
    360.0 9.874
    360.0 3.661
    360.0 1.016
    360.0 3.98
    360.0 4.694
    360.0 0.311
    360.0 0.909
    360.0 0.713
    360.0 0.387
    360.0 3.156
    360.0 0.554
    360.0 12.572
    360.0 2.193
    360.0 1.36
    360.0 0.714
    360.0 0.513
    360.0 2.228
    360.0 8.506
    360.0 1.8
    360.0 3.718
    360.0 4.799
    360.0 0.769
    360.0 360.0
    360.0 5.039
    360.0 6.451
    360.0 2.771
    360.0 4.019
    360.0 1.983
    360.0 1.84
    360.0 1.131
    360.0 0.789
    360.0 4.16
    360.0 25.624
    360.0 0.388
    360.0 2.465
    360.0 0.512
    360.0 3.863
    360.0 1.259
    360.0 0.659
    360.0 0.811
    360.0 0.766
    360.0 3.555
    360.0 4.773
    360.0 0.848
    360.0 0.361
    360.0 1.069
    360.0 0.665
    360.0 6.0
    360.0 360.0
    360.0 5.569
    360.0 4.737
    360.0 0.961
    360.0 3.944
    360.0 0.737
    360.0 2.248
    360.0 360.0
    360.0 0.531
    360.0 3.291
    360.0 0.626
    360.0 0.558
    360.0 1.676
    360.0 0.928
    360.0 360.0
    360.0 0.378
    360.0 155.018
    360.0 3.556
    360.0 0.689
    360.0 0.545
    360.0 10.717
    360.0 1.009
    360.0 2.597
    360.0 6.777
    360.0 2.632
    360.0 0.906
    360.0 0.361
    360.0 4.655
    360.0 2.12
    360.0 1.016
    360.0 4.842
    360.0 0.709
    360.0 0.655
    360.0 9.278
    360.0 360.0
    360.0 4.81
    360.0 0.613
    360.0 0.384
    360.0 360.0
    360.0 3.319
    360.0 0.861
    360.0 123.262
    360.0 0.801
    360.0 1.554
    360.0 3.401
    360.0 0.371
    360.0 0.869
    360.0 0.364
    360.0 3.236
    360.0 0.559
    360.0 3.826
    360.0 3.001
    360.0 0.841
    360.0 6.031
    360.0 1.036
    360.0 2.594
    360.0 25.036
    360.0 4.365
    360.0 0.58
    360.0 0.313
    360.0 0.645
    360.0 7.672
    360.0 3.1
    360.0 0.8
    360.0 0.75
    360.0 2.837
    360.0 360.0
    360.0 15.397
    360.0 0.71
    360.0 11.448
    360.0 19.241
    360.0 13.414
    360.0 13.267
    360.0 4.667
    360.0 0.953
    360.0 360.0
    360.0 5.51
    360.0 2.632
    360.0 0.905
    360.0 3.595
    360.0 4.104
    360.0 0.962
    360.0 0.421
    360.0 0.601
    360.0 0.891
    360.0 2.875
    360.0 0.34
    360.0 360.0
    360.0 3.22
    360.0 0.908
    360.0 0.863
    360.0 2.623
    360.0 0.52
    360.0 3.979
    360.0 4.485
    360.0 0.797
    360.0 0.896
    360.0 1.866
    360.0 0.76
    360.0 0.385
    360.0 0.731
    360.0 3.664
    360.0 0.371
    360.0 5.138
    360.0 0.757
    360.0 2.059
    360.0 360.0
    360.0 6.324
    360.0 1.447
    360.0 360.0
    360.0 0.171
    360.0 0.503
    360.0 0.869
    360.0 41.008
    360.0 1.19
    360.0 2.307
    360.0 6.261
    360.0 2.534
    360.0 0.876
    360.0 2.49
    360.0 0.414
    360.0 4.617
    360.0 5.508
    360.0 9.991
    360.0 1.09
    360.0 1.623
    360.0 0.876
    360.0 360.0
    360.0 0.588
    360.0 1.318
    360.0 5.217
    360.0 0.857
    360.0 0.744
    360.0 12.314
    360.0 0.994
    360.0 0.964
    360.0 0.66
    360.0 0.86
    360.0 0.81
    360.0 3.26
    360.0 0.856
    360.0 20.762
    360.0 0.628
    360.0 360.0
    360.0 1.134
    360.0 0.843
    360.0 3.604
    360.0 2.535
    360.0 2.706
    360.0 1.798
    360.0 4.536
    360.0 5.113
    360.0 0.289
    360.0 11.137
    360.0 5.032
    360.0 0.436
    360.0 11.144
    360.0 2.509
    360.0 30.797
    360.0 0.943
    360.0 360.0
    360.0 3.916
    360.0 3.146
    360.0 7.72
    360.0 360.0
    360.0 3.966
    360.0 0.604
    360.0 360.0
    360.0 360.0
    360.0 0.816
    360.0 1.809
    360.0 360.0
    360.0 18.111
    360.0 7.035
    360.0 0.984
    360.0 4.193
    360.0 4.065
    360.0 0.503
    360.0 5.613
    360.0 20.205
    360.0 0.708
    360.0 4.457
    360.0 13.407
    360.0 0.837
    360.0 0.56
    360.0 6.979
    360.0 6.243
    360.0 0.721
    360.0 0.672
    360.0 0.279
    360.0 360.0
    360.0 9.784
    360.0 2.112
    360.0 360.0
    360.0 3.548
    360.0 0.793
    360.0 5.186
    360.0 5.37
    360.0 0.406
    360.0 1.858
    360.0 33.142
    360.0 0.605
    360.0 2.936
    360.0 0.668
    360.0 5.436
    360.0 3.701
    360.0 1.511
    360.0 4.475
    360.0 0.823
    360.0 0.619
    360.0 3.715
    360.0 0.949
    360.0 1.122
    360.0 3.71
    360.0 3.148
    360.0 3.858
    360.0 1.203
    360.0 0.484
    360.0 5.095
    360.0 0.777
    360.0 12.196
    360.0 1.15
    360.0 0.678
    360.0 0.57
    360.0 0.579
    360.0 1.289
    360.0 360.0
    360.0 5.187
    360.0 0.843
    360.0 10.226
    360.0 4.981
    360.0 1.242
    360.0 2.494
    360.0 2.229
    360.0 0.385
    360.0 19.832
    360.0 6.044
    360.0 2.114
    360.0 0.71
    360.0 2.562
    360.0 0.634
    360.0 0.578
    360.0 2.541
    360.0 0.868
    360.0 1.318
    360.0 0.669
    360.0 0.552
    360.0 1.025
    360.0 17.627
    360.0 12.67
    360.0 2.523
    360.0 0.5
    360.0 0.613
    360.0 2.554
    360.0 2.182
    360.0 7.245
    360.0 360.0
    360.0 4.394
    360.0 0.693
    360.0 0.847
    360.0 19.996
    360.0 4.224
    360.0 360.0
    360.0 0.344
    360.0 8.401
    360.0 0.241
    360.0 1.018
    360.0 1.129
    360.0 14.013
    360.0 0.867
    360.0 4.625
    360.0 6.975
    360.0 2.376
    360.0 4.674
    360.0 2.301
    360.0 4.957
    360.0 2.727
    360.0 0.675
    360.0 1.0
    360.0 4.952
    360.0 3.471
    360.0 1.068
    360.0 0.909
    360.0 0.962
    360.0 7.372
    360.0 0.651
    360.0 0.737
    360.0 2.892
    360.0 3.154
    360.0 0.839
    360.0 0.367
    360.0 3.902
    360.0 3.96
    360.0 2.443
    360.0 4.959
    360.0 3.714
    360.0 2.351
    360.0 4.193
    360.0 0.469
    360.0 2.852
    360.0 0.852
    360.0 3.853
    360.0 3.344
    360.0 2.85
    360.0 0.907
    360.0 0.852
    360.0 1.558
    360.0 5.517
    360.0 0.698
    360.0 3.481
    360.0 4.271
    360.0 6.942
    360.0 4.867
    360.0 360.0
    360.0 1.025
    360.0 6.781
    360.0 360.0
    360.0 0.899
    360.0 3.965
    360.0 1.824
    360.0 0.355
    360.0 2.876
    360.0 7.158
    360.0 6.751
    360.0 3.994
    360.0 2.684
    360.0 3.447
    360.0 2.575
    360.0 4.975
    360.0 360.0
    360.0 0.469
    360.0 0.765
    360.0 3.614
    360.0 4.098
    360.0 0.999
    360.0 10.737
    360.0 360.0
    360.0 0.937
    360.0 27.273
    360.0 3.109
    360.0 0.485
    360.0 6.716
    360.0 0.358
    360.0 360.0
    360.0 1.284
    360.0 1.637
    360.0 1.846
    360.0 4.992
    360.0 1.22
    360.0 7.239
    360.0 0.875
    360.0 0.401
    360.0 8.879
    360.0 3.092
    360.0 1.218
    360.0 0.951
    360.0 0.657
    360.0 8.706
    360.0 0.714
    360.0 0.42
    360.0 6.711
    360.0 0.679
    360.0 4.912
    360.0 0.678
    360.0 0.635
    360.0 0.175
    360.0 8.501
    360.0 0.285
    360.0 3.232
    360.0 3.366
    360.0 13.931
    360.0 0.4
    360.0 0.926
    360.0 0.79
    360.0 0.398
    360.0 360.0
    360.0 1.411
    360.0 360.0
    360.0 0.666
    360.0 4.112
    360.0 2.919
    360.0 9.415
    360.0 360.0
    360.0 0.519
    360.0 0.261
    360.0 5.575
    360.0 360.0
    360.0 360.0
    360.0 0.319
    360.0 3.127
    360.0 360.0
    360.0 0.596
    360.0 0.658
    360.0 0.636
    360.0 0.643
    360.0 15.913
    360.0 360.0
    360.0 360.0
    360.0 2.878
    360.0 0.64
    360.0 3.0
    360.0 0.587
    360.0 360.0
    360.0 3.007
    360.0 3.596
    360.0 5.603
    360.0 6.817
    360.0 0.681
    360.0 0.697
    360.0 2.275
    360.0 1.348
    360.0 4.223
    360.0 3.377
    360.0 0.512
    360.0 0.834
    360.0 0.233
    360.0 0.868
    360.0 2.999
    360.0 0.738
    360.0 7.497
    360.0 0.379
    360.0 0.76
    360.0 1.779
    360.0 0.597
    360.0 360.0
    360.0 0.587
    360.0 0.409
    360.0 0.94
    360.0 360.0
    360.0 1.297
    360.0 3.78
    360.0 2.575
    360.0 14.803
    360.0 2.839
    360.0 0.868
    360.0 4.248
    360.0 3.476
    360.0 2.27
    360.0 3.18
    360.0 2.452
    360.0 360.0
    360.0 0.391
    360.0 0.32
    360.0 94.839
    360.0 4.352
    360.0 0.489
    360.0 0.685
    360.0 4.227
    360.0 2.977
    360.0 1.05
    360.0 3.943
    360.0 0.902
    360.0 10.092
    360.0 2.935
    360.0 38.951
    360.0 360.0
    360.0 2.442
    360.0 0.852
    360.0 360.0
    360.0 6.284
    360.0 0.798
    360.0 4.67
    360.0 1.353
    360.0 0.178
    360.0 0.661
    360.0 360.0
    360.0 0.618
    360.0 3.919
    360.0 4.285
    360.0 360.0
    360.0 0.841
    360.0 0.968
    360.0 31.894
    360.0 15.524
    360.0 1.062
    360.0 2.977
    360.0 0.665
    360.0 3.885
    360.0 4.61
    360.0 360.0
    360.0 2.819
    360.0 3.773
    360.0 2.793
    360.0 0.508
    360.0 0.537
    360.0 0.382
    360.0 3.109
    360.0 3.432
    360.0 3.172
    360.0 1.117
    360.0 4.796
    360.0 4.708
    360.0 0.672
    360.0 5.319
    360.0 0.761
    360.0 1.067
    360.0 2.903
    360.0 0.512
    360.0 0.835
    360.0 3.13
    360.0 1.174
    360.0 1.618
    360.0 0.74
    360.0 0.795
    360.0 0.563
    360.0 3.777
    360.0 360.0
    360.0 0.591
    360.0 0.897
    360.0 4.09
    360.0 4.699
    360.0 1.069
    360.0 0.519
    360.0 1.331
    360.0 360.0
    360.0 2.142
    360.0 0.379
    360.0 360.0
    360.0 4.14
    360.0 0.764
    360.0 360.0
    360.0 0.172
    360.0 4.21
    360.0 5.395
    360.0 5.582
    360.0 2.886
    360.0 0.412
    360.0 6.672
    360.0 0.808
    360.0 0.175
    360.0 0.662
    360.0 32.593
    360.0 0.942
    360.0 0.454
    360.0 0.407
    360.0 0.665
    360.0 0.423
    360.0 0.398
    360.0 3.324
    360.0 0.926
    360.0 3.004
    360.0 0.406
    360.0 3.435
    360.0 0.985
    360.0 0.29
    360.0 0.521
    360.0 0.392
    360.0 2.317
    360.0 2.847
    360.0 9.562
    360.0 3.165
    360.0 0.857
    360.0 0.35
    360.0 360.0
    360.0 0.692
    360.0 0.889
    360.0 360.0
    360.0 5.585
    360.0 5.133
    360.0 2.694
    360.0 0.297
    360.0 1.312
    360.0 0.637
    360.0 360.0
    360.0 0.423
    360.0 0.648
    360.0 0.784
    360.0 0.913
    360.0 2.659
    360.0 0.711
    360.0 72.721
    360.0 3.778
    360.0 1.238
    360.0 12.902
    360.0 4.188
    360.0 14.219
    360.0 2.159
    360.0 0.623
    360.0 3.59
    360.0 0.993
    360.0 0.567
    360.0 0.373
    360.0 0.399
    360.0 1.819
    360.0 0.957
    360.0 1.538
    360.0 6.174
    360.0 3.718
    360.0 6.904
    360.0 0.697
    360.0 5.284
    360.0 4.026
    360.0 6.546
    360.0 5.727
    360.0 0.4
    360.0 3.848
    360.0 3.012
    360.0 5.575
    360.0 360.0
    360.0 0.538
    360.0 0.926
    360.0 360.0
    360.0 4.457
    360.0 2.469
    360.0 0.639
    360.0 4.532
    360.0 10.854
    360.0 0.832
    360.0 0.514
    360.0 19.991
    360.0 4.97
    360.0 3.488
    360.0 3.817
    360.0 0.907
    360.0 61.273
    360.0 0.381
    360.0 3.482
    360.0 13.245
    360.0 0.824
    360.0 5.411
    360.0 5.4
    360.0 2.804
    360.0 9.455
    360.0 5.246
    360.0 1.347
    360.0 1.728
    360.0 3.22
    360.0 5.997
    360.0 0.9
    360.0 5.8
    360.0 3.673
    360.0 5.056
    360.0 1.021
    360.0 7.284
    360.0 0.696
    360.0 2.696
    360.0 3.587
    360.0 0.984
    360.0 3.288
    360.0 0.988
    360.0 2.507
    360.0 1.971
    360.0 0.334
    360.0 3.393
    360.0 1.681
    360.0 1.027
    360.0 5.718
    360.0 0.283
    360.0 0.838
    360.0 0.416
    360.0 1.367
    360.0 13.898
    360.0 0.402
    360.0 3.011
    360.0 0.848
    360.0 3.556
    360.0 2.261
    360.0 0.836
    360.0 0.959
    360.0 2.708
    360.0 3.552
    360.0 5.098
    360.0 9.648
    360.0 0.791
    360.0 0.671
    360.0 360.0
    360.0 6.221
    360.0 1.384
    360.0 1.199
    360.0 0.748
    360.0 3.085
    360.0 360.0
    360.0 1.066
    360.0 34.15
    360.0 1.11
    360.0 0.772
    360.0 1.733
    360.0 7.34
    360.0 3.734
    360.0 5.997
    360.0 0.767
    360.0 3.875
    360.0 3.77
    360.0 0.303
    360.0 0.453
    360.0 360.0
    360.0 4.921
    360.0 0.973
    360.0 1.011
    360.0 0.95
    360.0 360.0
    360.0 2.123
    360.0 24.322
    360.0 4.25
    360.0 48.652
    360.0 0.66
    360.0 3.277
    360.0 10.743
    360.0 2.488
    360.0 1.063
    360.0 0.37
    360.0 360.0
    360.0 3.253
    360.0 0.765
    360.0 3.281
    360.0 360.0
    360.0 0.604
    360.0 1.999
    360.0 2.965
    360.0 2.705
    360.0 4.261
    360.0 360.0
    360.0 360.0
    360.0 0.528
    360.0 3.166
    360.0 1.37
    360.0 0.275
    360.0 40.679
    360.0 0.489
    360.0 0.879
    360.0 360.0
    360.0 360.0
    360.0 17.171
    360.0 1.032
    360.0 0.386
    360.0 2.333
    360.0 1.429
    360.0 2.487
    360.0 2.982
    360.0 4.447
    360.0 1.316
    360.0 4.746
    360.0 0.847
    360.0 7.823
    360.0 4.168
    360.0 0.517
    360.0 0.99
    360.0 17.789
    360.0 1.051
    360.0 0.877
    360.0 0.374
    360.0 0.525
    360.0 360.0
    360.0 0.781
    360.0 2.644
    360.0 7.571
    360.0 360.0
    360.0 0.737
    360.0 14.197
    360.0 1.795
    360.0 2.985
    360.0 1.405
    360.0 4.784
    360.0 0.811
    360.0 1.491
    360.0 3.291
    360.0 3.774
    360.0 0.528
    360.0 0.598
    360.0 0.658
    360.0 0.566
    360.0 29.267
    360.0 0.387
    360.0 0.605
    360.0 4.654
    360.0 3.762
    360.0 3.126
    360.0 0.646
    360.0 5.509
    360.0 1.256
    360.0 15.241
    360.0 0.634
    360.0 2.144
    360.0 0.658
    360.0 0.651
    360.0 0.554
    360.0 0.652
    360.0 2.391
    360.0 3.109
    360.0 4.139
    360.0 5.167
    360.0 360.0
    360.0 18.851
    360.0 75.056
    360.0 0.398
    360.0 3.71
    360.0 6.995
    360.0 0.822
    360.0 360.0
    360.0 1.01
    360.0 0.369
    360.0 4.096
    360.0 360.0
    360.0 0.653
    360.0 2.352
    360.0 4.345
    360.0 2.735
    360.0 1.207
    360.0 360.0
    360.0 4.692
    360.0 1.313
    360.0 5.822
    360.0 1.978
    360.0 1.227
    360.0 0.882
    360.0 0.806
    360.0 0.865
    360.0 0.399
    360.0 1.372
    360.0 4.683
    360.0 2.816
    360.0 1.33
    360.0 1.915
    360.0 3.62
    360.0 360.0
    360.0 8.745
    360.0 5.389
    360.0 0.886
    360.0 3.583
    360.0 0.685
    360.0 4.099
    360.0 5.354
    360.0 1.006
    360.0 0.76
    360.0 326.65
    360.0 2.92
    360.0 0.809
    360.0 2.449
    360.0 0.929
    360.0 1.472
    360.0 0.343
    360.0 0.767
    360.0 0.744
    360.0 3.032
    360.0 0.695
    360.0 360.0
    360.0 0.67
    360.0 0.649
    360.0 360.0
    360.0 4.844
    360.0 0.632
    360.0 360.0
    360.0 0.762
    360.0 2.938
    360.0 360.0
    360.0 1.006
    360.0 1.296
    360.0 0.362
    360.0 0.821
    360.0 360.0
    360.0 1.146
    360.0 0.644
    360.0 0.817
    360.0 11.077
    360.0 2.786
    360.0 3.729
    360.0 15.517
    360.0 5.525
    360.0 0.279
    360.0 3.991
    360.0 0.869
    360.0 2.873
    360.0 6.034
    360.0 4.104
    360.0 3.738
    360.0 1.802
    360.0 360.0
    360.0 10.854
    360.0 2.577
    360.0 1.47
    360.0 0.673
    360.0 0.876
    360.0 21.391
    360.0 64.576
    360.0 0.764
    360.0 0.786
    360.0 0.358
    360.0 8.393
    360.0 19.793
    360.0 2.473
    360.0 0.583
    360.0 0.726
    360.0 0.834
    360.0 0.709
    360.0 0.739
    360.0 3.884
    360.0 4.281
    360.0 0.374
    360.0 0.513
    360.0 0.578
    360.0 2.45
    360.0 3.147
    360.0 4.405
    360.0 3.602
    360.0 0.749
    360.0 3.865
    360.0 0.411
    360.0 1.078
    360.0 0.849
    360.0 2.591
    360.0 1.085
    360.0 360.0
    360.0 0.867
    360.0 1.099
    360.0 2.249
    360.0 360.0
    360.0 360.0
    360.0 1.255
    360.0 3.205
    360.0 5.778
    360.0 0.742
    360.0 0.335
    360.0 2.404
    360.0 0.702
    360.0 4.295
    360.0 360.0
    360.0 0.662
    360.0 1.819
    360.0 0.994
    360.0 0.603
    360.0 2.688
    360.0 309.011
    360.0 0.398
    360.0 0.402
    360.0 3.244
    360.0 6.644
    360.0 0.622
    360.0 0.516
    360.0 3.234
    360.0 2.669
    360.0 4.852
    360.0 3.378
    360.0 6.563
    360.0 1.516
    360.0 0.347
    360.0 17.618
    360.0 6.925
    360.0 0.726
    360.0 360.0
    360.0 9.355
    360.0 2.507
    360.0 3.755
    360.0 360.0
    360.0 1.097
    360.0 0.385
    360.0 0.944
    360.0 1.045
    360.0 1.942
    360.0 360.0
    360.0 0.914
    360.0 0.781
    360.0 2.388
    360.0 0.724
    360.0 0.558
    360.0 0.363
    360.0 2.903
    360.0 0.915
    360.0 3.427
    360.0 4.105
    360.0 0.493
    360.0 3.77
    360.0 3.933
    360.0 4.801
    360.0 0.362
    360.0 0.391
    360.0 0.734
    360.0 360.0
    360.0 0.888
    360.0 2.342
    360.0 2.533
    360.0 1.0
    360.0 0.712
    360.0 3.851
    360.0 0.645
    360.0 6.093
    360.0 5.102
    360.0 1.906
    360.0 0.404
    360.0 2.649
    360.0 8.581
    360.0 4.399
    360.0 29.254
    360.0 2.997
    360.0 3.161
    360.0 3.041
    360.0 0.431
    360.0 5.38
    360.0 0.867
    360.0 4.774
    360.0 360.0
    360.0 22.175
    360.0 0.618
    360.0 7.08
    360.0 0.625
    360.0 0.848
    360.0 360.0
    360.0 5.28
    360.0 5.905
    360.0 0.657
    360.0 3.458
    360.0 4.312
    360.0 4.223
    360.0 0.599
    360.0 1.265
    360.0 1.22
    360.0 0.455
    360.0 0.623
    360.0 2.786
    360.0 0.66
    360.0 4.535
    360.0 360.0
    360.0 0.578
    360.0 360.0
    360.0 0.575
    360.0 0.611
    360.0 3.67
    360.0 0.996
    360.0 2.648
    360.0 2.25
    360.0 0.538
    360.0 4.261
    360.0 4.18
    360.0 3.385
    360.0 1.089
    360.0 0.451
    360.0 0.663
    360.0 2.989
    360.0 4.877
    360.0 0.424
    360.0 360.0
    360.0 0.865
    360.0 4.017
    360.0 0.386
    360.0 1.989
    360.0 12.949
    360.0 0.384
    360.0 0.451
    360.0 13.835
    360.0 0.733
    360.0 360.0
    360.0 360.0
    360.0 2.693
    360.0 360.0
    360.0 0.457
    360.0 0.792
    360.0 0.414
    360.0 3.811
    360.0 2.255
    360.0 0.706
    360.0 5.119
    360.0 13.222
    360.0 2.218
    360.0 1.012
    360.0 3.008
    360.0 360.0
    360.0 9.38
    360.0 3.597
    360.0 0.619
    360.0 360.0
    360.0 8.21
    360.0 2.299
    360.0 360.0
    360.0 6.397
    360.0 7.043
    360.0 3.123
    360.0 2.67
    360.0 0.485
    360.0 0.76
    360.0 0.942
    360.0 0.767
    360.0 0.872
    360.0 360.0
    360.0 6.892
    360.0 18.793
    360.0 7.885
    360.0 0.561
    360.0 6.864
    360.0 0.538
    360.0 1.023
    360.0 9.894
    360.0 0.431
    360.0 27.826
    360.0 2.82
    360.0 0.843
    360.0 360.0
    360.0 0.64
    360.0 0.614
    360.0 360.0
    360.0 4.67
    360.0 16.309
    360.0 0.666
    360.0 10.914
    360.0 360.0
    360.0 2.518
    360.0 0.851
    360.0 0.987
    360.0 2.527
    360.0 16.407
    360.0 36.814
    360.0 0.519
    360.0 0.67
    360.0 1.893
    360.0 0.464
    360.0 3.25
    360.0 360.0
    360.0 3.044
    360.0 360.0
    360.0 0.9
    360.0 4.676
    360.0 360.0
    360.0 0.844
    360.0 360.0
    360.0 75.935
    360.0 12.788
    360.0 0.791
    360.0 0.654
    360.0 0.727
    360.0 5.611
    360.0 34.237
    360.0 1.19
    360.0 3.88
    360.0 1.009
    360.0 2.762
    360.0 1.135
    360.0 4.747
    360.0 0.358
    360.0 0.572
    360.0 0.373
    360.0 4.484
    360.0 2.908
    360.0 0.944
    360.0 3.207
    360.0 360.0
    360.0 8.51
    360.0 3.353
    360.0 1.866
    360.0 0.547
    360.0 4.528
    360.0 0.9
    360.0 5.615
    360.0 0.364
    360.0 1.316
    360.0 0.408
    360.0 4.542
    360.0 2.209
    360.0 4.1
    360.0 6.861
    360.0 1.366
    360.0 3.025
    360.0 4.155
    360.0 4.195
    360.0 0.476
    360.0 1.273
    360.0 4.135
    360.0 0.308
    360.0 0.371
    360.0 5.266
    360.0 4.925
    360.0 1.536
    360.0 3.445
    360.0 5.268
    360.0 0.921
    360.0 7.763
    360.0 3.891
    360.0 0.654
    360.0 360.0
    360.0 0.302
    360.0 0.385
    360.0 0.655
    360.0 2.726
    360.0 360.0
    360.0 2.556
    360.0 4.494
    360.0 0.376
    360.0 4.72
    360.0 0.711
    360.0 0.68
    360.0 4.738
    360.0 0.327
    360.0 2.896
    360.0 0.357
    360.0 3.129
    360.0 3.267
    360.0 6.764
    360.0 4.394
    360.0 1.082
    360.0 2.915
    360.0 4.198
    360.0 2.313
    360.0 0.406
    360.0 2.929
    360.0 4.98
    360.0 6.804
    360.0 2.65
    360.0 2.85
    360.0 4.058
    360.0 0.59
    360.0 2.718
    360.0 2.017
    360.0 4.174
    360.0 0.775
    360.0 360.0
    360.0 9.964
    360.0 0.542
    360.0 6.581
    360.0 3.047
    360.0 2.945
    360.0 0.547
    360.0 0.983
    360.0 2.508
    360.0 0.687
    360.0 3.998
    360.0 3.277
    360.0 2.061
    360.0 6.48
    360.0 3.253
    360.0 0.847
    360.0 360.0
    360.0 0.787
    360.0 3.492
    360.0 1.15
    360.0 0.551
    360.0 5.584
    360.0 3.324
    360.0 0.444
    360.0 0.176
    360.0 0.77
    360.0 0.461
    360.0 0.313
    360.0 3.097
    360.0 5.593
    360.0 4.067
    360.0 1.265
    360.0 360.0
    360.0 2.555
    360.0 2.216
    360.0 15.975
    360.0 5.417
    360.0 4.562
    360.0 3.132
    360.0 0.661
    360.0 3.121
    360.0 5.826
    360.0 5.725
    360.0 2.078
    360.0 1.611
    360.0 1.396
    360.0 360.0
    360.0 1.266
    360.0 0.692
    360.0 3.012
    360.0 0.357
    360.0 1.668
    360.0 360.0
    360.0 3.982
    360.0 360.0
    360.0 3.199
    360.0 4.717
    360.0 360.0
    360.0 2.991
    360.0 1.128
    360.0 2.566
    360.0 0.28
    360.0 1.87
    360.0 0.229
    360.0 0.747
    360.0 0.434
    360.0 7.153
    360.0 1.505
    360.0 0.684
    360.0 360.0
    360.0 0.865
    360.0 5.156
    360.0 0.376
    360.0 2.086
    360.0 0.336
    360.0 0.88
    360.0 0.649
    360.0 360.0
    360.0 4.753
    360.0 0.704
    360.0 0.395
    360.0 2.19
    360.0 5.449
    360.0 292.914
    360.0 0.669
    360.0 0.371
    360.0 4.073
    360.0 0.808
    360.0 0.644
    360.0 1.569
    360.0 1.0
    360.0 0.973
    360.0 0.442
    360.0 0.387
    360.0 3.859
    360.0 1.718
    360.0 13.489
    360.0 1.33
    360.0 0.912
    360.0 4.809
    360.0 0.37
    360.0 0.381
    360.0 6.89
    360.0 0.727
    360.0 0.355
    360.0 3.864
    360.0 12.823
    360.0 3.434
    360.0 7.738
    360.0 0.796
    360.0 0.853
    360.0 0.481
    360.0 0.708
    360.0 197.548
    360.0 2.747
    360.0 0.915
    360.0 2.034
    360.0 2.692
    360.0 4.791
    360.0 0.637
    360.0 4.574
    360.0 4.228
    360.0 0.49
    360.0 1.311
    360.0 3.536
    360.0 1.218
    360.0 0.718
    360.0 3.792
    360.0 1.212
    360.0 0.503
    360.0 0.846
    360.0 3.585
    360.0 26.907
    360.0 0.584
    360.0 1.009
    360.0 2.222
    360.0 0.717
    360.0 360.0
    360.0 3.208
    360.0 360.0
    360.0 5.095
    360.0 12.486
    360.0 360.0
    360.0 0.9
    360.0 0.687
    360.0 0.682
    360.0 0.396
    360.0 2.057
    360.0 5.921
    360.0 3.847
    360.0 1.042
    360.0 4.942
    360.0 1.206
    360.0 1.279
    360.0 4.602
    360.0 2.542
    360.0 0.55
    360.0 0.698
    360.0 1.097
    360.0 11.647
    360.0 2.684
    360.0 360.0
    360.0 6.214
    360.0 3.714
    360.0 0.835
    360.0 5.423
    360.0 4.004
    360.0 0.796
    360.0 0.525
    360.0 2.893
    360.0 2.694
    360.0 3.607
    360.0 2.654
    360.0 1.289
    360.0 3.651
    360.0 2.994
    360.0 5.784
    360.0 5.696
    360.0 360.0
    360.0 360.0
    360.0 10.852
    360.0 0.955
    360.0 3.247
    360.0 2.491
    360.0 2.375
    360.0 0.429
    360.0 1.048
    360.0 0.857
    360.0 0.665
    360.0 360.0
    360.0 4.158
    360.0 1.098
    360.0 0.818
    360.0 1.382
    360.0 0.382
    360.0 4.806
    360.0 4.715
    360.0 0.873
    360.0 0.678
    360.0 0.88
    360.0 0.934
    360.0 4.306
    360.0 3.404
    360.0 360.0
    360.0 3.566
    360.0 0.998
    360.0 0.94
    360.0 1.058
    360.0 2.042
    360.0 0.371
    360.0 0.661
    360.0 2.451
    360.0 2.591
    360.0 1.165
    360.0 0.958
    360.0 1.058
    360.0 0.824
    360.0 7.064
    360.0 0.993
    360.0 49.616
    360.0 1.181
    360.0 2.672
    360.0 7.407
    360.0 5.905
    360.0 6.536
    360.0 0.543
    360.0 0.531
    360.0 4.697
    360.0 0.513
    360.0 4.33
    360.0 2.562
    360.0 0.894
    360.0 1.528
    360.0 4.477
    360.0 3.673
    360.0 0.652
    360.0 0.605
    360.0 0.831
    360.0 6.031
    360.0 1.011
    360.0 4.22
    360.0 2.68
    360.0 0.621
    360.0 2.156
    360.0 5.75
    360.0 0.627
    360.0 1.551
    360.0 0.807
    360.0 1.097
    360.0 21.934
    360.0 2.074
    360.0 360.0
    360.0 2.947
    360.0 5.348
    360.0 7.276
    360.0 5.559
    360.0 0.816
    360.0 4.132
    360.0 360.0
    360.0 0.377
    360.0 0.444
    360.0 0.388
    360.0 360.0
    360.0 0.643
    360.0 0.835
    360.0 4.226
    360.0 3.668
    360.0 3.165
    360.0 360.0
    360.0 1.753
    360.0 0.737
    360.0 0.801
    360.0 3.297
    360.0 2.558
    360.0 2.918
    360.0 0.672
    360.0 3.02
    360.0 46.463
    360.0 0.615
    360.0 0.697
    360.0 0.893
    360.0 0.361
    360.0 0.805
    360.0 19.91
    360.0 4.286
    360.0 10.313
    360.0 1.299
    360.0 4.727
    360.0 0.484
    360.0 1.278
    360.0 1.386
    360.0 1.267
    360.0 0.795
    360.0 1.909
    360.0 360.0
    360.0 3.957
    360.0 4.536
    360.0 4.319
    360.0 0.913
    360.0 3.125
    360.0 3.743
    360.0 4.454
    360.0 0.567
    360.0 360.0
    360.0 1.119
    360.0 0.783
    360.0 0.82
    360.0 3.229
    360.0 4.202
    360.0 2.688
    360.0 6.619
    360.0 3.935
    360.0 9.482
    360.0 2.715
    360.0 360.0
    360.0 4.057
    360.0 0.708
    360.0 3.984
    360.0 0.369
    360.0 0.371
    360.0 1.855
    360.0 360.0
    360.0 140.523
    360.0 177.815
    360.0 14.058
    360.0 0.507
    360.0 117.687
    360.0 5.94
    360.0 5.319
    360.0 1.061
    360.0 93.083
    360.0 2.866
    360.0 0.579
    360.0 2.49
    360.0 1.917
    360.0 0.644
    360.0 0.382
    360.0 0.741
    360.0 51.008
    360.0 7.887
    360.0 5.282
    360.0 5.274
    360.0 360.0
    360.0 1.256
    360.0 2.796
    360.0 3.366
    360.0 360.0
    360.0 4.186
    360.0 0.564
    360.0 0.68
    360.0 360.0
    360.0 2.114
    360.0 3.095
    360.0 1.569
    360.0 1.295
    360.0 1.47
    360.0 36.734
    360.0 360.0
    360.0 0.408
    360.0 3.017
    360.0 0.637
    360.0 0.758
    360.0 0.618
    360.0 1.344
    360.0 1.183
    360.0 0.788
    360.0 0.474
    360.0 360.0
    360.0 0.368
    360.0 4.152
    360.0 0.708
    360.0 0.894
    360.0 1.059
    360.0 0.295
    360.0 360.0
    360.0 1.05
    360.0 4.593
    360.0 0.951
    360.0 4.35
    360.0 0.758
    360.0 0.806
    360.0 0.443
    360.0 2.87
    360.0 0.825
    360.0 5.374
    360.0 0.399
    360.0 4.32
    360.0 0.469
    360.0 0.509
    360.0 7.617
    360.0 0.855
    360.0 0.87
    360.0 22.377
    360.0 5.361
    360.0 0.638
    360.0 1.453
    360.0 0.621
    360.0 243.522
    360.0 1.773
    360.0 0.71
    360.0 1.091
    360.0 0.588
    360.0 8.471
    360.0 0.752
    360.0 0.684
    360.0 0.928
    360.0 4.163
    360.0 2.727
    360.0 0.62
    360.0 2.614
    360.0 6.197
    360.0 3.062
    360.0 1.975
    360.0 0.747
    360.0 10.581
    360.0 0.679
    360.0 360.0
    360.0 0.708
    360.0 0.804
    360.0 360.0
    360.0 360.0
    360.0 0.752
    360.0 2.768
    360.0 1.768
    360.0 3.445
    360.0 2.419
    360.0 1.091
    360.0 2.597
    360.0 360.0
    360.0 5.479
    360.0 360.0
    360.0 360.0
    360.0 0.64
    360.0 360.0
    360.0 2.207
    360.0 360.0
    360.0 0.606
    360.0 0.903
    360.0 5.813
    360.0 0.81
    360.0 13.059
    360.0 3.312
    360.0 7.148
    360.0 0.271
    360.0 360.0
    360.0 360.0
    360.0 6.197
    360.0 4.062
    360.0 4.667
    360.0 0.751
    360.0 0.61
    360.0 0.992
    360.0 1.279
    360.0 3.506
    360.0 0.677
    360.0 0.358
    360.0 6.015
    360.0 3.143
    360.0 4.929
    360.0 2.185
    360.0 0.657
    360.0 4.754
    360.0 34.748
    360.0 1.592
    360.0 0.564
    360.0 0.576
    360.0 0.9
    360.0 37.828
    360.0 14.842
    360.0 3.721
    360.0 2.534
    360.0 3.385
    360.0 3.735
    360.0 5.495
    360.0 2.931
    360.0 0.391
    360.0 4.343
    360.0 360.0
    360.0 4.443
    360.0 0.467
    360.0 1.203
    360.0 0.971
    360.0 4.267
    360.0 7.397
    360.0 0.402
    360.0 360.0
    360.0 2.215
    360.0 0.383
    360.0 0.87
    360.0 3.896
    360.0 0.459
    360.0 0.977
    360.0 2.968
    360.0 5.298
    360.0 0.392
    360.0 3.436
    360.0 0.679
    360.0 0.386
    360.0 360.0
    360.0 1.244
    360.0 0.389
    360.0 7.902
    360.0 2.276
    360.0 1.476
    360.0 3.093
    360.0 3.315
    360.0 0.873
    360.0 2.81
    360.0 1.048
    360.0 12.803
    360.0 3.265
    360.0 2.405
    360.0 46.596
    360.0 2.132
    360.0 0.917
    360.0 4.116
    360.0 0.827
    360.0 3.522
    360.0 11.97
    360.0 1.816
    360.0 6.632
    360.0 1.198
    360.0 5.077
    360.0 12.685
    360.0 4.952
    360.0 1.502
    360.0 2.674
    360.0 0.949
    360.0 7.246
    360.0 3.075
    360.0 1.028
    360.0 0.747
    360.0 1.081
    360.0 2.804
    360.0 0.925
    360.0 5.982
    360.0 3.708
    360.0 0.769
    360.0 0.496
    360.0 2.563
    360.0 4.649
    360.0 2.427
    360.0 1.736
    360.0 360.0
    360.0 9.442
    360.0 3.931
    360.0 0.51
    360.0 2.508
    360.0 1.796
    360.0 5.505
    360.0 0.314
    360.0 0.216
    360.0 4.568
    360.0 0.635
    360.0 0.583
    360.0 0.864
    360.0 0.354
    360.0 4.547
    360.0 0.769
    360.0 2.654
    360.0 0.539
    360.0 0.983
    360.0 4.972
    360.0 0.509
    360.0 7.005
    360.0 4.492
    360.0 19.074
    360.0 0.703
    360.0 16.845
    360.0 2.867
    360.0 27.329
    360.0 1.06
    360.0 0.358
    360.0 1.239
    360.0 3.414
    360.0 4.08
    360.0 0.369
    360.0 360.0
    360.0 0.868
    360.0 0.484
    360.0 0.622
    360.0 2.234
    360.0 3.246
    360.0 360.0
    360.0 0.466
    360.0 360.0
    360.0 0.701
    360.0 0.234
    360.0 360.0
    360.0 5.645
    360.0 2.022
    360.0 3.434
    360.0 317.938
    360.0 17.827
    360.0 0.621
    360.0 4.926
    360.0 3.62
    360.0 2.763
    360.0 3.656
    360.0 0.557
    360.0 3.639
    360.0 0.868
    360.0 2.563
    360.0 4.693
    360.0 7.832
    360.0 360.0
    360.0 2.065
    360.0 0.674
    360.0 360.0
    360.0 360.0
    360.0 0.667
    360.0 0.836
    360.0 1.546
    360.0 0.461
    360.0 1.513
    360.0 0.606
    360.0 1.09
    360.0 4.406
    360.0 0.902
    360.0 0.841
    360.0 0.196
    360.0 360.0
    360.0 1.782
    360.0 360.0
    360.0 3.692
    360.0 0.885
    360.0 0.759
    360.0 1.293
    360.0 1.721
    360.0 0.749
    360.0 5.374
    360.0 0.373
    360.0 0.923
    360.0 2.788
    360.0 0.957
    360.0 0.382
    360.0 0.315
    360.0 2.726
    360.0 7.961
    360.0 360.0
    360.0 2.577
    360.0 2.394
    360.0 0.418
    360.0 4.987
    360.0 0.651
    360.0 0.804
    360.0 3.6
    360.0 360.0
    360.0 3.091
    360.0 3.321
    360.0 0.68
    360.0 0.468
    360.0 0.961
    360.0 0.368
    360.0 0.673
    360.0 2.967
    360.0 0.724
    360.0 2.766
    360.0 2.455
    360.0 3.279
    360.0 10.079
    360.0 2.247
    360.0 1.96
    360.0 0.73
    360.0 4.151
    360.0 0.733
    360.0 360.0
    360.0 13.786
    360.0 0.671
    360.0 3.121
    360.0 11.478
    360.0 2.627
    360.0 1.1
    360.0 360.0
    360.0 3.515
    360.0 3.504
    360.0 4.799
    360.0 1.015
    360.0 0.553
    360.0 3.278
    360.0 0.357
    360.0 12.957
    360.0 1.711
    360.0 0.671
    360.0 0.539
    360.0 3.484
    360.0 0.908
    360.0 0.263
    360.0 0.577
    360.0 7.969
    360.0 19.173
    360.0 5.258
    360.0 4.582
    360.0 4.565
    360.0 0.638
    360.0 3.11
    360.0 2.996
    360.0 360.0
    360.0 35.611
    360.0 3.742
    360.0 2.968
    360.0 219.176
    360.0 0.625
    360.0 1.026
    360.0 3.285
    360.0 2.012
    360.0 4.182
    360.0 3.133
    360.0 2.881
    360.0 0.507
    360.0 0.82
    360.0 0.722
    360.0 0.464
    360.0 3.897
    360.0 5.867
    360.0 0.847
    360.0 7.242
    360.0 0.493
    360.0 0.633
    360.0 312.354
    360.0 3.608
    360.0 3.395
    360.0 3.167
    360.0 6.035
    360.0 0.954
    360.0 0.367
    360.0 1.351
    360.0 0.337
    360.0 2.462
    360.0 0.477
    360.0 0.75
    360.0 0.851
    360.0 360.0
    360.0 4.373
    360.0 3.089
    360.0 360.0
    360.0 3.689
    360.0 4.059
    360.0 0.803
    360.0 360.0
    360.0 1.027
    360.0 3.734
    360.0 2.699
    360.0 3.217
    360.0 0.775
    360.0 0.482
    360.0 2.562
    360.0 0.881
    360.0 2.028
    360.0 360.0
    360.0 4.858
    360.0 4.931
    360.0 3.73
    360.0 360.0
    360.0 2.226
    360.0 25.299
    360.0 1.443
    360.0 0.955
    360.0 360.0
    360.0 4.888
    360.0 7.318
    360.0 0.918
    360.0 0.334
    360.0 6.025
    360.0 5.634
    360.0 0.714
    360.0 4.988
    360.0 29.358
    360.0 0.849
    360.0 4.161
    360.0 7.367
    360.0 6.078
    360.0 2.497
    360.0 360.0
    360.0 0.634
    360.0 5.085
    360.0 1.786
    360.0 0.675
    360.0 14.913
    360.0 0.989
    360.0 0.748
    360.0 1.665
    360.0 360.0
    360.0 5.82
    360.0 360.0
    360.0 3.815
    360.0 0.464
    360.0 0.552
    360.0 5.719
    360.0 2.294
    360.0 2.436
    360.0 360.0
    360.0 5.79
    360.0 1.225
    360.0 12.691
    360.0 0.828
    360.0 6.942
    360.0 2.568
    360.0 0.531
    360.0 6.365
    360.0 3.801
    360.0 46.845
    360.0 360.0
    360.0 0.546
    360.0 0.47
    360.0 3.904
    360.0 0.566
    360.0 360.0
    360.0 6.345
    360.0 1.308
    360.0 5.67
    360.0 6.381
    360.0 4.172
    360.0 0.927
    360.0 6.339
    360.0 3.299
    360.0 11.385
    360.0 10.099
    360.0 16.738
    360.0 0.737
    360.0 2.8
    360.0 0.372
    360.0 0.907
    360.0 1.319
    360.0 0.543
    360.0 1.645
    360.0 0.893
    360.0 1.446
    360.0 1.055
    360.0 0.496
    360.0 0.295
    360.0 253.486
    360.0 0.851
    360.0 0.929
    360.0 0.894
    360.0 1.719
    360.0 0.717
    360.0 1.281
    360.0 2.865
    360.0 360.0
    360.0 2.543
    360.0 360.0
    360.0 2.387
    360.0 2.249
    360.0 0.882
    360.0 0.563
    360.0 3.23
    360.0 3.609
    360.0 2.898
    360.0 1.224
    360.0 1.51
    360.0 360.0
    360.0 1.436
    360.0 10.393
    360.0 2.46
    360.0 11.707
    360.0 4.827
    360.0 1.178
    360.0 24.736
    360.0 1.138
    360.0 4.872
    360.0 1.416
    360.0 1.23
    360.0 6.123
    360.0 1.347
    360.0 8.08
    360.0 0.433
    360.0 0.587
    360.0 0.768
    360.0 1.159
    360.0 2.752
    360.0 1.516
    360.0 3.857
    360.0 0.609
    360.0 0.543
    360.0 0.841
    360.0 0.598
    360.0 0.812
    360.0 0.593
    360.0 4.79
    360.0 3.827
    360.0 49.81
    360.0 1.81
    360.0 0.253
    360.0 0.32
    360.0 0.363
    360.0 360.0
    360.0 12.598
    360.0 1.415
    360.0 0.988
    360.0 0.518
    360.0 0.602
    360.0 2.225
    360.0 0.312
    360.0 89.17
    360.0 1.483
    360.0 0.62
    360.0 0.459
    360.0 1.16
    360.0 0.638
    360.0 2.644
    360.0 0.382
    360.0 360.0
    360.0 3.132
    360.0 8.957
    360.0 2.691
    360.0 1.001
    360.0 6.247
    360.0 0.718
    360.0 0.826
    360.0 0.545
    360.0 2.485
    360.0 2.396
    360.0 0.572
    360.0 0.647
    360.0 0.713
    360.0 6.929
    360.0 0.528
    360.0 0.979
    360.0 2.87
    360.0 3.639
    360.0 0.639
    360.0 4.003
    360.0 360.0
    360.0 1.008
    360.0 4.751
    360.0 22.423
    360.0 0.706
    360.0 0.413
    360.0 3.506
    360.0 0.917
    360.0 5.845
    360.0 1.225
    360.0 0.643
    360.0 4.681
    360.0 1.563
    360.0 4.301
    360.0 0.388
    360.0 0.43
    360.0 0.376
    360.0 2.842
    360.0 0.713
    360.0 3.694
    360.0 0.707
    360.0 2.29
    360.0 3.678
    360.0 41.262
    360.0 360.0
    360.0 6.895
    360.0 7.486
    360.0 0.404
    360.0 360.0
    360.0 0.974
    360.0 4.495
    360.0 17.716
    360.0 0.462
    360.0 5.155
    360.0 0.649
    360.0 7.463
    360.0 3.85
    360.0 1.565
    360.0 8.53
    360.0 360.0
    360.0 360.0
    360.0 1.336
    360.0 360.0
    360.0 9.502
    360.0 3.824
    360.0 0.858
    360.0 23.677
    360.0 2.658
    360.0 0.372
    360.0 360.0
    360.0 2.285
    360.0 360.0
    360.0 9.38
    360.0 0.832
    360.0 9.206
    360.0 0.574
    360.0 2.573
    360.0 6.479
    360.0 2.197
    360.0 2.047
    360.0 0.973
    360.0 0.616
    360.0 0.63
    360.0 2.362
    360.0 0.763
    360.0 2.888
    360.0 360.0
    360.0 360.0
    360.0 4.157
    360.0 0.585
    360.0 7.281
    360.0 0.8
    360.0 2.096
    360.0 0.304
    360.0 7.36
    360.0 0.891
    360.0 1.839
    360.0 2.069
    360.0 2.647
    360.0 5.011
    360.0 0.703
    360.0 4.191
    360.0 4.383
    360.0 8.128
    360.0 6.191
    360.0 3.069
    360.0 22.527
    360.0 1.038
    360.0 4.51
    360.0 2.579
    360.0 1.245
    360.0 4.004
    360.0 12.771
    360.0 6.385
    360.0 0.489
    360.0 3.393
    360.0 2.104
    360.0 3.859
    360.0 2.384
    360.0 1.868
    360.0 4.978
    360.0 5.143
    360.0 0.797
    360.0 360.0
    360.0 0.346
    360.0 2.633
    360.0 0.556
    360.0 4.373
    360.0 360.0
    360.0 2.319
    360.0 1.654
    360.0 0.841
    360.0 1.0
    360.0 4.666
    360.0 360.0
    360.0 0.808
    360.0 4.149
    360.0 360.0
    360.0 0.907
    360.0 0.708
    360.0 0.883
    360.0 0.883
    360.0 360.0
    360.0 360.0
    360.0 5.045
    360.0 0.957
    360.0 0.409
    360.0 360.0
    360.0 0.313
    360.0 12.095
    360.0 3.002
    360.0 4.602
    360.0 0.828
    360.0 2.208
    360.0 1.42
    360.0 0.807
    360.0 10.12
    360.0 0.86
    360.0 2.628
    360.0 0.631
    360.0 1.108
    360.0 0.863
    360.0 360.0
    360.0 360.0
    360.0 0.949
    360.0 15.46
    360.0 0.906
    360.0 2.552
    360.0 6.89
    360.0 0.813
    360.0 1.73
    360.0 1.522
    360.0 0.897
    360.0 2.946
    360.0 5.269
    360.0 0.743
    360.0 0.404
    360.0 0.675
    360.0 1.345
    360.0 1.6
    360.0 4.771
    360.0 2.308
    360.0 1.061
    360.0 1.114
    360.0 16.67
    360.0 0.785
    360.0 0.968
    360.0 9.383
    360.0 0.308
    360.0 6.828
    360.0 0.864
    360.0 13.367
    360.0 1.944
    360.0 0.588
    360.0 0.859
    360.0 0.662
    360.0 3.437
    360.0 2.879
    360.0 0.514
    360.0 0.839
    360.0 5.407
    360.0 3.246
    360.0 3.384
    360.0 1.534
    360.0 0.909
    360.0 0.455
    360.0 4.123
    360.0 5.56
    360.0 51.096
    360.0 1.039
    360.0 4.655
    360.0 3.087
    360.0 0.977
    360.0 1.364
    360.0 2.021
    360.0 1.962
    360.0 0.816
    360.0 6.922
    360.0 4.954
    360.0 4.372
    360.0 360.0
    360.0 4.474
    360.0 0.408
    360.0 3.287
    360.0 1.014
    360.0 4.857
    360.0 0.295
    360.0 3.261
    360.0 0.894
    360.0 3.318
    360.0 0.854
    360.0 0.864
    360.0 2.752
    360.0 0.39
    360.0 4.392
    360.0 0.791
    360.0 115.321
    360.0 4.14
    360.0 4.465
    360.0 16.722
    360.0 1.014
    360.0 1.281
    360.0 0.584
    360.0 3.177
    360.0 2.678
    360.0 360.0
    360.0 0.774
    360.0 0.567
    360.0 3.078
    360.0 0.557
    360.0 4.36
    360.0 360.0
    360.0 3.786
    360.0 2.86
    360.0 3.215
    360.0 5.627
    360.0 0.875
    360.0 9.415
    360.0 8.015
    360.0 1.711
    360.0 1.184
    360.0 6.356
    360.0 3.806
    360.0 223.147
    360.0 1.073
    360.0 5.534
    360.0 1.523
    360.0 72.43
    360.0 3.774
    360.0 0.911
    360.0 0.64
    360.0 4.201
    360.0 3.151
    360.0 5.107
    360.0 1.652
    360.0 3.693
    360.0 4.131
    360.0 2.783
    360.0 0.551
    360.0 1.707
    360.0 0.689
    360.0 1.305
    360.0 8.796
    360.0 0.721
    360.0 2.056
    360.0 0.831
    360.0 22.74
    360.0 3.426
    360.0 360.0
    360.0 2.405
    360.0 0.688
    360.0 1.156
    360.0 3.143
    360.0 0.505
    360.0 8.537
    360.0 0.707
    360.0 5.807
    360.0 3.03
    360.0 0.729
    360.0 360.0
    360.0 1.182
    360.0 0.766
    360.0 3.172
    360.0 360.0
    360.0 0.827
    360.0 7.169
    360.0 29.238
    360.0 0.344
    360.0 3.386
    360.0 0.752
    360.0 1.039
    360.0 7.933
    360.0 3.603
    360.0 34.785
    360.0 3.455
    360.0 0.947
    360.0 0.468
    360.0 0.518
    360.0 0.616
    360.0 4.887
    360.0 10.816
    360.0 360.0
    360.0 1.804
    360.0 360.0
    360.0 1.913
    360.0 3.314
    360.0 2.029
    360.0 1.566
    360.0 360.0
    360.0 0.394
    360.0 0.756
    360.0 8.328
    360.0 2.197
    360.0 360.0
    360.0 0.861
    360.0 4.789
    360.0 12.385
    360.0 6.454
    360.0 0.555
    360.0 0.794
    360.0 360.0
    360.0 0.505
    360.0 0.887
    360.0 0.664
    360.0 3.982
    360.0 34.265
    360.0 8.857
    360.0 2.903
    360.0 0.9
    360.0 360.0
    360.0 0.442
    360.0 3.846
    360.0 6.042
    360.0 360.0
    360.0 0.706
    360.0 27.661
    360.0 10.284
    360.0 25.224
    360.0 1.502
    360.0 2.455
    360.0 0.983
    360.0 2.641
    360.0 1.13
    360.0 360.0
    360.0 0.492
    360.0 0.696
    360.0 15.591
    360.0 1.059
    360.0 0.638
    360.0 5.272
    360.0 360.0
    360.0 157.732
    360.0 360.0
    360.0 4.386
    360.0 0.81
    360.0 360.0
    360.0 25.22
    360.0 0.998
    360.0 1.68
    360.0 0.44
    360.0 62.404
    360.0 15.495
    360.0 1.306
    360.0 12.753
    360.0 3.429
    360.0 1.137
    360.0 360.0
    360.0 0.97
    360.0 2.749
    360.0 5.84
    360.0 1.151
    360.0 0.498
    360.0 0.525
    360.0 0.902
    360.0 2.288
    360.0 5.003
    360.0 0.943
    360.0 2.494
    360.0 360.0
    360.0 0.734
    360.0 0.96
    360.0 3.657
    360.0 0.792
    360.0 360.0
    360.0 360.0
    360.0 3.302
    360.0 1.339
    360.0 3.278
    360.0 0.858
    360.0 0.673
    360.0 2.96
    360.0 4.779
    360.0 1.732
    360.0 0.561
    360.0 5.789
    360.0 360.0
    360.0 3.125
    360.0 2.895
    360.0 2.638
    360.0 0.557
    360.0 2.817
    360.0 1.545
    360.0 2.38
    360.0 5.838
    360.0 1.674
    360.0 0.299
    360.0 4.497
    360.0 0.617
    360.0 360.0
    360.0 0.385
    360.0 7.294
    360.0 1.136
    360.0 3.533
    360.0 0.859
    360.0 360.0
    360.0 5.025
    360.0 0.907
    360.0 0.836
    360.0 1.414
    360.0 243.217
    360.0 0.898
    360.0 4.545
    360.0 3.368
    360.0 10.458
    360.0 0.669
    360.0 41.952
    360.0 0.749
    360.0 3.867
    360.0 3.018
    360.0 3.037
    360.0 0.766
    360.0 1.192
    360.0 3.22
    360.0 67.807
    360.0 1.19
    360.0 0.754
    360.0 4.278
    360.0 4.299
    360.0 0.237
    360.0 0.519
    360.0 3.972
    360.0 1.28
    360.0 1.318
    360.0 2.222
    360.0 1.849
    360.0 0.968
    360.0 1.872
    360.0 4.176
    360.0 360.0
    360.0 360.0
    360.0 0.341
    360.0 360.0
    360.0 0.91
    360.0 17.511
    360.0 3.127
    360.0 3.299
    360.0 10.314
    360.0 2.528
    360.0 3.234
    360.0 4.253
    360.0 0.814
    360.0 5.39
    360.0 60.957
    360.0 1.595
    360.0 0.673
    360.0 0.5
    360.0 0.847
    360.0 0.67
    360.0 1.109
    360.0 3.31
    360.0 50.362
    360.0 1.086
    360.0 0.996
    360.0 3.271
    360.0 0.899
    360.0 1.792
    360.0 1.732
    360.0 12.241
    360.0 1.112
    360.0 3.426
    360.0 2.706
    360.0 0.368
    360.0 0.464
    360.0 0.662
    360.0 0.695
    360.0 360.0
    360.0 0.666
    360.0 33.27
    360.0 2.325
    360.0 4.351
    360.0 2.38
    360.0 360.0
    360.0 4.185
    360.0 1.004
    360.0 4.785
    360.0 3.381
    360.0 1.494
    360.0 1.049
    360.0 360.0
    360.0 1.631
    360.0 4.337
    360.0 3.278
    360.0 1.211
    360.0 3.251
    360.0 10.495
    360.0 1.68
    360.0 0.759
    360.0 15.281
    360.0 4.992
    360.0 0.667
    360.0 1.5
    360.0 1.256
    360.0 8.012
    360.0 4.98
    360.0 26.943
    360.0 6.11
    360.0 1.056
    360.0 3.466
    360.0 1.06
    360.0 2.688
    360.0 14.019
    360.0 31.822
    360.0 0.375
    360.0 0.807
    360.0 122.717
    360.0 0.356
    360.0 0.776
    360.0 0.63
    360.0 5.879
    360.0 4.311
    360.0 4.852
    360.0 0.551
    360.0 4.534
    360.0 0.637
    360.0 4.106
    360.0 8.62
    360.0 0.749
    360.0 31.857
    360.0 5.417
    360.0 3.033
    360.0 5.686
    360.0 0.704
    360.0 360.0
    360.0 2.844
    360.0 5.049
    360.0 4.166
    360.0 0.536
    360.0 5.777
    360.0 0.982
    360.0 2.862
    360.0 1.803
    360.0 1.06
    360.0 2.906
    360.0 1.038
    360.0 2.838
    360.0 10.039
    360.0 0.677
    360.0 4.613
    360.0 360.0
    360.0 1.006
    360.0 3.122
    360.0 5.726
    360.0 1.352
    360.0 2.805
    360.0 4.17
    360.0 1.09
    360.0 4.248
    360.0 5.827
    360.0 3.195
    360.0 0.783
    360.0 0.691
    360.0 2.448
    360.0 11.327
    360.0 360.0
    360.0 360.0
    360.0 2.915
    360.0 0.966
    360.0 0.881
    360.0 6.854
    360.0 3.062
    360.0 2.344
    360.0 4.474
    360.0 3.885
    360.0 0.779
    360.0 5.242
    360.0 360.0
    360.0 360.0
    360.0 7.868
    360.0 0.932
    360.0 24.353
    360.0 0.727
    360.0 0.865
    360.0 1.069
    360.0 5.421
    360.0 5.014
    360.0 0.512
    360.0 0.57
    360.0 4.752
    360.0 6.561
    360.0 0.573
    360.0 0.597
    360.0 360.0
    360.0 360.0
    360.0 2.799
    360.0 4.05
    360.0 0.481
    360.0 360.0
    360.0 0.626
    360.0 13.796
    360.0 21.367
    360.0 0.316
    360.0 1.882
    360.0 0.84
    360.0 2.854
    360.0 0.478
    360.0 1.827
    360.0 30.535
    360.0 0.743
    360.0 0.427
    360.0 3.277
    360.0 0.998
    360.0 0.975
    360.0 0.29
    360.0 0.73
    360.0 360.0
    360.0 2.778
    360.0 0.747
    360.0 2.063
    360.0 0.272
    360.0 4.317
    360.0 1.339
    360.0 8.477
    360.0 0.67
    360.0 6.669
    360.0 1.698
    360.0 0.55
    360.0 0.607
    360.0 0.839
    360.0 2.699
    360.0 21.437
    360.0 120.996
    360.0 16.916
    360.0 5.837
    360.0 0.527
    360.0 3.239
    360.0 7.129
    360.0 4.625
    360.0 26.212
    360.0 2.425
    360.0 0.969
    360.0 0.778
    360.0 0.684
    360.0 7.369
    360.0 4.226
    360.0 0.372
    360.0 4.056
    360.0 2.546
    360.0 0.911
    360.0 0.878
    360.0 2.128
    360.0 0.596
    360.0 3.955
    360.0 2.976
    360.0 0.633
    360.0 360.0
    360.0 0.535
    360.0 3.409
    360.0 360.0
    360.0 2.827
    360.0 0.57
    360.0 4.146
    360.0 6.741
    360.0 1.154
    360.0 0.326
    360.0 0.886
    360.0 0.532
    360.0 0.823
    360.0 0.932
    360.0 1.604
    360.0 0.192
    360.0 2.774
    360.0 0.755
    360.0 0.177
    360.0 0.822
    360.0 0.962
    360.0 5.008
    360.0 3.996
    360.0 0.916
    360.0 0.904
    360.0 0.599
    360.0 1.513
    360.0 1.648
    360.0 7.556
    360.0 0.717
    360.0 9.133
    360.0 133.515
    360.0 360.0
    360.0 1.02
    360.0 5.925
    360.0 0.892
    360.0 360.0
    360.0 1.309
    360.0 11.383
    360.0 0.619
    360.0 0.868
    360.0 7.923
    360.0 6.409
    360.0 32.492
    360.0 0.697
    360.0 15.329
    360.0 360.0
    360.0 1.242
    360.0 8.746
    360.0 360.0
    360.0 1.523
    360.0 0.765
    360.0 360.0
    360.0 360.0
    360.0 0.782
    360.0 2.357
    360.0 0.361
    360.0 0.938
    360.0 0.82
    360.0 0.729
    360.0 0.966
    360.0 81.278
    360.0 1.276
    360.0 5.262
    360.0 360.0
    360.0 14.635
    360.0 3.194
    360.0 0.701
    360.0 1.105
    360.0 2.608
    360.0 4.481
    360.0 360.0
    360.0 0.48
    360.0 0.574
    360.0 2.964
    360.0 3.08
    360.0 0.793
    360.0 2.393
    360.0 360.0
    360.0 360.0
    360.0 1.645
    360.0 4.262
    360.0 360.0
    360.0 0.666
    360.0 0.585
    360.0 0.884
    360.0 360.0
    360.0 0.856
    360.0 0.681
    360.0 1.007
    360.0 43.265
    360.0 4.865
    360.0 360.0
    360.0 1.029
    360.0 360.0
    360.0 4.604
    360.0 0.914
    360.0 1.058
    360.0 0.658
    360.0 0.797
    360.0 0.65
    360.0 2.563
    360.0 0.956
    360.0 3.052
    360.0 5.025
    360.0 17.546
    360.0 3.546
    360.0 0.833
    360.0 0.636
    360.0 0.391
    360.0 11.594
    360.0 8.575
    360.0 0.401
    360.0 1.53
    360.0 2.61
    360.0 2.918
    360.0 1.348
    360.0 0.588
    360.0 3.325
    360.0 2.99
    360.0 9.992
    360.0 1.003
    360.0 5.362
    360.0 4.461
    360.0 3.912
    360.0 0.375
    360.0 6.545
    360.0 64.52
    360.0 1.375
    360.0 18.186
    360.0 0.906
    360.0 1.269
    360.0 0.981
    360.0 5.412
    360.0 360.0
    360.0 0.563
    360.0 3.728
    360.0 2.313
    360.0 3.933
    360.0 0.669
    360.0 3.554
    360.0 0.618
    360.0 0.888
    360.0 4.299
    360.0 0.718
    360.0 2.88
    360.0 4.061
    360.0 0.361
    360.0 0.727
    360.0 0.814
    360.0 4.574
    360.0 1.773
    360.0 1.658
    360.0 4.181
    360.0 4.667
    360.0 0.668
    360.0 4.249
    360.0 0.843
    360.0 0.994
    360.0 3.042
    360.0 3.461
    360.0 360.0
    360.0 4.506
    360.0 1.674
    360.0 3.066
    360.0 2.363
    360.0 2.668
    360.0 3.657
    360.0 5.816
    360.0 2.9
    360.0 0.511
    360.0 0.812
    360.0 1.68
    360.0 24.364
    360.0 2.009
    360.0 360.0
    360.0 2.646
    360.0 2.312
    360.0 0.794
    360.0 27.628
    360.0 4.03
    360.0 0.869
    360.0 6.161
    360.0 193.914
    360.0 4.231
    360.0 0.963
    360.0 0.56
    360.0 1.023
    360.0 0.857
    360.0 0.671
    360.0 12.561
    360.0 0.944
    360.0 360.0
    360.0 0.874
    360.0 30.129
    360.0 5.013
    360.0 0.893
    360.0 0.546
    360.0 2.423
    360.0 360.0
    360.0 0.726
    360.0 3.973
    360.0 5.182
    360.0 5.505
    360.0 4.222
    360.0 0.335
    360.0 0.835
    360.0 0.527
    360.0 6.374
    360.0 8.949
    360.0 4.169
    360.0 0.22
    360.0 0.706
    360.0 3.347
    360.0 5.971
    360.0 360.0
    360.0 0.507
    360.0 4.534
    360.0 1.198
    360.0 0.875
    360.0 1.555
    360.0 3.9
    360.0 0.404
    360.0 0.891
    360.0 0.82
    360.0 3.11
    360.0 3.58
    360.0 4.008
    360.0 0.788
    360.0 6.792
    360.0 0.902
    360.0 31.884
    360.0 0.613
    360.0 0.938
    360.0 1.708
    360.0 360.0
    360.0 3.238
    360.0 0.591
    360.0 10.281
    360.0 123.216
    360.0 1.565
    360.0 0.784
    360.0 3.079
    360.0 4.157
    360.0 2.447
    360.0 2.005
    360.0 360.0
    360.0 0.486
    360.0 63.936
    360.0 0.894
    360.0 0.61
    360.0 4.318
    360.0 360.0
    360.0 3.085
    360.0 0.859
    360.0 2.519
    360.0 5.382
    360.0 4.188
    360.0 0.53
    360.0 6.679
    360.0 9.102
    360.0 1.195
    360.0 0.696
    360.0 6.025
    360.0 2.34
    360.0 0.776
    360.0 1.165
    360.0 1.679
    360.0 0.841
    360.0 5.22
    360.0 3.463
    360.0 1.152
    360.0 0.389
    360.0 0.911
    360.0 0.297
    360.0 0.989
    360.0 1.636
    360.0 360.0
    360.0 5.008
    360.0 0.406
    360.0 1.014
    360.0 4.667
    360.0 360.0
    360.0 0.858
    360.0 1.3
    360.0 1.739
    360.0 0.682
    360.0 0.577
    360.0 4.596
    360.0 7.608
    360.0 1.621
    360.0 1.579
    360.0 1.527
    360.0 3.626
    360.0 3.598
    360.0 0.863
    360.0 9.675
    360.0 6.23
    360.0 360.0
    360.0 0.783
    360.0 1.242
    360.0 1.43
    360.0 5.439
    360.0 4.471
    360.0 360.0
    360.0 3.308
    360.0 5.932
    360.0 0.592
    360.0 3.661
    360.0 2.197
    360.0 0.858
    360.0 3.132
    360.0 29.033
    360.0 1.384
    360.0 2.977
    360.0 0.587
    360.0 360.0
    360.0 1.255
    360.0 0.789
    360.0 4.076
    360.0 9.599
    360.0 48.254
    360.0 1.287
    360.0 0.644
    360.0 0.712
    360.0 3.777
    360.0 2.353
    360.0 0.647
    360.0 2.323
    360.0 0.888
    360.0 4.124
    360.0 2.151
    360.0 3.559
    360.0 0.696
    360.0 360.0
    360.0 4.416
    360.0 0.875
    360.0 4.07
    360.0 4.725
    360.0 0.791
    360.0 2.557
    360.0 28.102
    360.0 0.738
    360.0 0.555
    360.0 1.81
    360.0 0.634
    360.0 2.677
    360.0 0.471
    360.0 3.13
    360.0 0.761
    360.0 2.275
    360.0 0.833
    360.0 360.0
    360.0 0.375
    360.0 2.5
    360.0 0.771
    360.0 0.833
    360.0 0.418
    360.0 0.638
    360.0 6.662
    360.0 0.591
    360.0 2.457
    360.0 1.312
    360.0 30.346
    360.0 3.534
    360.0 1.004
    360.0 4.185
    360.0 3.833
    360.0 0.78
    360.0 0.539
    360.0 35.355
    360.0 3.426
    360.0 0.523
    360.0 3.006
    360.0 0.777
    360.0 0.925
    360.0 75.624
    360.0 0.498
    360.0 0.385
    360.0 3.013
    360.0 5.15
    360.0 0.915
    360.0 2.299
    360.0 0.913
    360.0 8.065
    360.0 0.882
    360.0 0.756
    360.0 0.946
    360.0 360.0
    360.0 0.812
    360.0 0.539
    360.0 2.565
    360.0 0.871
    360.0 5.49
    360.0 0.358
    360.0 6.927
    360.0 0.523
    360.0 360.0
    360.0 27.629
    360.0 0.875
    360.0 0.78
    360.0 1.093
    360.0 2.976
    360.0 0.434
    360.0 0.508
    360.0 0.699
    360.0 11.375
    360.0 360.0
    360.0 5.221
    360.0 6.064
    360.0 0.883
    360.0 3.676
    360.0 1.304
    360.0 0.774
    360.0 13.681
    360.0 4.726
    360.0 0.364
    360.0 19.181
    360.0 10.189
    360.0 0.695
    360.0 1.101
    360.0 6.532
    360.0 3.325
    360.0 0.32
    360.0 360.0
    360.0 0.867
    360.0 4.03
    360.0 3.09
    360.0 2.073
    360.0 1.677
    360.0 26.952
    360.0 1.005
    360.0 6.297
    360.0 0.622
    360.0 0.917
    360.0 0.836
    360.0 2.569
    360.0 3.197
    360.0 1.5
    360.0 126.435
    360.0 1.36
    360.0 3.286
    360.0 2.129
    360.0 4.186
    360.0 3.784
    360.0 0.681
    360.0 0.78
    360.0 15.599
    360.0 0.894
    360.0 2.816
    360.0 0.829
    360.0 0.764
    360.0 0.788
    360.0 1.675
    360.0 1.011
    360.0 0.739
    360.0 26.132
    360.0 0.865
    360.0 1.085
    360.0 4.526
    360.0 0.738
    360.0 5.489
    360.0 0.384
    360.0 3.126
    360.0 1.911
    360.0 360.0
    360.0 0.582
    360.0 360.0
    360.0 360.0
    360.0 27.008
    360.0 0.882
    360.0 10.546
    360.0 11.543
    360.0 5.805
    360.0 10.971
    360.0 5.456
    360.0 0.936
    360.0 1.048
    360.0 1.107
    360.0 1.865
    360.0 2.025
    360.0 4.687
    360.0 1.027
    360.0 7.967
    360.0 0.559
    360.0 0.681
    360.0 0.591
    360.0 0.771
    360.0 9.096
    360.0 1.141
    360.0 360.0
    360.0 3.074
    360.0 13.348
    360.0 0.637
    360.0 3.539
    360.0 0.673
    360.0 11.789
    360.0 0.613
    360.0 2.988
    360.0 3.867
    360.0 3.738
    360.0 1.201
    360.0 52.426
    360.0 3.892
    360.0 0.607
    360.0 1.01
    360.0 6.607
    360.0 1.119
    360.0 0.961
    360.0 5.539
    360.0 4.46
    360.0 0.732
    360.0 1.053
    360.0 2.56
    360.0 0.907
    360.0 4.271
    360.0 2.993
    360.0 0.712
    360.0 11.612
    360.0 0.781
    360.0 5.196
    360.0 3.407
    360.0 3.539
    360.0 5.623
    360.0 0.667
    360.0 3.359
    360.0 0.381
    360.0 0.749
    360.0 0.941
    360.0 2.029
    360.0 4.577
    360.0 0.948
    360.0 0.775
    360.0 4.849
    360.0 2.504
    360.0 0.6
    360.0 0.585
    360.0 0.821
    360.0 7.151
    360.0 3.159
    360.0 0.681
    360.0 0.672
    360.0 4.18
    360.0 0.94
    360.0 0.612
    360.0 3.96
    360.0 1.004
    360.0 2.42
    360.0 18.142
    360.0 4.423
    360.0 0.488
    360.0 0.997
    360.0 0.307
    360.0 8.074
    360.0 13.142
    360.0 360.0
    360.0 2.034
    360.0 0.909
    360.0 4.554
    360.0 6.075
    360.0 360.0
    360.0 0.997
    360.0 3.103
    360.0 3.83
    360.0 2.771
    360.0 0.664
    360.0 2.891
    360.0 2.742
    360.0 0.634
    360.0 4.99
    360.0 46.45
    360.0 1.026
    360.0 0.292
    360.0 18.41
    360.0 1.284
    360.0 0.529
    360.0 0.305
    360.0 0.75
    360.0 0.957
    360.0 360.0
    360.0 2.373
    360.0 5.716
    360.0 1.637
    360.0 3.007
    360.0 9.383
    360.0 2.996
    360.0 12.41
    360.0 360.0
    360.0 3.263
    360.0 50.383
    360.0 1.858
    360.0 1.451
    360.0 6.49
    360.0 1.408
    360.0 3.514
    360.0 3.677
    360.0 0.176
    360.0 360.0
    360.0 0.589
    360.0 360.0
    360.0 360.0
    360.0 4.145
    360.0 38.82
    360.0 360.0
    360.0 0.887
    360.0 5.507
    360.0 2.298
    360.0 0.9
    360.0 0.701
    360.0 0.563
    360.0 3.076
    360.0 2.764
    360.0 360.0
    360.0 27.389
    360.0 29.055
    360.0 3.005
    360.0 0.411
    360.0 36.632
    360.0 0.288
    360.0 1.955
    360.0 0.538
    360.0 17.002
    360.0 42.353
    360.0 0.783
    360.0 3.536
    360.0 0.613
    360.0 4.901
    360.0 360.0
    360.0 0.478
    360.0 4.253
    360.0 1.977
    360.0 3.094
    360.0 4.674
    360.0 0.773
    360.0 0.539
    360.0 0.717
    360.0 1.878
    360.0 2.459
    360.0 0.533
    360.0 1.837
    360.0 2.723
    360.0 4.271
    360.0 0.534
    360.0 0.439
    360.0 1.65
    360.0 0.884
    360.0 0.398
    360.0 360.0
    360.0 3.158
    360.0 0.749
    360.0 1.025
    360.0 5.713
    360.0 0.793
    360.0 26.053
    360.0 0.684
    360.0 7.699
    360.0 2.799
    360.0 2.585
    360.0 10.178
    360.0 3.648
    360.0 6.867
    360.0 0.637
    360.0 4.355
    360.0 1.12
    360.0 4.388
    360.0 3.837
    360.0 1.337
    360.0 4.961
    360.0 1.694
    360.0 0.701
    360.0 0.608
    360.0 3.656
    360.0 4.522
    360.0 0.845
    360.0 0.398
    360.0 4.908
    360.0 0.69
    360.0 24.587
    360.0 2.78
    360.0 1.886
    360.0 0.698
    360.0 41.025
    360.0 2.738
    360.0 0.373
    360.0 0.474
    360.0 84.187
    360.0 3.511
    360.0 0.512
    360.0 3.613
    360.0 1.197
    360.0 1.062
    360.0 360.0
    360.0 0.534
    360.0 0.747
    360.0 1.313
    360.0 0.536
    360.0 1.292
    360.0 10.7
    360.0 0.653
    360.0 1.927
    360.0 0.946
    360.0 1.744
    360.0 5.971
    360.0 2.562
    360.0 0.788
    360.0 0.931
    360.0 28.91
    360.0 1.37
    360.0 3.236
    360.0 4.531
    360.0 0.482
    360.0 7.159
    360.0 1.971
    360.0 360.0
    360.0 2.538
    360.0 4.853
    360.0 360.0
    360.0 4.446
    360.0 258.148
    360.0 3.782
    360.0 0.342
    360.0 0.62
    360.0 360.0
    360.0 3.442
    360.0 6.99
    360.0 0.679
    360.0 2.099
    360.0 10.655
    360.0 233.076
    360.0 0.978
    360.0 0.906
    360.0 3.081
    360.0 0.703
    360.0 0.872
    360.0 4.446
    360.0 0.841
    360.0 0.757
    360.0 43.261
    360.0 0.585
    360.0 1.121
    360.0 3.804
    360.0 0.658
    360.0 2.671
    360.0 0.89
    360.0 0.742
    360.0 0.673
    360.0 0.546
    360.0 4.865
    360.0 0.379
    360.0 360.0
    360.0 360.0
    360.0 0.38
    360.0 0.35
    360.0 0.955
    360.0 1.009
    360.0 360.0
    360.0 24.208
    360.0 1.949
    360.0 7.21
    360.0 1.172
    360.0 0.421
    360.0 360.0
    360.0 2.531
    360.0 1.236
    360.0 2.419
    360.0 7.651
    360.0 11.567
    360.0 0.401
    360.0 0.714
    360.0 360.0
    360.0 0.438
    360.0 2.529
    360.0 3.632
    360.0 0.671
    360.0 5.801
    360.0 3.057
    360.0 5.639
    360.0 2.668
    360.0 3.546
    360.0 0.741
    360.0 0.896
    360.0 0.426
    360.0 11.343
    360.0 0.739
    360.0 0.684
    360.0 6.831
    360.0 3.09
    360.0 0.384
    360.0 2.259
    360.0 0.52
    360.0 0.873
    360.0 37.752
    360.0 0.838
    360.0 0.405
    360.0 360.0
    360.0 1.851
    360.0 5.504
    360.0 0.904
    360.0 360.0
    360.0 1.213
    360.0 0.25
    360.0 0.785
    360.0 4.099
    360.0 0.316
    360.0 0.373
    360.0 0.995
    360.0 3.56
    360.0 3.069
    360.0 1.12
    360.0 360.0
    360.0 30.892
    360.0 3.998
    360.0 0.686
    360.0 1.99
    360.0 1.301
    360.0 360.0
    360.0 0.772
    360.0 5.367
    360.0 0.376
    360.0 132.439
    360.0 0.818
    360.0 0.678
    360.0 360.0
    360.0 15.845
    360.0 1.086
    360.0 4.51
    360.0 360.0
    360.0 0.542
    360.0 360.0
    360.0 3.012
    360.0 1.855
    360.0 4.045
    360.0 0.286
    360.0 0.767
    360.0 360.0
    360.0 360.0
    360.0 3.901
    360.0 4.225
    360.0 4.745
    360.0 2.882
    360.0 0.622
    360.0 0.358
    360.0 1.01
    360.0 1.302
    360.0 1.989
    360.0 0.387
    360.0 3.57
    360.0 3.116
    360.0 76.093
    360.0 360.0
    360.0 0.698
    360.0 0.638
    360.0 360.0
    360.0 0.501
    360.0 0.618
    360.0 2.039
    360.0 9.646
    360.0 5.708
    360.0 3.788
    360.0 0.683
    360.0 0.937
    360.0 0.358
    360.0 2.699
    360.0 0.521
    360.0 1.074
    360.0 1.603
    360.0 0.875
    360.0 15.39
    360.0 3.75
    360.0 1.427
    360.0 2.987
    360.0 0.485
    360.0 3.517
    360.0 0.469
    360.0 4.641
    360.0 360.0
    360.0 0.712
    360.0 1.043
    360.0 3.112
    360.0 17.252
    360.0 0.745
    360.0 3.758
    360.0 0.428
    360.0 0.856
    360.0 1.283
    360.0 0.678
    360.0 3.038
    360.0 1.179
    360.0 0.344
    360.0 3.827
    360.0 0.756
    360.0 0.647
    360.0 1.79
    360.0 25.937
    360.0 2.763
    360.0 36.036
    360.0 360.0
    360.0 12.298
    360.0 0.734
    360.0 0.611
    360.0 1.696
    360.0 185.644
    360.0 2.912
    360.0 3.27
    360.0 3.75
    360.0 2.674
    360.0 10.535
    360.0 0.618
    360.0 3.034
    360.0 4.842
    360.0 0.923
    360.0 1.293
    360.0 2.877
    360.0 3.391
    360.0 0.669
    360.0 5.076
    360.0 360.0
    360.0 1.138
    360.0 2.732
    360.0 360.0
    360.0 0.544
    360.0 3.692
    360.0 1.152
    360.0 360.0
    360.0 3.837
    360.0 1.205
    360.0 7.593
    360.0 2.274
    360.0 0.702
    360.0 3.182
    360.0 1.824
    360.0 0.61
    360.0 360.0
    360.0 0.77
    360.0 360.0
    360.0 6.753
    360.0 3.23
    360.0 2.626
    360.0 1.769
    360.0 4.281
    360.0 0.592
    360.0 3.114
    360.0 9.878
    360.0 4.107
    360.0 5.575
    360.0 2.137
    360.0 46.036
    360.0 4.312
    360.0 70.554
    360.0 0.919
    360.0 4.356
    360.0 0.59
    360.0 0.372
    360.0 3.318
    360.0 360.0
    360.0 1.251
    360.0 1.652
    360.0 2.888
    360.0 0.768
    360.0 3.379
    360.0 2.945
    360.0 5.908
    360.0 0.612
    360.0 10.507
    360.0 0.879
    360.0 0.544
    360.0 0.64
    360.0 0.743
    360.0 6.067
    360.0 0.896
    360.0 7.144
    360.0 337.345
    360.0 7.356
    360.0 0.774
    360.0 360.0
    360.0 0.715
    360.0 0.372
    360.0 3.137
    360.0 0.578
    360.0 360.0
    360.0 62.826
    360.0 0.578
    360.0 4.244
    360.0 360.0
    360.0 0.434
    360.0 360.0
    360.0 0.519
    360.0 2.559
    360.0 0.659
    360.0 4.27
    360.0 0.595
    360.0 2.334
    360.0 50.622
    360.0 0.885
    360.0 4.592
    360.0 42.715
    360.0 3.276
    360.0 3.26
    360.0 0.686
    360.0 5.003
    360.0 0.443
    360.0 2.984
    360.0 360.0
    360.0 3.94
    360.0 7.109
    360.0 0.315
    360.0 360.0
    360.0 360.0
    360.0 1.06
    360.0 4.936
    360.0 0.756
    360.0 0.369
    360.0 0.709
    360.0 360.0
    360.0 2.504
    360.0 0.794
    360.0 6.727
    360.0 0.866
    360.0 5.431
    360.0 0.982
    360.0 85.367
    360.0 5.461
    360.0 1.239
    360.0 0.497
    360.0 0.925
    360.0 4.01
    360.0 0.866
    360.0 3.26
    360.0 0.87
    360.0 0.393
    360.0 4.501
    360.0 2.417
    360.0 4.963
    360.0 0.684
    360.0 3.747
    360.0 2.503
    360.0 3.466
    360.0 1.66
    360.0 0.253
    360.0 0.796
    360.0 4.671
    360.0 19.566
    360.0 2.516
    360.0 2.474
    360.0 5.085
    360.0 6.863
    360.0 2.786
    360.0 0.741
    360.0 4.161
    360.0 0.66
    360.0 0.432
    360.0 11.208
    360.0 11.828
    360.0 360.0
    360.0 0.537
    360.0 3.535
    360.0 0.607
    360.0 1.13
    360.0 0.841
    360.0 5.493
    360.0 3.301
    360.0 2.625
    360.0 0.564
    360.0 0.412
    360.0 5.669
    360.0 1.013
    360.0 1.297
    360.0 2.887
    360.0 1.282
    360.0 4.839
    360.0 4.49
    360.0 0.669
    360.0 2.844
    360.0 4.428
    360.0 3.941
    360.0 4.237
    360.0 2.412
    360.0 2.308
    360.0 3.458
    360.0 2.258
    360.0 4.1
    360.0 2.423
    360.0 0.628
    360.0 1.804
    360.0 12.834
    360.0 360.0
    360.0 1.003
    360.0 2.783
    360.0 3.056
    360.0 12.198
    360.0 0.32
    360.0 0.83
    360.0 7.081
    360.0 4.499
    360.0 2.101
    360.0 3.215
    360.0 0.761
    360.0 0.888
    360.0 1.059
    360.0 9.711
    360.0 0.857
    360.0 5.445
    360.0 0.997
    360.0 2.162
    360.0 0.69
    360.0 3.949
    360.0 360.0
    360.0 1.979
    360.0 0.312
    360.0 2.799
    360.0 4.069
    360.0 3.806
    360.0 1.348
    360.0 1.136
    360.0 2.572
    360.0 3.905
    360.0 1.03
    360.0 8.494
    360.0 234.025
    360.0 0.584
    360.0 0.804
    360.0 1.72
    360.0 0.921
    360.0 2.375
    360.0 8.369
    360.0 0.697
    360.0 0.535
    360.0 10.317
    360.0 0.663
    360.0 5.78
    360.0 0.961
    360.0 4.176
    360.0 0.365
    360.0 2.708
    360.0 0.928
    360.0 0.917
    360.0 360.0
    360.0 1.778
    360.0 2.271
    360.0 1.014
    360.0 0.507
    360.0 2.779
    360.0 0.395
    360.0 1.276
    360.0 4.446
    360.0 0.406
    360.0 0.616
    360.0 0.77
    360.0 0.679
    360.0 0.66
    360.0 2.67
    360.0 0.541
    360.0 360.0
    360.0 5.838
    360.0 46.972
    360.0 0.631
    360.0 1.386
    360.0 70.379
    360.0 2.794
    360.0 0.61
    360.0 4.329
    360.0 0.515
    360.0 4.336
    360.0 3.347
    360.0 1.693
    360.0 0.524
    360.0 27.597
    360.0 2.447
    360.0 0.886
    360.0 0.603
    360.0 3.584
    360.0 0.828
    360.0 0.98
    360.0 0.827
    360.0 0.917
    360.0 48.865
    360.0 2.29
    360.0 0.574
    360.0 0.69
    360.0 21.507
    360.0 0.384
    360.0 360.0
    360.0 0.741
    360.0 4.298
    360.0 2.128
    360.0 15.15
    360.0 0.95
    360.0 0.985
    360.0 5.959
    360.0 0.8
    360.0 0.86
    360.0 3.334
    360.0 0.36
    360.0 332.302
    360.0 4.535
    360.0 7.503
    360.0 5.879
    360.0 5.67
    360.0 0.957
    360.0 0.408
    360.0 5.428
    360.0 0.783
    360.0 1.709
    360.0 0.342
    360.0 0.54
    360.0 0.632
    360.0 9.906
    360.0 22.938
    360.0 0.952
    360.0 4.873
    360.0 2.236
    360.0 0.637
    360.0 1.557
    360.0 0.822
    360.0 1.158
    360.0 0.527
    360.0 1.13
    360.0 0.621
    360.0 0.798
    360.0 4.486
    360.0 1.149
    360.0 1.4
    360.0 1.292
    360.0 5.719
    360.0 7.733
    360.0 3.423
    360.0 1.028
    360.0 0.522
    360.0 8.242
    360.0 5.612
    360.0 0.738
    360.0 1.99
    360.0 0.561
    360.0 16.301
    360.0 0.597
    360.0 0.71
    360.0 360.0
    360.0 1.509
    360.0 360.0
    360.0 3.634
    360.0 4.974
    360.0 6.307
    360.0 3.469
    360.0 0.303
    360.0 0.557
    360.0 0.874
    360.0 19.997
    360.0 5.131
    360.0 0.904
    360.0 1.493
    360.0 0.553
    360.0 0.985
    360.0 360.0
    360.0 1.408
    360.0 18.099
    360.0 0.907
    360.0 0.71
    360.0 360.0
    360.0 3.933
    360.0 4.843
    360.0 0.946
    360.0 0.948
    360.0 0.797
    360.0 360.0
    360.0 1.042
    360.0 3.917
    360.0 3.689
    360.0 0.78
    360.0 0.749
    360.0 4.909
    360.0 360.0
    360.0 360.0
    360.0 360.0
    360.0 5.864
    360.0 1.548
    360.0 360.0
    360.0 0.788
    360.0 20.552
    360.0 0.986
    360.0 17.515
    360.0 0.63
    360.0 0.396
    360.0 0.797
    360.0 0.557
    360.0 3.101
    360.0 0.54
    360.0 1.764
    360.0 6.0
    360.0 23.992
    360.0 2.728
    360.0 0.72
    360.0 26.156
    360.0 4.061
    360.0 60.35
    360.0 2.788
    360.0 0.819
    360.0 7.607
    360.0 17.333
    360.0 4.664
    360.0 0.629
    360.0 0.767
    360.0 10.399
    360.0 0.942
    360.0 0.963
    360.0 6.256
    360.0 2.672
    360.0 0.939
    360.0 92.275
    360.0 0.758
    360.0 5.691
    360.0 4.844
    360.0 1.956
    360.0 1.133
    360.0 0.668
    360.0 360.0
    360.0 0.389
    360.0 3.759
    360.0 0.745
    360.0 0.831
    360.0 2.668
    360.0 0.476
    360.0 360.0
    360.0 3.077
    360.0 1.129
    360.0 3.292
    360.0 0.387
    360.0 4.196
    360.0 0.394
    360.0 3.205
    360.0 211.434
    360.0 0.931
    360.0 3.071
    360.0 1.429
    360.0 3.685
    360.0 2.406
    360.0 4.513
    360.0 0.582
    360.0 52.635
    360.0 3.92
    360.0 1.151
    360.0 8.379
    360.0 0.523
    360.0 2.03
    360.0 0.935
    360.0 1.129
    360.0 1.0
    360.0 5.788
    360.0 4.012
    360.0 2.637
    360.0 0.575
    360.0 7.088
    360.0 285.553
    360.0 0.664
    360.0 0.872
    360.0 0.628
    360.0 1.781
    360.0 1.516
    360.0 15.97
    360.0 29.817
    360.0 2.792
    360.0 360.0
    360.0 0.292
    360.0 0.998
    360.0 2.917
    360.0 17.806
    360.0 0.633
    360.0 18.058
    360.0 4.113
    360.0 0.627
    360.0 4.763
    360.0 0.389
    360.0 1.964
    360.0 0.57
    360.0 2.567
    360.0 5.194
    360.0 4.979
    360.0 2.787
    360.0 0.4
    360.0 22.653
    360.0 1.439
    360.0 1.038
    360.0 7.71
    360.0 0.907
    360.0 2.949
    360.0 0.637
    360.0 5.093
    360.0 1.471
    360.0 347.964
    360.0 2.347
    360.0 0.336
    360.0 360.0
    360.0 0.907
    360.0 4.867
    360.0 0.521
    360.0 0.987
    360.0 3.901
    360.0 0.739
    360.0 4.564
    360.0 0.789
    360.0 1.585
    360.0 0.727
    360.0 10.046
    360.0 2.727
    360.0 1.652
    360.0 1.45
    360.0 0.872
    360.0 7.966
    360.0 4.108
    360.0 0.52
    360.0 360.0
    360.0 0.459
    360.0 2.141
    360.0 2.2
    360.0 0.632
    360.0 4.027
    360.0 0.667
    360.0 0.495
    360.0 0.542
    360.0 2.441
    360.0 2.617
    360.0 0.988
    360.0 4.61
    360.0 3.174
    360.0 30.193
    360.0 0.44
    360.0 2.201
    360.0 0.597
    360.0 0.397
    360.0 6.677
    360.0 0.837
    360.0 0.958
    360.0 3.093
    360.0 1.9
    360.0 0.555
    360.0 0.628
    360.0 42.06
    360.0 3.127
    360.0 0.768
    360.0 6.099
    360.0 360.0
    360.0 0.173
    360.0 6.56
    360.0 0.727
    360.0 4.398
    360.0 4.854
    360.0 1.178
    360.0 3.883
    360.0 0.996
    360.0 4.738
    360.0 3.554
    360.0 3.213
    360.0 2.848
    360.0 2.581
    360.0 3.238
    360.0 0.242
    360.0 4.847
    360.0 7.651
    360.0 0.392
    360.0 3.696
    360.0 5.196
    360.0 2.54
    360.0 5.639
    360.0 4.382
    360.0 6.463
    360.0 0.584
    360.0 1.077
    360.0 0.385
    360.0 0.756
    360.0 0.688
    360.0 4.093
    360.0 9.804
    360.0 2.06
    360.0 4.095
    360.0 0.679
    360.0 360.0
    360.0 7.5
    360.0 6.888
    360.0 2.733
    360.0 10.507
    360.0 0.539
    360.0 0.347
    360.0 3.303
    360.0 41.789
    360.0 360.0
    360.0 4.064
    360.0 3.437
    360.0 3.422
    360.0 360.0
    360.0 0.418
    360.0 4.2
    360.0 0.609
    360.0 10.792
    360.0 0.986
    360.0 1.057
    360.0 3.353
    360.0 0.182
    360.0 3.221
    360.0 1.355
    360.0 1.33
    360.0 0.502
    360.0 0.419
    360.0 0.776
    360.0 4.87
    360.0 0.969
    360.0 0.357
    360.0 0.689
    360.0 0.561
    360.0 2.543
    360.0 12.179
    360.0 1.062
    360.0 1.172
    360.0 0.572
    360.0 4.072
    360.0 1.024
    360.0 3.236
    360.0 3.315
    360.0 6.295
    360.0 2.909
    360.0 4.565
    360.0 17.352
    360.0 0.291
    360.0 3.738
    360.0 360.0
    360.0 3.704
    360.0 5.437
    360.0 0.889
    360.0 0.798
    360.0 3.37
    360.0 2.969
    360.0 2.189
    360.0 1.054
    360.0 3.587
    360.0 3.147
    360.0 0.581
    360.0 360.0
    360.0 4.169
    360.0 0.962
    360.0 0.634
    360.0 1.725
    360.0 19.764
    360.0 12.23
    360.0 1.864
    360.0 0.415
    360.0 2.696
    360.0 4.06
    360.0 1.338
    360.0 2.547
    360.0 2.664
    360.0 0.69
    360.0 3.955
    360.0 4.28
    360.0 0.703
    360.0 4.118
    360.0 2.333
    360.0 6.745
    360.0 1.426
    360.0 0.505
    360.0 3.647
    360.0 1.69
    360.0 5.848
    360.0 53.757
    360.0 0.574
    360.0 0.402
    360.0 2.428
    360.0 4.289
    360.0 2.424
    360.0 3.167
    360.0 3.792
    360.0 1.172
    360.0 4.263
    360.0 9.06
    360.0 0.939
    360.0 360.0
    360.0 0.488
    360.0 0.624
    360.0 259.706
    360.0 0.69
    360.0 4.902
    360.0 4.402
    360.0 1.451
    360.0 0.256
    360.0 5.293
    360.0 2.185
    360.0 0.304
    360.0 0.624
    360.0 3.74
    360.0 1.632
    360.0 6.477
    360.0 1.083
    360.0 0.357
    360.0 5.563
    360.0 3.003
    360.0 10.739
    360.0 0.482
    360.0 1.858
    360.0 71.803
    360.0 0.232
    360.0 4.221
    360.0 18.827
    360.0 1.929
    360.0 3.239
    360.0 0.404
    360.0 1.583
    360.0 360.0
    360.0 1.838
    360.0 5.043
    360.0 3.586
    360.0 6.45
    360.0 1.122
    360.0 5.788
    360.0 0.486
    360.0 0.562
    360.0 1.312
    360.0 360.0
    360.0 360.0
    360.0 3.137
    360.0 3.834
    360.0 1.62
    360.0 1.008
    360.0 6.461
    360.0 0.867
    360.0 3.44
    360.0 0.575
    360.0 2.648
    360.0 0.887
    360.0 2.994
    360.0 0.515
    360.0 360.0
    360.0 360.0
    360.0 1.047
    360.0 0.277
    360.0 0.999
    360.0 2.977
    360.0 4.802
    360.0 33.053
    360.0 23.301
    360.0 1.259
    360.0 1.108
    360.0 2.522
    360.0 0.707
    360.0 1.128
    360.0 5.066
    360.0 0.839
    360.0 3.827
    360.0 0.552
    360.0 5.116
    360.0 4.273
    360.0 3.056
    360.0 2.965
    360.0 360.0
    360.0 2.79
    360.0 0.808
    360.0 360.0
    360.0 0.808
    360.0 6.952
    360.0 2.862
    360.0 0.717
    360.0 360.0
    360.0 4.644
    360.0 360.0
    360.0 0.716
    360.0 4.257
    360.0 0.399
    360.0 8.4
    360.0 1.756
    360.0 0.609
    360.0 20.094
    360.0 1.044
    360.0 0.944
    360.0 360.0
    360.0 0.793
    360.0 0.955
    360.0 3.698
    360.0 360.0
    360.0 0.367
    360.0 4.744
    360.0 0.577
    360.0 1.112
    360.0 0.527
    360.0 2.154
    360.0 0.961
    360.0 0.721
    360.0 0.8
    360.0 0.754
    360.0 14.127
    360.0 360.0
    360.0 0.666
    360.0 0.384
    360.0 0.479
    360.0 360.0
    360.0 0.815
    360.0 360.0
    360.0 26.002
    360.0 0.515
    360.0 0.591
    360.0 4.766
    360.0 0.839
    360.0 0.918
    360.0 4.706
    360.0 0.868
    360.0 6.222
    360.0 360.0
    360.0 1.015
    360.0 1.526
    360.0 9.849
    360.0 4.281
    360.0 360.0
    360.0 1.707
    360.0 0.871
    360.0 360.0
    360.0 5.058
    360.0 4.64
    360.0 360.0
    360.0 360.0
    360.0 3.598
    360.0 0.608
    360.0 2.125
    360.0 0.885
    360.0 2.466
    360.0 0.959
    360.0 0.955
    360.0 1.777
    360.0 9.639
    360.0 2.106
    360.0 1.34
    360.0 2.388
    360.0 2.401
    360.0 3.889
    360.0 3.163
    360.0 1.007
    360.0 1.066
    360.0 360.0
    360.0 2.541
    360.0 0.519
    360.0 5.101
    360.0 1.526
    360.0 0.602
    360.0 2.253
    360.0 3.524
    360.0 9.861
    360.0 0.645
    360.0 20.498
    360.0 0.513
    360.0 1.2
    360.0 2.607
    360.0 3.905
    360.0 0.926
    360.0 0.714
    360.0 360.0
    360.0 8.207
    360.0 1.036
    360.0 3.094
    360.0 0.621
    360.0 360.0
    360.0 5.211
    360.0 0.889
    360.0 360.0
    360.0 2.471
    360.0 5.246
    360.0 0.388
    360.0 2.089
    360.0 0.452
    360.0 7.508
    360.0 360.0
    360.0 3.408
    360.0 0.74
    360.0 0.8
    360.0 5.566
    360.0 360.0
    360.0 0.915
    360.0 6.197
    360.0 1.002
    360.0 0.286
    360.0 3.893
    360.0 0.944
    360.0 5.233
    360.0 12.587
    360.0 0.648
    360.0 1.052
    360.0 0.407
    360.0 16.203
    360.0 3.835
    360.0 0.719
    360.0 0.83
    360.0 0.658
    360.0 5.151
    360.0 7.325
    360.0 5.3
    360.0 6.373
    360.0 0.775
    360.0 1.591
    360.0 4.096
    360.0 1.849
    360.0 3.188
    360.0 2.663
    360.0 0.894
    360.0 3.858
    360.0 0.759
    360.0 0.628
    360.0 2.526
    360.0 360.0
    360.0 0.996
    360.0 5.458
    360.0 4.927
    360.0 0.618
    360.0 4.297
    360.0 0.331
    360.0 6.507
    360.0 1.595
    360.0 3.78
    360.0 0.363
    360.0 2.125
    360.0 0.997
    360.0 7.534
    360.0 0.704
    360.0 8.892
    360.0 0.516
    360.0 4.009
    360.0 3.213
    360.0 360.0
    360.0 4.871
    360.0 1.022
    360.0 0.976
    360.0 360.0
    360.0 0.637
    360.0 3.66
    360.0 4.979
    360.0 0.77
    360.0 360.0
    360.0 0.658
    360.0 5.873
    360.0 5.866
    360.0 3.274
    360.0 0.601
    360.0 0.415
    360.0 21.085
    360.0 1.061
    360.0 10.516
    360.0 2.529
    360.0 2.529
    360.0 3.682
    360.0 0.312
    360.0 7.135
    360.0 0.516
    360.0 0.807
    360.0 360.0
    360.0 0.866
    360.0 3.4
    360.0 28.867
    360.0 20.176
    360.0 2.651
    360.0 0.878
    360.0 360.0
    360.0 1.027
    360.0 17.944
    360.0 0.564
    360.0 0.277
    360.0 0.639
    360.0 0.886
    360.0 360.0
    360.0 360.0
    360.0 235.607
    360.0 3.324
    360.0 5.125
    360.0 0.78
    360.0 0.652
    360.0 0.672
    360.0 9.677
    360.0 2.706
    360.0 360.0
    360.0 1.057
    360.0 0.336
    360.0 48.927
    360.0 0.653
    360.0 1.478
    360.0 6.335
    360.0 13.905
    360.0 0.406
    360.0 1.113
    360.0 345.814
    360.0 0.84
    360.0 3.829
    360.0 360.0
    360.0 0.891
    360.0 3.453
    360.0 0.422
    360.0 0.652
    360.0 19.701
    360.0 1.482
    360.0 1.617
    360.0 0.181
    360.0 1.12
    360.0 1.117
    360.0 1.889
    360.0 5.651
    360.0 0.51
    360.0 0.669
    360.0 0.636
    360.0 7.526
    360.0 0.665
    360.0 9.053
    360.0 1.581
    360.0 1.009
    360.0 1.451
    360.0 360.0
    360.0 0.447
    360.0 4.169
    360.0 3.326
    360.0 360.0
    360.0 360.0
    360.0 1.076
    360.0 0.815
    360.0 3.14
    360.0 1.073
    360.0 0.695
    360.0 3.5
    360.0 0.379
    360.0 0.787
    360.0 360.0
    360.0 6.802
    360.0 7.861
    360.0 44.243
    360.0 3.035
    360.0 0.796
    360.0 3.045
    360.0 1.159
    360.0 360.0
    360.0 5.311
    360.0 360.0
    360.0 0.984
    360.0 15.831
    360.0 2.283
    360.0 1.596
    360.0 0.939
    360.0 0.545
    360.0 0.559
    360.0 0.696
    360.0 3.873
    360.0 360.0
    360.0 27.688
    360.0 2.766
    360.0 4.77
    360.0 3.715
    360.0 360.0
    360.0 4.528
    360.0 2.842
    360.0 0.734
    360.0 0.83
    360.0 6.697
    360.0 7.06
    360.0 0.676
    360.0 3.239
    360.0 0.223
    360.0 0.763
    360.0 0.984
    360.0 0.361
    360.0 3.495
    360.0 3.192
    360.0 24.367
    360.0 4.425
    360.0 3.764
    360.0 0.94
    360.0 8.74
    360.0 0.899
    360.0 0.885
    360.0 360.0
    360.0 0.6
    360.0 360.0
    360.0 5.456
    360.0 360.0
    360.0 0.504
    360.0 4.617
    360.0 0.411
    360.0 0.804
    360.0 360.0
    360.0 2.58
    360.0 51.572
    360.0 5.359
    360.0 0.884
    360.0 2.721
    360.0 0.371
    360.0 360.0
    360.0 5.241
    360.0 0.968
    360.0 4.653
    360.0 7.306
    360.0 9.343
    360.0 2.482
    360.0 360.0
    360.0 2.197
    360.0 0.589
    360.0 5.119
    360.0 0.7
    360.0 4.599
    360.0 1.11
    360.0 3.48
    360.0 0.826
    360.0 1.301
    360.0 0.626
    360.0 5.384
    360.0 0.729
    360.0 360.0
    360.0 4.418
    360.0 360.0
    360.0 15.715
    360.0 360.0
    360.0 360.0
    360.0 0.633
    360.0 1.955
    360.0 1.855
    360.0 360.0
    360.0 7.327
    360.0 1.885
    360.0 6.893
    360.0 0.692
    360.0 0.647
    360.0 2.023
    360.0 4.37
    360.0 0.797
    360.0 2.149
    360.0 0.651
    360.0 4.464
    360.0 1.319
    360.0 9.264
    360.0 1.192
    360.0 0.532
    360.0 8.993
    360.0 0.967
    360.0 0.626
    360.0 4.392
    360.0 1.26
    360.0 0.753
    360.0 360.0
    360.0 44.741
    360.0 0.54
    360.0 0.715
    360.0 5.457
    360.0 0.77
    360.0 0.742
    360.0 2.598
    360.0 360.0
    360.0 3.734
    360.0 0.42
    360.0 0.577
    360.0 31.078
    360.0 6.664
    360.0 4.014
    360.0 0.499
    360.0 360.0
    360.0 0.671
    360.0 1.074
    360.0 5.61
    360.0 2.688
    360.0 360.0
    360.0 2.284
    360.0 1.356
    360.0 0.864
    360.0 4.282
    360.0 87.718
    360.0 3.481
    360.0 3.151
    360.0 360.0
    360.0 0.69
    360.0 0.92
    360.0 0.7
    360.0 0.823
    360.0 1.087
    360.0 0.685
    360.0 0.968
    360.0 5.944
    360.0 360.0
    360.0 8.896
    360.0 0.62
    360.0 0.684
    360.0 2.621
    360.0 4.495
    360.0 6.02
    360.0 0.614
    360.0 0.609
    360.0 0.597
    360.0 18.293
    360.0 0.467
    360.0 1.41
    360.0 307.342
    360.0 0.773
    360.0 0.625
    360.0 5.363
    360.0 0.873
    360.0 3.708
    360.0 4.167
    360.0 14.249
    360.0 1.35
    360.0 3.645
    360.0 1.434
    360.0 5.738
    360.0 0.707
    360.0 14.82
    360.0 7.571
    360.0 0.896
    360.0 2.432
    360.0 0.368
    360.0 0.708
    360.0 45.784
    360.0 0.453
    360.0 0.326
    360.0 4.613
    360.0 2.408
    360.0 2.048
    360.0 48.417
    360.0 1.083
    360.0 360.0
    360.0 2.842
    360.0 0.671
    360.0 2.5
    360.0 360.0
    360.0 0.673
    360.0 0.689
    360.0 4.239
    360.0 3.714
    360.0 360.0
    360.0 0.752
    360.0 6.009
    360.0 0.635
    360.0 0.353
    360.0 12.695
    360.0 239.23
    360.0 0.906
    360.0 0.51
    360.0 0.498
    360.0 8.253
    360.0 1.598
    360.0 360.0
    360.0 0.812
    360.0 0.277
    360.0 4.188
    360.0 4.247
    360.0 360.0
    360.0 1.244
    360.0 360.0
    360.0 10.629
    360.0 0.982
    360.0 0.388
    360.0 360.0
    360.0 0.49
    360.0 3.654
    360.0 0.871
    360.0 2.693
    360.0 1.125
    360.0 3.279
    360.0 0.769
    360.0 0.856
    360.0 0.564
    360.0 0.526
    360.0 30.848
    360.0 3.538
    360.0 10.802
    360.0 1.08
       
        };
    \end{axis}
    \end{tikzpicture}
    \caption{Comparison of run times (in seconds) between Non-summary (x-axis) and Summary-based (y-axis) using a logarithmic scale.}
    \label{fig:plot-sum}
    \end{figure}
% \begin{figure}[t!]
%   \centering
%   \includegraphics[scale=0.30]{sum-nosum.pdf}
% %   \vspace{-0.2in}
% \caption{Comparison of run times (in seconds) between non-summary (x-axis) and summary-based (y-axis) (log-scale).}
% % \vspace{-0.2in}
% \label{fig:plot-sum}
% \end{figure}

% \definecolor{bblue}{HTML}{0064FF}
\definecolor{rred}{HTML}{C0504D}
\definecolor{ggreen}{HTML}{9BBB59}
\definecolor{ppurple}{HTML}{9F4C7C}

\definecolor{b1}{HTML}{BEE9E8}
\definecolor{b2}{HTML}{188FA7}

\begin{figure}[!t]
\centering
\scalebox{0.85}{
\begin{tikzpicture}
    \begin{axis}[
        width  = 8cm,
        height = 8cm,
        y=0.0015cm,
        x=2.8cm,
        major x tick style = transparent,
        ybar=2*\pgflinewidth,
        bar width=24pt,
        ymajorgrids = true,
        ylabel = {Number of instructions},
        ylabel style={yshift=-4mm},
        symbolic x coords={\batchoverflow,\reentrancy},
        ytick={0,500,1000,1500,2000,2500,3000},
        xtick = data,
        scaled y ticks = false,
        enlarge x limits=0.6,
        ymin=0,
        ymax=3000,
        legend cell align=left,
        legend entries={Approximate, Precise},
        legend style={
                at={(0.5,1.14)},
                legend columns=-1,
                anchor=north,
        %        column sep=1ex
        }
    ]
    %\hspace*{-2mm}
        \addplot[style={ggreen,fill=ggreen,mark=none}]
        coordinates{(\batchoverflow,743)(\reentrancy,451)};

        \addplot[style={ppurple,fill=ppurple,mark=none}]
        coordinates{(\batchoverflow,2976)(\reentrancy,2087)};
    \end{axis}
\end{tikzpicture}
}
% \vspace{-0.2in}
\caption{Average number of instructions evaluated under different settings}
\vspace{-0.2in}
\label{fig:eval-oyente}
\end{figure}

% \definecolor{bblue}{HTML}{0064FF}
\definecolor{rred}{HTML}{C0504D}
\definecolor{ggreen}{HTML}{9BBB59}
\definecolor{ppurple}{HTML}{9F4C7C}

\definecolor{b1}{HTML}{BEE9E8}
\definecolor{b2}{HTML}{188FA7}

\begin{figure}[!t]
\centering
\scalebox{0.85}{
\begin{tikzpicture}
    \begin{axis}[
        width  = 8cm,
        height = 8cm,
        y=0.03cm,
        x=2.8cm,
        major x tick style = transparent,
        ybar=2*\pgflinewidth,
        bar width=24pt,
        ymajorgrids = true,
        ylabel = {Running time in seconds},
        ylabel style={yshift=-4mm},
        symbolic x coords={\batchoverflow,\reentrancy},
        ytick={0,30,60,90,120},
        xtick = data,
        scaled y ticks = false,
        enlarge x limits=0.6,
        ymin=0,
        ymax=120,
        legend cell align=left,
        legend entries={Approximate, Precise},
        legend style={
                at={(0.5,1.14)},
                legend columns=-1,
                anchor=north,
        %        column sep=1ex
        }
    ]
    %\hspace*{-2mm}
        \addplot[style={ggreen,fill=ggreen,mark=none}]
        coordinates{(\batchoverflow,9)(\reentrancy,12)};

        \addplot[style={ppurple,fill=ppurple,mark=none}]
        coordinates{(\batchoverflow,99)(\reentrancy,67)};
    \end{axis}
\end{tikzpicture}
}
% \vspace{-0.2in}
\caption{Average running time under different settings}
\vspace{-0.2in}
\label{fig:eval-oyente}
\end{figure}


\begin{table}[]
\begin{tabular}{|l|l|l|l|l|}
\hline
\multicolumn{1}{|c|}{\multirow{2}{*}{Statistics}} & \multicolumn{2}{l|}{\reentrancy} & \multicolumn{2}{l|}{\batchoverflow} \\ \cline{2-5} 
\multicolumn{1}{|c|}{}                         & $\vulnerability$             & $\vulnerability^{\diamond}$             & $\vulnerability$                & $\vulnerability^{\diamond}$               \\ \hline \hline  
Avg time (seconds)                                          &      9          &    52            &         12         &      67           \\ \hline
Avg \#instructions      &     451           &    1762            &         743         &      2087           \\ \hline
FPs                                             &           3\%     &       1\%         &    8\%              &         5\%        \\ \hline
FNs                                             &         7\%       &          23\%      &   5\%               &       32\%          \\ \hline
\end{tabular}
  \caption{Effectiveness of summary-based evaluation under queries of different granularity.}
%   \vspace{-0.3in}
  \label{tbl:summary}
\end{table}


Table~\ref{tbl:summary} shows the results of running \toolname with 
different settings and a time limit of 10 minutes. In particular, for the \reentrancy client, the approximate query $\vulnerability$ discussed in Section~\ref{sec:vul} has an average running time of 9 seconds while a precise query $\vulnerability^{\diamond}$ has to run for 52 seconds on average. Looking into the results closely, the approximate query $\vulnerability$ significantly reduces the size of the summaries. Specifically, $\vulnerability$ evaluates about 451 instructions on average while $\vulnerability^{\diamond}$ has to evaluate 1762 instructions on average. As a result, $\vulnerability^{\diamond}$ generates more false negatives (i.e., 23\% vs. 7\%) because it fails to explore enough search space within the time limit. On the other hand, although $\vulnerability$ is less precise, its false positive rate is very close to the one in $\vulnerability^{\diamond}$ (i.e., 3\% vs. 1\%). In the \batchoverflow client, we observe a similar trend.
\looseness=-1

% Each dot in the 
% figure represents the pairwise running time of a specific benchmark under different 
% settings; a dot near the diagonal indicates that the performance of 
% two settings is similar. Our summary-based symbolic
% evaluation significantly outperforms the baseline (i.e., non-summary) in the vast majority of benchmarks.
% As shown in Table~\ref{fig:summary},  if we exclude the benchmarks that timeout
% in 10 minutes, the mean time of our summary-based symbolic evaluation is only 8
% seconds, while it takes 35 seconds without computing the summary. Furthermore,
% 1846 benchmarks time out for both settings, and only 548 benchmarks time out on
% $S^{\dagger}$ but not on $S^{\diamond}$. However, without computing the summary,
% 17454 (i.e., 69.8\%) benchmarks time out. The result confirms that the
% summary-based technique is key to the efficiency of \toolname.\looseness=-1
% the running time of each benchmark under two different settings, and 
% the x-axis and y-axis represent the running time with and without computing 
% the summary, respectively. In particular, without computing the summary, on average
% it takes our tool \todo{XX} seconds to solve each benchmark and \todo{YY}
% benchmarks fail to terminate within a \todo{6} mins timeout. On the other hand, 
% the performance of our summary-based technique \emph{significantly} outperforms
% the baseline, and on average it takes only \todo{16} to solve each benchmark
% and timeouts on \todo{XXX} benchmarks.

\fbox{
\begin{minipage}{0.9\linewidth}
{\bf Result for RQ2:}
Summary-based technique is key to the efficiency of \toolname. In the meantime, the approximate queries enable \toolname to generate smaller summaries, which leads to much better scalability in exchange of a minor loss in precision.
\end{minipage}
} \\

% \subsection{A case study on the BatchOverflow vulnerability}\label{sec:case}

To evaluate whether \toolname can express and discover new vulnerabilities in real world
smart contracts, we conduct a case study on the recent \batchoverflow
vulnerability. Exploits
due to this vulnerability have resulted in the creation of trillions of invalid
Ethereum Tokens in 2018~\cite{batch-news}, causing major exchanges to temporary
halt until all tokens could be reassessed. Note that generating exploits for
this vulnerability is quite challenging as it requires the tool to reason about
the combination of arithmetic operations, interference, and the read-write
semantics of the storage system in Solidity. 
% For instance, existing tools such
% as \oyente and \madmax~\cite{madmax} will simply mark a large number of arithmetic
% operations as \emph{potentially vulnerable}, and it turns out that most of the
% alarms are not exploitable.

Similar to our previous experiment, we first encode the \batchoverflow vulnerability
(Section~\ref{sec:vul}) in our query language and then run our tool on the 
\etherscan data set. In total, \toolname flags 16 vulnerable
contracts. To verify that the exploits are effective, we setup a private
blockchain using the Geth~\cite{geth} framework where we can run exploits on the
vulnerable contracts. We confirmed that 9 exploits are valid. 
The infeasible attacks come from the incompleteness of the query 
as well as imprecise control flow graphs from the Vandal decompiler. 
Running \teether on these 9 vulnerable contracts, we find 
that it fails to generate their exploits.

% \todo{To evaluate the effectiveness of \toolname on vetting the \batchoverflow vulnerability, we compare \toolname against \teether~\cite{teether} tool, the most recent
% tool using dynamic symbolic execution for generating exploits that would enable
% the attacker to control the money transactions of a victim contract. In
% particular, the \teether tool looks for so-called \emph{critical instructions}
% (i.e., \texttt{call}, \texttt{selfdestruct}, etc.) that include recipients' addresses, 
% which can be manipulated by the attacker. In the end, \teether takes an average of 31 seconds to analyze the \etherscan data set but fails to generate exploits for those 9 vulnerable contracts. The false negative rate in \teether is caused by  
% attack programs that require more than three method calls, or victim programs with over 3000 lines of source code with complex control flow.
% As a result, the \teether tool fails to explore sufficiently many \emph{concrete traces} to 
% find the exploits.}  


% !TEX root =  main.tex
\section{Related Work}
Smart contract security has been extensively studied in recent years. 
This section briefly discusses prior closely related work.

\paragraph{Smart Contract Analysis}
Many popular security analyzers for smart contracts are based on symbolic
execution~\cite{symbolic-e}. Well-known tools include Oyente~\cite{oyente},
Mythril~\cite{mythril} and Manticore~\cite{manticore}. Their key idea is to find
an execution path that satisfies a given property or assertion. While \toolname
also uses symbolic evaluation to search for attack programs, our system differs
from these tools in two ways. First, the prior tools adopt symbolic execution
for \emph{bug finding}. Our tool can be used not only for bug finding but also
for \emph{exploit generation}. Second, while symbolic execution is a powerful
and precise technique for finding security vulnerabilities, it does not
guarantee to explore all possible paths, which leads to false negative rates as
shown in Section~\ref{sec:teether}. In contrast, \toolname analyzes all
(bounded) paths through a contract using summary-based symbolic evaluation,
which significantly reduces the number of paths that the underlying Rosette
engine has to execute symbolically while maintaining the same precision.

To address the scalability and path explosion problems in symbolic execution,
researchers developed sound and scalable static
analyzers~\cite{ecf,securify,madmax,zeus}. Both Securify~\cite{securify} and
Madmax~\cite{madmax} are based on abstract interpretation~\cite{CousotC77},
which soundly overapproximates and merges execution paths to avoid path
explosion. The ZEUS~\cite{zeus} system takes the source code of a smart contract
and a policy as inputs, and then compiles them into LLVM IRs that will be
checked by an off-the-shelf verifier~\cite{smack}. The ECF~\cite{ecf} system is
designed to detect the DAO vulnerability. Similar to our tool, Securify also
provides a query language to specify the patterns of common vulnerabilities.
Unlike our tool, none of these systems can generate exploits. We could not
directly compare \toolname with Zeus as the tool and benchmarks
are not publicly available. However, we note that our system is complementary to
existing static analyzers such as Securify: in particular, we can use Securify
to filter out safe smart contracts and leverage \toolname to generate exploits
for vulnerable ones.

Some systems~\cite{Hirai17,GrishchenkoMS18,kframework} for reasoning about smart contracts
rely on formal verification. These systems prove security properties of smart
contracts using existing interactive theorem provers~\cite{coq}. They
typically offer strong guarantees that are crucial to smart contracts. However,
unlike our system, all of them require significant manual effort to encode the
security properties and the semantics of smart contracts.

% Finally, reverse engineering projects~\cite{porosity, madmax, Erays} aim to to
% lift EVM bytecode to an intermediate representation that is easy to analyze.
% Although \toolname uses the IRs from Vandal~\cite{madmax}, our technique is
% agnostic to the underlying language.

\paragraph{Automatic Exploitation}
Our work is also closely related to automatic
exploitation~\cite{AvgerinosCHB11,ChaARB12,teether,contractfuzzer}. While  
prior systems rely on constraint solvers to generate counterexamples as
potential exploits, we note that there are additional challenges in automatic
exploitation for smart contracts. First, the exploits in classical
vulnerabilities (e.g., buffer overflows, SQL injections) are typically program
inputs of a specific data type (e.g., integer, string) whereas the exploits in
our setting are adversarial smart contracts that faithfully model the execution
environment (storage, gas, etc.) of the EVM. Second, Keccak-256 hash is
ubiquitous in smart contract for accessing addresses in memory or storage. As
shown in Section~\ref{sec:teether}, basic symbolic execution will fail to
resolve the Keccak-256 hash, resulting in poor coverage. To address this
problem, the \teether~\cite{teether} system proposed a novel algorithm to infer
the memory addresses encoded as Keccak-256 hash. Unlike \teether, our system
directly synthesizes function calls that manipulate the memory and storage thus
avoids expensive computation to resolve the hash values. Our evaluation in
Section~\ref{sec:teether} shows that \toolname outperforms the \teether tool in
terms of both running time and false negatives.  
% The original teEther also does not mention its running time for getting the 
% exploits. 
Similar to \toolname, \contractfuzz~\cite{contractfuzzer} also generates exploits for 
a limited class of vulnerabilities based on the ABI specifications of smart contracts.
However, as shown in Section~\ref{sec:fuzz}, since \contractfuzz is 
based on random input generation, it is an order of magnitude slower than \toolname, 
resulting in many missed exploits compared to \toolname. Its assertion language is 
also less expressive than ours, leading to false positives that 
\toolname avoids.\looseness=-1

\paragraph{Symbolic Evaluation}
\toolname builds on the Rosette~\cite{rosette} symbolic evaluation engine with a
new summary-based technique for scaling symbolic evaluation to large programs in
the domain of smart contracts. As shown in Section~\ref{sec:expr}, this
technique is critical for performance. The idea of computing summaries to speed
up symbolic evaluation has also been explored in the context of symbolic
execution (see~\cite{BaldoniCDDF18} for a survey), leading to three main
approaches~\cite{AnandGT08,Godefroid07,BoonstoppelCE08}. Two of these
approaches~\cite{Godefroid07,AnandGT08} compute summaries path-by-path, so a
full summary that encodes all (bounded) paths through a program would be, in the
worst case, exponential in program size. Prior tools therefore avoid computing
full summaries, instead summarizing a subset of all paths for the purpose of
test generation. \toolname, in contrast, summarizes all (bounded) paths through
a procedure, and produces compact (polynomially-sized) summaries by employing 
a symbolic evaluator~\cite{rosette} that combines symbolic execution and bounded model
checking. Another summarization approach~\cite{BoonstoppelCE08} uses a caching scheme
that lets the underlying symbolic execution engine terminate the exploration of a
path as soon as it reaches a previously seen state. The scheme does not compute
explicit summaries of code; instead, it only stores enough information to
soundly decide when the symbolic execution of a path reaches a previously seen
state. In contrast, our approach computes an explicit and precise summary of a
procedure's semantics.  

\paragraph{Program Synthesis} 
\toolname uses syntax-guided synthesis~\cite{sygus} to search for attack
programs. Synthesizers of this kind (see~\cite{synthesis-survey} for a survey)
rely on either enumerative search (which can be stochastic or exhaustive) or
symbolic reasoning or a combination of the two. \toolname combines exhaustive
enumeration with symbolic synthesis (Section~\ref{sec:rosette}), and extends
this with a parallel symbolic evaluation technique (Section~\ref{sec:parallel}) for fast enumeration. Both optimizations are specialized to
the domain of smart contracts, and they are critical for performance: disabling
them renders the system unusable.\looseness=-1

% !TEX root =  main.tex
\section{Conclusion}\label{sec:concl}
This paper presented \toolname, a tool for automatic synthesis of adversarial
contracts that exploit vulnerabilities in a victim smart contract. To make
synthesis tractable, \toolname introduces \emph{summary-based symbolic
evaluation}, which enables our tool to perform precise all-paths analysis of
large real-world contracts, while significantly reducing the number of paths
that need to be executed symbolically. \toolname also introduces optimizations
to partition the synthesis search space for parallel exploration. %and to prune infeasible attack candidates earlier. 
Evaluating \toolname on the entire
\etherscan data set, we find that it significantly outperforms  state-of-the-art
analyzers in terms of precision and execution time.\looseness=-1

\section*{Acknowledgements}
%
This work has been supported in part by 
% NSF
the NSF Grants CCF-1651225, ACI OAC--1535191, FMitF CCF-1918027, OIA-1936731, SaTC-1908494,
%
by the Intel and NSF joint research center for Computer Assisted Programming for Heterogeneous Architectures (CAPA NSF CCF-1723352), 
%
the CONIX Research Center, one of six centers in JUMP, a Semiconductor Research Corporation (SRC) program sponsored by DARPA CMU 1042741-394324 AM01,
%
grants from DARPA FA8750--14--C--0011 and DARPA FA8750--16--2--0032, 
%
as well as gifts from Adobe, Facebook, Google, Intel, and Qualcomm. 
%
%
% ---- Bibliography ----
%
% BibTeX users should specify bibliography style 'splncs04'.
% References will then be sorted and formatted in the correct style.
%
\bibliographystyle{splncs04}
% \bibliography{mybibliography}
%
\bibliography{main}
\end{document}
