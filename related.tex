% !TEX root =  main.tex
\section{Related Work}
Smart contract security has been extensively studied in recent years. 
This section briefly discusses prior closely related work.

\paragraph{Smart Contract Analysis}
Many popular security analyzers for smart contracts are based on symbolic
execution~\cite{symbolic-e}. Well-known tools include Oyente~\cite{oyente},
Mythril~\cite{mythril} and Manticore~\cite{manticore}. Their key idea is to find
an execution path that satisfies a given property or assertion. While \toolname
also uses symbolic evaluation to search for attack programs, our system differs
from these tools in two ways. First, the prior tools adopt symbolic execution
for \emph{bug finding}. Our tool can be used not only for bug finding but also
for \emph{exploit generation}. Second, while symbolic execution is a powerful
and precise technique for finding security vulnerabilities, it does not
guarantee to explore all possible paths, which leads to false negative rates as
shown in Section~\ref{sec:teether}. In contrast, \toolname analyzes all
(bounded) paths through a contract using summary-based symbolic evaluation,
which significantly reduces the number of paths that the underlying Rosette
engine has to execute symbolically while maintaining the same precision.

To address the scalability and path explosion problems in symbolic execution,
researchers developed sound and scalable static
analyzers~\cite{ecf,securify,madmax,zeus}. Both Securify~\cite{securify} and
Madmax~\cite{madmax} are based on abstract interpretation~\cite{CousotC77},
which soundly overapproximates and merges execution paths to avoid path
explosion. The ZEUS~\cite{zeus} system takes the source code of a smart contract
and a policy as inputs, and then compiles them into LLVM IRs that will be
checked by an off-the-shelf verifier~\cite{smack}. The ECF~\cite{ecf} system is
designed to detect the DAO vulnerability. Similar to our tool, Securify also
provides a query language to specify the patterns of common vulnerabilities.
Unlike our tool, none of these systems can generate exploits. We could not
directly compare \toolname with Zeus as the tool and benchmarks
are not publicly available. However, we note that our system is complementary to
existing static analyzers such as Securify: in particular, we can use Securify
to filter out safe smart contracts and leverage \toolname to generate exploits
for vulnerable ones.

Some systems~\cite{Hirai17,GrishchenkoMS18,kframework} for reasoning about smart contracts
rely on formal verification. These systems prove security properties of smart
contracts using existing interactive theorem provers~\cite{coq}. They
typically offer strong guarantees that are crucial to smart contracts. However,
unlike our system, all of them require significant manual effort to encode the
security properties and the semantics of smart contracts.

% Finally, reverse engineering projects~\cite{porosity, madmax, Erays} aim to to
% lift EVM bytecode to an intermediate representation that is easy to analyze.
% Although \toolname uses the IRs from Vandal~\cite{madmax}, our technique is
% agnostic to the underlying language.

\paragraph{Automatic Exploitation}
Our work is also closely related to automatic
exploitation~\cite{AvgerinosCHB11,ChaARB12,teether,contractfuzzer}. While  
prior systems rely on constraint solvers to generate counterexamples as
potential exploits, we note that there are additional challenges in automatic
exploitation for smart contracts. First, the exploits in classical
vulnerabilities (e.g., buffer overflows, SQL injections) are typically program
inputs of a specific data type (e.g., integer, string) whereas the exploits in
our setting are adversarial smart contracts that faithfully model the execution
environment (storage, gas, etc.) of the EVM. Second, Keccak-256 hash is
ubiquitous in smart contract for accessing addresses in memory or storage. As
shown in Section~\ref{sec:teether}, basic symbolic execution will fail to
resolve the Keccak-256 hash, resulting in poor coverage. To address this
problem, the \teether~\cite{teether} system proposed a novel algorithm to infer
the memory addresses encoded as Keccak-256 hash. Unlike \teether, our system
directly synthesizes function calls that manipulate the memory and storage thus
avoids expensive computation to resolve the hash values. Our evaluation in
Section~\ref{sec:teether} shows that \toolname outperforms the \teether tool in
terms of both running time and false negatives.  
% The original teEther also does not mention its running time for getting the 
% exploits. 
Similar to \toolname, \contractfuzz~\cite{contractfuzzer} also generates exploits for 
a limited class of vulnerabilities based on the ABI specifications of smart contracts.
However, as shown in Section~\ref{sec:fuzz}, since \contractfuzz is 
based on random input generation, it is an order of magnitude slower than \toolname, 
resulting in many missed exploits compared to \toolname. Its assertion language is 
also less expressive than ours, leading to false positives that 
\toolname avoids.\looseness=-1

\paragraph{Symbolic Evaluation}
\toolname builds on the Rosette~\cite{rosette} symbolic evaluation engine with a
new summary-based technique for scaling symbolic evaluation to large programs in
the domain of smart contracts. As shown in Section~\ref{sec:expr}, this
technique is critical for performance. The idea of computing summaries to speed
up symbolic evaluation has also been explored in the context of symbolic
execution (see~\cite{BaldoniCDDF18} for a survey), leading to three main
approaches~\cite{AnandGT08,Godefroid07,BoonstoppelCE08}. Two of these
approaches~\cite{Godefroid07,AnandGT08} compute summaries path-by-path, so a
full summary that encodes all (bounded) paths through a program would be, in the
worst case, exponential in program size. Prior tools therefore avoid computing
full summaries, instead summarizing a subset of all paths for the purpose of
test generation. \toolname, in contrast, summarizes all (bounded) paths through
a procedure, and produces compact (polynomially-sized) summaries by employing 
a symbolic evaluator~\cite{rosette} that combines symbolic execution and bounded model
checking. Another summarization approach~\cite{BoonstoppelCE08} uses a caching scheme
that lets the underlying symbolic execution engine terminate the exploration of a
path as soon as it reaches a previously seen state. The scheme does not compute
explicit summaries of code; instead, it only stores enough information to
soundly decide when the symbolic execution of a path reaches a previously seen
state. In contrast, our approach computes an explicit and precise summary of a
procedure's semantics.  

\paragraph{Program Synthesis} 
\toolname uses syntax-guided synthesis~\cite{sygus} to search for attack
programs. Synthesizers of this kind (see~\cite{synthesis-survey} for a survey)
rely on either enumerative search (which can be stochastic or exhaustive) or
symbolic reasoning or a combination of the two. \toolname combines exhaustive
enumeration with symbolic synthesis (Section~\ref{sec:rosette}), and extends
this with a parallel symbolic evaluation technique (Section~\ref{sec:parallel}) for fast enumeration. Both optimizations are specialized to
the domain of smart contracts, and they are critical for performance: disabling
them renders the system unusable.\looseness=-1
