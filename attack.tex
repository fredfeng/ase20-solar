\subsection{Attack Synthesis}\label{sec:attack}

Given a vulnerability query, we are interested in synthesizing an attack 
program that can exploit this vulnerability in a victim contract. 
The basic building blocks of an attack program are called 
\emph{components}, and each component $\comp$ corresponds to a public method provided by the  
victim contract. We use $\comps$ to denote the union of all publicly available 
methods.

\begin{definition}{{(\bf Component)}}
A \emph{Component} $\comp$ from an ABI configuration is a pair $(f, \tau)$ where: 1) $f$ is $\comp$'s name, and 2) $\tau$ is the type signature of $\comp$.
% \begin{itemize}
% \item $f$ is $\comp$'s name. 
% \item $\tau$ is the type signature of $\comp$.
% \end{itemize}
\end{definition}

\begin{example}
Consider the ABI configuration in Figure~\ref{fig:batchcode}. Its first element 
declares a component for the problematic \texttt{batchTransfer} method. 
This component takes inputs as an 
array of \texttt{address} and a 256-bit integer (\texttt{uint256}). 
\end{example}


%Given a component $\comp(t_1,...,t_n)$ where $t_i$ are symbolic values, we write
%$\peval{\comp(t_1,...,t_n)}$ to represent the result of executing $\comp(t_1,...,t_n)$ on
%program state $\pstate$.

We represent a set of candidate attack programs as a \emph{symbolic program}, 
which is a sequence of \emph{holes} to be filled with components from $\comps$. 
The synthesizer fills these holes to obtain 
a \emph{concrete program} that exploits a given vulnerability.
\begin{definition}{{(\bf Symbolic Attack Program)}}\label{def:symbolic-attack}
Given a set of components $\comps=\{(f_1,\tau_1),\ldots,(f_N,\tau_N)\}$, a \emph{symbolic attack program} $\sketch$ 
for $\comps$ is a sequence of \emph{statement holes} of the form\looseness=-1
$$
\mathtt{choose}(f_1({\vec{v}_{\tau_1}}), \ldots, f_N({\vec{v}_{\tau_N}}));
$$
where $f_i({\vec{v}_{\tau_i}})$ stands for the application of the
$i$-th component to fresh symbolic values of types specified by $\tau_i$. 
\end{definition}
% Our synthesis procedure starts with the most general sketch $\sketch$ and 
% iteratively refines it to obtain concrete programs. Specifically, we express sketch using 
% a \emph{refinement list} whose elements and size can change dynamically. 
% Adequately, the list represents a partial program in our sketch
% language of which context-free grammar is defined in Figure~\ref{fig:sketch}.
% According to the grammar, we can apply two different rules on a sketch $\sketch$,
% namely, \emph{instantiation}, whose goal is to pick a component $\comp$ from $\comps$ 
% and set all the arguments to be symbolic, and \emph{expansion}, whose goal is to 
% expand current sketch. 
\begin{definition}{{(\bf Concrete Attack Program)}}
A \emph{concrete attack program} for a symbolic program $\sketch$ 
replaces each hole in $\sketch$ with one of the specified function calls, 
and each symbolic argument to a function call is replaced with a concrete value.
\end{definition}
\begin{example}
Here is a symbolic program that captures the attack candidate in
Fig~\ref{fig:batchcode}:
\begin{lstlisting}[numbers=none,frame=none,basicstyle=\footnotesize\ttfamily]
choose(makeFlag($x_1$), batchTransfer($y_1$,$z_1$));  
choose(makeFlag($x_2$), batchTransfer($y_2$,$z_2$)); 
\end{lstlisting}
And here is a concrete attack program for this symbolic attack:  
\begin{lstlisting}[numbers=none,frame=none,basicstyle=\footnotesize\ttfamily]
makeFlag(true); 
batchTransfer([0x123,0x345], $2^{256}-1$);  
\end{lstlisting}

\end{example}

The $\mathtt{choose}$ construct is a notational shorthand for a conditional statement that guards the specified choices with fresh
symbolic booleans. For example, $\mathtt{choose}(e_1, e_2)$ stands for the statement $\mathtt{if}\ b_1\ \mathtt{then}\ e_1\
\mathtt{else}\ e_2$, where $b_1$ is a fresh symbolic boolean value. A concrete
attack program therefore substitutes concrete values for the implicit $\mathtt{choose}$ guards and the
explicit function arguments of a symbolic attack program. 


% \begin{figure}[t]
% \[
% \begin{array}{lll}
% {\rm Term} \ t:\tau & := & (\texttt{define-symbolic freshId }  \tau?)  \\
% {\rm Sketch} \ \sketch & := &   \sketch \ | \ \sketch; \sketch \ \ {\rm -expansion}\\
% & & | \ \comp(t_1:\tau_1,...,t_n:\tau_n) \ \ (\comp \in \comps) \ \ {\rm -instantiation}
% \end{array}
% \]
% \caption{Context-free grammar for sketch}\label{fig:sketch}
% \end{figure}

% \begin{figure}
% \lstinputlisting[language=Java]{code.sol}
% \caption{A Smart Contract written in Solidity.}
% \label{fig:code}
% \end{figure}

% \begin{figure}
% \lstinputlisting[language=Java]{abi.json}
% \caption{Contract Application Binary Interface (ABI) for the example in Figure~\ref{fig:code}.}
% \label{fig:abi}
% \end{figure}



%Since attack programs are valid programs in our language, 
%their semantics is given by the rules in Figure~\ref{fig:sem}. 
%\begin{definition}{{(\bf State Transition over symbolic program $\sketch$)}}
%  Since a symbolic program $\sketch$ is a sequence of statements, a
%  \emph{State Transition} $\transition^{*}$ over $\sketch$ is obtained 
%  by sequentially invoking the rules in Figure~\ref{fig:sem}.
%  We denote the transition using $\pstate \reach_{\sketch} \pstate'$. 
%\end{definition} 

%We write $\peval{\sketch}$ to denote the result of executing 
%a symbolic attack program $\sketch$ from the program state $\pstate$.
%The result $\peval{\sketch}$
%represents the set of states reachable by all concrete programs for $\sketch$ starting from the state $\pstate$. 
The goal of attack synthesis is to  
find a concrete program $P$ for a given symbolic program $\sketch$ 
such that $P$ reaches a state satisfying a desired vulnerability query.


\begin{definition}{{(\bf Problem Specification)}}
The specification for our \emph{attack synthesis} problem is a tuple ($\pstate_0$, $\vulnerability$, $\sketch$) where:
\begin{itemize}
\item $\sketch$ is a symbolic attack program for the set of components $\comps$
of a victim contract $\victim$.
\item $\pstate_0$ is the initial state of the symbolic attack program, obtained by executing the victim's initialization code.
\item $\vulnerability$ is a first-order formula over the (symbolic) program state
$\peval{\sketch}$ reachable from $\pstate_0$ by the attack program $\sketch$.
\end{itemize}
\end{definition}
\begin{definition}{{(\bf Attack Synthesis)}}
Given a specification ($\pstate_0$, $\vulnerability$, $\sketch$), the 
\emph{Attack Synthesis problem} is to find a \emph{concrete attack program} $P$ for
$\sketch$ such that: 1) $\geval{ P }_{\pstate_0} = \pstate$, and 2) $\pstate \models \vulnerability$.
In other words, executing 
% the concrete attack 
$P$ from the initial state $\pstate_0$
results in a program state $\pstate$ that satisfies $\vulnerability$.
\end{definition}
