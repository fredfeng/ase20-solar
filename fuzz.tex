\paragraph{\bf{Comparison with \contractfuzz}}\label{sec:fuzz}
We further compared \toolname against \contractfuzz~\cite{contractfuzzer}, a recent 
smart contract analyzer based on dynamic fuzzing. 
\contractfuzz takes as input the ABI interfaces of smart
contracts and \emph{randomly} generates inputs invoking the public methods 
provided by the ABI. To verify the correctness of the exploits, \contractfuzz
implements oracles for different vulnerabilities by instrumenting
the Ethereum Virtual Machine (EVM) with extra assertions.

% \begin{figure}
%   \centering
%   \begin{subfigure}[b]{0.22\textwidth}
%     \includegraphics[width=\textwidth]{time-fuzz.pdf}
%     \caption{Timestamp Dependency}
%     \label{fig:eval-fuzz-time}
%   \end{subfigure}
%   %
%   \begin{subfigure}[b]{0.22\textwidth}
%     \includegraphics[width=\textwidth]{gasless.pdf}
%     \caption{Gasless Vulnerability}
%     \label{fig:eval-fuzz-gas}
%   \end{subfigure}
% \caption{Comparing \toolname against \contractfuzz}% on obfuscated apps}
% \label{fig:eval-fuzz}
% \end{figure}

\begin{table}[]
\centering
\begin{tabular}{|l|l|l|l|l|l|l|}
\hline
\multirow{2}{*}{Vulnerability} & \multicolumn{3}{l|}{\toolname} & \multicolumn{3}{l|}{\contractfuzz} \\ \cline{2-7} 
                               & No.       & FP       & FN      & No.        & FP        & FN        \\ \hline
Timestamp                      & 16        & 0        & 1       & 13         & 3         &7         \\ \hline
Gasless Send                   & 17        & 0        & 0       & 14         & 3         & 6         \\ \hline
Bad Random                     & 9        & 0        & 0       & 5          & 1         & 5         \\ \hline
\end{tabular}
\caption{Comparing \toolname against \contractfuzz}% on obfuscated apps}
\vspace{-0.2in}
\label{fig:eval-fuzz}
\end{table}

We use the docker image~\cite{fuzz-docker} provided by the author of \contractfuzz.
The original paper does not discuss the performance of the tool, 
but from our experience, \contractfuzz is slow, taking more than 
10 mins to fuzz a smart contract. Since it would be time-consuming to 
run \contractfuzz on the \etherscan data set, we evaluate both tools 
on the 33 benchmarks from the \contractfuzz artifact~\cite{fuzz-data} plus another
67 random samples from \etherscan for which we know the ground truth.

\paragraph{Summary of results}
The results of our evaluation are summarized in Table~\ref{fig:eval-fuzz}.
For the timestamp dependency, \contractfuzz flags 13 benchmarks 
as vulnerable. However, 3 of them are false alarms, and \contractfuzz fails to detect 7 
vulnerable benchmarks. On the other hand, \toolname detects most of 
the benchmarks with only one false negative, which is caused by a timeout of the Vandal
decompiler~\cite{madmax}.\looseness=-1

Similarly, for the Gasless-send vulnerability, 14 benchmarks are flagged by \contractfuzz.
However, 3 of them are false positives, and 6 vulnerable benchmarks can not be detected 
within 10 minutes. In contrast, \toolname successfully generates exploits for 
all the vulnerable benchmarks.

\paragraph{Performance}
On average, \contractfuzz takes 10 mins to analyze a smart contract. 
\toolname takes an average of 11 seconds on this data set.

\paragraph{Discussion}
The cause of false negatives in \contractfuzz is easy to understand as it is
based on random, rather than exhaustive, exploration of an extremely large
search space. So if there are relatively few inputs in this space that lead to
an attack, \contractfuzz is unlikely to find one in reasonable time. The false
positives in \contractfuzz are caused by the limited expressiveness of its
assertion language. For instance, the Time Dependency is defined as the
following assertion in \contractfuzz: 
\[
\textbf{TimestampOp} \wedge (\textbf{SendCall} \vee \textbf{EtherTransfer})
\]
The assertion raises a Time Dependency vulnerability if the smart contract
contains the \texttt{timestamp} and \texttt{call} instructions. It is
easy to raise false alarms with this assertion if the \texttt{call} instruction 
does not depend on \texttt{timestamp}. 
%On the other hand, the \texttt{interfere?} function enables \toolname to reason about this dependency precisely.

\fbox{
\begin{minipage}{0.9\linewidth}
{\bf Result for RQ1:}
\toolname outperforms three state-of-the-art analyzers in terms of running time, false positives, and false negatives.
\end{minipage}
} \\