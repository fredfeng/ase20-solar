% !TEX root =  main.tex
\section{Summary-based Symbolic Evaluation}\label{sec:sum}

Solving the attack synthesis problem involves searching for a concrete program
$P$ in the space of candidate attacks defined by a symbolic program $\sketch$.
\toolname delegates this search to an off-the-shelf SMT solver, by using
symbolic evaluation to reduce the attack synthesis problem to a satisfiability
query. Given a specification  $(\pstate_0, \vulnerability, \sketch)$,  \toolname
evaluates $\sketch$ on the state $\pstate_0$ to obtain the state
$\geval{\sketch}_{\pstate_0}$, and then uses the solver to check the
satisfiability of the  formula $\exists \vec{v} .
\vulnerability(\geval{\sketch}_{\pstate_0})$, where $\vec{v}$ denotes the
symbolic variables in $\sketch$. A model of this formula, if it exists, binds
every variable in $\vec{v}$ to a concrete value, and so represents a concrete
attack program $P$ for $\sketch$ that triggers the vulnerability
$\vulnerability$.\looseness=-1
\begin{figure}[!t]
    \centering
  \begin{minipage}{0.5\linewidth}
\begin{lstlisting}[linewidth=8cm] 
(define (get-summary s $\pc$)
  (match s
    [transfer(x, y, z) $\sumi{transfer}$($\pstate_S(x)$, $\pstate_S[y]$, $\pstate_S[z]$)@$\pc$]
    [sstore(x, y)  $\sumi{sstore}$(x, $\pstate_S[y]$)@$\pc$]
    [_             #f]))
		\end{lstlisting}
		\end{minipage}
		\vspace{-0.1in}
    \caption{Procedure for summary generation. }
    \vspace{-0.1in}
      \label{fig:sum-gen}
\end{figure}
But computing $\geval{\sketch}_{\pstate_0}$ is expensive as it relies on
symbolic evaluation~\cite{rosette}. In particular, evaluating a
$\mathtt{choose}$ statement in $\sketch$ involves symbolically evaluating each
function call in that statement. So, for a symbolic program of length $K$, every
public function in the victim contract must be symbolically executed $K$ times
on different symbolic arguments. As we will see in section~\ref{sec:eval}, this
direct approach to evaluating $\sketch$ does not scale to real contracts that
contain a large number of complex public functions. To mitigate this issue, we
use a summary-based symbolic evaluation that performs symbolic execution of each
public method only once. 

Our approach is based on the following insight. An attack program performs a
sequence of transactions---i.e., method invocations---that manipulate the
victim's persistent storage and global properties. The transactions that
comprise an attack exchange data and influence each other's control flow
exclusively through these two parts of the program state. So, if we can
faithfully summarize the effects of a public method on the persistent storage
and global properties, evaluating this summary on the symbolic arguments passed
to the method is equivalent to symbolically executing the method
itself.\looseness=-1

\begin{definition}
A summary $\summary$ in our system is a pair $s@\pc$ where $s$ represents a
statement that has a side effect on the persistent state (i.e., storage and
global properties) of a smart contract, and $\pc$ denotes the path condition of
executing $s$.
\end{definition}

We generate such faithful method summaries in two steps. First, we evaluate the
method on a program state $\pstate_S$ that maps every state variable (i.e.,
persistent storage location, global property, etc.) to a fresh symbolic variable
of the right type. This step produces a path condition and symbolic inputs for
each instruction that capture every possible way to reach and execute the
instruction within the given method. Next, we use the procedure in
Figure~\ref{fig:sum-gen} to generate the method summary.\footnote{We omit 
the details of other side-effecting instructions for
simplicity.\looseness=-1} Given a storage-store instruction \texttt{sstore(x,y)}
and its path condition, we generate a ``summary sstore" statement (i.e.,
$\sumi{sstore}$) that takes as input the name of the storage variable (i.e.,
$x$) and the symbolic expression $\pstate_S[y]$ held in the register $y$. Similarly,
given a \texttt{call(gas,addr,value)} instruction and path condition, we emit
its ``summary call" statement (i.e., $\sumi{call}$) that takes as input the
symbolic expressions of the instruction's gas consumption, recipient address,
and amount of cryptocurrency, respectively. All other instructions are omitted
from the summary since they have no effect on the persistent state. By
construction, our summary therefore precisely  captures all of the method's
effects on the persistent state, and the summaries are polynomially-sized as guaranteed by Rosette's symbolic evaluator~\cite{rosette}.\looseness=-1

\begin{example}
Recall that we introduce the following code snippet in Figure~\ref{fig:sum-interp}b:
\begin{lstlisting}[linewidth=9cm,frame=none, numbers=none]
  assert(_amount > 0);
  r1 (*\textbf{:=}*) _amount - 1;
  (*\textbf{sstore}*)(vesting.amount, _amount - 1);
  (*\textbf{transfer}*)(msg.sender, _to, _amount - 1);
  r2 (*\textbf{:=}*) amount - 15;
  r3 (*\textbf{:=}*) amount - 15;
  r4 (*\textbf{:=}*) (*\textbf{sload}*)(vesting.startTime);
  (*\textbf{no-op}*);
\end{lstlisting}
Then using the rule in Figure~\ref{fig:sum-gen}, \toolname generates the following summary: 
\begin{lstlisting}[linewidth=9cm,frame=none, numbers=none]
  $\sumi{sstore}(\mathtt{vesting.amount}, \pstate_S[\mathtt{\_amount}] - 1)@(\pstate_S[\mathtt{\_amount}]>0)$;
  $\sumi{transfer}(\pstate_S[\mathtt{msg.sender}], \pstate_S[\mathtt{\_to}], \pstate_S[\mathtt{\_amount}] - 1)@(\pstate_S[\mathtt{\_amount}]>0)$;
\end{lstlisting}
In particular, our
tool summarizes the side effects of the \texttt{call} and \texttt{sstore}
instructions at lines 2 and 3 in Figure~\ref{fig:sum-interp}b, respectively.  The remaining instructions (E.g., statements from line 5 to 8) are
omitted from the summary because they have no persistent side effects.\looseness=-1
\end{example}

Once \toolname generates the summary for each procedure, we still need to adjust the symbolic evaluation engine to cope with the summaries. In particular, given a method summary and a program state $\pstate$, we use the procedure in
Figure~\ref{fig:sum-inter} to reproduce the effects of executing the method
symbolically on $\pstate$ as follows. Recall that we generate the summary by
executing the method on a fully symbolic state $\pstate_S=\{x_1\mapsto
v_1,\ldots, x_n\mapsto v_n\}$, so every path
condition and symbolic expression in the summary is given in terms of the
symbolic variables $v_1,\ldots,v_n$. Our summary interpretation procedure works by
substituting each $v_i$ in an instruction's path condition and inputs with its
corresponding value in $\pstate$, i.e., $\pstate[x_i]$. The resulting instruction summary
$s_{\pstate}@\pc_{\pstate}$ is therefore expressed in terms of $\pstate$, so applying its
side effects $s_{\pstate}$ under the path condition $\pc_{\pstate}$ is equivalent to
executing the instruction $s$ in the original method on the state $\pstate$. 
Since we interpret every instruction in the summary in this way, the
combined effect on the persistent state is equivalent to executing the original
method symbolically on $\pstate$.\looseness=-1

% Because our rules
% for summary generation preserve each instruction's data- and
% control-dependencies (in the symbolic inputs and path condition, respectively),
% executing the summary in this way has the same effect on the persistent state as
% executing the original method symbolically on $E$.\looseness=-1

%Once our system obtains the summary of the original program, it will further apply the 
%abstract interpretation procedure in Figure~\ref{fig:sum-inter} to evaluate its side effects
%over the storage states, using the same operational semantics in Figure~\ref{fig:sem}. The only difference is,
%we also have to stitch the original path condition $\pc$ to its corresponding statement to 
%soundly model the control dependency for each instruction.
\begin{example}
Figure~\ref{fig:sum-interp}d shows an example for interpreting the summary in
Figure~\ref{fig:sum-interp}c by applying the procedure in
Figure~\ref{fig:sum-inter}. Specifically, given an environment $\pstate$ and the
\texttt{transfer} summary at line 2 in Figure~\ref{fig:sum-interp}c, we first
generate an \texttt{if} statement guarded by the path condition $\pc$ in $\pstate$, 
then in the body of the \texttt{if} statement we symbolically
evaluate the \texttt{call} statement in the environment $\pstate$.
\end{example}


%Thanks to the soundness of our summaries, we can state the following theorem:
%\begin{theorem}
% Suppose we have a program $\sketch$, its corresponding summary $\sumi{\sketch}$,
% initial state $\pstate_0$, states $\pstate_1$ and $\pstate_2$ by executing $\sketch$ and 
% $\sumi{\sketch}$, respectively, then if a query $\query$ holds on $\pstate_1$, it will
% also hold on $\pstate_2$.
%\end{theorem}

\begin{figure}[!t]
  \centering
  \begin{minipage}{0.5\linewidth}
\begin{lstlisting}[linewidth=8cm] 
(define (interpret-summary $s$@$\pc$ ${\pstate}$) 
  (define $s_{\pstate}$@$\pc_{\pstate}$ (substitute $s$@$\pc$ ${\pstate}$))
  (match $s_{\pstate}$
    [$\sumi{transfer}(x_{\pstate}, y_{\pstate}, z_{\pstate})$  (when $\pc_{\pstate}$ transfer($x_{\pstate}$, $y_{\pstate}$, $z_{\pstate}$))]
    [$\sumi{sstore}(x, y_{\pstate})$     (when $\pc_{\pstate}$ sstore(x, $y_{\pstate}$))]
    [_  no-op]))
\end{lstlisting}
\end{minipage}
\vspace{-0.1in}
		\caption{Procedure for summary interpretation}
		\vspace{-0.1in}
      \label{fig:sum-inter}
\end{figure}