% !TEX root =  main.tex
\section{Implementation}\label{sec:impl}

This section discusses the design and implementation of 
\toolname, as well as two key optimizations that enable our tool 
to efficiently solve the synthesis attack problem.\looseness=-1


\subsection{Symbolic Computation Using \rosette}\label{sec:rosette}

\toolname leverages \rosette~\cite{rosette} to symbolically search for attack
programs. \rosette is a programming language that provides facilities for
symbolic evaluation.  \rosette programs use
assertions and symbolic values to formulate queries about program behavior,
which are then solved with off-the-shelf SMT solvers. For example, the
\texttt{(solve expr)} query searches for a binding of symbolic variables to
concrete values that satisfies the assertions encountered during the symbolic
evaluation of the program expression \texttt{expr}. \toolname uses the \texttt{solve}
query to search for a concrete attack program.


\begin{figure}
\centering
  \begin{minipage}{0.5\linewidth}
\begin{lstlisting}[linewidth=8cm] 
(define (solar $\vulnerability$  $\comps$ $K$)
 (define program (for/list ([i K]) (apply choose* $\comps$)))
 (define i-pstate (get-initial-state $\comps$))
 (define o-pstate (interpret program i-state))
 (define binding (solve (assert ($\vulnerability$ o-pstate))))
 (evaluate program binding))
\end{lstlisting}
\end{minipage}
\vspace{-0.1in}
\caption{\toolname implementation in \rosette. 
 }
\label{fig:sketch-overview}
\vspace{-0.1in}
\end{figure}

Figure~\ref{fig:sketch-overview} shows the implementation of \toolname in Rosette. 
The tool takes as input a vulnerability specification
$\vulnerability$, the components $\comps$ of a victim program, and a bound $K$ 
on the length of the attack program. Given these inputs, line 2 uses $\comps$ to 
construct a symbolic attack \texttt{program} of length $K$. 
Next, lines 3 runs the victim's initialization code to obtain the 
initial program state, \texttt{i-pstate}, for the attack. 
Then, line 4 evaluates the symbolic attack \texttt{program} on the initial state to 
obtain a symbolic output state, \texttt{o-pstate}. 
Finally, lines 5-6 use the \texttt{solve} query to search for a concrete
attack program that satisfies the vulnerability assertion.

The core of our tool is the \emph{interpreter} for our smart contract language
(Figure~\ref{fig:grammar}), which implements the semantics from the EVM
yellow paper~\cite{evm-yellow}. We use this interpreter to compute the symbolic
summaries of the victim's public methods (Section~\ref{sec:sum}) and to evaluate
symbolic attack programs. The interpreter itself does not implement symbolic
execution; instead, it uses \rosette's symbolic evaluation engine to execute
programs in our language on symbolic values. 

Another key component of \toolname is the \emph{translator} that converts EVM
bytecode into our language (Figure~\ref{fig:grammar}). The translator leverages the Vandal
Decompiler~\cite{madmax} to soundly convert the stack-based EVM bytecode into
its corresponding three-address format in our language. The jump targets are resolved through abstract
interpretation~\cite{CousotC77}.  We use the translator to convert victim
contracts to the \toolname language for attack synthesis. Both the translator and the
interpreter support all the instructions defined in the Ethereum
specification~\cite{yellowpaper}.\looseness=-1



% To use \toolname, a security analyst expresses specifications in Racket as 
% boolean predicates over storage variables. For instance, the time dependency vulnerability
% from section~\ref{sec:vul} can be expressed as follows:
% \begin{lstlisting}
% (define (time-dependent $V_{time}$ $V_{ret}^{call}$)
% (apply ||
%   (for*/list ([x $V_{time}$][y $V_{ret}^{call}$])
%     (interferes? x y))))
% \end{lstlisting}

\subsection{Parallel Synthesis using Hoisting}\label{sec:parallel}

\toolname uses summary-based symbolic evaluation to efficiently reduce attack
synthesis problems to satisfiability queries. But the resulting queries can
still be too difficult %(for both \rosette and the underlying solver) 
to solve
in practice, especially when the victim contract has many public methods. To
further improve performance, \toolname exploits the structure of symbolic attack
programs (Definition~\ref{def:symbolic-attack}) to decompose the single
\texttt{solve} query in Figure~\ref{fig:sketch-overview} into multiple smaller
queries that can be solved quickly and in parallel, without missing any concrete
attacks.\looseness=-1

The basic idea is as follows. Given a set of $N$ components and a bound $K$ on 
the length of the attack, line 2 creates a symbolic attack program of the
following form:
\begin{lstlisting}[numbers=none,frame=none,basicstyle=\footnotesize\ttfamily]
  choose$_1$($f_1({\vec{v_1}_{\tau_1}}), \ldots, f_N({\vec{v_1}_{\tau_N}})$);  
  $\vdots$
  choose$_K$($f_1({\vec{v_K}_{\tau_1}}), \ldots, f_N({\vec{v_K}_{\tau_N}})$);   
\end{lstlisting}
This symbolic attack encodes a set of concrete attacks that can also be
expressed using $N^K$ symbolic programs that fix the choice of the method to
call at each line, but leave the arguments symbolic. So, we can enumerate these
$N^K$ programs and solve the vulnerability query for each of them, instead of
solving the single query at line 5. This approach essentially \emph{hoists} the
symbolic boolean guards out of the \texttt{choose} statements in the original query, and
\toolname explores all possible values for these guards explicitly, rather than
via SMT solving.\footnote{For practical efficiency, our implementation hoists 
the guards to generate $N^K/c$ symbolic programs, where $c$ is  
the number of available cores.} 
As we show in Section~\ref{sec:eval},
hoisting the guards leads to significantly faster synthesis, 
both because it enables parallel solving of the smaller queries, and 
because the smaller queries can be solved quickly.\looseness=-1



% \subsection{SMT-based Early Pruning}\label{sec:prune}

% In addition to hoisting, we also design a simple but effective \emph{early pruning}
% strategy that allows \toolname to prune \emph{infeasible} symbolic programs 
% before executing them. The intuition behind our strategy is that
% all attacks expressible in \toolname (e.g.,~\cite{attack1,attack2,attack3}) invoke at least one public method that 
% manipulates persistent storage and at least one public method that transfers 
% cryptocurrency using the \texttt{call} instruction. In other words,  
% a successful attack executes at least one store instruction followed 
% by at least one \texttt{call} instruction. We express our early pruning strategy using the following 
% \rosette program:\looseness=-1
% % \begin{lstlisting}[xleftmargin=.2\textwidth,linewidth=9.6cm]
% \begin{lstlisting}
% (define (may-store-and-call? p)
%  (solve (exists (list i j)
%   (and (< i j) (= (type p[i]) 'store)
% 	       (= (type p[j]) 'call)))))
% \end{lstlisting}
% This procedure queries the solver to find out if the given symbolic program $p$ contains 
% any concrete attack program that executes a \texttt{call} after a store. 
% This query is much faster to solve than a vulnerability query, so if $p$ contains no 
% feasible candidate, \toolname does not run the vulnerability query for it.\looseness=-1

\subsection{Practical EVM fragment}\label{sec:fragment}
In this section, we briefly illustrate how \toolname handles other challenging EVM fragments.

\paragraph{Loops} Similar to other analyzers that based on symbolic execution, for loops whose iterations are determined by potentially unbounded collections, \toolname will unroll the loops for $K$ time and $K$ is set to 2 by default.

\paragraph{SHA and Storage access}
\toolname directly analyzes the EVM bytecode. For complex data structures such as arrays and maps that are ubiquitous, the address of an element is determined by the follow function:
\[
a[i] := \textit{SHA-256(id(a))} + n \times i
\]
In other words, to access element \texttt{a[i]} whose size is $n$, the address is determined by the
SHA-256 hash of its identifier and size. \toolname leverages uninterpreted functions to model both the SHA-256 and address computation function. I.e., two addresses are the same as long as they share the same array identifier, index, and element size.

\paragraph{Gas consumption} \toolname's program state keeps tracking of the gas usage by accumulating the cost of instructions during symbolic evaluation. If a transaction runs out of gas in the middle of the evaluation,  \toolname terminates the current exploration with an “out of gas” assertion failure.