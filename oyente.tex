% !TEX root =  main.tex
% \section{Comparison with \oyente}\label{sec:oyente}
\paragraph{Comparison with \mythril}\label{sec:oyente}
% \begin{figure}
%   \centering
%   \begin{subfigure}[b]{0.22\textwidth}
%     \includegraphics[width=\textwidth]{time.pdf}
%     \caption{Timestamp Dependency}
%     \label{fig:eval-oyente-time}
%   \end{subfigure}
%   %
%   \begin{subfigure}[b]{0.22\textwidth}
%     \includegraphics[width=\textwidth]{dao.pdf}
%     \caption{Reentrancy}
%     \label{fig:eval-oyente-dao}
%   \end{subfigure}
% \caption{Comparing \toolname against \oyente}% on obfuscated apps}
% \label{fig:eval-oyente}
% \end{figure}

% \begin{table}[]
% \begin{tabular}{|l|l|l|l|l|l|l|l|l|}
% \hline
% \multirow{2}{*}{Vulnerability} & \multicolumn{4}{l|}{\toolname} & \multicolumn{4}{l|}{\oyente} \\ \cline{2-9} 
%                                & No.     & Chk    & FP   & FN   & No.    & Chk   & FP   & FN   \\ \hline
% Timestamp                      & 1245    & 20     & 0    & 1    & 660    & 20    & 18   & 20   \\ \hline
% Reentracy                      & 248     & 20     & 0    & 0    & 90     & 20    & 5    & 3    \\ \hline
% \end{tabular}
% \caption{Comparing \toolname against \oyente}% on obfuscated apps}
% \label{fig:eval-oyente}
% \end{table}

% We first compare with \oyente~\cite{oyente}, which takes as input 
% a smart contract and checks whether there are concrete traces that match
% the tool's predefined security properties. If so, the tool returns a counterexample
% as the exploit. We evaluate \oyente and \toolname on the \etherscan data set, and 
% both systems use a timeout of ten minutes.

We first compare with \mythril~\cite{mythril} by generating exploits for the reentrancy vulnerability. \mythril takes as input 
a smart contract and checks whether there are concrete traces that match
the tool's predefined security properties. If so, the tool returns a counterexample
as the exploit. We evaluate \mythril and \toolname on the \etherscan data set, and 
both systems use a timeout of 10 minutes.\looseness=-1

% The \oyente tool supports four different types of vulnerabilities, namely, 
% call-stack-limit, Timestamp dependency~\cite{attack4}, Reentrancy~\cite{attack1}, 
% and Transaction-Ordering dependency (TOD)~\cite{attack4}. Since the call-stack-limit
% vulnerability had already been fixed by the Solidity team and the TOD vulnerability
% requires synthesizing multiple programs, we will cover the 
% remaining two vulnerabilities.

\paragraph{Summary of results}
% The results of our evaluation are summarized in Table~\ref{fig:eval-oyente}. In particular, for the Timestamp dependency vulnerability, there
% are 485 benchmarks where both tools report a vulnerability and find the exploits.
% 39 benchmarks are flagged as vulnerable by \oyente but \toolname can not find 
% the exploits. We manually inspected the source code of those benchmarks and 
% confirm that 30 of them are false positives. On the other hand, 842
% benchmarks are flagged as safe by \oyente while \toolname manages to find
% their exploits. To verify the reports of our tool, we randomly select 
% 20 benmarks and confirm 18 of them are actually vulnerable. In the meantime,
% we also contacted the author of \oyente and confirmed our report.

% For the Reentrancy vulnerability, 49 benchmarks are flagged by both tools.
% 41 benchmarks are flagged as vulnerable by \oyente while \toolname cannot
% find the exploits. After manual inspection, we confirm all of them are false 
% positives. In contrast, 128 benchmarks are marked as safe but \toolname 
% successfully finds their exploits, and we manage to reproduce 102 of the attacks 
% in our testbed.

For 156 contracts flagged as vulnerable
by at least one tool, we manually determine the ground truth and summarize the results in Figure~\ref{fig:eval-oyente}. 
% As shown in Table~\ref{fig:eval-oyente-fp-fn}, for 
% the Timestamp vulnerability, the FN and FP rates of \toolname are 7\% and 10\%,
% while the FN and FP rates of \oyente on our selected data set are 36\% and 35\%.
% The result on the Reentrancy vulnerability is similar: 
The false negative (FN) and false positive (FP) rates of \toolname are 7\% and 3\%,
while the FN and FP rates of \mythril  are 26\% and 12\%.

\paragraph{Performance}
\mythril takes an average of 23 seconds to analyze a contract, 
while \toolname takes an average of 8 seconds for this data set.
\paragraph{Discussion} 
% To understand why \oyente has higher false positive and negative rates than
% \toolname, we manually inspected 20 randomly chosen samples from each category. 
% The results of this analysis are as follows.

The high false negative rate in \mythril is caused by low coverage on the
corresponding benchmarks. In the presence of large and complex
methods, \mythril fails to generate traces that trigger the vulnerability.
Moreover, \mythril does not support cross-function re-entrancy, i.e., re-entrancy attacks span over multiple functions of the
victim contract.\looseness=-1 

% The false positives in \mythril can be attributed to two root causes. The first
% is that the tool does not model the semantics of the gas system, and its query
% language cannot reason about gas consumption in a smart contract. For instance,
% \mythril will report spurious Reentrancy vulnerabilities even though the gas
% specified by the victim is insufficient for an attacker to generate the exploit.
% On the other hand, since \toolname precisely models the semantics of the gas
% system, we are able to achieve a low false positive rate. The second cause of
% false positives is due to the exploration of paths that an attacker cannot
% trigger. 
%The vulnerable functions of the second type of smart contracts
%strictly check whether the caller of them is the owner of the
%smart contract specified during contract creation. Since there is
%no way for an external account to invoke the function containing
%ether transfer, reentrant attack is also not possible. 
% For instance, \oyente marks the following code as Reentrancy vulnerability 
% even though an attacker has no permission to trigger it.
% \begin{lstlisting}[escapechar=@]
% public function mintETHRewards(
%   address _contract, uint256 _amount) 
%   @\textbf{onlyManager}@() {
%   require(_contract.call.value(_amount)());}
% \end{lstlisting}

We also investigated the cause of false positives 
reported by \toolname. It turns out that the false positives are 
caused by the imprecision of our queries. 
% Recall from Section~\ref{sec:vul}
% that we use a specific pattern of traces to \emph{overapproximate} the
In particular, we use a specific pattern of traces to \emph{overapproximate} the 
behavior of the Reentrancy attack. While effective and 
efficient in practice, our query may generate spurious 
exploits that are infeasible. To mitigate this limitation, one 
compelling approach for developing secure smart contracts is to 
ask the developers to provide invariants that the tool can use to 
rule out infeasible attacks. 
%for preventing the vulnerabilities, 
%and then use \toolname to search for exploits that violate the invariants. 
\definecolor{bblue}{HTML}{0064FF}
\definecolor{rred}{HTML}{C0504D}
\definecolor{ggreen}{HTML}{9BBB59}
\definecolor{ppurple}{HTML}{9F4C7C}

\definecolor{b1}{HTML}{BEE9E8}
\definecolor{b2}{HTML}{188FA7}

\begin{figure}[!t]
\centering
\scalebox{0.85}{
\begin{tikzpicture}
    \begin{axis}[
        width  = 8cm,
        height = 8cm,
        y=0.11cm,
        x=2.8cm,
        major x tick style = transparent,
        ybar=2*\pgflinewidth,
        bar width=24pt,
        ymajorgrids = true,
        ylabel = {Percentage \%},
        ylabel style={yshift=-4mm},
        symbolic x coords={FN,FP},
        ytick={0,10,20,30,40},
        xtick = data,
        scaled y ticks = false,
        enlarge x limits=0.6,
        ymin=0,
        ymax=40,
        legend cell align=left,
        legend entries={\toolname, \mythril},
        legend style={
                at={(0.5,1.14)},
                legend columns=-1,
                anchor=north,
        %        column sep=1ex
        }
    ]
    %\hspace*{-2mm}
        \addplot[style={ggreen,fill=ggreen,mark=none}]
        coordinates{(FN,3)(FP,7)};

        \addplot[style={ppurple,fill=ppurple,mark=none}]
        coordinates{(FN,26)(FP,12)};
    \end{axis}
\end{tikzpicture}
}
% \vspace{-0.2in}
\caption{Comparing \toolname against \mythril}
\vspace{-0.2in}
\label{fig:eval-oyente}
\end{figure}

% \begin{table}
% \centering
% \begin{tabular}{|c|c|c|c|c|}
% \hline
% \multirow{2}{*}{Vulnerability} & \multicolumn{2}{c|}{\toolname} & \multicolumn{2}{c|}{\oyente} \\ \cline{2-5} 
%                               & FP             & FN            & FP            & FN           \\ \hline
% Timestamp                      & 7\%            & 10\%            & 36\%             & 35\%            \\ \hline
% Reentrancy                     & 14\%             & 5\%           & 43\%             & 37\%            \\ \hline
% \end{tabular}
% \caption{Analysis of the results based on full inspection on 20 random samples 
% from $S \cup O$}
% \label{fig:eval-oyente-fp-fn}
% \end{table}

% \begin{table}
% \centering
% \begin{tabular}{|l|l|l|l|}
% \hline
% \multicolumn{1}{|c|}{\multirow{2}{*}{Vulnerability}} & \multicolumn{3}{l|}{Number of vulnerable contracts} \\ \cline{2-4} 
% \multicolumn{1}{|c|}{}                               & \multicolumn{1}{c|}{$S \land O$}            & \multicolumn{1}{c|}{$S - O$}           & \multicolumn{1}{c|}{$O - S$}           \\ \hline
% Timestamp   &\multicolumn{1}{c|}{485}  &\multicolumn{1}{c|}{842}   &\multicolumn{1}{c|}{39}                \\ \hline
% Reentracy   &\multicolumn{1}{c|}{49}  &\multicolumn{1}{c|}{128}   &\multicolumn{1}{c|}{41}                \\ \hline
% \end{tabular}
% \caption{Comparing \toolname ($S$) against \oyente ($O$).
% $S \land O$, $S - O$, and $O - S$ represent \# of benchmarks 
% reported by both tools, $S$ only, and $O$ only, respectively.}% on obfuscated apps}
% \label{fig:eval-oyente}
% % \vspace{-0.1in}
% \end{table}
