% !TEX root =  main.tex
\paragraph{\bf{Comparison with \teether}}\label{sec:teether}
We next compare \toolname against \teether~\cite{teether}, the most recent
tool using dynamic symbolic execution for generating exploits that would enable
the attacker to control the money transactions of a victim contract. In
particular, \teether looks for so-called \emph{critical instructions}
(i.e., \texttt{call}, \texttt{selfdestruct}, etc.) that include recipients' addresses, 
which can be manipulated by the attacker to withdraw
tokens from a vulnerable contract. 

\paragraph{Summary of results}
% The results of our evaluation are summarized in Figure~\ref{fig:plot-attack} and 
% Table~\ref{fig:eval-teether-fp-fn}. 
% In particular, as shown in Figure~\ref{fig:plot-attack}, \toolname and \teether flag
% 198 and 179 benchmarks as vulnerable, respectively. Here, the vulnerable benchmarks
% reported by \teether are a subset of the ones flagged by \toolname.
% To obtain the ground truth for those 198 contracts, we set up a private test 
% blockchain where we can deploy the contracts and validate their corresponding exploits.
% In the end, it turns out that 181 benchmarks are indeed vulnerable (i.e., true positives 
% in the second column of Table~\ref{fig:eval-teether-fp-fn}). Specifically,
% both tools maintain a low false positives with 17 in \toolname and 
% 19 in \teether. That is not very surprising because both tools are based on symbolic 
% execution and use the same query for driving the evaluation. On the other hand, 
% \toolname manages to find 21 exploits that cannot be generated by the \teether tool. 

In total, there are 198 contracts that are marked as vulnerable by at
least one tool. While \toolname covers all exploits generated by
\teether, \toolname also finds 21 \emph{extra} exploits that cannot be generated
by \teether. 

% \begin{table}[]
% \centering
% \begin{tabular}{|c|c|c|c|c|c|}
% \hline
% \multirow{2}{*}{Vulnerability}  &\multirow{2}{*}{\#TP}     & \multicolumn{2}{c|}{\toolname} & \multicolumn{2}{c|}{\teether} \\ \cline{3-6} 
%                                  &    & \#FP             & \#FN            & \#FP            & \#FN           \\ \hline
% Attack Control                 & 181  & 17            & 0            & 19             & 21            \\ \hline
% \end{tabular}
% \caption{Analysis of the results based on full inspection on 198 suspicious contracts from \etherscan}
% \label{fig:eval-teether-fp-fn}
% \end{table}

\paragraph{Performance}
\teether takes an average of 31 seconds to analyze the \etherscan data set, 
while \toolname takes an average of 8 seconds per contract.

\paragraph{Discussion} 
% To understand the cause of false positives and false negatives in \teether and
% \toolname, we manually inspected all the problematic benchmarks that lead
% to either false positives or false negatives.

The missing exploits in \teether are caused by low coverage on the
corresponding benchmarks. For the 21 benchmarks with exploits that cannot be generated by \teether, 
14 involve attack programs with four method calls, and each of the remaining 7 
benchmarks contains over 3000 lines of source code with complex control flow.
As a result,  \teether  fails to explore sufficiently many \emph{concrete traces} to 
find the exploits, even if we increase the timeout from 10 minutes to 1 hour.\looseness=-1  
% Moreover, since the Keccak-256 hash function is ubiquitous in smart contracts,
% and hard for the solver to reason about, \teether may fail to cover the code regions
% that have dependencies on the hash function. 

% The false positives in \teether can be attributed to several root causes. The major
% reason, which is also discussed in the \teether's paper, is due to the
% inconsistency of the persistent states between the initial exploit generation 
% and exploit validation. This issue can be further exacerbated if the exploits 
% depend on parts of the global state that are subject to implicit invariants 
% unknown to the exploit generator. 
% We note that \toolname also shares this limitation. To mitigate it, one 
% compelling approach for developing secure smart contracts is to 
% ask the developers to provide these invariants. % for preventing the vulnerabilities, 
% %and then use \toolname to search for exploits that violate the invariants. 
% Finally, the two extra false positives from \teether are caused by its imprecise 
% modeling of the gas system. In particular, \teether sets the gas price of all
% operations to 0, which leads to some infeasible exploits.

