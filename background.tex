% !TEX root =  main.tex
\section{Background}\label{sec:background}
% This section briefly reviews the background on blockchains and smart contracts.
We first review necessary background on smart contracts.

% \subsection{Blockchain and Ethereum}
% Blockchain, invented by Satoshi Nakamoto in 2008, is a distributed public ledger that stores transactions between different parties. A blockchain is comprised of a growing list of blocks, each of which contains the hash of the previous block, a timestamp when the block is appended, and transaction value. Due to the decentralized consensus protocol, each block is inherently resistant to modification once it is created.\looseness=-1 

% While Satoshi's original blockchain proposes a peer to peer e-cash system that offers secure transactions, the Ethereum~\cite{yellowpaper} blockchain provides a more powerful distributed computing platform that can execute custom code in the form of \emph{smart contracts}. In addition to the crypto tokens (i.e., Ether) that are transferred among parties during a transaction, Ethereum also implements a \emph{gas} scheme (explained in Section~\ref{sec:abi}) to incentivize \emph{miners} who perform the computationally expensive creation of new blocks. 
% Gas is simply a small amount of cryptocurrency paid to the miners to perform the computation. A user specifies the gas along with the executable code (i.e., smart contracts) during a transaction.

% \subsection{Smart Contract}
\paragraph{Smart Contract.}Smart contracts are programs that are stored and
executed on the blockchain. They are created through the transaction system on
the blockchain and are immutable once deployed. Each smart contract is
associated with a unique 160-bit address; a private persistent storage; a
certain amount of cryptocurrency, expressed as a balance (i.e., Ether in
Ethereum) held by the contract; and a piece of executable code that fulfills
complex computations to manipulate the storage and balance. The code is
typically written in a high-level Turing-complete programming language such as
Serpent~\cite{serpent}, Vyper~\cite{vyper}, and Solidity~\cite{solidity}, and
then compiled to the Ethereum Virtual Machine (EVM) bytecode~\cite{yellowpaper},
a low-level stack-based language. For instance, Figure~\ref{fig:motivate} shows
two smart contracts written in the Solidity programming language~\cite{solidity}.

% \subsection{ABI and Transactions}\label{sec:abi}
\paragraph{Application Binary Interface.}
In the Ethereum ecosystem, smart contracts communicate with each other using the
Contract Application Binary Interface (ABI), which defines the signatures of
public functions provided by the hosted contract. While ABI offers a flexible
mechanism for communication, it also creates an attack surface for exploits that
use  the ABI of a given smart contract. 
%We will elaborate on this in the following section. For instance, Figure~\ref{fig:attack-abi} shows the ABI for the smart contract in Figure~\ref{fig:attack-vic}.
% \begin{table}[]
% \centering
% \begin{tabular}{|l|l|}
% \hline
% ...         & ...                                                                \\ \hline
% From:       & 0x7d5c8c59837357e541bc7d87dee53fcbba55ba65                         \\ \hline
% To:         & 0x8811fffcfc266844e8c36418389f7cda76c77ab7 \\ \hline
% Value:      & 0.05 Ether                                                         \\ \hline
% Gas Limit:  & 31602                                                              \\ \hline
% Input Data: & 0x687474703a2f2f6c6f63616c686f73743a38353435                       \\ \hline
% \end{tabular}
% \caption{A sample transaction~\cite{sample-trans} obtained from \etherscan}
% \label{fig:trans-sample}
% \end{table}

% All interactions between smart contracts are fulfilled 
% by transactions. 
% Table~\ref{fig:trans-sample} shows a sample transaction obtained
% from \etherscan. Here, the important fields are \texttt{From},
% \texttt{To}, \texttt{Value}, \texttt{Gas Limit}, and \texttt{Input Data}. In 
% particular, \texttt{From} and \texttt{To} represent the sender and recipient, respectively.
% \texttt{Value} denotes the amount transferred from one smart contract to another. 
% \texttt{Input Data} contains the function's signature (obtained from the ABI) and its arguments. 
% Finally, the \texttt{Gas Limit} field specifies the amount of cryptocurrency which a miner 
% gets for conveying the transaction. The Ethereum protocol~\cite{yellowpaper} defines
% the gas cost for each bytecode instruction. For instance, an integer division operation 
% costs 5 units of gas while a store operation on the storage can cost up to 20000.
% As we will see in Section~\ref{sec:eval}, 
% the gas mechanism plays a key role in several different types of vulnerabilities.


% \subsection{Threat Model}
\paragraph{Threat Model.} To synthesize an adversarial contract,  we assume that
we can obtain the victim contract's bytecode and the ABI specifying its public
methods. To confirm an adversarial contract is indeed an exploit, we must also
be able to invoke public methods by submitting transactions over the Ethereum
Blockchain. These requirements are easy to satisfy in practice.
