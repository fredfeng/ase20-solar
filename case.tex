\subsection{A case study on the BatchOverflow vulnerability}\label{sec:case}

To evaluate whether \toolname can express and discover new vulnerabilities in real world
smart contracts, we conduct a case study on the recent \batchoverflow
vulnerability. Exploits
due to this vulnerability have resulted in the creation of trillions of invalid
Ethereum Tokens in 2018~\cite{batch-news}, causing major exchanges to temporary
halt until all tokens could be reassessed. Note that generating exploits for
this vulnerability is quite challenging as it requires the tool to reason about
the combination of arithmetic operations, interference, and the read-write
semantics of the storage system in Solidity. 
% For instance, existing tools such
% as \oyente and \madmax~\cite{madmax} will simply mark a large number of arithmetic
% operations as \emph{potentially vulnerable}, and it turns out that most of the
% alarms are not exploitable.

Similar to our previous experiment, we first encode the \batchoverflow vulnerability
(Section~\ref{sec:vul}) in our query language and then run our tool on the 
\etherscan data set. In total, \toolname flags 16 vulnerable
contracts. To verify that the exploits are effective, we setup a private
blockchain using the Geth~\cite{geth} framework where we can run exploits on the
vulnerable contracts. We confirmed that 9 exploits are valid. 
The infeasible attacks come from the incompleteness of the query 
as well as imprecise control flow graphs from the Vandal decompiler. 
Running \teether on these 9 vulnerable contracts, we find 
that it fails to generate their exploits.

% \todo{To evaluate the effectiveness of \toolname on vetting the \batchoverflow vulnerability, we compare \toolname against \teether~\cite{teether} tool, the most recent
% tool using dynamic symbolic execution for generating exploits that would enable
% the attacker to control the money transactions of a victim contract. In
% particular, the \teether tool looks for so-called \emph{critical instructions}
% (i.e., \texttt{call}, \texttt{selfdestruct}, etc.) that include recipients' addresses, 
% which can be manipulated by the attacker. In the end, \teether takes an average of 31 seconds to analyze the \etherscan data set but fails to generate exploits for those 9 vulnerable contracts. The false negative rate in \teether is caused by  
% attack programs that require more than three method calls, or victim programs with over 3000 lines of source code with complex control flow.
% As a result, the \teether tool fails to explore sufficiently many \emph{concrete traces} to 
% find the exploits.}  
